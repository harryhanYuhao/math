\documentclass[12pt, a4paper]{article}
\usepackage{blindtext, titlesec, amsthm, thmtools, amsmath, amsfonts, scalerel, amssymb, graphicx, titlesec, xcolor, multicol, hyperref, mathrsfs}
\usepackage[left=0.4in, right=0.4in, top=0.3in, bottom=0.6in]{geometry}
\usepackage[utf8]{inputenc}
\hypersetup{colorlinks,linkcolor={red!40!black},citecolor={blue!50!black},urlcolor={blue!80!black}}
\newtheorem{hypothesis}{Coniectura}
\theoremstyle{definition}
\newtheorem{theorem}{Theorema}[section]
\newtheorem{definition}{Definitio}[section]
\theoremstyle{remark}
\newtheorem{remark}{Observatio}[section]
\newtheorem{example}{Exampli Gratia}[section]
\newcommand{\bb}[1]{\mathbb{#1}}
\renewcommand\qedsymbol{Q.E.D.}
\renewcommand{\emph}[1]{\textbf{\textit{#1}}}
\newcommand{\dga}{\Omega^{\bullet}}
\renewcommand{\L}{\mathscr{L}}

\begin{document}
{\huge{Differential Geometry}}

\section{Notation}
\begin{multicols}{2}
	\begin{enumerate}
		\item $\mathscr{X}(M)$: set of all smooth vector fields on $M$.
		\item $\mathscr{T}^r_s$ : a $(r,s)$ tensor bundle (tangent space $s$ and cotangent space $r$).
		\item $C(M)$ smooth maps on $M$.
		\item $\Gamma(E)$: set of all sections on $E$, while $E$ is a vector bundle on $M$.
	\end{enumerate}
\end{multicols}
	

\section{Manifolds and Smooth Structure}

\begin{definition}[Manifold]
	An $n$ dimensional smooth manifold is a set $M$ with a atlas $\{(U_a, \phi_a)\}$, where $M = \bigcup U_a$ and $\phi_a: U_a \rightarrow R^n$ is a bijection into a open set of $R^n$. Moreover, $\phi_a \circ \phi_b^{-1}$ is a diffeomorphism, that is, smooth with smooth inverse.

	A manifold is \emph{compact} if every atlas has a finite subatlas.
\end{definition}

\begin{theorem}
	$F: R^n \rightarrow  R^m$ is a smooth, and $0 \in R^m$ is \emph{regular value}, that is, $DF_p: R^n \rightarrow R^m$ is surjective (full ranked) with $F(p) = 0$. Then $F^{-1}(0)$ is a $(n-m)$ dimensional manifold and a embedded submanifold of $R^n$.
\end{theorem}

\begin{example} [Calculation Jacobian] \label{eg:Jacobian_Matrix}
	Let $F: M^n \rightarrow N^m$ be a smooth map, $x_i$ local coordinates of $M$, and $y_i$ local coordinate of $N$.

	\[ 
		F_* \frac{\partial }{\partial x^i}\bigg\rvert_p 
		= \frac{\partial F^j}{\partial x^i} \frac{\partial }{\partial y^j}\bigg\rvert_{F(p)}
		,
		F_* = 
		\begin{bmatrix}
			\frac{\partial F^1}{\partial x^1}(p) & \cdots & \frac{\partial F^1}{\partial x^n}(p) \\
			\vdots & \ddots & \vdots \\ 
			\frac{\partial F^m}{\partial x^1}(p) & \cdots & \frac{\partial F^m}{\partial x^n}(p) \\
		\end{bmatrix}
	\]
\end{example}

\begin{definition}[Smooth map]
	$F: M^M \rightarrow N^n$ is a smooth map if for each $p \in M$ with corresponding chart $(U, \phi)$, $F(p) \in N$ with corresponding chart $(V, \psi)$ we have $\psi \circ F \phi^{-1}: \phi^{-1}(U \cap V) \rightarrow \psi(U \cap V)$ is smooth as function $R^m \rightarrow R^n$.

	A smooth map is a diffeomorphism if it is bijective with smooth inverse.
\end{definition}

\begin{theorem}[C(M) as rings]
	$C(M)$ forms a ring. For smooth map $F: M \rightarrow N$, $F^*: C(N) \rightarrow C(M)$ defined as $F^* f = f\circ F$ is a ring homomorphism.
\end{theorem}

\subsection{Submersion, immersion and Embedding}

\begin{definition}[Immersion, Submersion, Embedding]
	Let $F: M^m \rightarrow N^n$ be a smooth map.
	It is an immersion of $F_*$ is injective, submersion if $F_*$ is surjective. $F$ is an embedding if it is an immersion and a homeomorphism onto its image, which is a submanifold of $N$.
\end{definition}

\begin{theorem}
	Let $F: M^m \rightarrow N^n$ be a smooth map. A point $c \in N$ is a regular value of $F$ if $(F_*)(p)$ is surjective (full ranked and $m > n$) for all $p \in F^{-1}(c)$.
	In such case, $F^{-1}(c)$ is an embedded submanifold of $M$ with dimension $m-n$, and $T_pF^{-1}(c) = ker(F_*)_p \subset T_pM$. (The tangent space)
\end{theorem}


\section{Derivatives and Tangent Spaces}

\begin{definition}[Short Hand Notation For Derivative]
	Let $M$ be a manifold with chart $\phi$ at $p$ with local coordinates $x_i$. 
	At $p, M$ has a tangent space $T_pM$, with basis $\frac{\partial}{\partial x^i}\big\rvert_p = (\phi^{-1})_* \frac{\partial }{\partial x^i}\big\rvert_{\phi(p)}$. 
\end{definition}

\begin{theorem}[Change of coordinates]
	Let $x_i$ and $y_i$ be two local coordinates at $p$. Then, 
	$$
	\frac{\partial}{\partial y^i}\Big\rvert_p = \sum \frac{\partial x^j}{\partial y^i}(\tilde{p})\frac{\partial}{\partial x^j}\Big\rvert_p
	$$
\end{theorem}

\begin{theorem}[Push Forward]
	Let $F: M \rightarrow N$ be a smooth map. Then, $F_*: T_pM \rightarrow T_{F(p)}N$ is a linear map, defined as $F_*X_p(f) = X_p(f \circ F)$. 
	Moreover
	\begin{enumerate}
		\item $F_*$ is $C(M)$ linear 
		\item $(G \circ F)_* = G_* \circ F_*$ 
		\item $Id_* = Id$
		\item If $F$ is a diffeomorphism, then $F_*$ is an isomorphism.
	\end{enumerate}
\end{theorem}

\begin{definition}[F related]
	$F: M \rightarrow N$. Vector fields $X \in \mathscr{X}(M), Y \in \mathscr{X}{N}$ is $F$ related if $Y_{F(p)} = F_*X_p$.
\end{definition}

\begin{theorem}[F related]
	$X \in \mathscr{X}(M)$ is $F$ related to $Y$ iff for all functions $f \in C(N)$
	$$	
	X(f \circ F) = (Yf) \circ F
	$$

\end{theorem}

\subsection{Vector Bundles}

\begin{definition}[Vector Bundle]
	$E$ is a real vector bundle of rank $k$ over $M$ if there is a smooth surject submersion/projection map $\pi: E \rightarrow M$with the property that 1) $\pi^{-1}(c) \cong R^k$ 2) for each $p \in M$, there is a neighborhood $U$ of $p$ and a diffeomorphism $\phi: \pi^{-1}(U) \rightarrow U \times R^k$.

	This diffeomorphism is called a trivilisation.
\end{definition}

\begin{theorem}[Transition Function]
	Let $E$ be a vector bundle over $M$ with two trivilisation $(U, \phi), (V, \psi)$.
	Then the transition function $g_{UV} U \cap V \rightarrow GL(k, R)$ is defined as $g_{UV} = \phi \circ \psi^{-1}$.

	For the tangent bundle, the transition function is $g_{UV}$ is the Jacobian matrix of the change of coordinates. 
For cotangent bundle, the transition function is $(g_{UV}^{-1})^T$
\end{theorem}

\begin{theorem}[Cocycle Condition]
	Bundle maps satisfy the cocycle condition 
	\begin{enumerate}
		\item $g_{aa} = 1$
		\item $g_{ab}g_{ba} = 1$
		\item $g_{ab}g_{bc}g_{ca} = 1$
	\end{enumerate}

	A vector bundle can be reconstructed from the bundle map satisfying cocycle condition.
\end{theorem}

\begin{theorem}[Bundle Map]
	$E \rightarrow^p M$ is a vector bundle, $F \rightarrow^q N $ is a vector bundle, $\psi: M \rightarrow N$ is a smooth map. A bundle map $\Psi: E \rightarrow F$ is a smooth linear map such that $\Psi \circ q = p \circ \psi$.
	We say $\Psi$ covers $\psi$.
\end{theorem}

\subsection{Section}
	
\begin{definition}[Smooth Section]
	A smooth section of $E$ is a smooth map $s: M \rightarrow E$ s.t. $\pi \circ s = id$. 
	Let $\Gamma (E)$ denote the set of all smooth section.
	$\Gamma(E)$ is a real vector space.

	For $f \in C(M)$, define $fs \in \Gamma(E)$ as $fs(p) = f(p)s(p)$.
\end{definition}

\begin{definition}[Local Frame]
	Let $E \rightarrow M$ be a vector bundle. 
	Let $U \subset M$ be an open subset. 
	Local sections $\sigma_i$ of $E$ is an \emph{independent} section if their values $\sigma_i(p)$ are linearly independent for all $p \in P$. These sections is called a \emph{local frame}.
	If $U = M$, it is a \emph{global} frame. 
\end{definition}

\begin{theorem}[Frames]
	A bundle map is trivil, i.e., admitting global trivilisation iff there exists a global frame. 

	Every smooth local frame is associated with a smooth local frame and vice versa.
\end{theorem}
\section{Flows}

\begin{definition}[One-parameter group of diffeomorphism]
	A \emph{One parameter group of diffeomorphism} is a smooth map $\phi(t, p) = \phi_t(p): R \times M \rightarrow M$ such that $\phi_t \circ \phi_s = \phi_{t+s}$ and $\phi_0 = Id$, and $\phi_t$ is a diffeomorphism for all $t$.
\end{definition}

\begin{definition}[Integral Curve]
	An integral curve of a vector field $X$ is a smooth curve $\gamma: I \rightarrow M$ such that $\gamma'(t) = \gamma_*(\frac{d}{dt})\rvert_p = X_{\gamma(t)}$.
\end{definition}

\begin{definition}[Local Flow]
	If a one-parameter group of diffeomorphism $\phi_t$ exists that is the integral curve of a vector field $X$, then $X$ is a \emph{local flow} of $\phi_t$. 
	If this map is defined for all $t \in R$, it is a global flow.
\end{definition}

\begin{example}
	Let $M = R^2$ and $X = \frac{\partial }{\partial x}$. Assuming the flow is $g(t) = (x(t), y(t))$, then we have 
	\[
		g'(t) = g_*(\frac{d}{dt}) = \frac{dx}{dt} \frac{\partial }{\partial x} + \frac{dy}{dt} \frac{\partial }{\partial y} =  \frac{\partial }{\partial x}
		\]
	Solve the ODE, we have $x(t) = x_0 + t, y(t) = y_0$.
\end{example}

\section{Pull Back}

\begin{definition}[Pull Back]
	Let $F, G: M \rightarrow N$ be a smooth map between manifolds. 
	There exist push forward $F_*: T_pM \rightarrow T_{F(p)}N$. The Pull back, $F^*: T_{F(p)}^*N \rightarrow T_{p}^*M$, is the dual of push forward.
	Specifically, for $\omega \in T_{F(p)}^*N, X \in T_pM$,
	$$
	F^*(\omega) X = \omega(F_*X)
	$$

	For $\gamma \in \Omega^1(N)$, we can define the pull back of the one form (covector field) as $(G^* \gamma)_p = G^*(\gamma_{G(p)})$.

\end{definition}

\begin{theorem}[Properties of Pull back]
	Let $f \in C(M)$. 
	\begin{enumerate}
		\item Pull backs are dga morphism. 
		\item Pull backs are $C(M)$ linear.
		\item Pull backs are contravariant, i.e., $(F \circ G)^* = G^* \circ F^*$.
		\item $G^*df = dG^*f =d(f \circ G)$
		\item $G^*(f \omega) = (f \circ G)G^*\omega$
		\item For $k$-forms, the pull back is $F^*(\omega^1 \wedge \cdots \wedge \omega^k) = (F^*\omega^1)\wedge \cdots \wedge (F^*\omega^k)$.
	\end{enumerate}
\end{theorem}

\begin{example}
	$G^*\omega = (\omega_j \circ G)dG^j$. 

	$(u,v) = G(x,y,z) = (x^2y, y\sin{z}), \omega = udv + vdu$. 
	We have 
	$$
	G^*\omega = (x^2y)d(y\sin{z}) + (y\sin{z})d(x^2y) 
	$$

\end{example}


\section{DGA}

\begin{definition}[Differential]
	The differential of a smooth function $f$ on manifold $M$ is a covector field $df_p(X_p) = X_pf, X_p \in T_pM$.
\end{definition}

\begin{example}[Calculating Differential]
	Let $f$ be a smooth function on $M$ with $x_i, 1\leq i \leq n$ as its local coordinate. 
	$df = \frac{\partial f}{\partial x_i}dx^i$.

	Let $f(x,y,z) = x^2y + y\sin{z}$, then $df = 2xydx + x^2dy + \sin{z}dy + y\cos{z}dz$.
\end{example}

\begin{definition}[Differential Forms]
	Differential 1-form on manifold $M$ is a section of the cotangent bundle, i.e., a covector field, denoted as $\Omega^1(M)$.
	
	There is a canonical map $\omega: \mathscr{X}(M) \rightarrow  C(M)$, defined as $\omega(X) = \omega(X_p)$.

	A differential $k$-form is a section of the $k$-th exterior power of the cotangent bundle, denoted as $\Omega^k(M)$.
	That is, $\omega \in \Omega^k(M)$ is a map $\omega: \mathscr{X}(M)^k \rightarrow C(M)$.
\end{definition}


\begin{theorem}[Properties of Differential Form]
	Let $\alpha = \sum \alpha_Idx^I \in \Omega^k(U), \beta \in \Omega^l(U)$, $f \in C(N), F: M \rightarrow N$
	\begin{enumerate}
		\item $\alpha \wedge \beta = (-1)^{kl}\beta \wedge \alpha \in \Omega^{k+l}(U)$;
		\item $d \alpha = \sum d \alpha_I \wedge dx^I$,
		\item $d \alpha \wedge \beta = d \alpha \wedge \beta + (-1)^k \alpha \wedge d \beta$
		\item For all $\alpha \in \Omega^k(U), dd \alpha = 0$
			% TODO: Did not get the following 
		\item Let $F: R^m \rightarrow U$ be a smooth map. $dF^* \alpha = F^* d \alpha$ \emph{?}.
		\item $F^*df = d(F^*f)$, where $F^*f = f \circ F$.
		\item $df_p(X_p) = X_p f$
	\end{enumerate}
\end{theorem}

\begin{definition}[DGA]
	A \emph{Differential graded algebra} $(\Omega^{\bullet}, \wedge, d)$ satisfies 
	\begin{enumerate}
		\item $\wedge: \Omega^k(U) \times \Omega^l(U) -> \Omega^{k+l}(U)$,
		\item $\alpha \in \Omega^k(U), \beta \in \Omega^l(U)$, $\alpha \wedge \beta = (-1)^{kl}\beta \wedge \alpha$.
		\item $d: \Omega^{l}(U) \rightarrow \Omega^{l+1}(U)$, $d(\alpha \wedge \beta) = d \alpha \wedge \beta + (-1)^k \alpha \wedge d \beta$ ($\alpha$ is $k$ form).
		\item $dd = 0$
	\end{enumerate}

	A smooth map $F: R^m \rightarrow U$ is a dga morphism, as we have the pullback $F^*: \Omega^{\bullet}(U) \rightarrow  \Omega^{\bullet}(R^m)$ and $F^*(\alpha \wedge \beta) = (F^* \alpha)$
\end{definition}

\section{Lie Derivative}

\begin{definition}[Lie Derivative]
If $\phi_t$ is a local flow of $X$. For every $f \in C(M)$ we have 
$$
	X(f) = \frac{d}{dt}f \circ \phi_t \bigg\rvert_{t=0}
$$

We define Lie derivative of a vector field $Y$ along $X$ as $\mathscr{L}_XY = [X, Y]$.

For a $k$ form $\alpha \in \Omega^k(M)$, define 
$$
	\mathscr{L}_X \alpha = \frac{d}{dt} \phi_t^* \alpha \bigg\rvert_{t=0}
$$
\end{definition}

\begin{definition}[Derivations]
	An $R$ linear map $D: \dga(M) \rightarrow \dga(M)$ is a degree-k derivation if $D: \Omega^{l}(M) \rightarrow  \Omega^{l+m}(M)$, it satisfies the Leibniz rule, and $D(\alpha \wedge \beta) = D \alpha \wedge \beta + (-1)^{kl} \alpha \wedge d \beta$.
\end{definition}

\begin{theorem}[Lie Bracket]
	let $D_1 \in Der_k(M), D_2 \in Der_l(M)$. 
	$$
	[D_1, D_2] = D_1 \circ D_2 - (-1)^{kl}D_2 \circ D_1 \in Der_{k+l}(M)
	$$
\end{theorem}

\begin{theorem}
	$d \in Der_1(M)$, $\mathscr{L} \in Der_0(M)$, 
	\begin{enumerate}
		\item Lie bracket if $R$ linear but not $C(M)$ linear. $[X, fY] = X(f)Y + f[X, Y]$,
		\item Leibniz Rules: $\L(X \otimes Y) = \L X \otimes Y + X \otimes \L Y$
		\item $\L_X f = X(f)$
		\item $\L_XY = [X, Y]$
		\item $\L_X \alpha (Y) = \L_X (\alpha Y) - \alpha(\L_{X}Y) = X(\alpha(Y)) - \alpha([X,Y])$
		\item $\L \circ d = d \circ \L$
		\item $[d,[d, D]] = 0$ for any $D \in Der_k$
		\item $i_X \circ i_X = 0$
		\item $i_{fX} = fi_X $
		\item $\L_X = [d, i_X] \iff \L_X \alpha = i_X d \alpha + di_X \alpha$
		\item $[\L_X, i_Y] = i_{[X, Y]} \iff \L_X i_Y \alpha = i_Y \L_x \alpha + i_{[X, Y]} \alpha$
		\item $[\L_x, \L_y] = \L_{[X, Y]} \iff  \L_x \L_y  \alpha= \L_x \L_y \alpha +  \L_{[X, Y]} \alpha $
	\end{enumerate}
\end{theorem}
\begin{definition}[Closed and Exact]
	A form $\alpha$ is \emph{closed} if $d \alpha = 0$, and \emph{exact} if $\alpha = d \beta$ for some $\beta$.
\end{definition}

\begin{theorem}[Lie Derivative On vector bundles]
	Let $T \in \mathscr{T}^r_s(M)$ and $X \in \mathscr{X}(M)$.
	$\L_X(T) \in \mathscr{T}^r_s(M)$ is defined as, for $Y_i \in \mathscr{X}(M), \alpha_i \in \omega^1(M)$ 
	\begin{equation}
		\begin{split}
			(\L_XT)(Y_1, \cdots, Y_s, \alpha_1, \cdots, \alpha_r) &= XT(Y_1, \cdots, Y_s, \alpha_1, \cdots, \alpha_r) \\ 
			&- \sum T(Y_1, \cdots, [X, Y_i], \cdots, Y_s, \alpha_1, \cdots, \alpha_r) \\ 
			&- \sum T(Y_1, \cdots, Y_s, \alpha_1, \cdots, \L_X\alpha_i, \cdots, \alpha_r)
		\end{split}
	\end{equation}
\end{theorem}

\end{document}

