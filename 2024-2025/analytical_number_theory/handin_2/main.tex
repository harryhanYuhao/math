\documentclass{article}

% \usepackage[tmargin=2.5cm,rmargin=3cm,lmargin=3cm,bmargin=3cm]{geometry} 
% Top margin, right margin, left margin, bottom margin, footnote skip
\usepackage[utf8]{inputenc}
\usepackage{biblatex}
\addbibresource{./references.bib}
% linktocpage shall be added to snippets.
\usepackage{hyperref,theoremref}
\hypersetup{
	colorlinks, 
	linkcolor={red!40!black}, 
	citecolor={blue!50!black},
	urlcolor={blue!80!black},
	linktocpage % Link table of content to the page instead of the title
}

\usepackage{blindtext}
\usepackage{titlesec}
\usepackage{amsthm}
\usepackage{thmtools}
\usepackage{amsmath}
\usepackage{amssymb}
\usepackage{graphicx}
\usepackage{titlesec}
\usepackage{xcolor}
\usepackage{multicol}
\usepackage{hyperref}
\usepackage{import}


\newtheorem{theorem}{Theorema}[section]
\newtheorem{lemma}[theorem]{Lemma}
\newtheorem{corollary}{Corollarium}[section]
\newtheorem{proposition}{Propositio}[theorem]
\theoremstyle{definition}
\newtheorem{definition}{Definitio}[section]

\theoremstyle{definition}
\newtheorem{axiom}{Axioma}[section]

\theoremstyle{remark}
\newtheorem{remark}{Observatio}[section]
\newtheorem{hypothesis}{Coniectura}[section]
\newtheorem{example}{Exampli Gratia}[section]

% Proof Environments
\newcommand{\thm}[2]{\begin{theorem}[#1]{}#2\end{theorem}}

%TODO mayby proof environment shall have more margin
% \renewenvironment{proof}{\vspace{0.4cm}\noindent\small{\emph{Demonstratio.}}}{\qed\vspace{0.4cm}}
% \renewenvironment{proof}{{\bfseries\emph{Demonstratio.}}}{\qed}
\renewcommand\qedsymbol{Q.E.D.}
% \renewcommand{\chaptername}{Caput}
% \renewcommand{\contentsname}{Index Capitum} % Index Capitum 
\renewcommand{\emph}[1]{\textbf{\textit{#1}}}
\renewcommand{\ker}[1]{\operatorname{Ker}{#1}}

%\DeclareMathOperator{\ker}{Ker}

% New Commands
\newcommand{\bb}[1]{\mathbb{#1}} %TODO add this line to nvim snippets
\newcommand{\orb}[2]{\text{Orb}_{#1}({#2})}
\newcommand{\stab}[2]{\text{Stab}_{#1}({#2})}
\newcommand{\im}[1]{\text{im}{\ #1}}
\newcommand{\se}[2]{\text{send}_{#1}({#2})}
\renewcommand{\hat}[1]{\widehat{#1}}
\newcommand{\G}{\mathcal{G}}
\newcommand{\g}{\mathfrak{g}}
\newcommand{\hG}{\hat{G}}
\newcommand{\Log}{\text{Log}}


\begin{document}

\section{Q1}

\subsection{Part 1} 
Claim: $L(z; \chi)$ is holomorphic and never vanishes for $z \in X, $ where $X \:= \{z: \Re(z) > 1 \}$.

\begin{proof}
	Define $t_n$ and $P_n$ as the $n$th term and the $n$th partial sum of $L(z; \chi)$ respectively, i.e., $t_n = \frac{\chi(n)}{n^z}$, $P_n = \sum_{k=1}^{n} t_k$.
	Note
	$$
	|t_n| = \left|\frac{\chi(n)}{n^z} \right| \leq \frac{1}{n^{\Re(z)}}
	$$

	$\Re(z) > 1$ for all $z \in X$, so $\sum_{n=1}^{\infty} \frac{1}{n^{\Re(z)}}$ converges absolutely by $p$ test. 
And by Weierstrass M-test, $P_n$ converges uniformly to $L(z; \chi)$ on $X$.
The fact that $|L(z, \chi)| \leq \sum_{n=1}^{\infty} \frac{1}{n^{\Re(z)}}$ also shows that $L(z, \chi)$ converges absolutely for all $z \in X$.

Each $t_n$ is holomorphic, so each $P_n$ as finite sum of holomorphic functions is holomorphic. 
As $P_n$ converges uniformly to $L(z; \chi)$ for $z \in X$, $L(z; \chi)$ is holomorphic.

The fact that $\chi$ and is completely multiplicative leads $f(n) = \frac{\chi(n)}{n^z}$ to be completely multiplicative, as 

\begin{equation}\label{eq:completely_multiplicative}
	f(m)f(n) = \frac{\chi(m)}{m^z} \frac{\chi(n)}{n^z} = \frac{\chi(m)\chi(n)}{(mn)^z} = \frac{\chi(mn)}{(mn)^z} = f(mn)
\end{equation}

As $f(n)$ is multiplicative and $P_n$ converge absolutely for all applicable $z$, by proposition 2.3 from the lecture notes $L(z; \chi) = \prod_{p} \frac{1}{1- \chi(p)p^{-z}} = \prod_p (1 + u_p)$ where $u_p = \frac{\chi(p)p^{-z}}{1 - \chi(p) p ^{-z}}$. 
If $u_p = -1$, $\chi(p) p ^{-z} = -1 +  \chi(p) p ^{-z} \implies -1 = 0$, which is absurd, so $u_p \neq -1 $ for all $p$ and $z \in X$. 
Applying proposition 2.1 from lecture notes, which states that $\prod_{n=1}^{\infty} (1+a_n) = 0$ iff some $a_n = -1$, we conclude $L(z; \chi) \neq 0 $ for all $z \in X.$
\end{proof}

\subsection{Part 2}

We want to prove that 

\begin{equation}
	\frac{L'(z; \chi)}{L(z; \chi)} = - \sum_p \frac{\chi(p) \log(p)}{p^z - \chi(p)}
\end{equation}
where $z \in X$.

As shown in lemma \ref{lm:log_to_Log} the following defines a branch of logarithm for $L(z; \chi)$ such that 
\begin{equation}\label{log_l}
\log(L(z; \chi)) = - \sum_p \Log(1-\chi(p)p^{-z})
\end{equation}

and this series converges uniformly in the desired domain.
So the derivative of the LHS of \ref{log_l} equals to the the derivative of the series summed up terms by term. 

Let us calculate the derivative 

\begin{align}
	\frac{L'(z; \chi)}{L(z; \chi)}
	&=\frac{d}{dz} \log(L(z; \chi)) 
	\\&= \frac{d}{dz} \left(- \sum_p \Log(1-\chi(p)p^{-z})\right) 
	\\&=  - \sum_p  \frac{d}{dz} \Log(1-\chi(p)p^{-z}) 
	\\&=  - \sum_p \frac{\chi(p) \log(p) p^{-z}}{1-\chi(p) p^{-z}} 
	= - \sum_p \frac{\chi(p) \log(p)}{p^z - \chi(p)}
\end{align}

in calculating the derivative we used that fact that $\chi(p)$ is a constant independent of $z$ and $\frac{d}{dz}p^{z} = \log(p) p^z$.
This gives the desired equality.

\begin{lemma}\label{log_ineq}
	For $|w| < \frac{1}{2}$, $|\Log(1 - w)| < 2|w|$
\end{lemma}

\begin{proof}
	This inequality follows from the manipulations of the terms of the taylor expansions of $\Log$. 

	\begin{align}
		|\Log(1-w)| 
		&= \left|\sum_{n=1}^{\infty}\frac{w^n}{n}\right| \\
		&\leq \sum_{n=1}^{\infty}\left|\frac{w^n}{n}\right| \\
		&< \sum_{n=1}^{\infty}\left|w^n\right| \\
		&< \sum_{n=0}^{\infty}|w|(\frac{1}{2})^n \\
		&= 2 |w|
	\end{align}

	Also note that the Taylor series for $\Log(1-w)$ converges absolutely for $-1< |w| < 1$, so all the manipulations make sense.
\end{proof}

% TODO: 
\begin{lemma}\label{lm:log_to_Log}
The following equation defines a branch of logarithm for $L(z; \chi)$ such that 
\begin{equation}\label{log_1}
	\log(L(z; \chi)) = - \sum_p \Log(1-\chi(p)p^{-z})
\end{equation}

and it converges uniformly for $z \in X$.

\end{lemma}

\begin{proof}
	As shown in part one of question 1 $f(n) = \frac{\chi(n)}{n^{z}}$ is completely multiplicative and the function $L(z; \chi)$ defined as the series $\sum_{n >0} f(n)$ converges absolutely for $z \in X$. 
	Applying lemma 2.3 from the lecture notes, $L(z; \chi) = \prod_{p} \frac{1}{1- \chi(p)p^{-z}} = \prod_p (1 + u_p)$ where $u_p = \frac{\chi(p)p^{-z}}{1 - \chi(p) p ^{-z}}$.

	Define $\log(P_n) = \sum_{p \in \{\text{first n primes}\}} \Log(1 + u_p)$. 
	Note 
	
	\begin{align*}
		\exp(\log(P_n))
		=& \prod_{ {p \in \{\text{first n primes}\}}} \exp(\Log (1+ u_p))\\
		=& \prod_{ {p \in \{\text{first n primes}\}}}  (1+ u_p)
	\end{align*}

	Thus by continuity of $\exp$ and $\log$ functions, 
	\begin{align*}
		L(z; \chi) 
		= & \prod_{p} (1 + u_p) \\
		=& \lim_{n \rightarrow \infty} \prod_{ p \in \{\text{first n primes}\} } 1 + u_p \\
		=& \lim_{n \rightarrow \infty} \exp(\log(P_n)) \\
		=& \exp(\lim_{n \rightarrow \infty} \log(P_n)) \\
		=& \exp(\lim_{n \rightarrow \infty} \sum_{p \in \{\text{first n primes}\}} \Log(1 + u_p)) \\
		=& \exp(\sum_{p} \Log(1 + u_p)) \\
		=& \exp( - \sum_{p} \Log(1 - \chi(p)p^{-z})) \\
	\end{align*}

	So $- \sum_{p} \Log(1 - \chi(p)p^{-z})$ is indeed a branch of logarithm for $L(z; \chi)$ and equation \ref{log_1} holds.

	To show \ref{log_1} converges uniformly, note that all prime $p \geq 2$, $|\chi(p) p^{-z}| \leq | p^{-z}| < \frac{1}{2}$ while $z \in X$.
	So $|\Log(1- \chi(p)p^{-z})| \leq 2|\chi(p)p^{-z}| \leq 2 |p^{-z}| $ by lemma \ref{log_ineq}.
	
	Since $0 < \sum_{p} 2|p^{-z}| < 2\sum_{n \geq 1} n^{-z} < \infty$ for all $z \in X$, by Weierstrass M-test the series $-\sum_{p} \Log(1-\chi(p)p^{-z})$ converges uniformly for $z \in X$.
\end{proof}

\section{Q2}

$G$ is finite abelian group. I claim $\#\hat{G} \leq \#G $.

Label the set of all maps $G \rightarrow \bb{C}$ as $\G$. 

We shall prove that $\G$ is a vector space over $\bb{C}$ whose dimension is $\#G$, and $\hG$ is a set of linearly independent vectors of this vector space, and our claims ensues.

\begin{proof}[Proof: $\G$ is a vector space over $\bb{C}$]

	Let $u, v \in \G$. 
	Define $ u + v: G \rightarrow \bb{C}$ as $(u + v)(g) = u(g) + v(g)$.
	This makes $\G$ an abelian group under addition, with identity element being the zero map, $i_{\G} (g) = 0$ for all $g \in G$, and inverse of $u$ being $-u$ defined as $(-u)(g) = -u(g)$.
	Closure of group is obvious. 
	The associtivity follows as addition in $\bb{C}$ is associtive.

	For $c \in \bb{C}, g\in G, u \in \G$, define the scalar multiplication $cu: G \rightarrow \bb{C}$ as $(cu)(g) = c \cdot u(g)$. 
	This scalar multiplication together with addition defined above makes $\G$ a vector space over the field $\bb{C}$.
	
	Let $c_1, c_2 \in \bb{C}, u, v \in \G$, and $g \in G$.
	Validating each of the vector space axioms is straghtforward. 

	\begin{enumerate}
		\item $(c_1c_2)u = c_1(c_2u)$ as for all $g \in G$, $((c_1c_2)u)(g) = (c_1c_2)(u(g)) = c_1(c_2u(g)) = c_1(c_2u)(g)$
		\item $c(u + v) = cu + cv$ as for all $g \in G$, $c(u + v)(g) = c(u(g) + v(g)) = cu(g) + cv(g) = (cu + cv)(g)$
		\item $(c_1 + c_2)u = c_1u + c_2u$ as for all $g \in G$, $((c_1 + c_2)u)(g) = (c_1 + c_2)u(g) = c_1u(g) + c_2u(g) = (c_1u + c_2u)(g)$
		\item $1u = u$ as for all $g \in G$, $1u(g) = 1 \cdot u(g) = u(g)$
	\end{enumerate}
\end{proof} 

\begin{proof}[Proof: $\G$ has dimension $\#G$]
	For each $g_i \in G$, $1 \leq i \leq \#G$, define a vector $\g_i \in \G$ as $\g_i(g_j) = \delta^i_j$, where $\delta^i_j$ is the Kronecker delta. 
 
	Note that any $\mathfrak{v} \in \G$ can be written as $\mathfrak{v} = \sum_{i=1}^{\#G} \mathfrak{v}(g_i) \g_i$, so $\mathcal{B}$ is a spanning set for $\G$.
	Assuming that $\sum_{i=1}^{\#G} c_i\g_i = 0$ for some $c_i \in \bb{C}$, by construction $(\sum_{i=1}^{\#G} c_i\g_i) (g_j) = c_j = 0$, so each of $c_i$ is zero and $\mathcal{B}$ is linearly independent.
	As a result, as long as $G$ is finite, the set $\mathcal{B} = \{\g_i: 1\ \leq i \leq \#G\}$ is both spanning and linearly independent thus a basis for $\G$, and the dimension of $\G$ is $\#\mathcal{B} = \#G$.
\end{proof}

\begin{proof}[Proof: $\hG$ is linearly independent]
	As noted in the lecture there exists a bilinear form on $\G$ $\langle -, - \rangle: \G \times \G \rightarrow \bb{C}$ defined thus

	\begin{equation}\label{eq:inner_product}
		\langle u, v \rangle = \frac{1}{\#G} \sum_{g \in G} u(g) \overline{v(g)},
	\end{equation}
	where $u, v \in \G$.
	The important property of \ref{eq:inner_product} as shown in lecture is that 
	$$
	 \langle e, f \rangle = 
	 \begin{cases}
		1 & \text{if } e = f\\ 
		0 & \text{if } e \neq f
	 \end{cases}
	$$

	where $e, f \in \hG$. 

	To show $\langle -, - \rangle$ is bilinear is also straightforward. 
	$$
	\begin{aligned}
		\langle u + v, w \rangle &= \frac{1}{\#G} \sum_{g \in G} (u(g) + v(g)) \overline{w(g)} \\
		&= \frac{1}{\#G} \sum_{g \in G} u(g) \overline{w(g)} + v(g) \overline{w(g)} \\
		&= \langle u, w \rangle + \langle v, w \rangle
	\end{aligned}
	$$

	$$
	\begin{aligned}
		\langle cu, w \rangle &= \frac{1}{\#G} \sum_{g \in G} c u(g) \overline{w(g)} \\
		&= c \frac{1}{\#G} \sum_{g \in G} u(g) \overline{w(g)} \\
		&= c \langle u, w \rangle 
	\end{aligned}
	$$

	As $\langle u, v \rangle = \overline{\langle v, u \rangle}$, the bilinearity in the second term follows. 

	Any bilinear form taken one value of 0 must produce 0, this is because $\langle 0, f \rangle = \langle 0 + 0, f \rangle = 2 \langle 0, f \rangle \implies \langle 0, f \rangle = 0 $ for all $f$.

	To prove $\hG$ is linearly independent is now a standard exercise on applying the definition

	\begin{align*} 
		\sum_{f \in \hG} c_f f = 0 
		&\implies \langle \sum_{f \in \hG} c_f f, f' \rangle = \langle 0, f' \rangle = 0 \\
		&\implies \sum_{f \in \hG} c_f \langle f, f' \rangle =
		 c_{f'} = 0
	\end{align*}

	Where $f' \in \hG$. This implies all of the coefficients $c_f$ are zero, and $\hG$ is linearly independent.
\end{proof}

By the standard theorem of linear algebra, the number of elements in a linearly independent set is less than or equal to the dimension of the vector space, so $\#\hG \leq \#G$.


\section{Q3}

Consider the finite group $G_9$ consists of all the invertible elements of $Z/9Z$. 

Let us first find all the elements of $G_9$. Since Euclidean algorithm applies for any $z \in \bb{Z}$, the smallest positive integer $n$ such that the following equation holds is the greatest common divisor of $z$ and $9$.  

$$
z k + 9 k' = n = gcd(z, 9)
$$

where $k, k' \in \bb{N}$. 
Notice this equation means that if $gcd(z, 9) = 1$, $\exists k, z k \equiv 1 \mod 9$ and $z$ is invertible, and if $gcd(z, 9) \neq 1$, there does not exist $k$ such that $z k \equiv 1 \mod 9$, so is $z$ is not invertible.

This means the elements of $G_9$ are precisely $[n]$, where $n$ is coprime to $9$.

$$
G_9 = \{[1], [2], [4], [5], [7], [8]\}
$$

The last step is to verify that $G_9$ is indeed a group. 
All elements of $G_9$ are invertible, and since each element is coprime to $9$, the product of two elements is also coprime to $9$, so the product of two elements is also in $G_9$. 
$G_9$ is group then follows.

Note that $[2]^2 = [4], [2]^3 = [8], [2]^4 = [7], [2]^5 = [5], [2]^6 = [1]$. So $G_9$ is generated by $[2]$, and $G_9 \cong Z / 6Z$.

For any $u \in \hat{G_9}$,
$u([2])^6 = u([2]^6) = u([1]) = 1$. 
As shown in \ref{lm:homo} in any group homomorphism the identity must map to identity, so $u([2]) $  is one of the sixth root of unity, i.e. $ u([2]) = \exp(2 k \pi / 6)$, $0 \leq k \leq 5$,
Once $u([2])$ is defined, $u$ is uniquely determined as any elements of $G_9$ can be written as $[2]^l$ for some $l$, and $u([2]^l) = u([2])^l$.
In this way six elements of $\hat{G_9}$, $u_i, 1 \leq i \leq 6$, are found.
Since $\# \hat{G_9} \leq \# G_9 = 6$, we have found all the elements of $\hat{G_9}$.

\begin{lemma}\label{lm:homo}
	In a group homomorphism $\phi: G \rightarrow H$, $\phi(1_G) = 1_H$
\end{lemma}

\begin{proof}
	This is a standard application of definition 
	\begin{align*}
		\phi(1_g) 
		= \phi(1_g 1_g)
		= \phi(1_g) \phi(1_g) 
		\rightarrow \phi(1_g) = 1_h
	\end{align*}
\end{proof}

\section{Q4}

The real valued Dirichelet character,$\chi$, must sent $[2]$ into a real number. Among all sixth roots of unity, only $1$ and $-1$ are real.

If $\chi[2] = 1$, $\chi$ will send everything to $1$ and $\chi$ is the principle character. 
So the only non principle real character send $[2]$ to $-1$.

Apply the definition $\chi(916) = \chi ([7]) = \chi([2]^4) = 1$

\end{document}
