\documentclass{article}

% \usepackage[tmargin=2.5cm,rmargin=3cm,lmargin=3cm,bmargin=3cm]{geometry} 
% Top margin, right margin, left margin, bottom margin, footnote skip
\usepackage[utf8]{inputenc}
\usepackage{biblatex}
% \addbibresource{./reference/reference.bib}
% linktocpage shall be added to snippets.
\usepackage{hyperref,theoremref}
\hypersetup{
	colorlinks, 
	linkcolor={red!40!black}, 
	citecolor={blue!50!black},
	urlcolor={blue!80!black},
	linktocpage % Link table of content to the page instead of the title
}

\usepackage{bbold}
\usepackage{blindtext}
\usepackage{titlesec}
\usepackage{amsthm}
\usepackage{thmtools}
\usepackage{amsmath}
\usepackage{amssymb}
\usepackage{graphicx}
\usepackage{titlesec}
\usepackage{xcolor}
\usepackage{multicol}
\usepackage{hyperref}
\usepackage{import}


\newtheorem{theorem}{Theorema}
\newtheorem{lemma}[theorem]{Lemma}
\newtheorem{corollary}{Corollarium}[section]
\newtheorem{proposition}{Propositio}[theorem]
\theoremstyle{definition}
\newtheorem{definition}{Definitio}[section]

\theoremstyle{definition}
\newtheorem{axiom}{Axioma}[section]

\theoremstyle{remark}
\newtheorem{remark}{Observatio}[section]
\newtheorem{hypothesis}{Coniectura}[section]
\newtheorem{example}{Exampli Gratia}[section]

% Proof Environments
\newcommand{\thm}[2]{\begin{theorem}[#1]{}#2\end{theorem}}

%TODO mayby proof environment shall have more margin
% \renewenvironment{proof}{\vspace{0.4cm}\noindent\small{\emph{Demonstratio.}}}{\qed\vspace{0.4cm}}
% \renewenvironment{proof}{{\bfseries\emph{Demonstratio.}}}{\qed}
\renewcommand\qedsymbol{Q.E.D.}
\renewcommand\mod{\text{ mod }}
% \renewcommand{\chaptername}{Caput}
% \renewcommand{\contentsname}{Index Capitum} % Index Capitum 
\renewcommand{\emph}[1]{\textbf{\textit{#1}}}
\renewcommand{\ker}[1]{\operatorname{Ker}{#1}}

%\DeclareMathOperator{\ker}{Ker}

% New Commands
\newcommand{\bb}[1]{\mathbb{#1}} %TODO add this line to nvim snippets
\newcommand{\orb}[2]{\text{Orb}_{#1}({#2})}
\newcommand{\stab}[2]{\text{Stab}_{#1}({#2})}
\newcommand{\im}[1]{\text{im}{\ #1}}
\newcommand{\se}[2]{\text{send}_{#1}({#2})}

\title{Analytical Number Theorey Handin One}
\author{Harry Han} 
\date{\today}

\begin{document}
\maketitle
% \tableofcontents

\section*{Q1}

Define functions $\mathbb{1}, \chi_0, \chi: \bb{N} \rightarrow \{-1,0,1\}$ thus 

\begin{equation}
	\mathbb{1}(n) = 
	\begin{cases}
		1,& n \equiv 1 \mod 3 \\ 
		0,& \text{ otherwise }
	\end{cases},
	\chi_0 (n) = 
	\begin{cases}
		1, & \gcd(n, 3) = 1 \\
		0, & \gcd(n, 3) \neq 1
	\end{cases},
	\chi(n) = 
	\begin{cases}
		0, & n \equiv 0 \mod 3 \\ 
		1, & n \equiv 1 \mod 3 \\ 
		-1, & n \equiv 2 \mod 3
	\end{cases},
\end{equation}
where $\mathbb{1}$ is the indicator function, and $\chi_0$ is the principle Dirichlet function mod 3.

We claim that $\chi$ and $ \chi_0$ are completely multiplicative, i.e., $\chi(xy) = \chi(x)\chi(y)$ (similar for $\chi_0$), and $\frac{1}{2}(\chi + \chi_0) = \mathbb{1}$.

\begin{proof}
Since $3$ is a prime number, and $\bb{N}$ is a unique factorisation domain, $\gcd(3, n) = 1$ if and only if $n$ is not a multiple of three. 

Consider $\chi_0 (xy)$. 
There are two cases: none of $x,y$ are multiples of $3$, or at least one of them is. 
In the former case, as $3$ is prime, $xy$ is not a multiple of three either, thus $\gcd(xy, 3) = 1$, and $\chi_0(xy) = 1 = 1 \times 1 = \chi_0(x) \chi_0(y)$. 
In the later case, one of $\chi_0(x), \chi_0(y)$ will be zero, and $xy$ will be multiple of $3$, thus $\chi_0(xy) = 0 = \chi_0(x)\chi_0(y)$.

Multiplicativity of $\chi$ is obvious from the fact that $\mathbb{Z} \backslash 3 \mathbb{Z}$ is a field.
To demonstrate this, let us invoke the fact that $\bb{N}$ is a completely ordered set by the canonical ordering and, subsequently, Euclidean Algorithm applies. 
This means for all $n \in \bb{N}$, there exist $q \in \bb{N}, r \in \{-1,0,1\}$ such that $n = 3q + r$. 
Notice $\chi(n) = r$ by construction. 
Breakdown $n, m$ into $n = 3q + r, m = 3q' + r'$, then 
\begin{equation}
nm = 3(3qq' + rq' + r'q) + rr'
\end{equation}
Since $r, r' \in \{-1,0,1\}$, $rr' \in \{-1,0,1\}$, and $\chi(mn) = rr' = \chi(n) \chi(m) $.

Last be not least let us show that $\mathbb{1}(n) = \frac{1}{2}(\chi(n) + \chi_0(n))$ by considering the all three possible cases

\begin{equation}
	\begin{cases}
		n \equiv 0 \mod 3 &\rightarrow \mathbb{1}(n) = 0 = \frac{1}{2}(0+0) = \frac{1}{2}(\chi_0(n) + \chi(n)) \\
		n \equiv 1 \mod 3 &\rightarrow \mathbb{1}(n) = 1 = \frac{1}{2}(1+1) = \frac{1}{2}(\chi_0(n) + \chi(n)) \\
		n \equiv 2 \mod 3 &\rightarrow \mathbb{1}(n) = 0 = \frac{1}{2}(1-1) = \frac{1}{2}(\chi_0(n) + \chi(n)) \\
	\end{cases}
\end{equation}
\end{proof}

\section*{Q2}
For $x > 1$ define the $L$ function 
\begin{equation}
	L(x; \chi) = \sum_{n = 1}^{\infty}\frac{\chi(n)}{n^x},
	L(x; \chi_0) = \sum_{n = 1}^{\infty}\frac{\chi_0(n)}{n^x},
\end{equation}

{\large\textbf{Part 1}: $\frac{1}{2} \leq L(x; \chi) \leq 1$ for all $1 < x$}

\begin{proof}
	Assume $ x > 1$, notice 
	\begin{equation}
		L(x, \chi) = 1 - \frac{1}{2^x} + \frac{1}{4^x} - \frac{1}{5^x} + \cdots
	\end{equation}
	Let $t_n$ denote the $n$-th term; observe that 
	\begin{enumerate}
		\item the sign of $t_n$ is $(-1)^{n+1}$, and 
		\item as $x > 1$, $|t_n| > |t_{n+1}|$ \
		\item $\lim_{n \rightarrow \infty} |t_n| = 0$, thus, by alternating test, the series converges, and $L(x, \chi)$ is well-defined.
	\end{enumerate}

	\textbf{Proof for Upper Bound}:

	Denote the partial sum $s_1 = \sum_{n=1}^k \frac{\chi(n)}{n^x}$. 
	We shall prove that $s_k \leq 1$ for all $k \in \bb{N}$ by induction.\\
	\textit{base case}: when $k = 1, s_1 = 1 \leq 1$. \\ 
	\textit{induction step}: Assuming $s_i \leq 1$ for $1\leq i \leq k$.
	There are two cases, $k$ is odd and $k$ is even. 
	If $k$ is odd, $k+1$ is even and, by our first observation, $t_{k+1} < 0$. 
	This means $s_{k+1} = s_{k} - |t_{k+1}| < s_{k} \leq 1$.
	If $k$ is even, $k+1$ is odd, and $|t_{k+1}| < |t_{k}|$ by our second observation. 
	By first observation $t_k < 0$ and $t_{k+1} > 0 \rightarrow  t_{k} + t_{k+1} < 0 \rightarrow s_{k+1} = s_{k-1} + t_k + t_{k+1} < s_{k-1}\leq 1$.

	Now, recall the definition of series that $L(x, \chi) = \lim_{n \rightarrow \infty} s_n$. 
	Observation 3 states that the limit exists, and as each of $s_n \leq 1$, it must also be smaller than or equal to $1$ as desired.

	\textbf{Lower Bound}:

	To prove the lower bound is similar. 
	We induct on $\frac{1}{4^x} - \frac{1}{5^x} + \cdots$. (Note this not $L(x, \chi)$!)
	Again, denote the $n$-th partial sum as $\sigma_n$, and the $n$-th term as $\tau_n$. 
	Our observations 1, 2, and 3 still hold, but our induction hypothesis is changed to $\sigma_k \geq 0$.\\
	\textit{base case}: when $k = 1, \sigma_1 > 0$ by the first observation. \\ 
	\textit{induction step}: Assuming $\sigma_i \geq 0$ for $1\leq i \leq k$.
	There are two cases, $k$ is odd and $k$ is even. 
	If $k$ is even, $k+1$ is odd and, by our first observatiop, $\tau_{k+1} > 0$. 
	This means $\sigma_{k+1} = \sigma_{k} + |\tau_{k+1}| \geq \sigma_{k} \geq 0$.
	If $k$ is odd, $k+1$ is even, and $|t_{k+1}| < |t_{k}|$ by our second observation. 
	By first observation $t_k > 0$ and $t_{k+1} < 0 \rightarrow  \tau_{k} + \tau_{k+1} > 0 \rightarrow \sigma_{k+1} = \sigma_{k-1} + |\tau_k + \tau_{k+1}| > \sigma_{k-1} \geq 0$.

	Now, notice $1 - \frac{1}{2^x} \geq \frac{1}{2}$ for all $x > 1$, and $\sum_{n=1}^k = 1 - \frac{1}{2^x} + \sum_{n=1}^{k-2} \tau_n  \geq \frac{1}{2}$ for all $k$. 

	Again, recall the definition of series that $L(x, \chi) = \lim_{k \rightarrow \infty} \sum_{n=1}^{k}\frac{\chi(n)}{n^x}$. 
	As each of $ \sum_{n=1}^{k}\frac{\chi(n)}{n^x} \geq \frac{1}{2}$ for all $n$, and the limit exists by obesrvation 3, it must also be greater than or equal to $\frac{1}{2}$ as desired.
\end{proof}
	

{\large\textbf{Part 2}: As $x \searrow 1, \frac{1}{2} \log(L(x; \chi_0)) + \frac{1}{2}\log( L(x; \chi)) \rightarrow \infty$}

As $\frac{1}{2} \leq L(x; \chi) \leq 1$ for all $x > 1$, and $\log$ is a monotonically increasing function, $\log \frac{1}{2} \leq \log(L(x; \chi)) \leq \log(1)$. 
So we only need to show that $\lim_{x \rightarrow 1^+}L(x; \chi_0) = \infty$, which would imply that $\lim_{x \rightarrow 1^+}\log(L(x; \chi_0)) = \infty$, as $\lim_{x \rightarrow \infty}\log(x) = \infty$.

To prove this we need the following lemma. 

\begin{lemma}\label{Q6}
	For strictly increasing positive integers $n_k, 1 \leq k <  \infty$,
	\begin{equation}
		\sum_{k=1}^{\infty} \frac{1}{n_k} = \infty \iff \lim_{x \rightarrow 1^+} \sum_{k=1}^{\infty} \frac{1}{n_k^x} = \infty
	\end{equation}
\end{lemma}

This is the problem 6 in the workshop 1. 

Notice that
$$
L(x, \chi_0) = \sum_{k=1}^{\infty} \frac{1}{(3k+1)^x} + \sum_{k=1}^{\infty} \frac{1}{(3k+2)^x} >
\sum_{k=1}^{\infty} \frac{1}{(3k+1)^x}
$$

Moreover, 
$$
	\sum_{k=1}^{\infty} \frac{1}{3k+1} > \frac{1}{9}\sum_{k=1}^{\infty} \frac{1}{k} = \infty
$$

So by Lemma \ref{Q6} 
$\lim_{x \rightarrow 1^+} L(x; \chi_0) >
\lim_{x \rightarrow 1^+} \sum_{k=1}^{\infty} \frac{1}{(3k+1)^x} 
= \infty
$ as desired.

\begin{proof}[Proof of Lemma \ref{Q6}]
	\textbf{Forward Direction}:
	Define function $f(x) = \sum_{k=1}^{\infty} \frac{1}{n_k^x} $.
	Assuming $ \lim_{x \rightarrow 1^+} f(x) \neq \infty$. 
	Notice $f(x) > 0$ for all $x>1$, and as $x$ decrease $f(x)$ will increase.

	An increasing sequence of real numbers can be bounded or not bounded. 
	If it is non bounded it diverges to infinity.
	If it is bounded by completeness of real number it must converge to some number, $C$. 

	Apply this reasoning to $f(x)$ we see $ \lim_{x \rightarrow 1^+} f(x) \neq \infty$ implies that $ \lim_{x \rightarrow 1^+} f(x) = C$ for some $C \in \bb{R}$. 

	Notice

	$$
	\sum_{k=1}^{N} \frac{1}{n_k} = \lim_{x \rightarrow  1^+} \sum_{k=1}^N \frac{1}{n_k^x} < \lim_{x \rightarrow 1^+} \sum_{k=1}^{\infty}\frac{1}{n_k^x} = C
	$$

	So $\sum_{k=1}^{N} \frac{1}{n_k}$ is bounded for all $N$, meaning $\sum_{k=1}^{\infty}\frac{1}{n_k}$ converges.

	In summary we proved that 
	$
		 \lim_{x \rightarrow 1^+} \sum_{k=1}^{\infty} \frac{1}{n_k^x} < \infty \implies \sum_{k=1}^{\infty} \frac{1}{n_k} < \infty 
	$.
	The forward direction is its contrapositive.

	\textbf{Backword Direction} 
	If
	$
		\lim_{x \rightarrow 1^+} \sum_{k=1}^{\infty} \frac{1}{n_k^x} = \infty
	$, note
	$\sum_{k=1}^{\infty} \frac{1}{n_k}$, being necessarily greater than the former for $x > 1$, must also diverge.

\end{proof}

\section*{Q3}

We claim that 

\begin{equation}
\frac{1}{2} \log(L(x; \chi_0)) + \frac{1}{2} \log(L(x; \chi)) = \sum_p \sum _{m=1}^{\infty} \frac{\mathbb{1}(p^m)}{m p^{mx}} 
\end{equation}

where $x>1$.

The proof of this question is more tricky and we need the following facts.

\begin{equation}\label{sum_to_product}
	\sum_{n=1}^{\infty} f(n) = \prod_p \frac{1}{1-f(p)}
\end{equation}
where $f$ is a multiplicative function, and $\sum_{n=1}^{\infty} f(n)$ converges absolutely. This is proposition 2.3 in the lecture notes.

Also recall the Taylor expansion for $\log(x)$ is 

\begin{equation}
	\log(x) = \sum_{n=1}^{\infty} (-1)^{n-1}\frac{(x-1)^n}{n},
\end{equation}
which converges absolutely for $0 \leq x \leq 2$.

This means 
\begin{equation}\label{taylor_log}
	\log(1-x) = - \sum_{n=1}^{\infty} \frac{x^n}{n},
\end{equation}
and it will converge absolutely for $-1<x< 1$.

Last, recall the following fact from Honours Analysis.

\begin{theorem}\label{order_of_sum}
	Changing the order of the sum will not change the value of the infinite series if the series converges absolutely.
\end{theorem}

\begin{proof}[Proof of Q3]
	By question one, both of $\chi$ and $\chi_0$ are multiplicative. 
	Consider $f(n) = \frac{\chi(n)}{n^x}$, $g(n) = \frac{\chi_0(n)}{n^x}$ we have 
	
	$$
		f(mn) = \frac{\chi(nm)}{(nm)^x} = \frac{\chi(n)}{n^x} \frac{\chi(m)}{m^x} = f(n)f(m),
	$$ 
	and
	$$
		g(mn) = \frac{\chi_0(nm)}{(nm)^x} = \frac{\chi_0(n)}{n^x} \frac{\chi_0(m)}{m^x} = g(n)g(m),
	$$

	so both of $g, f$ are multiplicative. 

	Moreover, for all $x > 1$, there exists some $ 1 < \epsilon < x$ so that for each $n$

	$$
	 |f(n)| = \left|\frac{\chi(n)}{n^x}\right| \leq \frac{1}{n^x} < \frac{1}{n^\epsilon} ,
	 |g(n)| = \left|\frac{\chi_0(n)}{n^x}\right| \leq \frac{1}{n^x} < \frac{1}{n^\epsilon} 
	$$

	Since $\sum_{n=1}^{\infty} \frac{1}{n^\epsilon}$ converges absolutely as it is a harmonic series and $\epsilon > 1$,
	by Weiestrass M test $\sum_{n=1}^{\infty} f(n) = L(x; \chi)$ and $\sum_{n=1}^{\infty} g(n) = L(x; \chi_0)$
	also converge absolutely, and
	by applying \ref{sum_to_product}

\begin{equation}
	L(x; \chi) = \prod_p (1-\chi(n)n^{-x})^{-1}, 
\end{equation}	

Similarly 
\begin{equation}
	L(x; \chi_0) = \prod_p (1-\chi_0(n)n^{-x})^{-1}, 
\end{equation}

Now, as log of product is the sum of log 

\begin{equation}\label{eq:log_before_sub}
	\frac{1}{2}\log(L(x; \chi)) = - \sum_{p} \log(1-\chi(p)p^{-x})
\end{equation}

Each of the terms of the Taylor series converges absolutely for $-1<\chi(n)p^{-x} < 1$.
Since $-1< \chi(n) < 1$ and $0 < p^x < 1$ for all $x > 1$, $\sum_{m=1}^{\infty} \frac{\chi(p^m)}{mp^{mx}}$  converges absolutely for all $x > 1$ and $p$ being prime.
So we can safely substitute \ref{eq:log_before_sub} into its taylor expansion

\begin{equation}\label{eq_log_subbed}
	\frac{1}{2}\log(L(x; \chi)) = 
	\frac{1}{2}\sum_{p} \sum_{m=1}^{\infty} \frac{\chi(p^m)}{mp^{mx}}=
	\sum_{p} \sum_{m=1}^{\infty} \frac{1}{2}\frac{\chi(p^m)}{mp^{mx}}
\end{equation}

Note that $ \frac{1}{2}\log(L(x; \chi)) < L(x; \chi)$ for all $x > 1$, and since $\sum_{m=1}^{\infty} \frac{\chi(p^m)}{mp^{mx}} > 0$ for all $p$ being prime and $x > 1$, the series \ref{eq_log_subbed} also converges absolutely.

For the exact same arguments
\begin{equation}
	\frac{1}{2}\log(L(x; \chi_0)) = 
	\frac{1}{2}\sum_{p} \sum_{m=1}^{\infty} \frac{\chi_0(p^m)}{mp^{mx}}=
	\sum_{p} \sum_{m=1}^{\infty} \frac{1}{2}\frac{\chi_0(p^m)}{mp^{mx}},
\end{equation}
which converges absolutely for $x > 1$.

By summing them all up
\begin{align}
		\label{eq_terrible_1}
		\frac{1}{2} \log(L(x; \chi_0)) + \frac{1}{2} \log(L(x; \chi)) =& 
	\sum_{p} \sum_{m=1}^{\infty} \frac{1}{2}\frac{\chi(p^m)}{mp^{mx}} +
		\sum_{p} \sum_{m=1}^{\infty} \frac{1}{2}\frac{\chi_0(p^m)}{mp^{mx}} \\ 
		\label{eq_terrible_2}
		=&
		\sum_{p} \sum_{m=1}^{\infty} \frac{\frac{1}{2}(\chi_0(p^m) + \chi(p^m))}{mp^{mx}} \\
		\label{eq_terrible_3}
		=&
		\sum_p \sum _{m=1}^{\infty} \frac{\mathbb{1}(p^m)}{m p^{mx}} 
	\end{align}

	Where from \ref{eq_terrible_1} to \ref{eq_terrible_2}, we used theorem \ref{order_of_sum} to reordered the terms, and from \ref{eq_terrible_2} to \ref{eq_terrible_3} we used the identity $\frac{1}{2}((\chi_0(x) + \chi(x)) = \mathbb{1}(x)$ proved in the first question.
\end{proof}

\section*{Q4}
This is the pinnacle of this homework, we are to prove that 

\begin{equation}
	\sum_{p \equiv 1 \mod 3} \frac{1}{p} = \sum_{p} \frac{\mathbb{1}(p)}{p} = \infty
\end{equation}

\begin{proof}
In the previous question we have proved that,  for $x > 1$

	$$
\frac{1}{2} \log(L(x; \chi_0)) + \frac{1}{2} \log(L(x; \chi)) =
\sum_p \sum _{m=1}^{\infty} \frac{\mathbb{1}(p^m)}{m p^{mx}} 
	$$

Notice that 
\begin{align}
	\lim_{x \rightarrow 1^+} \frac{1}{2} \log(L(x; \chi_0)) + \frac{1}{2} \log(L(x; \chi)) 
	= \lim_{x \rightarrow 1^+}\sum_p \sum _{m=1}^{\infty} \frac{\mathbb{1}(p^m)}{m p^{mx}} \\
	=
	\underbrace{\lim_{x \rightarrow 1^+}\sum_p \frac{\mathbb{1}(p)}{p^x}}_{1} +
	\underbrace{\lim_{x \rightarrow 1^+} \sum_p \sum _{m=2}^{\infty} \frac{\mathbb{1}(p^m)}{m p^{mx}}}_{2} 
	= \infty
\end{align}

The limit diverges, as proven in question two part two.

Let us scrutinise the second term

\begin{align}
 \lim_{x \rightarrow 1^+} \sum_p \sum _{m=2}^{\infty} \frac{\mathbb{1}(p^m)}{m p^{mx}} 
<&  \sum_p \sum _{m=2}^{\infty} \frac{\mathbb{1}(p^m)}{p^{m}} 
<  \sum_p \sum _{m=2}^{\infty} \frac{1}{p^{m}} 
\\  
=&  \sum_p \frac{p^{-2}}{1-p^{-1}} \text{ (evaluating the geometric series)}\\ 
=&  \sum_p \frac{1}{p^2-p} 
<  \sum_{n = 2}^{\infty} \frac{1}{p(p-1)} < \infty
\end{align}

This means the second term converges, so the first term must diverges. 
Now invoke lemma \ref{Q6} we claim that 

$$
	\sum_p \frac{\mathbb{1}(p)}{p} = \infty
$$

\end{proof}


\end{document}
