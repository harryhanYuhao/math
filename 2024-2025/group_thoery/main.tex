\documentclass[twocolumn]{article}
\usepackage[a4paper, margin=0.3in]{geometry} % Top margin, right margin, left margin, bottom margin, footnote skip


\usepackage[utf8]{inputenc}
\usepackage{biblatex}
\addbibresource{./references.bib}
% linktocpage shall be added to snippets.
\usepackage{hyperref,theoremref}
\hypersetup{
	colorlinks, 
	linkcolor={red!40!black}, 
	citecolor={blue!50!black},
	urlcolor={blue!80!black},
	linktocpage % Link table of content to the page instead of the title
}

\usepackage{afterpage}

\newcommand\blankpage{%
    \null
    \thispagestyle{empty}%
    \addtocounter{page}{-1}%
    \newpage}


\usepackage{blindtext}
\usepackage{mathrsfs}
\usepackage{yhmath}
\usepackage{breqn}
\usepackage{titlesec}
\usepackage{keytheorems}
\usepackage{amsthm}
\usepackage{amsmath}
\usepackage{amssymb}
\usepackage{graphicx}
\usepackage{titlesec}
\usepackage{xcolor}
% \usepackage{multicol}
\usepackage{hyperref}
\usepackage{import}

% \renewenvironment{proof}{\vspace{0.4cm}\noindent\small{\emph{Demonstratio.}}}{\qed\vspace{0.4cm}}
% \renewenvironment{proof}{{\bfseries\emph{Demonstratio.}}}{\qed}
\renewcommand\qedsymbol{Q.E.D.}
% \renewcommand{\chaptername}{Caput}
% \renewcommand{\contentsname}{Index Capitum} % Index Capitum 
% \renewcommand{\emph}[1]{\textbf{\textit{#1}}}
\renewcommand{\emph}[1]{{\color{blue!70!black}\sffamily\bfseries #1}}
\renewcommand{\ker}[1]{\operatorname{Ker}{#1}}

% \DeclareMathOperator{\ker}{Ker}

\newcommand{\X}{\mathscr{X}}
\newcommand{\n}{\nabla}
\newcommand{\R}{\mathbb{R}}
\newcommand{\C}{C^{\infty}}
\newcommand{\CY}{\mathfrak{C}}
\newcommand{\p}{\partial}
\newcommand{\G}{\Gamma}
% mathscr is from mathrsfs
\renewcommand{\L}{\mathscr{L}}
\newcommand{\g}{\gamma}
\newcommand{\dg}{\dot{\gamma}}
\newcommand{\N}{\textit{\underline{NOTE} }}
\newcommand{\W}{\Omega}
\newcommand{\ot}{\otimes}
\renewcommand{\dg}{\dot{\gamma}}

\usepackage[most]{tcolorbox}

\tcbuselibrary{theorems}
\newtcbtheorem{fdefi}{Definitio}%
{fonttitle=\bfseries}{th}
\newtcbtheorem{fthm}{Theorema}%
{fonttitle=\bfseries}{th}
\newtcbtheorem{eg}{\textit{E.G.}}
{colback=white,              % background of the box
    colframe=black,             % dark frame color
    coltitle=black,             % title text color
	colbacktitle=gray!10!white,
	colback=gray!10!white,
    fonttitle=\bfseries,        % title font
    boxrule=1pt,              % border thickness
    arc=2mm,                    % rounded corners
    boxsep=0.7mm,                 % inner spacing
    enhanced,
}{th}

\theoremstyle{definition}
\newtheorem{thm}{THM}
\newtheorem{prop}[thm]{PROP}
\newtheorem{lemma}[thm]{LEM}
\newtheorem{corollary}[thm]{Corollarium}
\newtheorem{example}[thm]{Exampli Gratia}
\newtheorem{remark}[thm]{Remark}
\newtheorem{defi}[thm]{DEF}


\begin{document}
% \tableofcontents

\begin{defi}
	Let $H \leq G$, $g \in G$. 
	\emph{Left coset} of $H$ in $G$ is the set $g H := \{gh: h in H\}$.
	\emph{Right coset} of $H$ in $G$ is the set $H g := \{hg: h in H\}$.
\end{defi}

\begin{defi}[Center]
	Center $Z(G)$ of the group $G$ is the set of all $z \in G$ such that $zg = gz$ for all $g$.
\end{defi}

\begin{defi}[Index of Group]
	Let $H \leq G$. 
	The index of the group is the number of distinct left cosets of $H$ in $G$. 
\end{defi}

\begin{thm}[First Isomorphism Theorem]
	Let $\phi: G \to H$ be a group homomorphism. 
	Then $G / \ker(\phi) \cong H$
\end{thm}

\begin{thm}[Second Isomorphism Theorem]
	Let $H, S \triangleleft G$. 
	Then 
	$$
	\frac{HS}{S} \cong \frac{H}{H\cap S}
	$$
\end{thm}

\begin{thm}[Third Isomorphism Theorem]
	$$
		\frac{G / N}{H / N} \cong \frac{G}{H}
	$$
\end{thm}

\begin{prop}
	If $H, S \lhd G$ and $H \cap S = \{e\}$, then $HK \cong H \times K$.
\end{prop}

\begin{prop}
	Let $G$ be a finite group and $H, K$ its subgroups, with at least one of which is normal. 
	Moreover, if $[H]$ and $[G:K]$ are coprime, then $H \leq K$.

	If $|G| $ has prime decomposition $ p^I q^J$ (regarding $I, J$ as multindeces, or vectors). 
	The condition is precisely $|H| = p^{I'}$ and $|K| = p^I c$, where $I'$ is strictly less than or equal to $I$ in every index.
	That is, $|K|$ contains all the prime factor of $H$ to the maximum possible degree.
\end{prop}

\begin{proof}
	First consider $K \lhd G$.

	The proof exploits the facts that $HK \leq G$, when $K \lhd G, HK / K \leq G / K$, and Legrange theorem. 

	Also note, If $h \in H$ and $h \notin K$, as $G$ is a finite group, there will be equally amount of elements of $H$ in each coset of $K$ that contains some element of $H$.
\end{proof}

\begin{thm}[Free Group Representation of Dihedral Group]
	$$D_5 \cong \langle a, b | a^2, b^5, (ab)^2 \rangle \cong \langle a, b | a^2, b^2, (ab)^5 \rangle$$
\end{thm}

\begin{thm}[Sylow Theorem]
	Let $|G| = n$ and suppose $n = p^q r$
	\begin{enumerate}
		\item There exists Sylow subgroup
		\item Any Sylow $p$ subgroups are conjugate 
		\item $n_p | r$ and $n_p \equiv 1 \mod p$.
	\end{enumerate}	
\end{thm}

To prove Sylow theorem involves heavily on group actions.

\begin{proof}[Proof of 1]
	Let $X$ be the subset of $A$ with $|A| = p^m$. 
	$G$ acts on $X$ by left translation.

	If there is an orbit whose order does not divide $p$, by orbit stabilizer theorem there is a the kernel must be a group of order $p^m$.

	We figure out $|X|$ is not divisible by $p$. Therefore there must be such an orbit.
\end{proof}


\begin{lemma}
	Let $p$ be a prime and $G$ a finite $p$ group acting on set $X$.
	The number of fixed point in $X \equiv |X| \mod p$. 
\end{lemma}

\begin{defi}
	Let $G$ be a group and $H \leq G$. 
	Normalizer of $H$ is 
	$$
	N_G(H) = \{g \in G: gHg^{-1} \in H\}
	$$
\end{defi}

$N_G(H)$ is the largest subgroup $N$ of $G$ such that $H \lhd N$. 

\begin{thm}
	Let $G$ be a finite group. 
	\begin{enumerate}
		\item For any subgroup $H \leq G$, we have 
			$$
				[G: N_G{H}] = \text{ The number of distinct conjugate of } H
			$$
	\end{enumerate}
\end{thm}

\begin{thm}
	A group of order $p^2$ is abelian. 
\end{thm}
\section*{Permuation Group}

\begin{lemma}
Let $\sigma = (a_1 a_2 a_3 \cdots a_k)$ then 
$$
	\tau \sigma \tau^{-1} = (\tau(a_1) \tau(a_2) \cdots \tau(a_k))
$$
\end{lemma}

\begin{lemma}[Subgroups of $S_4$]
	Let $N = \{(), (12)(34), (13)(24), (14)(23)\}$. 
	$S_4 / N \cong D_3$. 
\end{lemma}

\begin{example}
Conjugacy class of $S_5$

% use a table 
\begin{tabular}{|c|c|c|}
	\hline
	Cycle Type & \# perms & Even / Odd \\
	\hline 
	5    & 24 & E \\
	\hline
	4,1    & 30 & O \\
	\hline
	3, 2    & 20 & O \\
	\hline
	3, 1, 1    & 20 & E \\
	\hline
	2, 2, 1    & 15 & E \\
	\hline
	2, 1, 1 ,1    & 10 & O \\
	\hline
	1, 1, 1 ,1, 1    & 1 & E \\
	\hline
\end{tabular}

In particular, the class equation for $A_5$ is $1 + 12 + 12 + 15 + 20$

\end{example}

\begin{thm}
	For $n \geq 3$, $A_n$ is generated by 3 cycles.
\end{thm}


\section*{Jordan Holder}

\begin{defi}[Composition series]
	Composition series for $G$ is a chain of subgroups $\{e\} = G_0 \lhd G_1 \lhd \cdots \lhd G_s = G$ 

	Such that $G_i \neq G_{i+1}$ and $G_{i+1} / G_i$ is simple for all $i$.
\end{defi}

\begin{thm}[Jordan Holder Theorem]
	Let $G$ be a finite group. 
	$G$ has a composition series, and any two composition series of $G$ has the same length and same composition factors upto rearrangements.
\end{thm}

\begin{thm}
	Composition series of group $G$ can be built from $N \lhd G$ and $G / N$ by the correspondence theorem.
\end{thm}


\subsection*{Solvable Groups}

\begin{defi}[Subnormal Series]
	Let $G$ be a group. 
	A subnormal series is a series of subgroups 
	$\{e\} = G_0 \lhd G_1 \lhd \cdots \lhd G_s = G$
\end{defi}

\begin{defi}
	A group is solvable if it has subnormal series such that the adjacent quotents are abelian, that is $G_{i+1} / G_{i}$ is abelian.
\end{defi}

\begin{defi}
	A group is solvable if and only if the composition factors are all cyclic.
\end{defi}

\begin{thm}
	Let $G$ be a group and $N \lhd G$. $G$ is solvable if and only if both $G$ and $G/N$ are. 
\end{thm}

\begin{thm}
	$G$ is solvable, then any subgroup of $G$ also is. They do not need to be normal.
\end{thm}

\begin{defi}[Derived Series]
	The derived series of the group $G$ is the series of commutator $G'$, where 
	$$G' = \langle aba^{-1}b^{-1} | a, b \in G \rangle $$
\end{defi}

Lable $[a, b] = aba^{-1}b^{-1}$.
Note that $[a, b][b, a]  =e $ and $z[a, b]z^{-1} = [zaz^{-z}, zbz^{-1}]$.

\begin{thm}
	Let $G$ be a group and $H \lhd G$. $G / H$ is abelian if and only if $G' \subset H$.
\end{thm}

\begin{thm}
	Group $G$ is solvable if and only if its derived series terminate at $\{e\}$.
\end{thm}

\section*{Memorandum}

\subsection*{Small simple groups}

\begin{itemize}
	\item 60: $A_5 \cong {\rm PSL}_2(4) \cong {\rm PSL}_2(5)$.
	\item 168: ${\rm PSL}_2(7) \cong {\rm PSL}_3(2)$.
	\item 360: $A_6 \cong {\rm PSL}_2(9)$.
	\item 504: ${\rm PSL}_2(8)$.
	\item 660: ${\rm PSL}_2(11)$.
	\item 1092: ${\rm PSL}_2(13)$.
	\item 2448: ${\rm PSL}_2(17)$.
\end{itemize}

\subsection*{Outline of Proof of Jordan Holder Theorem}

Step 1: Composition series exists for finite group by induction.\\
Step 2: Assuming there are two composition series of $G$, $H_1 \lhd H_2 \lhd \cdots \lhd H_{n-1} \lhd H_{n} = G$ and $S_1 \lhd S_2 \lhd \cdots \lhd S_{n-1} \lhd S_n = G$.

There are two cases: 
1. $H_{n-1} = S_{n-1}$ we are done.  

2. $H_{n-1} \neq S_{n-1}$.

Notice $H_{n-1}S_{n-1}$ is another normal subgroup of $G$, containing $H_{n-1}$ and $S_{n-1}$. This means it must be $G$.

Lable $H_{n-1} \cap S_{n-1} = K$, which is a normal subgroup of $G$.
Now, $\frac{S_{n-1}}{K} = \frac{S_{n-1}H_{n-1}}{H_{n-1}}$, $\frac{H_{n-1}}{K} = \frac{S_{n-1}H_{n-1}}{S_{n-1}} = \frac{G}{S_{n-1}}$.

Until this step the proof is almost done.
Just write down four distinct composition seris of $H, K, S$ and compare them, and use induction.

\subsection*{Outline of Proof of Sylow's theorem}

Sylow 2: 
Let $P$ be a $p$ group, $H$ another one. 
$H$ acts on $G / P$, the left coset, by left translation. 
$|G / P |  = r$, so there must be at least one fixed point, $xP$, say, that is $hxP = xP \implies P, H$ are conjugate.


\afterpage{\blankpage}
\end{document}


