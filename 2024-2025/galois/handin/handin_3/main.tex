\documentclass{article}

\usepackage[tmargin=2.5cm,rmargin=3cm,lmargin=3cm,bmargin=3cm]{geometry} 
% Top margin, right margin, left margin, bottom margin, footnote skip
\usepackage[utf8]{inputenc}
\usepackage{biblatex}
\addbibresource{./references.bib}
% linktocpage shall be added to snippets.
\usepackage{hyperref,theoremref}
\hypersetup{
	colorlinks, 
	linkcolor={red!40!black}, 
	citecolor={blue!50!black},
	urlcolor={blue!80!black},
	linktocpage % Link table of content to the page instead of the title
}

\usepackage{blindtext}
\usepackage{titlesec}
\usepackage{amsthm}
\usepackage{thmtools}
\usepackage{amsmath}
\usepackage{amssymb}
\usepackage{graphicx}
\usepackage{titlesec}
\usepackage{xcolor}
\usepackage{multicol}
\usepackage{hyperref}
\usepackage{import}


\newtheorem{theorem}{Theorema}[section]
\newtheorem{lemma}[theorem]{Lemma}
\newtheorem{corollary}{Corollarium}[section]
\newtheorem{proposition}{Propositio}[theorem]
\theoremstyle{definition}
\newtheorem{definition}{Definitio}[section]

\theoremstyle{definition}
\newtheorem{axiom}{Axioma}[section]

\theoremstyle{remark}
\newtheorem{remark}{Observatio}[section]
\newtheorem{hypothesis}{Coniectura}[section]
\newtheorem{example}{Exampli Gratia}[section]

% Proof Environments
\newcommand{\thm}[2]{\begin{theorem}[#1]{}#2\end{theorem}}

%TODO mayby proof environment shall have more margin
\renewenvironment{proof}{\vspace{0.4cm}\noindent\small{\emph{Demonstratio.}}}{\qed\vspace{0.4cm}}
% \renewenvironment{proof}{{\bfseries\emph{Demonstratio.}}}{\qed}
\renewcommand\qedsymbol{Q.E.D.}
% \renewcommand{\chaptername}{Caput}
% \renewcommand{\contentsname}{Index Capitum} % Index Capitum 
\renewcommand{\emph}[1]{\textbf{\textit{#1}}}
\renewcommand{\ker}[1]{\operatorname{Ker}{#1}}

%\DeclareMathOperator{\ker}{Ker}

% New Commands
\newcommand{\bb}[1]{\mathbb{#1}} %TODO add this line to nvim snippets
\newcommand{\orb}[2]{\text{Orb}_{#1}({#2})}
\newcommand{\stab}[2]{\text{Stab}_{#1}({#2})}
\newcommand{\im}[1]{\text{im}{\ #1}}
\newcommand{\se}[2]{\text{send}_{#1}({#2})}

\title{Galois Thoery}
\author{Harry Han, S2162783} 
\date{\today}

\begin{document}
\maketitle
% \tableofcontents
\section*{Q1}

The Degree of the extension of $\bb{Q}(2^{\frac{1}{4}})$ is $4$. 

$2^{\frac{1}{4}}$ is the root of the polynomial $x^4 - 2 = 0$, which is irreducible by Eisenstein's criterion (by plugging in $2$); thus it is its minimal polynomial.
As a result, we can apply theorem 5.1.5 to conclude that this field extension is of degree 4 and the basis are $1, 2^{\frac{1}{4}}, 2^{\frac{1}{2}}, 2^{\frac{3}{4}}$.

\section*{Q2}
The Galois group, $Gal(\bb{Q}(2^{\frac{1}{4}}): \bb{Q})$ is isomorphic to $C_2$.

\begin{proof}
	We want to classify the automorphism of $Q(2^{\frac{1}{4}})$ over $\bb{Q}$.
	Let $\theta$ be such an automorphism. 
	Note that 
	$$
	\theta(2^{\frac{1}{4}})^4 = \theta(2) = 2
	$$

	So $\theta(2^{\frac{1}{4}})$ is a root of $x^4 - 2 = 0$.
	Since all basis of $\bb{Q}(2^{\frac{1}{4}})$ are elements of $\bb{R}$, $\bb{Q}(2^{\frac{1}{4}}) \subset \bb{R}$, so we conclude that $\theta(2^{\frac{1}{4}}) = 2^{\frac{1}{4}}$  or $\theta(2^{\frac{1}{4}}) = -2^{\frac{1}{4}}$.
	Once $\theta(2^{\frac{1}{4}})$ is determined, $\theta$ is determined as well by Lemma 4.3.6.
	So there are only two elements in $Gal(\bb{Q}(2^{\frac{1}{4}}), \bb{Q})$, and it must be isomorphic to $Z / 2Z \cong C_2 $.
\end{proof}

\end{document}
