\documentclass{article}

\usepackage[tmargin=2.5cm,rmargin=3cm,lmargin=3cm,bmargin=3cm]{geometry} 
% Top margin, right margin, left margin, bottom margin, footnote skip
\usepackage[utf8]{inputenc}
\usepackage{biblatex}
\addbibresource{./references.bib}
% linktocpage shall be added to snippets.
\usepackage{hyperref,theoremref}
\hypersetup{
	colorlinks, 
	linkcolor={red!40!black}, 
	citecolor={blue!50!black},
	urlcolor={blue!80!black},
	linktocpage % Link table of content to the page instead of the title
}

\usepackage{blindtext}
\usepackage{titlesec}
\usepackage{amsthm}
\usepackage{thmtools}
\usepackage{amsmath}
\usepackage{amssymb}
\usepackage{graphicx}
\usepackage{titlesec}
\usepackage{xcolor}
\usepackage{multicol}
\usepackage{hyperref}
\usepackage{import}


\newtheorem{theorem}{Theorema}[section]
\newtheorem{lemma}[theorem]{Lemma}
\newtheorem{corollary}{Corollarium}[section]
\newtheorem{proposition}{Propositio}[theorem]
\theoremstyle{definition}
\newtheorem{definition}{Definitio}[section]

\theoremstyle{definition}
\newtheorem{axiom}{Axioma}[section]

\theoremstyle{remark}
\newtheorem{remark}{Observatio}[section]
\newtheorem{hypothesis}{Coniectura}[section]
\newtheorem{example}{Exampli Gratia}[section]

% Proof Environments
\newcommand{\thm}[2]{\begin{theorem}[#1]{}#2\end{theorem}}

%TODO mayby proof environment shall have more margin
% \renewenvironment{proof}{{\bfseries\emph{Demonstratio.}}}{\qed}
\renewcommand\qedsymbol{Q.E.D.}
% \renewcommand{\chaptername}{Caput}
% \renewcommand{\contentsname}{Index Capitum} % Index Capitum 
\renewcommand{\emph}[1]{\textbf{\textit{#1}}}
\renewcommand{\ker}[1]{\operatorname{Ker}{#1}}

%\DeclareMathOperator{\ker}{Ker}

% New Commands
\newcommand{\bb}[1]{\mathbb{#1}} %TODO add this line to nvim snippets
\newcommand{\orb}[2]{\text{Orb}_{#1}({#2})}
\newcommand{\stab}[2]{\text{Stab}_{#1}({#2})}
\newcommand{\im}[1]{\text{im}{\ #1}}
\newcommand{\se}[2]{\text{send}_{#1}({#2})}

\title{Ars Amatoriae}
\author{Publius Ovidius Naso} 
\date{\today}

\begin{document}
\section*{Exercise 1}

Let $p$ be a prime. List the orders of all subfields of $\mathbb{F}_{p^8}$.

\textbf{Answer}:
The orders of subfields are $p, p^2, p^4,$ and $p^8$.

\section*{Exercise 2}
Let $K$ be a field and $n \geq 1$. Show that $\operatorname{Gal}_K(t^n - 1)$ is abelian.

\begin{proof}
	The splitting field $SF_k(t^n - 1)$ is generated by the roots of unity. 
	Let us investigate the set of roots of unity denoted as $U$, which has $n$ elements.

	It is clear that $1 \in U$. 
	If $v, u \in U$, then $(uv)^n = u^n v^n = 1$, meaning that $uv \in U$.

	If $v \in U$, then there is some $a$ such that $v^a = 1$. 
	The inverse of $v$ is precisely $v^{a-1}$, which, by the previous point, is also part of $U$. 

	This shows that $U$ is a multiplicative and finite subgroup of the field $SF_K(t^n-1)$, which must be cyclic. 

	Let $U$ be generated by the element $\xi$. 
	If $\xi \in K$, the galois group is trivial and we are done.
	If $\xi \notin K$, and field automorphism $\sigma$, when acting on $\xi$, must send it to another root of unity, as $\sigma(\xi)^n = \sigma(\xi^n) = 1$. 
	Thus, $\sigma(\xi) = \xi^k$ for some integer $k$.

	For any two automorphisms $\sigma, \tau$, assuming $\sigma(\xi) = \xi^s$ and $\tau(\xi) = \xi^t$, then $\sigma(\tau(\xi)) = \theta(\xi^t) = \xi^{ts} =  \tau(\xi^s) = \tau(\theta(\xi))$.
	This proves that the galois group is abelian.
\end{proof}

\section*{Exercise 3}
Show that the following polynomials are solvable by radicals:
\begin{enumerate}
    \item $t^5 + 5$.
    \item $t^5 + 4t^3 + 5$.
\end{enumerate}

\subsection*{Part 1}
Lable $M = SF_{\bb{Q}}(t^n + 5), L =  SF_{\bb{Q}}(t^n + 1)$.
Note that $L$ is generated by $\omega$, where $\omega = e^{\frac{2\pi i}{5}}$.
Also note that both $\sqrt{5}$ and $\omega \sqrt{5}$ are contained in $M$, thus their quotient, $\omega$, must be part of $M$. 
As a result, $L \subset M$, and there is a tower of field extension $M : L : \bb{Q}$.

The corresponding Galois groups of this tower are  
$Gal(M : M) - Gal(M : L) - Gal(M : \bb{Q})$. 
By theorem 7.1.5, $L:\bb{Q}$ is normal, therefore by the fundamental theorem 
$ Gal(M: L) \triangleleft Gal(M: \bb{Q})$, where $\triangleleft$ denotes normal subgroup.
Therefore we have the chain of groups:
$\{1\} \cong Gal(M : M) \triangleleft Gal(M: L) \triangleleft Gal(M: \bb{Q})$

By the fundamental theorem $\frac{Gal(M: \bb{Q})}{Gal(M: L)} = Gal(L: Q)$, which by lemma 9.1.6 (as well as the previous exercise) is abelian.
Now, by lemma 9.1.8, $Gal(M:L)$ is abelian. 
This precisely means that the group $G(M: \bb{Q})$ is solvable, thus the polynomial $t^5 + 5$ is solvable by radicals.


\subsection*{Part 2}
Note $-1$ is one of the roots of $t^5 +4^3 + 5$, so this polynomial equals $(t+1)q$, where $q$ is a degree 4 polynomial, which is solvable by radicals. So we conclude the original polynomial is also solvable by radicals.

\end{document}
