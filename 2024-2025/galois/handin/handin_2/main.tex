\documentclass{article}

% \usepackage[tmargin=2.5cm,rmargin=3cm,lmargin=3cm,bmargin=3cm]{geometry} 
% Top margin, right margin, left margin, bottom margin, footnote skip
\usepackage[utf8]{inputenc}
\usepackage{biblatex}
\addbibresource{./references.bib}
% linktocpage shall be added to snippets.
\usepackage{hyperref,theoremref}
\hypersetup{
	colorlinks, 
	linkcolor={red!40!black}, 
	citecolor={blue!50!black},
	urlcolor={blue!80!black},
	linktocpage % Link table of content to the page instead of the title
}

\usepackage{blindtext}
\usepackage{titlesec}
\usepackage{amsthm}
\usepackage{thmtools}
\usepackage{amsmath}
\usepackage{amssymb}
\usepackage{graphicx}
\usepackage{titlesec}
\usepackage{xcolor}
\usepackage{multicol}
\usepackage{hyperref}
\usepackage{import}


\newtheorem{theorem}{Theorema}[section]
\newtheorem{lemma}[theorem]{Lemma}
\newtheorem{corollary}{Corollarium}[section]
\newtheorem{proposition}{Propositio}[theorem]
\theoremstyle{definition}
\newtheorem{definition}{Definitio}[section]

\theoremstyle{definition}
\newtheorem{axiom}{Axioma}[section]

\theoremstyle{remark}
\newtheorem{remark}{Observatio}[section]
\newtheorem{hypothesis}{Coniectura}[section]
\newtheorem{example}{Exampli Gratia}[section]

% Proof Environments
\newcommand{\thm}[2]{\begin{theorem}[#1]{}#2\end{theorem}}

%TODO mayby proof environment shall have more margin
% \renewenvironment{proof}{\vspace{0.4cm}\noindent\small{\emph{Demonstratio.}}}{\qed\vspace{0.4cm}}
% \renewenvironment{proof}{{\bfseries\emph{Demonstratio.}}}{\qed}
\renewcommand\qedsymbol{Q.E.D.}
% \renewcommand{\chaptername}{Caput}
% \renewcommand{\contentsname}{Index Capitum} % Index Capitum 
\renewcommand{\emph}[1]{\textbf{\textit{#1}}}
\renewcommand{\ker}[1]{\operatorname{Ker}{#1}}
\renewcommand{\c}{\cos(\theta)}
\newcommand{\s}{\sin(\theta)}

%\DeclareMathOperator{\ker}{Ker}

% New Commands
\newcommand{\bb}[1]{\mathbb{#1}} %TODO add this line to nvim snippets
\newcommand{\orb}[2]{\text{Orb}_{#1}({#2})}
\newcommand{\stab}[2]{\text{Stab}_{#1}({#2})}
\newcommand{\im}[1]{\text{im}{\ #1}}
\newcommand{\se}[2]{\text{send}_{#1}({#2})}

\date{\today}

\begin{document}
\section{Q1}
We claim $\cos(\frac{ \pi}{9})$ is algebraic over $\bb{Q}$. 

\begin{proof}
	Let us begin by finding the formula for $\cos(3 \theta)$
	\begin{align*}
		\cos(3 \theta) 
		&= \cos(2 \theta + \theta) \\	
		&= \cos(2 \theta) \cos(\theta) - \sin(2 \theta) \sin(\theta) \\	
		&= (2\cos(\theta)^2 - 1) \c - 2\s\c\s \\
		&= 2 \c^3 - \c - 2\s^2 \c \\ 
		&= 2 \c^3 - \c + 2(\c^2 -1)\c \\
		&= 4 \c^3 - 3 \c ^2
	\end{align*}

	We know that $\cos(\frac{\pi}{3}) = \frac{1}{2}$, so, setting $\cos(\frac{\pi}{9}) $ as $x$, using the formula just deduced, we find it is the solution to the polynomials

	\begin{equation}\label{min_pol}
		8x^3 - 6x -1
	\end{equation}

	Let us check if this polynomial is reducible. 
	Since it is of degree three, it is reducible iff it has a root in $\bb{Q}$. 
	The leading coefficients of the polynomial is $8$ and the constant term of the polynomial is $1$, meaning any rational roots of it must be $\pm 1, \pm \frac{1}{2}, \pm \frac{1}{4}$ or $\pm \frac{1}{8}$. 
	Plugging in these values we find none of them are roots, so we conclude that the minimal polynomial is \ref{min_pol}.
\end{proof}

\section{Q2}

We want to prove that for any $n \in \bb{N}$, $n > 0$, there is a field extension of $\bb{Q}$ of degree $n$.

\begin{proof}
	For $n=1$, $\bb{Q}$ itself is a field extension of degree $1$.

	For $n > 1$, $\bb{Q}((2)^{\frac{1}{n}})$ is a field extension of degree $n$. 
	To prove this, note that an annihilating polynomial of $(2)^\frac{1}{n}$ is an $n$ degree polynomial $f =x^n - 2$. 
	Applying Eisenstein's criterion  we found $f$ is irreducible, thus the minimal polynomial of $(2)^{\frac{1}{n}}$.
	Again apply theorem 5.1.5 from lecture notes we conclude $\bb{Q}((2)^{\frac{1}{n}})$ is of degree $n$ over $\bb{Q}$.
\end{proof}

\end{document}
