\documentclass{article}

\usepackage[tmargin=2.5cm,rmargin=3cm,lmargin=3cm,bmargin=3cm]{geometry} 
% Top margin, right margin, left margin, bottom margin, footnote skip
\usepackage[utf8]{inputenc}
\usepackage{biblatex}
% \addbibresource{./reference/reference.bib}
% linktocpage shall be added to snippets.
\usepackage{hyperref,theoremref}
\hypersetup{
	colorlinks, 
	linkcolor={red!40!black}, 
	citecolor={blue!50!black},
	urlcolor={blue!80!black},
	linktocpage % Link table of content to the page instead of the title
}

\usepackage{blindtext}
\usepackage{bm}
\usepackage{titlesec}
\usepackage{amsthm}
\usepackage{thmtools}
\usepackage{amsmath}
\usepackage{amssymb}
\usepackage{graphicx}
\usepackage{titlesec}
\usepackage{xcolor}
\usepackage{multicol}
\usepackage{hyperref}
\usepackage{import}


\newtheorem{theorem}{Theorema}[section]
\newtheorem{lemma}[theorem]{Lemma}
\newtheorem{corollary}{Corollarium}[section]
\newtheorem{proposition}{Propositio}[theorem]
\theoremstyle{definition}
\newtheorem{definition}{Definitio}[section]

\theoremstyle{definition}
\newtheorem{axiom}{Axioma}[section]

\theoremstyle{remark}
\newtheorem{remark}{Observatio}[section]
\newtheorem{hypothesis}{Coniectura}[section]
\newtheorem{example}{Exampli Gratia}[section]

% Proof Environments
\newcommand{\thm}[2]{\begin{theorem}[#1]{}#2\end{theorem}}

%TODO mayby proof environment shall have more margin
% \renewenvironment{proof}{\vspace{0.4cm}\noindent\small{\emph{Demonstratio.}}}{\qed\vspace{0.4cm}}
% \renewenvironment{proof}{{\bfseries\emph{Demonstratio.}}}{\qed}
\renewcommand\qedsymbol{Q.E.D.}
% \renewcommand{\chaptername}{Caput}
% \renewcommand{\contentsname}{Index Capitum} % Index Capitum 
\renewcommand{\emph}[1]{\textbf{\textit{#1}}}
\renewcommand{\ker}[1]{\operatorname{Ker}{#1}}

%\DeclareMathOperator{\ker}{Ker}

% New Commands
\newcommand{\bb}[1]{\mathbb{#1}} %TODO add this line to nvim snippets
\newcommand{\orb}[2]{\text{Orb}_{#1}({#2})}
\newcommand{\stab}[2]{\text{Stab}_{#1}({#2})}
\newcommand{\im}[1]{\text{im}{\ #1}}
\newcommand{\se}[2]{\text{send}_{#1}({#2})}

\title{Ars Amatoriae}
\author{Publius Ovidius Naso} 
\date{\today}

\begin{document}
% \tableofcontents

\subsection*{Q1}
Let $k\in \bb{N}, \sigma \in S_k$ and $p \in \bb{Q}[t_1, \cdots, t_k]$. We are to prove that $\sigma p = p(t_{\sigma(1)}, \cdots, t_{\sigma(k)})$ is a left group action. 

\begin{proof}
	This directly follows from the fact that $S_k$ is a group, and each element of it is a bijection from $1, \cdots, k$ to itself.

	Above all we shall check $p(t_{\sigma(1)}, \cdots, t_{\sigma(k)})$ is a polynomial of $k$ variable over $\bb{Q}$, which of couse is as $\sigma$ is bijection from $1, \cdots,k$ to itself.

	Next, notice for $e \in S_k$, $e p = p$. Secondly, for $a,b \in S_k$ we have 
	\begin{equation}
		(ab) p
		= p(t_{ab(1)}, \cdots, t_{ab(k)}) 
		=p(t_{a(b(1))}, \cdots, t_{a(b(k))})
		=a(b p)
	\end{equation}	

	This is the definition of left group action.
\end{proof}

\subsection*{Q2}
We want to prove the Galois group is indeed a group. That is, for a polynomial, $f$, with distinct roots written in the tuple form $\bm{r} = (r_1, \cdots, r_n)$, $Gal(f)$, which consists of all those $\sigma \in S_k$ such that $(r_1, \cdots, r_n)$ are conjugate to $\sigma \bm{r} = ( r_{\sigma(1)}, \cdots,  r_{\sigma{(k)}})$, is a subgroup of $S_k$.

\begin{proof}

	Observe first that if $\bm{r}$ is the root of $f$, $\sigma\bm{r}$ is necessarily a root of $\sigma f$. Similarly, if $\sigma f$ has root $\sigma \bm{r}$, applying $\sigma^{-1}$ to them we see $\sigma^{-1} \sigma f = f$ must have $\sigma^{-1} \sigma \bm{r} = \bm{r}$ as root.
	In summary:
	\begin{equation}
		f(\bm{r}) = 0 \iff   (\sigma f)(\sigma \bm{ r}) = 0
	\end{equation}

	% This is because, if $f(r_1, \cdots, r_k) = 0$, $\sigma f = f(x_{\sigma 1}, \cdots, x_{\sigma_k})$, and if we plug in $(a_{\sigma_1}, \cdots, a_{\sigma_k})$, $x_{\sigma1}$ will be substituted for $a_{\sigma_1}$, etc, and the value necessarily is 0.

	Let $\bm{r}$ is a root of $f$ and let it be conjugate to $\sigma \bm{r}$ and $\tau \bm{r}$,i.e., $\sigma, \tau\in Gal(f)$. 
	Let $p \in \bb{Q}[t_1, \cdots, t_k]$. 
	Note:
	\begin{equation}
		\begin{split}
		p(\tau \sigma \bm{r}) = 0 
		&\iff (\sigma \sigma^{-1} \tau^{-1}p)(\sigma \bm{r}) = (\tau^{-1} p)(\sigma \bm{r})    \text{ (by observation 2 and previous exercise)}\\
		&\iff (\tau^{-1} p)(\bm{r}) = 0 \text{(As $\bm{r}$ and $\sigma \bm{r}$ are conjugate )} \\
		&\iff  (\tau \tau^{-1}p)(\tau \bm{r}) = p(\tau \bm{r})= 0 \text{ (by observation 2 and previous exercise)} \\
		&\iff p(\bm{r}) = 0 \text{(As $\bm{r}$ and $\tau \bm{r}$ are conjugate )} \\
		\end{split}
	\end{equation}

	Unpacking the definition, we have proved that if $\sigma, \tau$ are conjugate to $\bm{r}$ so is $\sigma \tau$, which is equivelant to the statement $\sigma, \tau \in Gal(f) \rightarrow \sigma \tau \in Gal(f)$.
		
	Each $\sigma \in Gal(f)$ by definition is an element of $S_k$.
	As $k$ is finite integer, $S_k$ is a finite group and the inverse of $\sigma$ must be $\sigma^m$ for some positive integer $m$. 
	This means the statement $\sigma, \tau \in Gal(f) \implies \sigma \tau \in Gal(f)$ alone will be sufficient to prove $Gal(f)$ is a subgroup of $S_k$.


\end{proof}

\end{document}
