\documentclass[twocolumn]{article}

\usepackage[tmargin=2cm,rmargin=2.5cm,lmargin=2.5cm,bmargin=2cm,footskip=0.4cm]{geometry} 
% Top margin, right margin, left margin, bottom margin, footnote skip
\usepackage[utf8]{inputenc}
\usepackage{biblatex}
\addbibresource{./reference/reference.bib}
% linktocpage shall be added to snippets.
\usepackage{hyperref,theoremref}
\hypersetup{
	colorlinks, 
	linkcolor={red!40!black}, 
	citecolor={blue!50!black},
	urlcolor={blue!80!black},
	linktocpage % Link table of content to the page instead of the title
}

\usepackage{blindtext}
\usepackage{titlesec}
\usepackage{amsthm}
\usepackage{thmtools}
\usepackage{amsmath}
\usepackage{amssymb}
\usepackage{graphicx}
\usepackage{titlesec}
\usepackage{xcolor}
\usepackage{multicol}
\usepackage{hyperref}
\usepackage{import}
\usepackage{bm}


\newtheorem{theorem}{Theorema}[section]
\newtheorem{lemma}[theorem]{Lemma}
\newtheorem{corollary}{Corollarium}[section]
\newtheorem{proposition}{Propositio}[theorem]
\theoremstyle{definition}
\newtheorem{definition}{Definitio}[section]

\theoremstyle{definition}
\newtheorem{axiom}{Axioma}[section]

\theoremstyle{remark}
\newtheorem{remark}{Observatio}[section]
\newtheorem{hypothesis}{Coniectura}[section]
\newtheorem{example}{Exampli Gratia}[section]

% Proof Environments
\newcommand{\thm}[2]{\begin{theorem}[#1]{}#2\end{theorem}}

%TODO mayby proof environment shall have more margin
\renewenvironment{proof}{\vspace{0.4cm}\noindent\small{\emph{Demonstratio.}}}{\qed\vspace{0.4cm}}
% \renewenvironment{proof}{{\bfseries\emph{Demonstratio.}}}{\qed}
\renewcommand\qedsymbol{Q.E.D.}
\renewcommand{\chaptername}{Caput}
\renewcommand{\contentsname}{Index Capitum} % Index Capitum 
\renewcommand{\emph}[1]{\textbf{\textit{#1}}}
\renewcommand{\ker}[1]{\operatorname{Ker}{#1}}

%\DeclareMathOperator{\ker}{Ker}

% New Commands
\newcommand{\bb}[1]{\mathbb{#1}} %TODO add this line to nvim snippets

% ALGEBRA
\newcommand{\orb}[2]{\text{Orb}_{#1}({#2})}
\newcommand{\stab}[2]{\text{Stab}_{#1}({#2})}
\newcommand{\im}[1]{\text{im}{\ #1}}
\newcommand{\se}[2]{\text{send}_{#1}({#2})}

%STATISTICS
\newcommand{\var}[1]{\text{Var}(#1)}
\newcommand{\ud}[1]{\underline{#1}}
\newcommand{\cor}[1]{\text{Cor}(#1)}
\newcommand{\std}[1]{\text{Std}(#1)}
\newcommand{\ste}[1]{\text{S.E.}(#1)}


\begin{document}

\begin{fdefi}{Notation}{}
	\begin{enumerate}
		\item $W, U, V, W_i, U_i, V_i$, $i \in \Z$, denote open set;
		\item $X, Y$ denotes topological space;
		\item At most countable means countable or finite, countable means there is bijection to $\N$
	\end{enumerate}
\end{fdefi}
\hrulefill

\section{Basics}

\begin{defi}[limit point]
	A \emph{limit point} of a set $A$ is the \emph{accumulation point}.
	Let $A'$ denote the set of limit points of $A$.
\end{defi}

\begin{defi}
	\begin{enumerate}
		\item \emph{closure} of set $A$ is 
			$$
				\closure{A} = \bigcap \{ U \mid U \text{ is closed and } A \subseteq U \}
			$$
		\item \emph{interior} of set $A$ is 
			$$
				\interior{A} = \bigcup \{ U \mid U \text{ is open and } U \subseteq A \}
			$$
		\item \emph{boundary} of set $A$ is 
			$$
			\partial A = \closure{A} \smallsetminus \interior{A} 
			$$
		\item A set $A$ in topological space $X$ is dense if $\closure{A} = X$.
	\end{enumerate}
\end{defi}

\begin{thm}[On closure and interior]
	\begin{enumerate}
		\item $\closure{A} = A \cup A'$.
	\end{enumerate}
\end{thm}

\begin{thm}
	Suppose $f : X \rightarrow Y$
	\begin{enumerate}
		\item $f$ is continuous if and only if $f(\closure{A}) \subseteq \closure{f(A)}$  for all $A \subseteq X$
		\item $f$ is closed if and only if $f(\closure{A}) \supseteq  \closure{f(A)}  $ for all $A \subseteq X$
		\item $f$ is continuous if and only if $f^{-1}(\interior{B}) \subseteq \interior{f^{-1}(B)}$ for all $B \subseteq Y$ 
		\item $f$ is open if and only if $f^{-1}(\interior{B}) \supseteq \interior{(f^{-1}(B))}$ for all $B \subseteq Y$
	\end{enumerate}
\end{thm}

\begin{thm}
	A subset of $\R$ is open if and only if it is the at most countable union of disjoint non-empty open interval.
\end{thm}

\begin{remark}
	Any family of non-empty open intervals in $\R$ is at most countable. 
	This is because any non-empty open interval in $\R$ contains at least one rational number.
\end{remark}

\begin{defi}
	A metric space is complete if every Cauchy sequence converges.
\end{defi}

\begin{defi}
	A collection of subset $\B$ of $X$ ($\B \subseteq \P(X)$ ) is a basis for topology on $X$ if 
	\begin{enumerate}
		\item $\cup \B = X$
		\item For any $U, V \in \B$ there exists $W \in \B$ such that $W \subset  U \cap V$
	\end{enumerate}
\end{defi}

\subsection{Countability}

\begin{defi}
	A topological space is \emph{first countable} if there exists a countable neighborhood bases at each point.
	A neighborhood basis at a point $p$ is a collection of set $\B_p$ such that every neighborhood at $p$ is contained in some $U \in \B_p$.
\end{defi}

\begin{thm}
	Suppose $X$ is a first countable space, $A$ is any subset of $X$, and $x$ is any point of $X$.
	\begin{enumerate}
		\item $x \in \closure{A}$ if and only if $x$ is a limit of a sequence of points in $A$.
		\item $x \in \interior{A}$ if and only if every sequence in $X$ converging to $x$ is eventually in $A$. 
		\item $A$ is closed in $X$ if and only if $A$ contains every limit of every convergent sequence of points in $A$.
		\item $A$ is open in $X$ if and only if every sequence in $X$ converging to a point of $A$ is eventually in $A$.
	\end{enumerate}
\end{thm}

\begin{defi}
	A topological space is \emph{second countable} if it has a countable bases.
\end{defi}

\begin{thm}
	Supppose $X$ is a second countable space.
	\begin{enumerate}
		\item $X$ is first countable.
		\item $X$ contains a countable dense set.
		\item Every open cover of $X$ has countable subcover.
	\end{enumerate}
\end{thm}

\section{New Topologies From Old}

\begin{thm}
	If $f$ is a continuous map. 
	Restricting thge domain, codomain, or expanding the ccodomain does not affect the continuity.
\end{thm}

\begin{example}
	Let $M$ be a subspace of $X$ with subspace topology.
	\begin{enumerate}
		\item If $X$ is hausdorff so is $M$.
	\end{enumerate}
\end{example}

\begin{defi}[Weak Topology]
	Assuming $X$ is a set and $X_i, T_i$ are topologies and $f_i: X \rightarrow X_i$. 
	The weak topology of $X$ generated by $f_i$ is the smallest topology on $X$ which make all of the $f_i$ continuous.
\end{defi}

\begin{defi}[Quotient Topology]
	Let $X$ be a topological space, $Y$ any set, and $f: X \rightarrow Y$ a surjective map.
	Define a subset $U \subset Y$ to be open if and only if $f^{-1}(U)$ is open in $X$. 
	This is the quotient topology.
\end{defi}

\begin{defi}[Equivalence relationship]
	An relationship $\sim$ is an equivalence relationship if it is 
	\begin{enumerate}
		\item reflexive: $x \sim x$ for all $x$
		\item symmetric: $x \sim y \implies y \sim x$
		\item transitive: if $x \sim y $ and $y \sim z$, then$ \implies x \sim z$
	\end{enumerate}
\end{defi}

\begin{example}
	Real projective space $\R P^n$ is $\S^n \slash \sim$, where $x \sim y $ if and only if $ x = y$ or $x = -y$.
\end{example}

\section{Compact}

\begin{thm}[Closed Map Theorem]
	Let $X$ be compact and $Y$ be Hausdorff. $f: X \rightarrow Y$ continuous, then
	\begin{enumerate}
		\item $f$ is a closed map
		\item If $f$ is injective it is a topological embedding
		\item If $f$ is surjective it is a quotient map 
		\item If $f$ is bijective it is a homeomorphism
	\end{enumerate}
\end{thm}

\begin{thm}
	$X \times Y$ is compact if and only if both $X$ and $Y$ are comapct.
\end{thm}

\begin{thm}
	If $X$ is a compact Hausdorff space and $q: X \rightarrow Y$ is a quotient map, then $Y$ is hausdorff if and only if $f$ is a closed map.
\end{thm}

\begin{defi}[One Point Compactification]
	Let $X$ abe a topological space. 
	The one point compacitifaction of $X$ is the topological space $X^* = X \cup \{\infty\}$ such with open sets either 
	\begin{enumerate}
		\item $U \subseteq X$ is open in $X$ or
		\item $U = X^* \smallsetminus K$ where $K$ is compact and closed in $X$.
	\end{enumerate}
\end{defi}

\begin{defi}[Locally compact]
	A topological space is locally compact if and only if for every point $x$ there exists a neighborhood $U$ of $x$ such that $\closure{U}$ is compact, or equivalently, $U$ is contained in a compact set.
\end{defi}

\begin{thm}
	The one point compactification of $X$ is hausdorff if and only if $X$ is hausdorff and locally compact.
\end{thm}

\section{Normal}

\begin{defi}[Normal]
	A topological space $X$ is normal if and only if every two disjoint closed sets can be separated by disjoint open sets.
\end{defi}

\begin{thm}
	$X$ if normal if and only if for all $A \subset B$, where $A$ is closed and $B$ is normal, there exists open set $B'$ and closed set $A'$ such that 
	$$
		A \subset B' \subset A' \subset B
	$$
\end{thm}

\begin{thm}[Urysohn's Lemma]
	Let $X$ be a normal space and $A, B$ be disjoint closed sets in $X$. 
	Then there exists a continuous function $f: X \rightarrow [0, 1]$ such that $f(A) = 0$ and $f(B) = 1$.	
\end{thm}

\begin{thm}[Tietze Extension Theorem]
	Let $X$ be a normal space and $A$ be a closed subset of $X$. 
	Then every continuous function $f: A \rightarrow [0, 1]$ can be extended to a continuous function $F: X \rightarrow [0, 1]$.
\end{thm}

\begin{thm}[Stone-Weierstrass Theorem]
	Let $X$ be a compact Hausdorff space and $f: X \rightarrow \R$ be a continuous function. 
\end{thm}

\end{document}

