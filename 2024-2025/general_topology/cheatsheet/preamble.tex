\usepackage[margin=0.25in]{geometry} % Top margin, right margin, left margin, bottom margin, footnote skip
\usepackage[utf8]{inputenc}
\usepackage{biblatex}
\addbibresource{./references.bib}
% linktocpage shall be added to snippets.
\usepackage{hyperref,theoremref}
\hypersetup{
	colorlinks, 
	linkcolor={red!40!black}, 
	citecolor={blue!50!black},
	urlcolor={blue!80!black},
	linktocpage % Link table of content to the page instead of the title
}

\usepackage{blindtext}
\usepackage{titlesec}
\usepackage{keytheorems}
\usepackage{amsthm}
\usepackage{amsmath}
\usepackage{amssymb}
\usepackage{graphicx}
\usepackage{titlesec}
\usepackage{xcolor}
% \usepackage{multicol}
\usepackage{hyperref}
\usepackage{import}




%TODO mayby proof environment shall have more margin
\renewenvironment{proof}{\vspace{0.4cm}\noindent\small{\emph{Demonstratio.}}}{\qed\vspace{0.4cm}}
% \renewenvironment{proof}{{\bfseries\emph{Demonstratio.}}}{\qed}
\renewcommand\qedsymbol{Q.E.D.}
% \renewcommand{\chaptername}{Caput}
% \renewcommand{\contentsname}{Index Capitum} % Index Capitum 
\renewcommand{\emph}[1]{\textbf{\textit{#1}}}
\newcommand{\Z}{\mathbb{Z}}
\newcommand{\N}{\mathbb{N}}
\newcommand{\R}{\mathbb{R}}
\renewcommand{\S}{\mathbb{S}}
\newcommand{\B}{\mathcal{B}}
\renewcommand{\P}{\mathcal{P}}
\newcommand*\interior[1]{#1^{\circ}}
\newcommand*\closure[1]{\overline{#1}} % (or \bar{#1})

\usepackage[most]{tcolorbox}

\renewcommand{\emph}[1]{{\color{blue!70!black}\sffamily\bfseries #1}}

\tcbuselibrary{theorems}
\newtcbtheorem{fdefi}{Definitio}%
{fonttitle=\bfseries}{th}
\newtcbtheorem{fthm}{Theorema}%
{fonttitle=\bfseries}{th}
\newtcbtheorem{eg}{\textit{E.G.}}
{colback=white,              % background of the box
    colframe=black,             % dark frame color
    coltitle=black,             % title text color
	colbacktitle=gray!10!white,
	colback=gray!10!white,
    fonttitle=\bfseries,        % title font
    boxrule=1pt,              % border thickness
    arc=2mm,                    % rounded corners
    boxsep=0.7mm,                 % inner spacing
    enhanced,
}{th}

\theoremstyle{definition}
\newtheorem{thm}{THM}
\newtheorem{lemma}{Lemma}
\newtheorem{corollary}[thm]{Corollarium}
\newtheorem{example}[thm]{Exampli Gratia}
\newtheorem{defi}[thm]{DEF}

\theoremstyle{remark}
\newtheorem{remark}[thm]{Remark}

