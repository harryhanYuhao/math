\documentclass[twocolumn]{article}
\usepackage[a4paper, margin=0.3in]{geometry} % Top margin, right margin, left margin, bottom margin, footnote skip


\usepackage[utf8]{inputenc}
\usepackage{biblatex}
\addbibresource{./references.bib}
% linktocpage shall be added to snippets.
\usepackage{hyperref,theoremref}
\hypersetup{
	colorlinks, 
	linkcolor={red!40!black}, 
	citecolor={blue!50!black},
	urlcolor={blue!80!black},
	linktocpage % Link table of content to the page instead of the title
}

\usepackage{blindtext}
\usepackage{mathrsfs}
\usepackage{yhmath}
\usepackage{breqn}
\usepackage{titlesec}
\usepackage{keytheorems}
\usepackage{amsthm}
\usepackage{amsmath}
\usepackage{amssymb}
\usepackage{graphicx}
\usepackage{titlesec}
\usepackage{xcolor}
% \usepackage{multicol}
\usepackage{hyperref}
\usepackage{import}

\renewenvironment{proof}{\vspace{0.4cm}\noindent\small{\emph{Demonstratio.}}}{\qed\vspace{0.4cm}}
% \renewenvironment{proof}{{\bfseries\emph{Demonstratio.}}}{\qed}
\renewcommand\qedsymbol{Q.E.D.}
% \renewcommand{\chaptername}{Caput}
% \renewcommand{\contentsname}{Index Capitum} % Index Capitum 
% \renewcommand{\emph}[1]{\textbf{\textit{#1}}}
\renewcommand{\emph}[1]{{\color{blue!70!black}\sffamily\bfseries #1}}
\renewcommand{\ker}[1]{\operatorname{Ker}{#1}}

\DeclareMathOperator{\sech}{sech}
\DeclareMathOperator{\csch}{csch}

\newcommand{\X}{\mathscr{X}}
\newcommand{\n}{\nabla}
\newcommand{\R}{\mathbb{R}}
\newcommand{\C}{C^{\infty}}
\newcommand{\CY}{\mathfrak{C}}
\newcommand{\p}{\partial}
\newcommand{\G}{\Gamma}
% mathscr is from mathrsfs
\renewcommand{\L}{\mathscr{L}}
\newcommand{\g}{\gamma}
\newcommand{\dg}{\dot{\gamma}}
\newcommand{\N}{\textit{\underline{NOTE} }}
\newcommand{\W}{\Omega}
\newcommand{\ot}{\otimes}
\renewcommand{\dg}{\dot{\gamma}}

\usepackage[most]{tcolorbox}

\tcbuselibrary{theorems}
\newtcbtheorem{fdefi}{Definitio}%
{fonttitle=\bfseries}{th}
\newtcbtheorem{fthm}{Theorema}%
{fonttitle=\bfseries}{th}
\newtcbtheorem{eg}{\textit{E.G.}}
{colback=white,              % background of the box
    colframe=black,             % dark frame color
    coltitle=black,             % title text color
	colbacktitle=gray!10!white,
	colback=gray!10!white,
    fonttitle=\bfseries,        % title font
    boxrule=1pt,              % border thickness
    arc=2mm,                    % rounded corners
    boxsep=0.7mm,                 % inner spacing
    enhanced,
}{th}

\theoremstyle{definition}
\newtheorem{thm}{THM}
\newtheorem{lemma}{Lemma}
\newtheorem{corollary}[thm]{Corollarium}
\newtheorem{example}[thm]{Exampli Gratia}
\newtheorem{remark}[thm]{Remark}
\newtheorem{defi}[thm]{DEF}




\begin{document}
% \tableofcontents
\begin{fdefi}{Notation}{}
	\begin{enumerate}
		\item $X, Y, Z, W \in \X(M)$ denotes vector fields. 
		\item $\alpha, \beta \in \W^1$ is one forms, unless specified otherwise.
		\item (1,2)-tensors fields are section of bundle $TM \otimes T^{*}M^{\ot 2}$. 
		It is equivalent to a a trilinear map $\W^1(M) \times \X(M) \times \X(M) \rightarrow C^{\infty}(M)$, or a bilinear map $\X(M) \times \X(M) \to \X(M)$.
	\end{enumerate}
\end{fdefi}

\hrulefill
\section{Connections}

\begin{defi}
	An \emph{affine connection} on $M$ is an $\R$ bilinear map 
	$$
		\n: \X (M) \times \X(M) \to \X (M), (X,Y) \mapsto \n_X Y
	$$
	satisfying for all $X, Y \in \X(M)$ and $f \in \C(M)$: 
	\begin{enumerate}
		\item $\n_{fX} Y = f \n_X Y$
		\item $\n_{X} fY = X(f) Y + f \n_X Y$
	\end{enumerate}

	Define connection coefficients $\G^k_{ij}$ by $\n_{\p_i}\p_j = \G_{ij}^k \p_k$.
\end{defi}
\N Affine connection is not (1,2) tensor field. But different between affine connection is.

\N A vector field $Y$ is parallel if $\n_X Y = 0$ for all $X$.

\begin{defi}
	Let $\n$ be an affine connection on $M$.
	The \emph{torsion} of $\n$ is $(1,2)$ tensor field $T$ defined by
	$$
	(X, Y) \mapsto T(X, Y) = \n_X Y - \n_Y X - [X, Y]
	$$
	
	\emph{Curvature} is $(1,3)$ tensor field $R$ defined by
	$$
	(X,Y,Z) \mapsto R(X, Y)Z = \n_X \n_Y Z - \n_Y \n_X Z - \n_{[X, Y]} Z
	$$

	An affine connection with zero curvation is called \emph{flat}, for which
	$$
		\n_X \n_Y Z-  \n_Y \n_X Z = \n_{[X, Y]} Z \text{ (flat connection) }
	$$
	whereas one with zero torsion is called \emph{symmetric} or \emph{torsion-free}, for which
	$$
	\n_X Y - \n_Y X = [X, Y] \text{ (torsion free)}
	$$

	Define $ T(\p_i, \p_j) := T_{ij}^k \p_k$, and 
	$$T_{ij}^k = \G_{ij}^k - \G_{ji}^k$$
	Define $R_{ijk}^l \p_l = R(\p_i, \p_j)\p_k$, and 
	$$R_{ijk}^l = \p_i \G_{jk}^l - \p_j \G_{ik}^l + \G_{jk}^m\G_{im}^l - \G_{ik}^m \G_{jm}^l$$
\end{defi}
\begin{defi}
	The \emph{Ricci} tensor is a (0,2) tensor defined by $Ric(X,Y) = tr(Z \mapsto R(Z,X)Y)$, or 
	$$
	R_{ij} = R_{kij}^k = g^{kl} R_{kijl}
	$$
	Ricci tensor may not be symmetric. 
	Ricci tensor of Levi-Civita connection is symmetric.
\end{defi}


\begin{defi}[Extension of Connection]
	For $f \in \C(M)$. Define 
	$$
		\n_X f = X(f)
	$$
	For one form $\alpha \in \W^1(M)$, define 
	$$
		\n_X(\alpha (Y)) = (\n_X \alpha)(Y) - \alpha(\n_X Y)
	$$

	For $\Phi$, which is $(1,1)$ tensor, define $\n_X \Phi$ to be an $(1,1)$ tensor such that 
	$$
		\n_X \Phi(Y) = \n_X(\Phi(Y)) - \Phi(\n_X Y)
	$$
	For $(1,2)$ tensor $T$, define $\n_X T$ to be a $(1,2)$ tensor such that
	$$
		\n_X T(Y, Z) = \n_X(T(Y, Z)) - T(\n_X Y, Z) - T(Y, \n_X Z)
	$$

	Note 
	$$
		X g(Y, Z) = (\n_X g)(Y, Z) + g(\n_X Y, Z) + g(Y, \n_X Z)
	$$
\end{defi}

\begin{defi}[Cyclic Operator]
	Define the cyclic operator $\CY$ be 
	$$
		\CY(X, Y, Z) = A(X, Y, Z) + A(Y, Z, X) + A(Z, X, Y)
	$$
\end{defi}

\begin{fthm}{Bianchi Identities}{}
	\begin{enumerate}
		\item Algebraic Bianchi identity 
			$$
				\CY R(X, Y) Z = \CY(\n_X T)(Y, Z) + \CY T (T (X, Y),Z)
			$$
		\item Differential Bianchi identity 
			$$
				\CY((\n_Z R)(X, Y))W + \CY R(T(X, Y), Z)W = 0
			$$
	\end{enumerate}
\end{fthm}

\begin{defi}[Koszul Connections]
	A \emph{Koszul connection} on a vector bundle $E \rightarrow  M$ is an $\R$ bilinear map 
	$$
		\n: \X (M) \times \G(E) \to \G (E), (X,s) \mapsto \n_X s
	$$
	satisfying for all $X \in \X(M)$, $s \in \G(E)$, and $f \in \C(M)$: 
	\begin{enumerate}
		\item $\n_{fX} s = f \n_X s$
		\item $\n_{X} fs = X(f) s + f \n_X s$
	\end{enumerate}

\end{defi}

\N Koszul connection is not tensorial, but their difference is.

\N A section $s$ is parallel if $\n_X s = 0$ for all $X$.

\N Koszul Connection may not have a torsion tensor, but will have a curvature tensor defined as 
$$
	R(X, Y)s = \n_X \n_Y s - \n_Y \n_X s - \n_{[X, Y]} s
$$
A Koszul connection is \emph{flat} if the curvature is zero.

\begin{defi}[Connection Coefficients]
	The \emph{connection coefficients}, $\omega_{i\ a}^{\ b} \in \C(M)$ relative a local frame of $s$, $e_a$, is $\n_{\p_i} e_a = \omega_{i\ a}^{\ b} e_b$
\end{defi}

\begin{defi}[Covariant Derivative]
	Let $\n$ be a Koszul connection on vector bundle $E \rightarrow M$. 
	\emph{Covariant derivative} on $\G(\g^{*} E)$ induced by $\n$ is the $\R$ linear map 
	$$
	\frac{D}{dt} : \G(\g^* E) \rightarrow \G(\g^* E)
	$$
	satisfying 
	\begin{enumerate}
		\item For all $s \in G(E)$, where $\dot{\g} = \gamma_* (\frac{d}{dt})$
			$$
			\frac{D}{dt} \g^*s = \g^* \n_{\dot{\g}} s
			$$
		\item for all $f \in \C(I)$ and $\sigma \in \G(\g^*E)$
			$$
			\frac{D}{dt}(f \sigma) = \frac{df}{dt} \sigma + f \frac{D \sigma}{dt}
			$$
	\end{enumerate}

	$\sigma \in \G(\g^*E)$ is \emph{parallel} if $\frac{D \sigma}{dt} = 0$.
\end{defi}

\begin{corollary}
	Suppose $s \in \G(E)$ such that $\n_X s = 0$ for all $X$, and if $s(p) = 0$ for some $p$ then $s \equiv 0$.
\end{corollary}

\begin{defi}
	Let $\n$ be an affine connection on $M$, and a smooth curve $\gamma: I \rightarrow M$, is a \emph{geodesic} if its velocity $\dg = \g_{*}(\frac{d}{dt})$, which is a section of $\g^* TM \rightarrow I$, is self parallel, that is 
	$$
		\frac{D}{dt} \dg = 0
	$$
	Writing in local coordinate, $\dg = \dg^i \p_i$ is geodesic if and only if the functions $\dg^i(t)$ satisfy the ODE
	$$
	\ddot{\g}^k(t) + \G_{ij}^{\ \ k}(\g(t)) \dg^i(t) \dg^j(t) = 0
	$$

\end{defi}

\section{Riemannian Metric}

\begin{defi}[Flat and Sharp]
	 For a vector space $V$ with inner product $\langle -, - \rangle$. 
	 The musical isomorphism is defined as
	 \begin{align*}
		 \flat &: V \rightarrow V^*, v \mapsto v^{\flat} = \langle v, - \rangle \\
		 \sharp &:  V^* \rightarrow V, \lambda \rightarrow  \lambda^{\sharp} = \langle \lambda^{\sharp}, v \rangle = \lambda(v) \ \forall v
	 \end{align*}
\end{defi}

\begin{eg}{}{}
	Let $(e_1, \cdots, e_n)$ denote bases and $(\theta^1, \cdots, \theta)$ denote canonical dual bases. 
	Let $\langle e_i, e_j\rangle = \eta_{ij}$, and $[\eta^{ij}]$ be the inverse of $[\eta_{ij}]$.
	Let $v = v^ie_i$, we have 
	$$
	v^{\flat}(e_j) = v^i \eta_{ij} := v_i
	$$
	and $v^{\flat}= v_i \theta^{i}$.
	Similarly, let $\alpha = \alpha_i \theta^i$, and $\alpha^i = \eta^{ij}\alpha_j$, and we have $\alpha^{\sharp} = \alpha^i e_i$.
\end{eg}

\begin{example}[Example of calculating musical isomorphism]
	Assuming we have metric $g = 2du dv + u^2(dx^2 + dy^2)$.
	Note $\p_u ^{\flat} = \langle \p_u, - \rangle$, and $\p_d^{\flat}(\p_v) = 1$ and otherwise $0$. Thus $\p_u^{\flat} = dv$.
	Similarly, $\p_v^{\flat} = du, \p_x^{\flat} = u^2 dx, \p_y^{\flat} = u^2 dy$.
\end{example}

\begin{thm}[Sylvester Law of Inertia]
	Let $E^{s,t}$ be a $t = s+t$ dimensional inner product space, with $s$ keys positive inner product and $t$ negative. That is, the inner product takes the form $(dx^1)^2 + \cdots + (dx^s)^2 - (dx^{s+1})^2 - \cdots - (dx^n)^2$
	Every $n$ dimensional inner product space is isometric to $E$.
	$(s, t)$ is callled the signature.
\end{thm}

\begin{defi}[Metric]
	A metric of signature $(s,t)$ is a section $g \in \G(\odot ^2 T^*M)$, such that for each $p$, $g_p$ is a inner product on $T_pM$ of signature $(s, t)$.
	In particular, a metric needs to be non-degenerate, bilinear, and symmetric.
	\emph{Non-degenerate} means that if there exists $w$ suchg that for all $v \in T_pM$, $g(v, w) = 0$, then $w = 0$.
\end{defi}

\begin{defi}
	A \emph{lorentzian manifold} is one whose metric hass the signature $(n-1,1)$.
	The $(3,1)$ lorentzian manifold $(M, g)$ with metric $-dt^2 + dx^2 + dy^2 + dz^2$ is \emph{Minkowski spacetime}.
\end{defi}

\begin{defi}
	Let $(M, g)$ and $(M', g')$ be two Riemannian manifolds, and $f: M \rightarrow M'$ a diffeomorphism. 
	Then $f$ is \emph{isometric} if $f^*g' = g$.
	That is 
	$$
	g_p(X_p, Y_p) = f^*g'(X_p, Y_p) = g'_{F(p)}((F_*)_p(X_p), (F_*)_p(Y_p))
	$$
\end{defi}

\begin{example}
	Let $(M, g)$ and $(N, g)$ be riemannian manifolds and let $i: N \rightarrow M$ be an isometric embedding. Any isometry $\Psi$ of $(M, g)$ which preserves $i(N)$ induces isometry of $(N, h)$.

	\textbf{solution}:
	The isometry $\Psi$ induce diffeomorphism $\psi$ on $N$ such that $\Psi \circ i = i \circ \psi$. 
	Assuming $\Psi$ preserves $g$, we want to show that $\psi$ preserves $h$.
	$\psi^*h = \psi^*i^*g = (i \circ \psi)^* g = (\Psi \circ i)^*g = i^* \Psi^* g = i^* g = h$.
\end{example}

\begin{fthm}{Levi-Civita Connection}{}
	Each riemannian maniford admits a torsionless, metric compatible connection, called Levi-Civita Connection.
	That is
	$$
		\n_X Y - \n_Y X = [X, Y], \ \n g = 0
	$$
	The connection coefficients of this connection is \emph{Christoffel symbols}.
	$$
		\G_{ij}^m = \frac{1}{2} g^{km}(\p_i g_{jk} + \p_j g_{ik} - \p_k g_{ij} )
	$$
\end{fthm}

\begin{example}
	\textbf{Proof of the Theorem}

	$\n g = 0$ means 
	\begin{align*}
		0 &= (\n_X g)(Y, Z) = Xg(Y, Z) - g(\n_X Y, Z) - g(Y, \n_X Z)\\
		0 &= (\n_Y g)(X, Z) = Yg(X, Z) - g(\n_Y X, Z) - g(X, \n_Y Z)\\
		0 &= (\n_Z g)(Y, X) = Zg(Y, X) - g(\n_Z Y, X) - g(Y, \n_Z X)
	\end{align*}
	Substract the third from the first two we geometry

	\begin{dmath}
		 (\n_X g)(Y, Z) (\n_Y g)(X, Z) + - (\n_Z g)(Y, X) 
		 =
 Xg(Y, Z) - g(\n_X Y, Z) - g(Y, \n_X Z) +
 Yg(X, Z) - g(\n_Y X, Z) - g(X, \n_Y Z)
		 - Zg(Y, X) + g(\n_Z Y, X) + g(Y, \n_Z X)
		 = g(Z, \n_X Y + \n_Y X) + g(X, \n_Y Z - \n_Z Y) + g(Y, \n_X Z - \n_Z X)
	\end{dmath}
	Using torsion free, that is 
	$$
		\n_X Y - \n_Y X = [X, Y]
	$$
	We can rewrite the right-handside of the equation as 
	$$
	2g(\n_X Y,Z) + g(X, [Y,Z]) + g(Y, [X,Z]) - g(Z, [X,Y])
	$$
	Plugging in, we are done.
\end{example}

\begin{thm}
	Writing in local coordinate, a vector field $\xi = \xi^{\mu} \p_{\mu} $ is \emph{parallel} ($\n_X \xi = 0$ for all $X \in \X(M)$) if and only if it satisfies the following PDE. (${\rho}$ is a dummy variable, just showing we need to iterate over all $\rho$ for each of the $i$ equation.)
	$$
	(\p_i \xi^{\rho} + \xi^{\mu} \G_{i\mu}^{ \rho})\p_{\rho} = 0 \text{ for all } i
	$$
\end{thm}

\begin{defi}[Killing Vector]
	A vector field $X$ is a Killing vector field if $\L_X(g) = 0$, that is 
	$$
	Xg(Y, Z) = g([X, Y], Z) + g(Y, [X, Z])
	$$
\end{defi}

\N If $g$ does not depend on variable $u$, $\p_u$ is a Killing vector field.

\N Killing vector field is a real vector space.

\begin{fthm}{Killing Equation}{}
	Let $(M, g)$ be riemannian manifold with Levi-Civita connection $\n$. A Killing vector field $X$ obeys the killing equation 
	$$
		g(\n_Y X, Z) + g(Y, \n_Z X) = 0
	$$
	Relabing $\n_i X = (\n_i X^k) \p_k$, we have 
	$$
		g(\n_i X, \p_j) = \n_i X
	$$
	Thus killing equation is equivalent to 
	$$
		\n_i X_j + \n_j X_i = 0
	$$
	Write in local coordinate, we have 
	$$
		\p_u X_v + \p_v X_u = 2 \G_{uv}^p X_p
	$$
	Where $X^v$ denotes $dv(X)$ and $X_v = g_{uv}X^u$.
	% TODO: UNCLEAR
\end{fthm}

\begin{thm}
	For all sections $U, V \in \G(\g^* TM)$
	$$
	\frac{d}{dt} g(U, V) = g(\frac{DU}{dt} , V) + g(U, \frac{DV}{dt})
	$$
\end{thm}

\begin{thm}[Geodesics of Levi-Civita]
	Let $\g$ be an affinely parametrised geodiscis of the manifold $(M, g)$.
	Then $g(\dg, \dg)$ is constant.
	The geodesic is \emph{timelike} if $g(\dg, \dg) < 0$; \emph{null} if $g(\dg, \dg) = 0$; \emph{spacelike} if $g(\dg, \dg) > 0$.

	Let $X$ be a Killing vector field, then $g(\dg, X)$ is constant along the geodesic $\g$.
\end{thm}

\section{Riemann Curvature}
\begin{defi}
	\emph{Riemann curvature} tensor is a $(0, 4)$ tensor defined by 
	$$
	Riem(X, Y, Z, W) = g(R(X, Y)Z, W)
	$$
	Note 
	\begin{align*}
		Riem(X, Y, Z, W) &= -Riem(Y, X, Z, W) \\
		Riem(X, Y, Z, W) &= Riem(X, Y, W, Z) \\
		Riem(X, Y, Z, W) &= Riem(W, Z, X, Y) \\
		Riem(X, Y, Z, W) &+ Riem(Y, Z, X, W) + Riem(Z, X, Y, W) = 0
	\end{align*}
\end{defi}

\N $R_{ijkl} = R_{ijk}^{\ \ \ m} g_{ml}$

\begin{eg}{Calculation of Ricci}{}
	$R_{ij} = R_{kij}^k = g^{kl}R_{kijl} = g^{lk}R_{ljik} = R_{lji}^l = R_{ji} $ 
\end{eg}

\begin{defi}[Ricci Scalar]
	Ricci scalar is $R := g^{ij}R_{ij}$
\end{defi}

\begin{fdefi}{Einstein tenser}{}
	\emph{Einstein tensor} of a Levi-Civita connection is 
	$$
	Ein(X, Y) = Ric(X, Y) - \frac{1}{2}g(X, Y)R
	$$
\end{fdefi}

\begin{defi}[Einstein Manifold]
	A riemannian manifold is an \emph{Einstein Manifold} if 
	$$
	Ric = \lambda g, \text{ equivalently } R_{ij} = \lambda g_{ij}
	$$
	For some constant $\lambda$.
\end{defi}

Let $G$ denote Einstein tensor, so $G_{ij} = R_{ij} - \frac{1}{2}g_{ij}R$.

\begin{thm}
	$$
\n ^{j} G_{jk} := g^{ij}\n_{i} G_{jk} = \n^j(R_{ij} - \frac{1}{2}g_{ij}R)= 0
	$$
\end{thm}

\begin{eg}{Schur's Lemma}{}
	In an connected $n$ dimensional manifold with $n \geq 3$, if $Ric = fg$ for some $f \in \C(M)$, then $f$ is constant function.

	\tcblower
	Since $R_{ij} = fg_{ij}$, taking the trace $R = g^{ij}R_{ij} = g^{ij}g_{ij}f = nf$, so 
	$$R_{ij} = \frac{1}{n} g_{ij} R$$
	Differentiating and trace (recall $ \n g = 0$)
	$$
		g^{jk} \n_{k}R_{ij} = \frac{1}{n} \p_i R 
	$$
	However, we know that $G$ has zero divergence, that is, 
	$$
		g^{jk} \n_{k}R_{ij} = \frac{1}{2} \p_i R 
	$$
	So $\p_i R = 0$. That is, $R$ is constant, so if $f$.
\end{eg}

\begin{fthm}{Cartan's Method of Moving Frame}{}
	Let $(e_a)$ be a local orthonormal frame and $(\theta ^a)$ its dual frame.
	Relative to a local chart, $e_a = e^i_a \p_i$ and $\theta^a = \theta^a_i dx^i$.

	We can show that $e^i_a \theta^b_i = \delta^a_b$, and $e^i_a \theta^a_j = \delta^i_j$.
	We can define connection one form as 
	$$
	\n_X e_a= \omega^b_{\ a}(X) e_b
	$$
	Define $g(e_a, e_b) = \eta_{ab}$, and 
	$$
	\omega_{ab} = \eta_{ac} \omega^c_{\ b} = - \omega_{ba}
	$$
	\emph{First Structure Equation} is
	\begin{align}
		d \theta^a + \omega^a_{\ b} \wedge \theta^b = 0
	\end{align}

	\emph{Second Structure Equation} is
	\begin{align}
		d \omega^a_{\ b} + \omega^a_{\ c} \wedge \omega^c_{\ b} = \Omega^a_{\ b}
	\end{align}
	Define 
	$$
		- \W_{ba} = \W_{ab} = \eta_{ac}\W^c_{\ b} 
	$$
	Where $\W^a_{\ b} \in \W^2(U)$ is curvature two form defined by 
	$$
		\W(X,Y)^a_{\ b} e_a = R(X, Y) e_b
	$$
	Riemann curvature can be calculated by 
	\begin{align}
		\frac{1}{2} \W_{ab}(\theta^a \wedge \theta^b) = - \frac{1}{4} R_{ijkl}(dx^i \wedge dx^j)(dx^k \wedge dx^l)
	\end{align}
\end{fthm}

\N On \emph{Notation}
$v_1 \wedge v_2 := v_1 \otimes v_2 - v_2 \otimes v_1$; $v_1v_2 = \frac{1}{2}(v_1\otimes v_2 + v_2\otimes v_1)$.

\begin{example}
	For metric $g: 2dudv + u^2(dx^2 + dy^2)$, use Witt frame $\theta^+ = du, \theta^- = dv, \theta^1 = udx, \theta^2 = udy$, $g = 2 \theta^2+\theta^{-} + (\theta^1)^2 + (\theta^2)^2$, and the matrix is 
	$ 
	\begin{pmatrix}
		0 & 1 & 0 & 0 \\
		1 & 0 & 0 & 0 \\
		0 & 0 & 1 & 0 \\
		0 & 0 & 0 & 1 \\
	\end{pmatrix}
	$
\end{example}

\begin{defi}[Hodge Star Operator]
	For an inner product space $V$, $\wedge^{\bullet}V = \bigoplus^n_{k=0} \wedge^k V$, which is the exterior algebra of $V$, where $\wedge^0 V = \R$ and $\wedge^1 V = V$, are equipped with an inner product defined thus:
	\begin{enumerate}
		\item for $\lambda, \mu \in \wedge^0 V = \R$, $\langle \lambda, \mu \rangle = \lambda \mu$;
		\item $\wedge^k V$ and $\wedge^l V$ is orthogonal (inner product is zero) if $k \neq l$;
		\item $\langle v_1 \wedge v_2 \cdots \wedge v_k, w_1 \wedge \cdots \wedge w_k \rangle = det\langle v_i, w_j \rangle$
	\end{enumerate}
	Given an orthonormal bases $(e_1, \cdots, e_n)$ of $V$, define the volume form $vol = e_1\wedge e2 \wedge \cdots \wedge e_n$. 
	Define a linear transformation $\star: \wedge^k V \mapsto \wedge^{n-k}V$ for each $\beta \in \wedge^{k} V$ thus
	$$
	\alpha \wedge \star \beta = \langle \alpha, \beta \rangle vol, \text{ for all } \alpha \in \wedge^{k} V
	$$
\end{defi}

\begin{example}
	For $dx^i$ being orthonormal bases in $n$ dimensional manifold.
	$\langle dx^1 \wedge dx^2, dx^1 \wedge dx^2 \rangle = 1$

	Indeed for any $I$, permutation of indeces of length $k$,
	$\langle dx^I , dx^I \rangle = 1$
	If $I,J$ are two permutation of different length, or there does not exists any permutation $\sigma$ such that $\sigma I = J$, then $\langle dx^I, dx^J \rangle = 0$.
\end{example}
\begin{example}
In Riemann 4 dimensional manifold with signature $(4, 0)$ or $(2,2)$, $\star \star = id$, while those with signature $(3,1)$, $\star \star = - id$.

Take 4 dimensional Riemann manifold with metric $(4,0)$ and bases $dx^1, dx^2, dx^3, dx^4$ for example.
\begin{align*}
	\star dx^1 &= dx^2 \wedge dx^3 \wedge dx^4 \\
	\star dx^2 &= - dx^1 \wedge dx^3 \wedge dx^4 \\
	\star dx^3 &=  dx^1 \wedge dx^2 \wedge dx^4 \\
	\star dx^4 &= - dx^1 \wedge dx^2 \wedge dx^3 \\
	\star dx^1 \wedge dx^2 &=  dx^2 \wedge dx^3 \\
	\star dx^1 \wedge dx^3 &=  - dx^2 \wedge dx^4 \\
\end{align*}

\end{example}

\section{General Relativity}
\begin{fthm}{Postulates of GR}{}
	\begin{description}
		\item[\textbf{Postulate 1}] The space time is connected four-dimensional lorentzian manifold $(M, g)$.
		\item[\textbf{Postulate 2}] Free particles in the spacetime follow timelike or null geodesics of Levi-Civita connection.
		\item[\textbf{Postulate 3}] The distribution of matter in the spacetime is described by the energy-momentum tensor, $T$, a symmetric $(0,2)$ tensor field which as zero divergence, $\n^u T_{uv} = g^{iu} \n_i T_{uv}= 0$
		\item[\textbf{Postulate 4}] The curvature of the spacetime is related to the energy-momentum tensor via \emph{Einstein Equation}
			$$
			G_{uv} = R_{uv} - \frac{1}{2}R g_{uv} = 8\pi G T_{uv},
			$$
		where $G$ is Newton's gravitational constant.
	\end{description}
	Since $T$ is divergence free, any space time satisfying Einstein equation is an Einstein manifold. 
	(Obvious by taking the trace $\n^i = g^{ij} \n_j$ on both sides)
\end{fthm}

\section{Differential Geometry}

\begin{defi}[Closed and Exact]
	A form $\alpha$ is \emph{closed} if $d \alpha = 0$, and \emph{exact} if $\alpha = d \beta$ for some $\beta$.
\end{defi}

\begin{example}[Jacobian and Pushforward]
	Assuming there is local coordinate $(x^1, x^2, \cdots, x^n) \subset M$ and $F : M \rightarrow N$ defined as $f(x^I) = (f_1(x^I), f_2(x^I), \cdots, f_m(X^I))$.
	The push forward of $f$ acting on $\frac{\p}{\p_{x^1}}$ is 
	$$
	F_* |_p \left.\frac{\p }{\p x^i}\right|_p = \frac{\p f_j}{\p x^i}(p) \left.\frac{\p}{\p y^j}\right|_{F(p)}
	$$

	For example, with $(r, \theta)  \mapsto (r\cos(\theta), r \sin(\theta)) = (x, y)$, 
	$$
		\frac{\p}{\p \theta} = -r \sin(\theta) \frac{\p}{\p x} + r \cos(\theta) \frac{\p}{\p y} = -y \frac{\p}{\p x} + x \frac{\p }{\p y}.
	$$
\end{example}
\begin{thm}
	\begin{enumerate}
		\item $d \alpha(X, Y) = X \alpha (Y) + Y \alpha(X) - \alpha([X, Y])$.
		\item 
			\begin{dmath}
				d \alpha(X_0, X_1, \cdots, X_k) 
				= \sum_{i=0}^k (-1)^i X_i \alpha(X_0, \cdots, \widehat{X_i}, \cdots, X_k) + \sum_{0\leq i < j \leq k} (-1)^{i+j} \alpha([X_i, X_j], X_0, \cdots, \widehat{X_i}, \cdots, \widehat{X_j}, \cdots, X_k)	
			\end{dmath}
		\item $\L_X f = X(f)$
		\item $\L_XY = [X, Y]$
		\item $\L_X \alpha (Y) = \L_X (\alpha Y) - \alpha(\L_{X}Y) = X(\alpha(Y)) - \alpha([X,Y])$
		\item $\L_X g (Y, Z) = \L_X (g(Y, Z)) - g([X, Y], Z) - g(Y, [X, Z])$
		\item $d F^* \alpha = F^* d \alpha$
	\end{enumerate}
\end{thm}

\begin{thm}
	Let $T \in \mathscr{T}^r_s(M)$ and $X \in \mathscr{X}(M)$.
	$\L_X(T) \in \mathscr{T}^r_s(M)$ is defined as, for $Y_i \in \mathscr{X}(M), \alpha_i \in \omega^1(M)$ 
	\begin{dmath}
		(\L_XT)(Y_1, \cdots, Y_s, \alpha_1, \cdots, \alpha_r) = XT(Y_1, \cdots, Y_s, \alpha_1, \cdots, \alpha_r) 
		- \sum T(Y_1, \cdots, [X, Y_i], \cdots, Y_s, \alpha_1, \cdots, \alpha_r)
		- \sum T(Y_1, \cdots, Y_s, \alpha_1, \cdots, \L_X\alpha_i, \cdots, \alpha_r)
	\end{dmath}
\end{thm}

\begin{thm}
	$d \in Der_1(M)$, $\mathscr{L} \in Der_0(M)$, 
	\begin{enumerate}
		\item Lie bracket if $R$ linear but not $C(M)$ linear. $[X, fY] = X(f)Y + f[X, Y]$,
		\item Leibniz Rules: $\L(X \otimes Y) = \L X \otimes Y + X \otimes \L Y$
		\item $\L \circ d = d \circ \L$
		\item $[d,[d, D]] = 0$ for any $D \in Der_k$
		\item $i_X \circ i_X = 0$
		\item $i_{fX} = fi_X $
		\item $\L_X = [d, i_X] \iff \L_X \alpha = i_X d \alpha + di_X \alpha$
		\item $[\L_X, i_Y] = i_{[X, Y]} \iff \L_X i_Y \alpha = i_Y \L_x \alpha + i_{[X, Y]} \alpha$
		\item $[\L_X, \L_Y] = \L_{[X, Y]} \iff  \L_X \L_Y  \alpha= \L_X \L_Y \alpha +  \L_{[X, Y]} \alpha $
	\end{enumerate}
\end{thm}

\begin{example}[Calculation of Lie Derivative]
	To calculate Lie derivative relies on the following four formulae:
	\begin{enumerate}
		\item $\L_X f = X (f)$
		\item $\L_X df = d \L_X f = d(Xf)$
		\item $\L_X Y = [X, Y]$
		\item $\L_X f \alpha \otimes \beta = (\L_X f) \alpha \otimes \beta + f (\L_X \alpha) \otimes \beta + f \alpha \otimes \L_X(\beta)$.
	\end{enumerate}

	When writing in the reduced form, the usual Leibniz rule still follows, for example 
	$$
		\L du^2 = 2du \L du
	$$
	
	\emph{Explicit calculation}: take $g = 2dudv + (\lambda x^2 + \mu y^2)du^2 + dx^2 + dy^2$, and $X = a(u)\p_x + b(u)x\p_v$.

	Note 
	$ \L_X du = d\L_X u  = d(X(u)) = 0$, $ \L_X dv = d\L_X v = d(bx) = b'du + bdx$, $\L_X dx = d \L_X x = da = a'du$, $\L_X dy = 0$.

	Now
	\begin{dmath}
		\L_X g = 
		2\L_X(du)dv + 2 \L_X(dv)du + \L_X(\lambda x^2 + \mu y^2)du^2 + 2\L_X(du) + 2\L_X(dx) + 2\L_X(dy)
		= 2(b'du + dbx)du + 2 \lambda ax du^2 + 2a'dxdu
	\end{dmath}
\end{example}

\section{Trig Identities}

\textbf{Trig}
\begin{align}
	\sin(3x) &= 3 \sin x - 4 \sin^3 x \\
	\cos(3x) &= 4 \cos^3 x - 3 \cos x \\ 
	\tan(3x) &= \frac{3 \tan x - \tan^3 x}{1 - 3 \tan^2 x} \\
\end{align}
\begin{align*}
\tan(x \pm y) &= \frac{\tan x \pm \tan y}{1 \mp \tan x \tan y}\\
\tan^2 x &= \frac{1 - \cos(2x)}{1 + \cos(2x)}
\end{align*}
\begin{align*}
	d \tan(x) &= \sec^2(x) dx\\
	d \cot(x) &= - \csc^2(x) dx\\
	d \sec(x) &= \sec(x)\tan(x) dx\\
	d \csc(x) &= - \csc(x)\cot(x) dx\\
\end{align*}

\textbf{Hyperbolic}
\begin{align*}
\cosh^2 x - \sinh^2 x &= 1 \\
\sinh(x \pm y) &= \sinh x \cosh y \pm \cosh x \sinh y \\
\cosh(x \pm y) &= \cosh x \cosh y \pm \sinh x \sinh y \\
\tanh(x \pm y) &= \frac{\tanh x \pm \tanh y}{1 \pm \tanh x \tanh y}
\end{align*}

\begin{align*}
	d \tanh(x) &= \sech^2(x) dx\\
	d \sech(x) &= \sech(x)\tanh(x) dx\\
\end{align*}


\end{document}

