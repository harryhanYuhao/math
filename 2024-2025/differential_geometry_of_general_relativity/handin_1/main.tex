\documentclass{article}

\usepackage[tmargin=2.5cm,rmargin=3cm,lmargin=3cm,bmargin=3cm]{geometry} 
% Top margin, right margin, left margin, bottom margin, footnote skip
\usepackage[utf8]{inputenc}
\usepackage{biblatex}
\addbibresource{./references.bib}
% linktocpage shall be added to snippets.
\usepackage{hyperref,theoremref}
\hypersetup{
	colorlinks, 
	linkcolor={red!40!black}, 
	citecolor={blue!50!black},
	urlcolor={blue!80!black},
	linktocpage % Link table of content to the page instead of the title
}

\usepackage{blindtext}
\usepackage{mathtools}
\usepackage{breqn}
\usepackage{titlesec}
\usepackage{amsthm}
\usepackage{thmtools}
\usepackage{amsmath}
\usepackage{amssymb}
\usepackage{graphicx}
\usepackage{titlesec}
\usepackage{xcolor}
\usepackage{multicol}
\usepackage{hyperref}
\usepackage{import}


\newtheorem{theorem}{Theorema}[section]
\newtheorem{lemma}[theorem]{Lemma}
\newtheorem{corollary}{Corollarium}[section]
\newtheorem{proposition}{Propositio}[theorem]
\theoremstyle{definition}
\newtheorem{definition}{Definitio}[section]

\theoremstyle{definition}
\newtheorem{axiom}{Axioma}[section]

\theoremstyle{remark}
\newtheorem{remark}{Observatio}[section]
\newtheorem{hypothesis}{Coniectura}[section]
\newtheorem{example}{Exampli Gratia}[section]

% Proof Environments
\newcommand{\thm}[2]{\begin{theorem}[#1]{}#2\end{theorem}}

%TODO mayby proof environment shall have more margin
% \renewenvironment{proof}{\vspace{0.4cm}\noindent\small{\emph{Demonstratio.}}}{\qed\vspace{0.4cm}}
% \renewenvironment{proof}{{\bfseries\emph{Demonstratio.}}}{\qed}
\renewcommand\qedsymbol{Q.E.D.}
% \renewcommand{\chaptername}{Caput}
% \renewcommand{\contentsname}{Index Capitum} % Index Capitum 
\renewcommand{\emph}[1]{\textbf{\textit{#1}}}
\renewcommand{\ker}[1]{\operatorname{Ker}{#1}}

%\DeclareMathOperator{\ker}{Ker}

% New Commands
\newcommand{\bb}[1]{\mathbb{#1}} %TODO add this line to nvim snippets
\newcommand{\orb}[2]{\text{Orb}_{#1}({#2})}
\newcommand{\stab}[2]{\text{Stab}_{#1}({#2})}
\newcommand{\im}[1]{\text{im}{\ #1}}
\newcommand{\se}[2]{\text{send}_{#1}({#2})}
\newcommand{\X}{\mathfrak{X}}
\newcommand{\n}{\nabla}
\newcommand{\G}{\Gamma}
\newcommand{\End}{\text{End}}
\newcommand{\C}{C^{\infty}}
\newcommand{\id}{\text{id}}
\renewcommand{\d}{d^{\nabla}}
\newcommand{\R}{R^{\nabla}}
\renewcommand{\n}{\nabla}

\title{GGR Handin 1}
\author{Harry Han, S2162783} 
\date{\today}

\begin{document}
\maketitle

\section*{Q1}

Let $\nabla, \nabla'$ be Koszul connections on $E$. We want to show that their difference $\kappa = \nabla - \nabla' \in \Omega^1(M; \End(M))$.

\begin{proof}
	Each $\nabla$ has signature $\mathfrak{X}(M) \times \Gamma(E) \rightarrow  \Gamma(E)$, which is equivalent to saying $\X(M) \rightarrow (\G(E) \rightarrow \Gamma(E))$, i.e., $\X(M) \rightarrow \G(\End(M))$.
	$\kappa$ as their difference has the same signature; therefore we only need to show that $\kappa$ is $\C(M)$ bilinear, which can be verified by the calculations below. 

	Let $f \in \C(M), X \in \X(M), s \in \G(E)$, note 
	\begin{align*}
		\kappa(fX, s) 
		&= \nabla_{fX} s - \nabla'_{fX} s \\
		&= f\nabla_{X} s - f\nabla'_{X} s \\ 
		&= f\kappa(X, s)
	\end{align*}
	and 
	\begin{align*}
		\kappa(X, fs) 
		&= \nabla_{X} fs - \nabla'_{X} fs \\
		&= f\nabla_{X} s + X(f)x - f\nabla'_{X} s - X(f)s\\
		&= f\nabla_{X} s - f\nabla'_{X} s \\ 
		&= f\kappa(X, s)
	\end{align*}
\end{proof}

\section*{Q2}

Let $\nabla $ be a Koszul connection on $E$, and $\phi \in \G(\End(E))$, define 

\begin{equation}
	(\nabla_{X}\phi)s =  \nabla_{X}\phi(s) - \phi(\nabla_X s)
\end{equation}
where $X \in \X(M), s \in \G(E)$.

We want to show that it is a Koszul connection on $\End(E)$.

\begin{proof}
	Let us first verify the signature of $\nabla$ defined above is $\X(M) \times \G(\End(E)) \rightarrow \G(\End(E))$.
	Note we have plugged in $X \in \X(M)$ and $\phi \in \G(\End(E))$ into $\nabla$ to get $\nabla_X \phi$, which we expected to be $\G(\End(E))$. 
	$\G(\End(E))$ is equivlent to $\G(E) \rightarrow \G(E)$. 
	So all we need to verify is that 	
	$(\nabla_{X}\phi)s$ is a section of $E$, which of course it is, as $\nabla_X \phi(s)$ by definition is a section of $E$, and, $\nabla_X s$, being a section of $E$, after applying $\phi$, is still a section of $E$. 
	So $(\n_X \phi) s$, as their difference, is also a section of $E$.

	Next we verify the two properties of the Koszul connection

	\begin{align*}
		(\nabla_{fX}\phi) s 
		&= \nabla_{fX} (\phi(s)) - \phi(\nabla_{fX}s)  && \text{by definition} \\ 
		&= f\nabla_{X} (\phi(s)) - \phi(f\nabla_{X}s)  && \nabla \text{ is Koszul connection on }E \\ 
		&= f\nabla_{X} (\phi(s)) - f\phi(\nabla_{X}s)  && \phi \text{ is } \C \text{ linear} \\ 
		&= f(\nabla_{X}\phi) s 
	\end{align*}

	\begin{align*}
		(\nabla_{X}f\phi) s 
		&= \nabla_{X} (f\phi(s)) - f\phi(\nabla_{X}s)  && \text{by definition} \\ 
		&= f\nabla_{X}(\phi(s)) + X(f)\phi(s) - f\phi(\nabla_{X}s)  && \nabla \text{ is Koszul connection on }E \\ 
		&= f(\nabla_{X}\phi) s  + X(f)\phi(s)
	\end{align*}

	As both identities are satisfied, $\nabla$ defined above is indeed a Koszul connection on $\End(E)$.
\end{proof}

For the following question, we need the formulae below

For $s \in \Omega^0(M; E)$, that is, $s\in \G(E)$
\begin{equation}
	\label{d0}
	\d(s)(X) = \nabla_X s 
\end{equation}

For $s \in \Omega^1(M; E)$
\begin{equation}
	\label{d1}
	\d(s)(X, Y) = \nabla_X s(Y) - \nabla_Y s(X) - s([X, Y])
\end{equation}

For $s \in \Omega^2(M; E)$

\begin{dmath}
	\label{d2}
	\d s (X, Y, Z) = \nabla_X s (Y, Z) + \nabla_Y s(Z, X) + \nabla_Z s (X, Y) - s([X, Y], Z) - s([Y, Z], X) - s([Z, X], Y)
\end{dmath}

\section*{Q3}
We want to show that when $\id$ is considered as one form $\Omega^1(M, TM)$, 

$$
d^{\nabla} \id = T^{\nabla}
$$

\begin{proof}
	Let us evaluate $d^{\nabla}\id$, which is a two form, with $X, Y \in \X(M)$

	\begin{align*}
		d^{\nabla} \id (X, Y) 
		&= \nabla_X(\id(Y)) - \nabla_Y(\id(X)) - \id([X, Y])  && \text{ applying \eqref{d1}}\\ 
		&= \nabla_X(Y) - \nabla_Y(X) - [X, Y]  && \id \text{ is identity  }\\ 
		&= T^{\nabla}(X, Y) && \text{ by definition}\\ 
	\end{align*}
\end{proof}

\section*{Q4}

We want to show 

\begin{equation}
	(\d(\d s)(X, Y)) = \R(X, Y)s
\end{equation}

where $s \in \Omega^0(M; E)$ and $X, Y \in X(M)$.

\begin{proof}
	We just need to apply the definition and compute

	\begin{align*}
		(\d(\d s)(X, Y)) 
		&= \nabla_X(\d(s)(Y)) - \nabla_Y(\d(s)X) - \d(s)([X, Y]) && \text{ applying \eqref{d1}} \\ 
		&= \nabla_X \nabla_Y s - \nabla_Y \nabla_X s - \nabla_{[X, Y]}s  && \text{ applying \eqref{d0}} \\ 
		&= \R(X, Y)s && \text{ By definition }
	\end{align*}
\end{proof}

\section*{Q5}

We want to verify the Bianchi Identity 

\begin{equation}
	\d\R = 0
\end{equation}

where $\R$ is the curvature, and $\d: \Omega^2(M; \End(E)) \rightarrow \Omega^3(M; \End(E))$

\begin{proof}
	This involves a bit of calculations. 
	Let us evaluate $\d \R(X, Y, Z)s$ where $X, Y, Z \in \X(M)$ and $s \in \G(E)$.


\begin{dmath}
	\d \R(X, Y, Z)s 
	=
	\n_X \R(Y, Z) s + \n_Y \R(Z, X) s + \n_Z \R(X, Y) s \nonumber 
    -\R([X, Y], Z) s - \R([Y, Z], X) s - \R([Z, X], Y) s 
	=
	\n_X(R(Y,Z)s) - R(Y, Z)\n_Xs + \n_Y(R(Z,X)s) - R(Z, X)\n_Ys + \n_Z(R(X,Y)s) - R(X, Y)\n_Zs   \nonumber 
    -\R([X, Y], Z) s - \R([Y, Z], X) s - \R([Z, X], Y) s 
	=
	\n_X\n_Y\n_Z s - \n_X\n_Z\n_Y s - \n_X \n_{[Y, Z]}s - \n_Y \n_Z \n_X s + \n_Z \n_Y \n_X s + \n_{[Y, Z]}\n_X s 
	+ \n_Y\n_Z\n_X s - \n_Y\n_X\n_Z s - \n_Y \n_{[Z, X]}s - \n_Z \n_X \n_Y s + \n_X \n_Z \n_Y s + \n_{[Z, X]}\n_Y s
	+ \n_Z\n_X\n_Y s - \n_Z\n_Y\n_X s - \n_Z \n_{[X, Y]}s - \n_X \n_Y \n_Z s + \n_Y \n_X \n_Z s + \n_{[X, Y]}\n_Z s 
	- \n_{[X, Y]}\n_Z s + \n_Z\n_{[X, Y]}s + \n_{[[X, Y], Z]}s
	- \n_{[Y, Z]}\n_X s + \n_X\n_{[Y, Z]}s + \n_{[[Y, Z], X]}s
	- \n_{[Z, X]}\n_Y s + \n_Y\n_{[Z, X]}s + \n_{[[Z, X], Y]}s
\end{dmath}

All terms shall cancel, except 

\begin{align}
\n_{[[X, Y], Z]}s + \n_{[[Y, Z], X]}s + \n_{[[Z, X], Y]}s
&= \n_{[[X, Y], Z] + [[Y, Z], X] + [[Z, X], Y]} s &&\\ 
&= \n_{0} s  && \text{ Jacobi identity} \\
&= 0
\end{align} 

So we conclude $\d \R = 0$ as desired.
\end{proof}
\end{document}
