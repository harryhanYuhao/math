\documentclass{article}

\usepackage[tmargin=2.5cm,rmargin=3cm,lmargin=3cm,bmargin=3cm]{geometry} 
% Top margin, right margin, left margin, bottom margin, footnote skip
\usepackage[utf8]{inputenc}
\usepackage{biblatex}
\addbibresource{./references.bib}
% linktocpage shall be added to snippets.
\usepackage{hyperref,theoremref}
\hypersetup{
	colorlinks, 
	linkcolor={red!40!black}, 
	citecolor={blue!50!black},
	urlcolor={blue!80!black},
	linktocpage % Link table of content to the page instead of the title
}

\usepackage{blindtext}
\usepackage{breqn}
\usepackage{titlesec}
\usepackage{amsthm}
\usepackage{thmtools}
\usepackage{amsmath}
\usepackage{amssymb}
\usepackage{graphicx}
\usepackage{titlesec}
\usepackage{xcolor}
\usepackage{multicol}
\usepackage{hyperref}
\usepackage{import}


\newtheorem{theorem}{Theorema}[section]
\newtheorem{lemma}[theorem]{Lemma}
\newtheorem{corollary}{Corollarium}[section]
\newtheorem{proposition}{Propositio}[theorem]
\theoremstyle{definition}
\newtheorem{definition}{Definitio}[section]

\theoremstyle{definition}
\newtheorem{axiom}{Axioma}[section]

\theoremstyle{remark}
\newtheorem{remark}{Observatio}[section]
\newtheorem{hypothesis}{Coniectura}[section]
\newtheorem{example}{Exampli Gratia}[section]

% Proof Environments
\newcommand{\thm}[2]{\begin{theorem}[#1]{}#2\end{theorem}}

%TODO mayby proof environment shall have more margin
\renewenvironment{proof}{\vspace{0.4cm}\noindent\small{\emph{Demonstratio.}}}{\qed\vspace{0.4cm}}
% \renewenvironment{proof}{{\bfseries\emph{Demonstratio.}}}{\qed}
\renewcommand\qedsymbol{Q.E.D.}
% \renewcommand{\chaptername}{Caput}
% \renewcommand{\contentsname}{Index Capitum} % Index Capitum 
\renewcommand{\emph}[1]{\textbf{\textit{#1}}}
\renewcommand{\ker}[1]{\operatorname{Ker}{#1}}

%\DeclareMathOperator{\ker}{Ker}

% New Commands
\newcommand{\bb}[1]{\mathbb{#1}} %TODO add this line to nvim snippets
\newcommand{\orb}[2]{\text{Orb}_{#1}({#2})}
\newcommand{\stab}[2]{\text{Stab}_{#1}({#2})}
\newcommand{\im}[1]{\text{im}{\ #1}}
\newcommand{\se}[2]{\text{send}_{#1}({#2})}

\title{GGR Handin 2}
\author{Harry Han; S2162783} 
\date{\today}

\begin{document}
\maketitle

\section*{Q1}
By setting $w =  \int_{c}^z \exp(-\phi(t))dt $ as a function of $z$	for some constant $c$, taking exterior derivative, and applying the fundamental theorem of calculus give
$$
	dw = \exp(-\phi(z))dz.
$$
Let $\Omega(w) = \exp(\phi(z))$ as an implicit function of $w$, by substitution 
\begin{equation}\label{eq:1}
	g = dz^2 + \exp(2\phi(z)) (-dt^2 + dx^2 + dy^2) = \Omega(w)^2(-dt^2 + dx^2 + dy^2 + dw^2)
\end{equation}

The metric $g$ written in this way is manifestly conformal flat.

\section*{Q2}

\subsection*{(a)}

Let $\theta^0 = e^{\phi}dt, \theta^1 = e^{\phi} dx, \theta^2 = e^{\phi} dy, \theta^3 = dz$. 
The constants $\eta_{00} = - 1, \eta_{11} = \eta_{22} = \eta_{33} = 1$, and all others are zero.
Therefore $\omega_i^0 = -\omega_{0i}$, and $\omega^1_i = \omega_{1i}$, $\omega^2_i = \omega_{2i}$, $\omega^3_i = \omega_{3i}$.
In particular, $\omega^0_{i} = \omega^i_0$, $\omega^1_i = - \omega^i_1$, $\omega^2_i = - \omega^i_2$, $\omega^3_i = - \omega^i_3$.

Lable $\frac{d \phi}{dz} = \phi'$, let us first caclulate $d \theta^i$

\begin{align*}
	d\theta^0 &= e^{\phi} \phi' dz \wedge dt \\
	d\theta^1 &= e^{\phi} \phi' dz \wedge dx \\
	d\theta^2 &= e^{\phi} \phi' dz \wedge dy \\
	d\theta^3 &= 0
\end{align*}

Now apply the first structure equations

\begin{align*}
	d\theta^0 &= e^{\phi} \phi' dz \wedge dt = \omega^0_1 \wedge \theta^1 - \omega^0_2 \wedge \theta^2 - \omega^0_3 \wedge \theta^3 \\
	d\theta^1 &= e^{\phi} \phi' dz \wedge dx = -\omega^1_0 \wedge \theta^0 -  \omega^1_2 \wedge \theta^2 - \omega^1_3 \wedge \theta^3 \\ 
	d\theta^2 &= e^{\phi} \phi' dz \wedge dy = -\omega^2_0 \wedge \theta^0 - \omega^2_1 \wedge \theta^1  - \omega^2_3 \wedge \theta^3 \\ 
	d\theta^3 &= 0 = -\omega^3_0 \wedge \theta^0 - \omega^3_1 \wedge \theta^1 - \omega^3_2 \wedge \theta^2 
\end{align*}
We may guess that $\omega^0_1, \omega^0_2, \omega^1_2$ to be zero, and consequently
\begin{align*}
	\omega_{13} = e^{\phi} \phi' dx\\ 
	\omega_{23} = e^{\phi} \phi' dy\\
	\omega_{03} = - e^{\phi} \phi' dt\\
\end{align*}

\subsection*{(b)}
We want to use the second structure equation $\Omega_{ab} = d \omega_{ab} + \omega^a_c \wedge \omega_{cb}$ to find the curvature two forms. 
First let us differentiate the $\omega$s.

\begin{align*}
	d\omega_{13} = - e^{\phi}((\phi')^2 + \phi'')dx \wedge dz\\
	d\omega_{23} = - e^{\phi}((\phi')^2 + \phi'')dy \wedge dz\\
	d\omega_{03} = e^{\phi}((\phi')^2 + \phi'')dt \wedge dz\\
\end{align*}

Now 
\begin{align*}
	\Omega_{01} &= d\omega_{01} + \omega^{~b}_{0} \wedge w_{b1} = w_{03} \wedge w_{31} = e^{2\phi} (\phi')^2 dt \wedge dx \\
	\Omega_{12} &= d\omega_{12} + \omega^{~b}_1 \wedge w_{b2} = w_{13} \wedge w_{32} = - e^{2\phi} (\phi')^2 dx \wedge dy \\
	\Omega_{13} &= d\omega_{13} + \omega^{~b}_{1} \wedge w_{b3} = d\omega_{13} =  - e^{\phi}((\phi')^2 + \phi'')dx \wedge dz \\
	\Omega_{02} &= d\omega_{02} + \omega^{~b}_{0} \wedge w_{b2} = w_{03} \wedge w_{32} = e^{2\phi} (\phi')^2 dt \wedge dy \\
	\Omega_{23} &= d\omega_{23} + \omega^{~b}_{2} \wedge w_{b3} = d\omega_{23} =  - e^{\phi}((\phi')^2 + \phi'')dy \wedge dz \\
	\Omega_{03} &= d\omega_{03} + \omega^{~b}_{0} \wedge w_{b3} = d\omega_{03} =  e^{\phi}((\phi')^2 + \phi'')dt \wedge dz \\
\end{align*}

\subsection*{(c)}
Note 
\begin{align*}
	\theta^0 \wedge \theta^1 = e^{2\phi} dt \wedge dx \\ 
	\theta^0 \wedge \theta^2 = e^{2\phi} dt \wedge dy \\ 
	\theta^0 \wedge \theta^3 = e^{\phi} dt \wedge dz \\ 
	\theta^1 \wedge \theta^2 = e^{2\phi} dx \wedge dy \\
	\theta^1 \wedge \theta^3 = e^{\phi} dx \wedge dz \\
	\theta^2 \wedge \theta^3 = e^{\phi} dy \wedge dz \\
\end{align*}

As a result, 
\begin{dmath}
\frac{1}{2} \Omega_{ab} \theta^a \theta^b =
e^{4\phi}(\phi')^2 (dt \wedge dx)^2 
- e^{4\phi}(\phi')^2 (dx \wedge dy)^2
+ e^{4\phi}(\phi')^2 (dt \wedge dy)^2 
- e^{2\phi}((\phi')^2 + \phi'') (dx \wedge dz)^2 
- e^{2\phi}((\phi')^2 + \phi'') (dy \wedge dz)^2 
+ e^{2\phi}((\phi')^2 + \phi'') (dt \wedge dz)^2
= -\frac{1}{4} R_{\mu \nu \rho \sigma} (dx^{\mu} \wedge dx^{\nu}) ( dx^{\rho} \wedge dx^{\sigma})
\end{dmath}

We can read off the coefficients 
\begin{align*}
	R_{txtx} &= - e^{4\phi}(\phi')^2 \\ 
	R_{tyty} &= - e^{4\phi}(\phi')^2 \\ 
	R_{tztz} &= - e^{2\phi}((\phi')^2 + \phi'') \\
	R_{xyxy} &= e^{4\phi}(\phi')^2 \\
	R_{xzxz} &= e^{2\phi}((\phi')^2 + \phi'') \\ 
	R_{yzyz} &= e^{2\phi}((\phi')^2 + \phi'') \\ 
\end{align*}

\subsection*{(d)}
Since $g_{\mu \nu}$ is diagonal, $g^{\mu \nu}$ is simply its inverse and equals to $g^{\mu \nu} = \text{diag}(-e^{-2\phi}, e^{-2\phi}, e^{-2\phi}, 1)$.

The formula for Ricci Tensor is 
$$
	R_{\mu \nu} = g^{\rho \sigma} R_{\rho \mu \nu \sigma}
$$

So 
\begin{dmath}
	R_{tt} 
	= g^{xx} R_{xttx} + g^{yy}R_{ytty} + g^{zz} R_{zttz} 
	= e^{2\phi}(\phi')^2 + e^{2\phi}(\phi')^2 + e^{2\phi}((\phi')^2 + \phi'')
	= 2 e^{2\phi}(\phi')^2 + e^{2\phi}((\phi')^2 + \phi'') 
	= 3 e^{2\phi}(\phi')^2 + e^{2\phi} \phi''
\end{dmath}
Upon calculation $-R_{xx} = - R_{yy} = R_{tt}$, and
\begin{dmath}
	R_{zz} 
	= g^{tt}R_{tzzt} g^{xx} R_{xzzx} + g^{yy}R_{yzzy} +
	= - (\phi')^2 + \phi'' - (\phi')^2 - \phi'' - (\phi')^2 - \phi''
	= -3 (\phi')^2 - 3 \phi''
\end{dmath}
Apply Einstein's equation $R_{\mu \nu} = - \Lambda g_{\mu \nu}$
\begin{align*}
	\Lambda g_{tt} &=  e^{2\phi}\Lambda = 3 e^{2\phi}(\phi')^2 + e^{2\phi} \phi'' \\
	\Lambda g_{zz} &= - \Lambda = -3 (\phi')^2 - 3 \phi''
\end{align*}

Upon solving the equations, we find $\frac{1}{3}\Lambda =  (\phi')^2$ as desired.

\subsection*{(e)}

If $\Lambda = 0$, then $\phi' = 0$ so $\phi(z) = C$ for some constant $C$.
Apply equation \eqref{eq:1}, the metric becomes 
$$
g = D (-dt^2 + dx^2 + dy^2 + dw^2)
$$
where $D = e^{2C}$ is a constant. 
This shows that the metric is isometric to Minkowski spacetime.

\subsection*{(f)}

Let $\Lambda > 0$, recall $g_{\mu \nu} = 0$ if $\mu \neq \nu$. 
In this case, moreover, $g_{\mu \nu} = g_{\nu \mu}$.
Since $(\phi')^2 = \frac{1}{3} \Lambda$ is a constants, $\phi'$ must also be a constant, meaning $\phi'' = 0$.

In such cases
\begin{align*}
	R_{txtx} &= - e^{4\phi}(\phi')^2 = (\phi')^2 (-e^{2\phi} e^{2\phi}) = -(\phi')^2 (g_{xt}g_{tx} - g_{xx} g_{tt} ) \\
	R_{tyty} &= - e^{4\phi}(\phi')^2 = (\phi')^2 (-e^{2\phi} e^{2\phi}) = -(\phi')^2 (g_{ty}g_{yt}- g_{yy} g_{tt} ) \\
	R_{tztz} &= - e^{2\phi}(\phi')^2 = (\phi')^2 (-e^{2\phi}) = -(\phi')^2 (g_{tz}g_{zt}- g_{tt} g_{zz} ) \\
	R_{xyxy} &= - e^{4\phi}(\phi')^2 = (\phi')^2 (e^{2\phi} e^{2\phi}) = -(\phi')^2 (g_{xy}g_{yx}- g_{yy} g_{xx} ) \\
	R_{xzxz} &= e^{2\phi}(\phi')^2 = (\phi')^2 (-e^{2\phi}) = -(\phi')^2 (g_{xz}g_{zx}- g_{xx} g_{zz} ) \\
	R_{yzyz} &= e^{2\phi}(\phi')^2 = (\phi')^2 (-e^{2\phi}) = -(\phi')^2 (g_{yz}g_{zy}- g_{yy} g_{zz} ) \\
\end{align*}
Define 
$$
	T_{\mu \nu \rho \sigma} = g_{\nu \rho} g_{\mu \sigma} - g_{\mu \rho} g_{\nu \sigma}
$$
If three of the indeces are the same  
$$
T_{\nu \nu \nu \sigma} = g_{\nu \nu} g_{\nu \sigma} - g_{\nu \nu} g_{\nu \sigma} = 0
$$
all other combinations are up to a permutation, so the term $T$ must be zero.
If three of the indeces are distinct, one of the $g_{\nu \rho}$ and $g_{\mu \sigma}$ must have distinct indeces, and one of the $g_{\mu\rho}$ and $g_{\nu \sigma}$ must also have distinct indeces. 
Recall $g_{\mu \nu} = 0$ if $\mu \neq \nu$, so the term $T$ must be zero too.

So we conclude 
$$
R_{\mu \nu \rho \sigma} = -(\phi')^2 T_{\mu \nu \rho \sigma} = -(\phi')^2 (g_{\nu \rho} g_{\mu \sigma} - g_{\mu \rho} g_{\nu \sigma})
$$
Substituting $R = -4 \Lambda = -12 (\phi')^2$ gives the desried result.

\section*{Q3}

Put $\Lambda = 3$ and $\phi(z) = z$, the metric becomes 
$$
g = dz^2 + e^{2z}(-dt^2 + dx^2 + dy^2)
$$
We want to show this metric is not geodesically complete.

Consider a null geodesic $\gamma(s)$ in $t,z$ plane where $x,y$ are constant and $s$ is the affine parameter.
Write $\dot{\gamma(s)} = \dot{z} \frac{\partial}{\partial z} + \dot{t} \frac{\partial}{\partial t}$, the geodesic equation becomes 
$$
g(\dot{\gamma}, \dot{\gamma}) = 0 \implies \dot{z}^2 - e^{2z} \dot{t}^2 = 0
$$
Since $\frac{\partial }{\partial t}$ is a Killing vector, we have 
$$
g(\dot{\gamma}, \frac{\partial}{\partial t}) = E \implies -e^{2z} \dot{t} =E
$$
Substitute this back to the geodesic equation gives
\begin{align*}
	(\dot{z}^2) + E \dot{t} = 0 
	& \implies \dot{t} = C \dot{z}^2
\end{align*}
where $C$ is the constant $-\frac{1}{E}$.
Substituting back to geodesic equation again gives 
$$
\dot{z}^2 - e^{2z} C^2 \dot{z}^4 = 0 
$$
We may assume $\dot{z} \neq 0$, so it becomes
$$
1 = e^{2z} C^2 \dot{z}^2 \implies \pm 1 = C \dot{z} e^{z}
$$
Choosing arbitrarily the positive sign and integrate both sides respect to $s$ gives 
$$
	s + D = \int C e^{z} dz = C e^{z}
$$
We may choose appropriate coordinate so that the constant $C = 1$. 
Choose the solution with $D = 0$ and solve for $z$ we get $z = \log(s)$, which diverges to infinity when $s$ approach $0$ from the positive.
As there exists a null geodesic that diverges to infinity with respect to finite affine parameter, we conclude that this metric is not geodesically complete.
\end{document}
