\documentclass[12pt, a4paper]{article}
\usepackage{blindtext, titlesec, amsthm, thmtools, amsmath, amsfonts, scalerel, amssymb, graphicx, titlesec, xcolor, multicol, bm}
\usepackage[utf8]{inputenc}
% \hypersetup{colorlinks,linkcolor={red!40!black},citecolor={blue!50!black},urlcolor={blue!80!black}}
\usepackage[margin=0.25in]{geometry}
\newtheorem{theorem}{Theorema}[section]
\newtheorem{lemma}[theorem]{Lemma}
\newtheorem{corollary}[theorem]{Corollarium}
\newtheorem{hypothesis}{Coniectura}
\theoremstyle{definition}
\newtheorem{definition}{Definitio}[section]
\theoremstyle{remark}
\newtheorem{remark}{Observatio}[section]
\newtheorem{example}{Exampli Gratia}[section]
\newcommand{\bb}[1]{\mathbb{#1}}
\renewcommand\qedsymbol{Q.E.D.}
\renewcommand{\emph}[1]{\textbf{\textit{#1}}}
\newcommand{\gl}{\mathfrak{gl}}
\renewcommand{\sl}{\mathfrak{sl}}
\newcommand{\so}{\mathfrak{so}}
\renewcommand{\sp}{\mathfrak{sp}}
\newcommand{\GL}{\text{GL}}
\newcommand{\rad}{\text{rad}}
\newcommand{\mf}[1]{\mathfrak{#1}}
\begin{document}
\emph{\huge{Lie Group}}
%\tableofcontents
\section{Some Facts}
\begin{table}[h]
	\centering
	\begin{tabular}{|c|c|c|c|c|c|c|}
		\hline
		$G$                                & Lie Algebra      & dim               &cpt& $\pi_0$    & $\pi_1$ & $Z(G)$\\
		\hline                                                                    
		$GL(1, \bb{R})$                    & $\gl(1, \bb{R})$ & $1$               & N & $\bb{Z}_2$ & $\{e\}$        & $R^{\times}$ \\
		$GL(2, \bb{R})$                    & $\gl(2, \bb{R})$ & $4$               & N & $\bb{Z}_2$ & $\bb{Z}$       & $R^{\times}$ \\
		$GL(n, \bb{R}), n \geq 3$          & $\gl(n, \bb{R})$ & $n^2$             & N & $\bb{Z}_2$ & $\bb{Z}_2$     & $R^{\times}$ \\
		$GL(n, \bb{C})$                    & $\gl(n, \bb{C})$ & $n^2$ (C)         & N & $\{e\}$    & $\bb{Z}$       & $C^{\times}$ \\
		$SL(2, \bb{R})$                    & $\sl(2, \bb{R})$ & $3$               & N & $\{e\}$    & $\bb{Z}$,      & $\bb{Z}_2$ \\
		$SL(2n + 1, \bb{R}), n \geq 3$     & -                & -                 & N & $\{e\}$    & $\bb{Z}_2 $,   & $\{e\}$ \\
		$SL(2n,\bb{R}), n \geq 3$          & -                & -                 & N & $\{e\}$    & $\{e\}$        & $\bb{Z}_2$ \\
		$SL(n, \bb{C})$                    & $\sl(n, \bb{C})$ & $n^2-1$ (C)       & N & $\{e\}$    & $\bb{Z}$       & $Z_n$ \\
		$O(n, \bb{R})$                     &  $\mf{so}(n, R)$ &$\frac{n(n-1)}{2}$ & Y & $\bb{Z}_2$ & $\bb{Z}_2$     & $\bb{Z}_2$  \\
		$O(n, \bb{C})$                     & $\mf{so}(n, C)$  &$\frac{n(n-1)}{2}$ & N & $\bb{Z}_2$ & $\bb{Z}_2$     & $\bb{Z}_2$  \\
		$SO(n, \bb{R}) n > 1$              &  -               &$\frac{n(n-1)}{2}$ & Y & - & -& - \\
		$SO(2, \bb{R})$                    & $\mf{so}(2, R)$  &1                  & Y & $\{e\}$    & $\bb{Z}$       & self \\
		$SO(2n+1, \bb{R}) n > 1$           &  -               &                -  & Y & $\{e\}$    & $\bb{Z}_2$     & $\{e\}$\\
		$SO(2n, \bb{R}) n > 2$             &                - &                -  & Y & $\{e\}$    & $\bb{Z}_2$     & $\{e\}$\\
		$SU(n) $                           &                - &$n^2-1$(R)           & Y & $\{e\}$    &$\{e\}$         & $Z_n$\\
		$U(n) $                            &                - &$n^2$(R)             & Y & $\{e\}$    & $\bb{Z}$       & $U(1)$\\
		\hline
	\end{tabular}
\end{table}

\begin{enumerate}
	\item $\mf{su}(2, C)$ is a real Lie Algebra different from $\sl(2, R)$. Its complexification is $\mf{sl}(2, C)$.
	\item $O(n, R)/O(n-1, R) \cong S^{n-1}$
	\item $SO(n, R)/SO(n-1, R) \cong S^{n-1}$
	\item $U(n)/U(n-1) \cong S^{2n-1}$
	\item $SU(n)/SU(n-1) \cong S^{2n-1}$
	\item $USp(n)/U(n-1) \cong S^{4n-1}$
	\item $Spin(3) \cong SU(2) \cong USp(1)$
	\item $Spin(4) \cong SU(2) \times SU(2)$
	\item $Spin(5) \cong USp(2)$
	\item $Spin(6) \cong SU(4)$
	\item $\mf{so}(3) \times \mf{so}(3) \cong \mf{so}(4)$
\end{enumerate}

\begin{theorem}[Differentiating Matrices]
	\ 
	\begin{enumerate} 
		\item $d(A(t)^T) = A'(t)^T$
		\item $d(A(t)^*) = A'(t)^*$
		\item $d(A(t)^{-1}) = -A(t)^{-1}A'(t)A(t)^{-1}$
		\item $d(\det(A(t))) = \det(A(t))\text{tr}(A(t)^{-1}A'(t))$
	
	\end{enumerate}
\end{theorem}

\begin{theorem}[$\bm{C}$ and $\bm{H}$]
	$GL(n,C)$ is a subgroup of $GL(2n, R)$ of invertible matrices $A \in GL(2n, R)$ s.t. $AJ_1 = J_1A$. 
	$GL(n, H)$ is a subgroup of $GL(2n, C)$ of invertible matrices $B \in GL(2n, C)$ s.t. $BJ_2 = J_2\overline{B}$.
	$
		J_1 = \begin{bmatrix}
			0 & I_n \\
			-I_n & 0
		\end{bmatrix}
		,
		J_2 = \begin{bmatrix}
			0 & -I_n \\
			I_n & 0
		\end{bmatrix}.
		$
\end{theorem}

\begin{enumerate}
	\item The universal cover of $SL(2,R)$ is not a matrix Lie group.
\end{enumerate}


\begin{theorem}[Adol's]
Over a field $k$ of characteristic 0. Any finite dimensional Lie Algebra is a sub-algebra of $gl(V)$, where $V$ is a vector space of $k$.  
\end{theorem}

\begin{theorem}[Lie's Third Theorem]
Every finite dimensional real Lie Algebra is isomorphic to a Lie algebra of a matrix Lie Group.
\end{theorem}
Reverse is \emph{Not} true. Not every Lie Group is a subgroup of matrix Lie Group.

\section{Representation}

\begin{definition}[Representation]
	\ 
	\begin{enumerate}	
		\item A \emph{representation}, $(V, \rho)$, of a Lie Group $G$ is Lie Group morphism, $\rho$, to $GL(V)$, where $V$ is a vector space. $\rho(g)v $ is denoted as $gv$.
		\item A representation of \emph{Lie Algebra} $\mathfrak{g}.$ is a Lie Algebra morphism, $\rho$, to $\mathfrak{gl}(V)$.
		\item \emph{Subrepresentation} of $(V, \rho)$ is a subspace, $W \subset V$, such that $\forall g \in G, w \in W, gw \in W$.
		\item A representation $(V, \rho)$ is \emph{irreducible} if the only subrepresentations are $0$ and $V$.
		\item \emph{Direct sum} of Lie Group and Lie algebra representation, $(V_1, \rho_1)$, $(V_2, \rho_2)$, denoted as $V_1 \oplus V_2$ is a representation, with $g(v_1, v_2) = (gv_1, gv_2)$
		\item A representation $(V, \rho)$ is \emph{completely reducible} if it is a direct sum of irreducible representations.
		\item \emph{Tensor product} of Lie Group representation is a representation. $g(v_1 \otimes v_2)=gv_1 \otimes gv_2 $
		\item Tensor product of \emph{Lie Algebra} representation is a representation. $x(v_1 \otimes v_2) = xv_1\otimes v_2 + v_1\otimes xv_2$
		\item For representation $(V_1, \rho), (V_2, \rho)$, an interwiner is a linear map $f: V_1 \rightarrow V_2$, such that $g(f(v_1)) = f(gv_1)$ for all $g \in G$.
		\item Two representation $V_1, V_2$ are isomorphic if there exists an invertible linear map $f$ such that $f(gv_1) = gv_2$ for all $g \in G$.
	\end{enumerate}
\end{definition}

\begin{theorem}
	Finite dimensional representation of compact Lie Group is completely reducible.
\end{theorem}

\begin{theorem}[Schur's Lemma]
	Interwiner between irreducible representations is either trivial or an isomorphism.
	
	Let $V$ be an irreducible representation. 
	If the ground field is $\bb{C}$, $End_G(V) \cong \bb{C}$.
	If the ground field is $\bb{R}$, $End_G(V) \cong \bb{R}, \bb{C}$, or $\bb{H}$.

	Irreducible representation of abelian Lie Algebra or Lie Group are $1$-dimensional.
\end{theorem}

\subsection{Representations of $\sl(2, \bb{C})$}

Recall $\sl(2, C)$ has basis $\{h, e, f\}$, where 
$h = \begin{bmatrix}
	1 & 0 \\
	0 & -1
	\end{bmatrix}, 
e = \begin{bmatrix}
	0 & 1 \\ 
	0 & 0 
	\end{bmatrix}, 
f = \begin{bmatrix} 
	0 & 0 \\ 
	1 & 0 
\end{bmatrix}$.

Notice $[e,f] = h, [h,e] = 2e, [h,f]=-2f$.

\begin{theorem}
	Every finite dimensional representation of $\sl(2, \bb{C})$ is completely reducible.
\end{theorem} 

\begin{definition}
	Let $V$ be a representation of $\sl(2, \bb{C})$. 
	\begin{enumerate}
		\item A \emph{weight} of $V$ is an eigenvalue of $h$, i.e. $hv = \lambda v$. The space of vector of weight $\lambda$ is denoted as $V[\lambda]$.
		\item Character of $V$ is $\chi_V(q) = \sum_{\lambda} \dim V[\lambda]q^{\lambda}$.
	\end{enumerate}
\end{definition}

\begin{theorem}
	\ 
	\begin{enumerate}
		\item $e: V[\lambda] \rightarrow  V[\lambda + 2], f: V[\lambda] \rightarrow  V[\lambda - 2]$.
			All weights of $V$ are integers.
		\item Representations of same character are isomorphic.
		\item All irreducible representation of $\sl(2, C)$ are $L(n)$, whose dimension is $n+1$, with weights $-n, -n+2, \cdots, n-2, n$. 
		\item $L(m) \otimes L(n) \cong L(m+n) \oplus L(m+n-2) \oplus \cdots \oplus L(m-n)$.
		\item Irreducible representation of $SO(3)$ are $L(n)$ with even $n$. Irreducible representation of $Spin(3)$ are $L(n)$.
		\item $\chi_V(L(n-1)) = \frac{q^n - q^{-n}}{q-q^{-1}}$.
	\end{enumerate}
\end{theorem}

\section{Structure Theory of Lie Algebra}

\begin{definition}[Derived Series]
	Let $\mathfrak{g}$ be a Lie Algebra. The \emph{derived series} of $\mathfrak{g}$ is defined as 
	$D^0\mathfrak{g} = \mathfrak{g}, D^1\mathfrak{g} = [\mathfrak{g}, \mathfrak{g}], D^{n+1}\mathfrak{g} = [D^n\mathfrak{g}, D^n\mathfrak{g}].$ 
\end{definition}

\begin{remark}
	$[\mf{g}, \mf{g}]$ is an ideal. $\mf{g}/ [\mf{g}, \mf{g}]$ is abelian.
	The derived series must terminate, either at $0$ or at some ideal.
\end{remark}

\begin{definition}[Solvable]
	A Lie algebra is solvable if its derived series terminates with $0$.
\end{definition}

\begin{example}

	The following Lie algebra are solvable.

	$\mf{b} \subset \gl(n, R)$ be subalgebra of upper triangular matrices. 
	All abelian Lie algebra.

	Semi-simple Lie algebra are \emph{not} solvable. (Unless it is trivial Lie Algebra)
\end{example}

\begin{definition}[Radical]
	$\rad(\mf{g})$ is the maximal solvable ideal of $\mf{g}$.
\end{definition}

\begin{definition}[Semisimple]
	A Lie algebra is semisimple if its radical is $0$.
\end{definition}

\begin{definition}[Simple]
	A Lie algebra is simple if it is not abelian and only has $0$ and itself as ideal.
\end{definition}

\begin{theorem}[Simple Lie Algebras]
	All finite dimensional simple Lie Algebras are isomorphic to one of the following:
		 $\sl(n, k)$, $\so(n, k)$, $\sp(n, k)$, $G_2$, $F_4$, $E_6$, $E_7$, or $E_8$.
	Each of them has an associated Dynkin diagram.

\end{theorem}

\begin{theorem}
	A semisimple Lie algebra is a direct sum of simple Lie algebras.
	$$\mf{g} \cong \mf{g}_1 \oplus \cdots \oplus \mf{g}_n$$
	where $\mf{g}_i$ are simple Lie algebras.
	Moreover, $[\mf{g}, \mf{g}] = \mf{g}$ and all ideals are direct sum of $\mf{g}_i$.
\end{theorem}

\begin{theorem}
	A Lie algebra is simple iff the adjoint representation is irreducible.
\end{theorem}

\begin{definition}[Semidirect Product]
	Let $\mf{g}, \mf{h}$ be Lie algebras with morphism $f: \mf{g} \rightarrow Der)\mf{h}$. 
	Their \emph{semidirect product} is the vector space $\mf{g} \oplus \mf{h}$ with the bracket $[(x, y), (x', y')] = ([x, x'], f(x)y' - f(x')y + [y, y'])$.
	It is labeled as $\mf{g} \ltimes_f \mf{h}$.
\end{definition}

\begin{theorem}[Levi]
	$\mf{g}$ be a Lie algebra over a field of characteristic 0. $\mf{g} \cong \mf{g}/\rad(\mf{g}) \ltimes \rad(\mf{g})$
\end{theorem}

\begin{definition}[Reducitive]
	A Lie algebra is reductive if $\rad(\mf{g}) = \mf{z}(\mf{g})$.
\end{definition}

\begin{example}
	$\gl(n, C)$ is reductive with one dimensional center.
\end{example}

\begin{theorem}
	A reductive Lie algebra $\mf{g} \cong \mf{g}_{ss} \oplus \mf{z}(g)$, where $\mf{g}_{ss}$ is semisimple.
\end{theorem}

\begin{theorem}
	A Lie algebra is reductive iff its adjoint representation is completely reducible.
\end{theorem}

\subsection{Killing Form}
\begin{definition}[Killing Form]
	Let $\mf{g}$ be a Lie algebran and $V$ its representation.
	The killing form $B_V: \mf{g} \times \mf{g} \rightarrow  k = tr(\rho(x)\rho(y))$ 
	Define $K_{\mf{h}}(x,y) = tr_{\mf{h}}(ad_x, ad_y)$.
\end{definition}

\begin{theorem}
	$B_v([x,y],z) + B_v(y, [x,z]) = 0$
\end{theorem}

\begin{theorem}
	$K$ non-degenerate $\iff$ $\mf{g}$ semisimple.
	$K([\mf{g}, \mf{g}], \mf{g}) =0 \iff \mf{g}$ reductive. 
\end{theorem}

\begin{theorem}
	Let $G$ be a compact Lie group. Its Lie algebra is reductive whose Killing form is negative semi-definite.

	Let $\mf{g}$ be a real semisimple Lie algebra with negative-definite Killing form. There is a unique compact connected and simply connected Lie group $G$ s.t. $Lie(G) \cong \mf{g}$.
\end{theorem}

\begin{theorem}
	Let $\mf{g}$ be a complex semisimple Lie algebra. There is a unique real semisimple Lie algebra, $\mf{g}_R$ with negative-definite Killing form such that $\mf{g}_R \otimes C \cong \mf{g}$.
\end{theorem}

\begin{theorem}
	Let $\mf{g}$ be a complex semisimple Lie algebra. Then every $\mf{g}$ representation is completely reducible.
\end{theorem}

\section{Root System}

\begin{definition}[Diagonalisable]
	A linear map $V \rightarrow V$ is diagonlisable if it has a basis of eigenvectors on $V$.

	For $x \in \mf{g}$. It is ad-diagonalisable if $ad_x$ is diagonalisable.
\end{definition}

\begin{definition}
	Toral subalgebra is a subalgebra of $\mf{g}$ whose elements are ad-diagonalisable.
	Cartan subalgebra is a maximal toral subalgebra.
\end{definition}

\begin{remark}
	All toral subalgebra are abelian.
\end{remark}

\begin{definition}
	A $\mf{g}$ invariant bilinear form is a form $B: \mf{g} \times \mf{g} \rightarrow k$ such that $B([x,y], z) + B(y, [x,z]) = 0$.
\end{definition}

\begin{theorem}
	Let $\mf{g}$ be a semisimple Lie algebra with a non-degenerate symmetric invariant bilinear form $(-,-)$.
	$\mf{h}\subset \mf{g}$ its Cartan subalgebra. 
	There is a finite set of vectors $\alpha_i \in \mf{h}^*$, called roots, denoted as $\Delta$, such that

\begin{enumerate}
	\item $\mf{g} = \mf{h} \oplus \bigoplus \mf{g}_{\alpha}$, where $\mf{g}_{\alpha} $ is eigensapce of $\mf{h}$, which is also one dimensional. That is $[h, x_{\alpha}] = \alpha(h)x_{\alpha}$.
	\item $[\mf{g}_{\alpha}, \mf{g}_{\beta}] = \mf{g}_{\alpha + \beta}$
	\item If $\alpha \neq -\beta$, they are orthogonal respect to the bilinear form.
	\item $\Delta$ spans $\mf{h}^*$.
	\item For any two roots $\alpha, \beta$, $\frac{2(\alpha, \beta)}{(\alpha, \alpha)}$ is an integer.
	\item Define reflection $s_{a}(l) = l - \frac{2(a,l)}{(a,a)}a$. $s_a(b)$ is a root. In particular, $s_a(a) = -a$ is root.
	\item The only multiples of $a$ that is also a root is $\pm a$.
	\item Define $H_a$ as $(H_a, h) =a(h)$
\end{enumerate}
\end{theorem}

\begin{definition}[Abstract Root System]
	Let $E$ be a real vector space with an inner product. 
	A root system is a finite set of vectors $\Delta \subset E$ such that 
	1) $\Delta$ spans $E$.
	2) For any $\alpha \in \Delta$, $n_{\alpha \beta} = \frac{2(\alpha, \beta)}{(\alpha, \alpha)}$ is an integer.
	3) For any $\alpha \in \Delta$, $s_{\alpha}(\beta) = \beta - \frac{2(\alpha, \beta)}{(\alpha, \alpha)}\alpha \in \Delta$.

	A root system is reduced if for any $\alpha$ the only roots parallel to $\alpha$ in $\Delta$ is $\pm \alpha$.

	Weyl group is the group generated by reflections $s_{\alpha}$.

	Coroot $\alpha ^{\wedge} \in E^{*}$ is defined as $\alpha^{\wedge}(x) = \frac{2(\alpha, x)}{(\alpha, \alpha)}$.
	$n_{\alpha,\beta}=\beta^{\wedge}(\alpha)$.
	$s_{\alpha}(\lambda) = \lambda - a^{\wedge}(\lambda)\alpha$.

	An \emph{Isomorphism} of root system $\Delta_{1} \in E_1, \Delta_2 \in E_2$ is a linear map $f: E_1 \rightarrow E_2$ such that $f(\Delta_1) = \Delta_2$ and $n_{f(\alpha), f(\beta)} = n_{\alpha, \beta}$.
\end{definition}

\begin{lemma}
Weyl group is a finite subgroup of $O(E)$ that preserves the root system. $s_{w{\alpha}} = ws_{\alpha}w^{-1}$.
\end{lemma}

\begin{theorem}
	There are only 4 rank 2 root systems, $A_1 \times A_1 \cong D_2, A_2, B_2, G_2$.
\end{theorem}

\begin{definition}
	Pick a vector $v \in E$. Positive roots are those forms a positive inner product with $v$. Simple roots are those that cannot be written as sum of other positive roots.
\end{definition}

\begin{definition}[Fundamental Weight]
	For $\alpha_i$ being simple roots, the fundamental weight $\varpi_i$ are the unique vector satisfying $\frac{2(\varpi_i, \alpha_j)}{\alpha_j, \alpha_j} = \delta^i_j$
\end{definition}

\begin{theorem}
	$Q$ is the root lattice, generted by $\alpha_i$. $P$ is the weight lattice, generated by $\varpi_i$.
	$Q\subset P$.
\end{theorem}

\begin{theorem}
	Let $\mf{g}$ be a complex semisimple Lie algebra with root system $\Delta$.
Let $G$ be the unique connected and simply connected Lie group such that $Lie(G) \otimes_R C \cong \mf{g}$.
Then $Z(G) = P/Q$.

Thus, by property of the universal cover, there is one to one correspondence between compact connected Lie group $G$, s.t. $Lie(G) \otimes_R C \cong \mf{g}$ and subgroups of $P/Q$.
\end{theorem}

\subsection{Dynkin Diagrams}

\begin{definition}[Cartan Matrix]
	Let $\Delta$ be a root system with simple roots $\alpha_i$. 
	The Cartan matrix is defined as $A_{ij} = \frac{2(\alpha_i, \alpha_j)}{(\alpha_i, \alpha_i)}$.
\end{definition}

\begin{lemma}
	The diagonal entries of Cartan matrix are all $2$. Off-diagonal entries are one of $0, -1, -2, -3$.
\end{lemma}

\begin{definition}[Coxeter Matrix]
	Let $\Delta$ be reduced root system of rank $n$ and $A$ be its Cartan matrix.	
	The Coxeter matrix is the symmetric matrix $m$, s.t.
	1) $m_{ii} = 1$;
	2) If $A_{ij} = 0, m_{ij} =2$;
	3) If $A_{ij}A_{ji} = 1, m_{ij} = 3 $;
	4) If $A_{ij}A_{ji} = 2, m_{ij} = 4 $;
	5) If $A_{ij}A_{ji} = 3, m_{ij} = 6 $;
\end{definition}

\begin{lemma}
	The Weyl group is generated by $s_1, \cdots, s_r, (s_is_j)^{m_{ij}} = 1$
\end{lemma}

\begin{definition}[Dynkin Diagram]
	Let $\Delta$ be a reduced root system of rank $n$ with simple roots $\alpha_i$ and Cartan matrix $a$.
	1) For each simple root, draw a node.
	2) For each pair of simple roots, connect them by $a_{ji}a_{ij}$ edges.
	3) If $|a_i| \neq |a_j|$, and they are not orthogonal, draw an arrow from longer root to the shorter.
\end{definition}[Dynkin Diagram]

\section{Miscellaneous}

\begin{theorem}[Jordan Normal Form]
	Let $k = \bb{R}$ or $\bb{C}$. For all matrix $A \in \gl(n, k)$, there exists $B \in \GL(n,k)$, such that $M = B^{-1}AB$ is in the Jordan Normal Form, which is diagonal with blocks like
	\[
		\begin{bmatrix}
			C & 1 & 0 & \cdots & 0 \\
			0 & C & 1 & \cdots & 0 \\
			\vdots & \vdots & \vdots & \ddots & \vdots \\
			0 & 0 & 0 & \cdots & C 
		\end{bmatrix},
		C_{\bb{R}} = \begin{bmatrix}
		a & b \\
		-b & a
	\end{bmatrix},
		 C_{\bb{C}} \in \bb{C}
	\]
\end{theorem}

\end{document}

