\documentclass[twocolumn,10pt]{article}
\usepackage[utf8]{inputenc}
\usepackage{amsmath}
\usepackage{amssymb}
\usepackage{amsthm}
\usepackage{xcolor}
\usepackage{marginnote}
\usepackage{graphicx}
\usepackage[margin=0.25in]{geometry}
\usepackage{setspace}
\setstretch{0.1}

\newcommand{\ideal}{\mathrel{\ooalign{$\leq$\cr\raise.22ex\hbox{$\lhd$}\cr}}}
\newcommand{\dom}{\text{dom}}
\newcommand{\wt}{\text{wt}}

\begin{document}

\section*{Metric Spaces Exam Notes}

\section{Definitions and examples}

\textbf{D1:} A \textbf{metric space} is a non-empty set equipped with a metric. Let $X$ be a non-empty set. A function $d:X\times X\to\mathbb{R}$ is called a \textbf{metric} $\Leftrightarrow$ $\forall x,y,z\in X$,
\begin{enumerate}
    \item $d(x,y)\geq0$ and $d(x,y)=0\Leftrightarrow x=y$;
    \item $d(x,y)=d(y,x)$;
    \item $d(x,y)\leq d(x,z)+d(z,y)$ (triangle inequality).
\end{enumerate}
\textbf{D6:} Let $X$ be a real vector space. An \textbf{inner product} on $X$ is a function that assigns to every pair $(x,y)$ in $X\times X$ a real number denoted by $\langle x,y\rangle$ and has the following properties:
\begin{enumerate}
    \item $\langle x,x\rangle\geq0$ and $\langle x,x\rangle=0\Leftrightarrow x=0$;
    \item $\langle x,y\rangle=\langle y,x\rangle$;
    \item $\langle ax+by,z\rangle=a\langle x,z\rangle + b\langle y,z\rangle$.
\end{enumerate}
A \textbf{real inner product space} is a real vector space equipped with an inner product. If $\langle-,-\rangle$ is an inner product on $X$, then $||x||=\sqrt{\langle x,x\rangle}$ defines a norm and $d(x,y)=||x-y||$ defines a metric.
\begin{equation*}
    \langle x,y\rangle = x_1y_1 + ... + x_ny_n.
\end{equation*}
Let $V$ be a vector space over $\mathbb{R}$. A \textbf{norm} on $V$ is a function which has the properties (for $a\in\mathbb{R}$):
\begin{enumerate}
    \item $||x||\geq0$ and $||x||=0\Leftrightarrow x=0$;
    \item $||ax|| = |a|||x||$;
    \item $||x+y||\leq ||x||+||y||$ (triangle inequality).
\end{enumerate}

\subsection*{Spaces of real sequences}

\textbf{D3:} We denote by $\ell^1$ the set of all real sequences $(x_n)_{n\in\mathbb{N}}$ for which the series $\sum_{n=1}^{\infty}|x_n|$ converges. For $x=(x_1,...,x_n,...)\in\ell^1$ and $y=(y_1,...,y_n,...)\in\ell^1$ we define
\begin{equation*}
    \textbf{Norm:} \hspace{0.5cm} ||x||_1 = \sum_{n=1}^{\infty}|x_n|, \hspace{0.5cm} \textbf{Metric:} \hspace{0.5cm} d_1(x,y) = \sum_{n=1}^{\infty}|x_n-y_n|.
\end{equation*}
We denote by $\ell^2$ the set of all real sequences $(x_n)_{n\in\mathbb{N}}$ for which the series $\sum_{n=1}^{\infty}|x_n|^2$ converges. For $x=(x_1,...,x_n,...)\in\ell^2$ and $y=(y_1,...,y_n,...)\in\ell^2$ we define
\begin{equation*}
    \textbf{N: }||x||_2 = \left(\sum_{n=1}^{\infty}|x_n|^2\right)^{1/2}, \hspace{0.5cm} \textbf{M: } d_2(x,y) = \left(\sum_{n=1}^{\infty}|x_n-y_n|^2\right)^{1/2}.
\end{equation*}

\textbf{T4:} $\ell^2$ is a real vector space.

We denote by $\ell^{\infty}$ the set of all bounded sequences or real numbers. For $x=(x_1,...,x_n,...)\in\ell^{\infty}$ and $y=(y_1,...,y_n,...)\in\ell^{\infty}$ we define
\begin{align*}
    &\textbf{Norm:} \hspace{0.5cm} ||x||_{\infty} = \sup\{|x_1|,...,|x_n|,...\},\\
    &\textbf{Metric:} \hspace{0.5cm} d_{\infty}(x,y) = \sup\{|x_1-y_1|,...,|x_n-y_n|,...\}.
\end{align*}

\subsection*{Spaces of continuous functions}

$X=C([a,b])$ is \textbf{the set of all continuous functions} $f:[a,b]\to\mathbb{R}$.
\begin{align*}
    &\textbf{Norm: } ||f||_{\infty} = \max\{|f(x)| : a\leq x\leq b\},\\
    &\textbf{Metric: } d_{\infty}(f,g) = ||f-g|| = \max\{|f(x)-g(x)| : a\leq x\leq b\}.
\end{align*}
\textbf{The $L^1$ metric:} $X$ as before.
\begin{align*}
    &\textbf{Norm: } ||f||_1 = \int_a^b |f(x)| dx,\\
    &\textbf{Metric: } d_1(f,g) = ||f-g||_1 = \int_a^b|f(x)-g(x)| dx.
\end{align*}
\textbf{The $L^2$ metric:} $X$ as before.
\begin{align*}
    &\textbf{Inner product: } \langle f,g\rangle = \int_a^b f(x)g(x) dx,\\
    &\textbf{Norm: } ||f||_2 = \langle f,f\rangle^{1/2} = \left(\int_a^b|f(x)|^2dx\right)^{1/2},\\
    &\textbf{Metric: } d_2(f,g) = ||f-g||_2 = \left(\int_a^b|f(x)-g(x)|^2dx\right)^{1/2}.
\end{align*}

\textbf{E7:} Let $(X,d)$ be a metric space and $Y$ be a non-empty subset of $X$. Define
\begin{equation*}
    d_Y: \text{ } Y\times Y\to\mathbb{R}, \hspace{0.5cm} d_Y(y,y')=d(y,y').
\end{equation*}
Then $d_Y$ is a metric on $Y$ called the \textbf{induced} or \textbf{inherited metric}, and $(Y,d_Y)$ is said to be a metric \textbf{subspace} of the metric space $(X,d)$.

\subsection*{Inequalities and identities}

The \textbf{Cauchy-Schwarz inequality} is $\langle x,y\rangle\leq||x||_2||y||_2$.

\section{Open balls, Sequences}

\textbf{D8:} Let $(X,d)$ be a metric space, $c\in X$ be a point and $r>0$. Then \textbf{open ball} with centre $c$ and radius $r$ is defined by
\begin{equation*}
    B(c,r) = \{x\in X : d(x,c) < r\}.
\end{equation*}

\textbf{Convergent sequences in metric spaces:} On the real line, $x_n\to x$ $\leftrightarrow$ $\forall\epsilon>0$, $\exists N\in\mathbb{N}$ such that $\forall n\geq N$ we have $|x_n-x|<\epsilon$.

\textbf{D15:} Let $(X,d)$ be a metric space, $(x_n)_{n\in\mathbb{N}}$ be a sequence in $X$, and $x\in X$. We say that $(x_n)_{n\in\mathbb{N}}$ converges to $x$ $\Leftrightarrow$ $\forall\epsilon>0$, $\exists N\in\mathbb{N}$ such that $\forall n\geq N$ we have $d(x_n,x)<\epsilon$.

Observe that $d(x_n,x)<\epsilon$ is equivalent to $x_n\in B(x,\epsilon)$.

$x_n\to x$ in $(X,d)$ $\Leftrightarrow$ $d(x_n,x)\to0$ on the real line.

\textbf{T16 Uniqueness of the limit:} Let $(X,d)$ be a metric space and $x,x'\in X$, $x\neq x'$. Then there exists a positive radius $r$ such that $B(x,r)\cap B(x',r)=\varnothing$. A sequence in a metric space can have at most one limit.

\textbf{D19:} A sequence in a metric space is said to be \textbf{bounded} $\Leftrightarrow$ there exists an open ball that contains all of its terms. \textbf{T20:} Every convergent sequence is bounded.

\textbf{D21:} A sequence $(x_n)_{n\in\mathbb{N}}$ in a metric space $(X,d)$ is said to be a \textbf{Cauchy sequence} $\Leftrightarrow$ $\forall\epsilon>0$, $\exists N\in\mathbb{N}$ such that $\forall n,m\geq N$, we have $d(x_n,x_m)<\epsilon$.

\textbf{T22:} If a sequence in a metric space converges, then it is a Cauchy sequence. \textbf{But Cauchy does not imply convergence}.

\textbf{E14:} If a Cauchy sequence has a convergent subsequence then the sequence itself is convergent.

\textbf{D24:} A metric space is said to be \textbf{complete} $\Leftrightarrow$ every Cauchy sequence is convergent.


\section{Open and closed sets}

\textbf{D26:} Let $(X,d)$ be a metric space. A subset $G\subseteq X$ is said to be \textbf{open} $\Leftrightarrow$ $\forall x\in G$ $\exists r>0$ such that $B(x,r)\subseteq G$. A subset $F\subseteq X$ is said to be \textbf{closed} $\Leftrightarrow$ $F^c$ is open.

\color{red}
\textbf{Example:} If $d$ is the discrete metric on a non-empty set $X$, then every subset of $X$ is both open and closed. Let $G\subseteq X$ and $x\in G$. Let $r=1/2$. Then $B(x,r)=\{x\}\subseteq G$. Therefore $G$ is open. Let $F\subseteq X$. Then $F^c$ is open, therefore $F$ is closed.
\color{black}

\textbf{D31:} A metric space is called \textbf{discrete} $\Leftrightarrow$ all its subsets are open (equiv. all subsets are closed).

\textbf{T34:} Let $(X,d)$ be a metric space. The union of any family of opens sets is an open set. The intersection of finitely many open sets is an open set.

\textbf{Remark:} The intersection of infinitely many open sets is not always an open set. \color{red} For example, let $G_n=(-1/n,1/n)$, $n=1,2,...$, on the real line with the standard metric. Each $G_n$ is open but $\bigcap_{n=1}^{\infty}G_n=\{0\}$ is not open.\color{black}

\textbf{T36:} Let $(X,d)$ be a metric space, $(x_n)_{n\in\mathbb{N}}$ be a sequence in $X$ and $x\in X$ be a point. $x_n\to x$ $\Leftrightarrow$ every open set that contains $x$ contains eventually all terms of the sequence.

An \textbf{open neighbourhood} of a point $x$ is any open set that contains $x$. $x_n\to x$ $\Leftrightarrow$ every open neighbourhood of $x$ contains eventually all terms of the sequence.

A \textbf{neighbourhood} of a point $x$ is any set that contains an open neighbourhood of $x$. $x_n\to x$ $\Leftrightarrow$ every neighbourhood of $x$ contains eventually all terms of the sequence.

\textbf{T37:} Let $(X,d)$ be a metric space. The intersection of any family of closed sets is a closed set. The union of finitely many closed sets is a closed set.

\textbf{Remark:} The union of infinitely many closed sets is not always a closed set. \color{red} For example, let $F_n=[1/n,1]$, $n=1,2,...$, on the real line with the standard metric. Each $F_n$ is closed but $\bigcup_{n=1}^{\infty}F_n=(0,1]$ is not closed.\color{black}

\textbf{T41:} A subset $F$ of a metric space is \textbf{closed} $\Leftrightarrow$ the limit of every convergent subsequence of elements of $F$ belongs to $F$.

\textbf{D43:} Let $(X,d)$ be a metric space and $A\subseteq X$. The \textbf{closure} of $A$, denoted by $\overline{A}$, is the smallest closed subset of $X$ that contains $A$.

Let $A\subseteq X$. The \textbf{boundary} of $A$, $\partial A$, is the set of all $x\in X$ such that, for all $r$
\begin{equation*}
    B(x,r) \cap A \neq \varnothing \hspace{0.5cm}\text{and}\hspace{0.5cm} B(x,r)\cap A^c\neq\varnothing.
\end{equation*}
In any metric space, $\partial\varnothing=\varnothing$ and $\partial X=\varnothing$. $\partial A=\overline{A}\setminus\text{Int}(A)$.

\textbf{T44:} Let $(X,d)$ be a metric space and $A,B\subseteq X$.
\begin{align*}
    &\overline{\varnothing} = \varnothing, \hspace{1cm} \overline{X}=X, \hspace{1cm} A\subseteq\overline{A} \text{ and } \overline{A} \text{ is closed}\\
    &A \text{ is closed } \Leftrightarrow A=\overline{A}, \hspace{1cm} \overline{\overline{A}} = \overline{A}\\
    &\text{if } A\subseteq B, \text{ then } \overline{A}\subseteq\overline{B}, \hspace{1cm} \overline{A\cup B}=\overline{A}\cup\overline{B}.
\end{align*}
\color{red}
\textbf{Example:} Let $X=\mathbb{R}$, $d(x,y)=|x-y|$. We claim that $\overline{\mathbb{Q}}=\mathbb{R}$. If there was a real number $x$ with $x\not\in\overline{\mathbb{Q}}$, then $x\in\mathbb{R}\setminus\overline{\mathbb{Q}}$. $\mathbb{R}\setminus\overline{\mathbb{Q}}$ is open, therefore there would be a positive $r$ such that $B(x,r)=(x-r,x+r)\subseteq\mathbb{R}\setminus\overline{\mathbb{Q}}$, therefore $(x-r,x+r)\subseteq\mathbb{R} \setminus\overline{\mathbb{Q}}$; contradiction.
\color{black}

\textbf{D49:} Let $(X,d)$ be a metric space. A subset $D\subseteq X$ is said to be \textbf{dense} $\Leftrightarrow$ $\overline{D}=X$. We showed in the previous example that $\mathbb{Q}$ is dense in $\mathbb{R}$.

\textbf{T50:} Let $(X,d)$ be a metric space, $A\subseteq X$, $x\in X$. The following are equivalent:
\begin{enumerate}
    \item $x\in\overline{A}$;
    \item For every positive $r$, $B(x,r)\cap A\neq\varnothing$;
    \item There exists a sequence $(a_n)_{n\in\mathbb{N}}$ with $a_n\in A$ for all $n$, such that $a_n\to x$.
\end{enumerate}
A point $x$ with any of these properties is called an \textbf{adherent point} of $A$. So $\overline{A}$ is the set of all adherent points of $A$.

\color{red}
\textbf{Example:} $X=\mathbb{R}$, $d(x,y)=|x-y|$, $A=(0,1)\cup\{2\}$, $\overline{A}=[0,1]\cup\{2\}$. $2$ is an adherent point of $A$. $0$ is an adherent point of $A$. Observe $2\in A$, $0\not\in A$.
\color{black}

\textbf{D52:} Let $(X,d)$ be a metric space, $A\subseteq X$ and $x\in X$. We say that $x$ is a \textbf{limit point} or an \textbf{accumulation point} of $A$ $\Leftrightarrow$ every open ball centered at $x$ contains an element of $A$ distinct from $x$, i.e.
\begin{equation*}
    \forall r>0, \hspace{0.5cm} (B(x,r) \setminus \{x\}) \cap A \neq \varnothing.
\end{equation*}
The set of all limit points of $A$ is called the \textbf{derived set} of $A$ and is denoted $A'$ or $A^{\sim}$.

\color{red}
\textbf{Example:} On the real line with the standard metric, let $A=(0,1)\cup\{2\}$. Then $\overline{A}=[0,1]\cup\{2\}$, so $0,2\in\overline{A}$. $0$ is a limit point of $A$, $2$ is not a limit point of $A$.
\color{black}

\textbf{E21:} Let $(X,d)$ be a metric space and $A\subseteq X$. Then $\overline{A}=A\cup A'$.

\section{Continuous functions between MSs}

\textbf{D54:} Let $(X,d_X)$ and $(Y,d_Y)$ be metric spaces and $f:X\to Y$ be a function. We say that $f$ is \textbf{continuous at a point $x_0$} in $X$ $\Leftrightarrow$ $\forall\epsilon>0$, $\exists\delta>0$ such that $\forall x\in X$ with $d_X(x,x_0)<\delta$, we have $d_Y(f(x),f(x_0))<\epsilon$.

$f$ is continuous at a point $x_0\in X$ $\Leftrightarrow$ $\forall\epsilon>0$, $\exists\delta>0$ such that $\forall x\in B_X(x_0,\delta)$ we have $f(x)\in B_Y(f(x_0),\epsilon)$.

\textbf{D55:} Let $(X,d_X)$ and $(Y,d_Y)$ be metric spaces. A function $f:X\to Y$ is said to be \textbf{continuous} $\Leftrightarrow$ it is continuous at every point in $X$.

\color{red}
\textbf{Example:} Let $(X,d)$ be a metric space and $p\in X$ be a point. Define $f:X\to\mathbb{R}$ by $f(x)=d(x,p)$. We claim that $f$ is continuous. Fix a point $x_0$. $\forall x\in X$,
\begin{equation*}
    |f(x) - f(x_0)| = |d(x,p) - d(x_0,p)| \leq d(x,x_0).
\end{equation*}
Given any $\epsilon>0$, take $\delta=\epsilon$. For all $x$ with $d(x,x_0)<\delta$ we have $|f(x)-f(x_0)|<\epsilon$.
\color{black}

\textbf{T57:} Let $(X,d_X)$, $(Y,d_Y)$ be metric spaces, $f:X\to Y$ be a function and $x_0\in X$ be a point. Then $f$ is continuous at $x_0$ $\Leftrightarrow$ for every open nbhd $G$ of $f(x_0)$ there exists an open nbhd $O$ of $x_0$ such that, for all $x\in O$, we have $f(x)\in G$.

\textbf{T58:} Let $(X,d_X)$, $(Y,d_Y)$ be metric spaces, $x_0\in X$ be a point and $f:X\to Y$ be a function. The following are equivalent:
\begin{enumerate}
    \item $f$ is continuous at $x_0$;
    \item For every sequence $(x_n)_{n=1}^{\infty}$ in $X$, if $x_n\to x_0$ in $(X,d_X)$, then $f(x_n)\to f(x_0)$ in $(Y,d_Y)$.
\end{enumerate}
\textbf{T59:} Let $(X,d_X)$ and $(Y,d_Y)$ be two metric spaces. A function $f:X\to Y$ is continuous $\Leftrightarrow$ the inverse image $f^{-1}(G)$ of any open subset $G$ of $Y$ is an open subset of $X$.

\textbf{D60:} A \textbf{topological space} is a set $X$ together with a family $\mathcal{T}$ of subsets of $X$ that has the following properties:
\begin{itemize}
    \item $\varnothing,X\in\mathcal{T}$;
    \item Any union of elements of $\mathcal{T}$ is an element of $\mathcal{T}$;
    \item Any finite intersection of elements of $\mathcal{T}$ is an element of $\mathcal{T}$.
\end{itemize}
$\mathcal{T}$ is called a \textbf{topology} and the elements of $\mathcal{T}$ are called \textbf{open sets}.

\textbf{D61:} Let $(X,\mathcal{T}_X)$ and $(Y,\mathcal{T}_Y)$ be two topological spaces. A function $f:X\to Y$ is said to be \textbf{continuous} $\Leftrightarrow$ for every $G$ in $\mathcal{T}_Y$ the pre-image $f^{-1}(G)$ is an element of $\mathcal{T}_X$.

$f$ is said to be a \textbf{homeomorphism} $\Leftrightarrow$ it is a continuous bijection and its inverse is continuous. If such a homeomorphism exists then $(X,\mathcal{T}_X)$ and $(Y,\mathcal{T}_Y)$ are said to be \textbf{homeomorphic}.

If $(X,\mathcal{T}_X)$ and $(Y,\mathcal{T}_Y)$ are homeomorphic and one of them is compact or connected or separable or ..., then so is the other. Properties that are preserved by homeomorphisms are called topological properties.

\textbf{T66:} Let $(X,d)$ be a metric space. The function $d:X\times X\to\mathbb{R}$ is continuous ($\mathbb{R}$ is equipped with the standard metric. $X\times X$ is equipped with the product metric).

\textbf{D67:} A linear operator $T:X\to Y$ is said to be \textbf{bounded} $\Leftrightarrow$ there exists a positive constant $C$ such that, for all $x\in X$, $||T(x)||_Y\leq C||x||_X$.

\textbf{T68:} Let $T:X\to Y$ be a linear operator. The following are equivalent:
\begin{enumerate}
    \item $T$ is continuous;
    \item $T$ is continuous at $0$;
    \item $T$ is bounded.
\end{enumerate}
\textbf{D70:} Let $(X,d_X)$ and $(Y,d_Y)$ be metric spaces. A function $f:X\to Y$ is said to be \textbf{Lipschitz} function, $\Leftrightarrow$ there exists a constant $L$ such that $\forall x,x'\in X$
\begin{equation*}
    d_Y(f(x),f(x')) \leq Ld_X(x,x').
\end{equation*}
If $L<1$, $f$ is said to be a contraction.

\textbf{T71:} Every Lipschitz function is continuous.

Lipschitz function $\Rightarrow$ \textbf{its derivative is bounded.}

\textbf{D72:} A \textbf{fixed point} of a function $f:S\to S$, where $S$ is a non-empty set, is any element $x$ of $S$ such that $f(x)=x$.

\textbf{T75 (Banach's fixed point theorem):} Let $(X,d)$ be a complete metric space and let $f:X\to X$ be a contraction. Then $f$ has a unique fixed point.

\textbf{D76:} Two metrics on the same non-empty set $X$ are said to be equivalent $\Leftrightarrow$ they have the same open sets.

\textbf{T77:} Let $d_1$ and $d_2$ be metrics on the same non-empty set $X$. If there exist positive constants $C,C'$ such that, $\forall x,y\in X$,
\begin{equation*}
    Cd_1(x,y) \leq d_2(x,y) \leq C'd_1(x,y),
\end{equation*}
then $d_1$ and $d_2$ are equivalent.

\color{red}
\textbf{Proof:} Let $G\subseteq X$ be open w.r.t $d_1$. Let $x\in G$. There is a positive $r$ such that $B_{d_1}(x,r)\subseteq G$. Observe that $B_{d_2}(x,Cr)\subseteq B_{d_1}(x,r)\subseteq G$ because
\begin{equation*}
    d_2(y,x) < Cr \Rightarrow d_1(y,x) \leq \frac{1}{C}d_2(x,y) < \frac{1}{C}Cr = r.
\end{equation*}
\color{black}

\textbf{D78:} Let $(X,d_X)$ and $(Y,d_Y)$ be metric spaces, $x_0$ be a limit point of $X$, $y_0\in Y$ and $f:X\to Y$ be a function. We say that $\lim_{x\to x_0}f(x)=y_0$ $\Leftrightarrow$
\begin{equation*}
    \forall\epsilon>0, \text{ } \exists\delta>0, \text{ } \forall x\in B_X(x_0,\delta)\setminus\{x_0\}, \text{ } f(x)\in B_Y(y_0,\epsilon).
\end{equation*}

\section{Completeness}

\textbf{T79:} $(\mathbb{R}^n,d_2)$ is a complete metric space.

\color{red}
\textbf{Proof:} Let $(x_k)_{k=1}^{\infty}$ be a Cauchy sequence in $\mathbb{R}^n$.
\begin{align*}
    x_1 &= (x_{11},x_{12},...,x_{1n})\\
    x_2 &= (x_{21},x_{22},...,x_{2n})\\
    &\vdots\\
    x_k &= (x_{k1},x_{k2},...,x_{kn}).
\end{align*}
On the $j$-th column, we have that for all $k,l$
\begin{equation*}
    |x_{kj}-x_{lj}| \leq \left(\sum_{i=1}^n(x_{ki}-x_{li})^2\right)^{1/2} = d_2(x_k,x_l).
\end{equation*}
Therefore the sequence $(x_{kj})_{k=1}^{\infty}$ is Cauchy in $\mathbb{R}$. Let $l_j=\lim_{k\to+\infty}x_{kj}$, ($j=1,...,n)$, and define $x=(l_1,...,l_n)$.
\begin{align*}
    x_1 &= (x_{11},x_{12},...,\color{blue} x_{1j} \color{red},...,x_{1n})\\
    x_2 &= (x_{21},x_{22},...,\color{blue} x_{2j} \color{red},...,x_{2n})\\
    &\vdots\\
    x_k &= (x_{k1},x_{k2},...,\color{blue} x_{kj} \color{red},...,x_{kn})\\
    x &= (l_1,l_2,..., \color{blue} l_j \color{red},...,l_n).
\end{align*}
Then $x_k\to x$ (as $k\to+\infty$) in $\mathbb{R}^n$ because
\begin{equation*}
    d_2(x_k,x) = \sqrt{|x_{k1}-l_1|^2 + ... + |x_{kn}-l_n|^2} \to \sqrt{0^2+...+0^2}=0
\end{equation*}
as $k\to+\infty$.
\color{black}

\textbf{T80:} $\ell^2$ is a complete metric space.

\textbf{Remark:} $\ell^p$, where $1\leq p<\infty$ is complete.

\textbf{T82:} Let $a,b\in\mathbb{R}$, $a<b$. Then $C([a,b])$ with its usual metric
\begin{equation*}
    d(f,g) = \max\{|f(x)-g(x)| : a\leq x\leq b\}
\end{equation*}
is a complete metric space.

\textbf{D83:} Let $(X,d)$ be a metric space and $n\in\mathbb{N}$. Define $D:X^n\to\mathbb{R}$ by
\begin{equation*}
    D(x_1,x_2) = d(x_{11},x_{21}) + d(x_{12},x_{22}) + ... + d(x_{1n},x_{2n}).
\end{equation*}
$D$ is a metric and a sequence converges in $(X^n,D)$ $\Leftrightarrow$ it converges component-wise.

\textbf{E34:} If $(X,d)$ is complete then $(X^n,D)$ is complete.

\textbf{D84:} $B^A$, where $A,B$ are sets, is the set of all functions from $A$ to $B$.

\textbf{D85:} Let $(X,d)$ be a metric space. Define a metric $D:X^{\mathbb{N}}\times X^{\mathbb{N}}\to\mathbb{R}$ by
\begin{equation*}
    D(x_1,x_2) = \sum_{n=1}^{\infty} \frac{1}{2^n}\frac{d(x_{1n},x_{2n})}{1+d(x_{1n},x_{2n})},
\end{equation*}
where, as usual, we have used the notation $x_1=(x_{11},...,x_{1n},...)$, $x_2=(x_{21},...,x_{2n},...)$. $(X^{\mathbb{N}},D)$ is called a \textbf{product space}.

\textbf{T86:} Let $(X,d)$ be a metric space, let $(x_k)_{k=1}^{\infty}$ be a sequence in $X^{\mathbb{N}}$ and let $x\in X^{\mathbb{N}}$. Write $x_k=(x_{k1},...,x_{kn},...)$ and $x=(I_1,...,I_n,...)$. Then,
\begin{equation*}
    x_k \underset{k\to+\infty}{\overset{(X^{\mathbb{N}},D)}{\longrightarrow}} x \hspace{0.5cm}\Leftrightarrow\hspace{0.5cm} \text{for all } n, \hspace{0.5cm} x_{kn} \underset{k\to+\infty}{\overset{(X,d)}{\longrightarrow}} I_n
\end{equation*}
\textbf{T87:} Let $(X,d)$ be a complete metric space. Then the product space $(X^{\mathbb{N}},D)$ is complete.

\textbf{The least upper bound principle:} Every non-empty bounded above subset of $\mathbb{R}$ has a least upper bound.

\textbf{T88:} Every bounded monotone sequence of real numbers as a limit.

\textbf{T89:} Every Cauchy sequence of real numbers is convergent.

\textbf{D90:} Let $(x_n)_{n=1}^{\infty}$ be a Cauchy sequence in $\mathbb{R}$. Then $(x_n)_{n=1}^{\infty}$ is bounded. Define $(I_n)_{n=1}^{\infty}$ and $(S_n)_{n=1}^{\infty}$ by
\begin{equation*}
    I_n = \inf\{x_n,x_{n+1},...\}, \hspace{1cm} S_n = \sup\{x_n,x_{n+1},...\}.
\end{equation*}
The limit of the sequence $(I_n)_{n=1}^{\infty}$ is called the \textbf{limit inferior} of $(x_n)_{n=1}^{\infty}$ and is denoted by $\liminf x_n$. The limit of the sequence $(S_n)_{n=1}^{\infty}$ is called the \textbf{limit superior} of $(x_n)_{n=1}^{\infty}$ and is denoted by $\limsup x_n$. So
\begin{align*}
    \liminf x_n &= \lim_{n\to+\infty}I_n = \lim_{n\to+\infty}\inf\{x_n,x_{n+1},...\}\\
    \limsup x_n &= \lim_{n\to+\infty}S_n = \lim_{n\to+\infty}\sup\{x_n,x_{n+1},...\}.
\end{align*}
$(x_n)_{n=1}^{\infty}$ converges if its limit inferior and superior equal.

\section{Compactness}

\textbf{D91:} Let $X=\mathbb{R}$ and $d$ be a standard metric. A subset $K$ of $\mathbb{R}$ is said to be \textbf{compact} $\Leftrightarrow$ every sequence of elements of $K$ has a subsequence that converges to an element of $K$.

\color{red}
\textbf{Examples:}
\begin{itemize}
    \item $[a,b]$ is compact. Let $(x_n)_{n=1}^{\infty}$ be a sequence of elements of $[a,b]$. By Bolzano-Weierstrass, it has a subsequence $(x_{n_k})_{k=1}^{\infty}$ that converges to some real $x$. For all $k$, $a\leq x_{n_k}\leq b$, therefore $a\leq x\leq b$, therefore $x\in[a,b]$.
    \item $(0,1)$ is not compact. The sequence $(1/2,1/3,...,1/n,...)$ does not have a subsequence that converges to an element of $(0,1)$.
    \item $\varnothing$ is compact.
    \item $\mathbb{R}$ is not compact. The sequence $(1,2,...,n,...)$ does not have a convergent subsequence.
    \item The extended real number system $\mathbb{R}\cup\{-\infty,\infty\}$ is compact.
\end{itemize}
\color{black}

\textbf{T93 (Heine-Borel theorem):} On the real line with the standard metric, a set is compact $\Leftrightarrow$ it is closed and bounded.

\color{red}
\textbf{Proof:} Let $K$ be a subset of $\mathbb{R}$.

($\Rightarrow$) Assume that $K$ is compact. $K$ is closed: Let $(x_n)_{n=1}^{\infty}$ be a sequence of elements of $K$ that converges to a real $x$. We want $x\in K$. By compactness, there is a subsequence $(x_{n_k})_{k=1}^{\infty}$ that converges to some $x'\in K$. But $x=x'$. Therefore $x\in K$.

$K$ is bounded above: If not, then there exists a sequence $(x_n)_{n=1}^{\infty}$ of elements of $K$ such that, for all $n$, $x_n>n$. This sequence does not have a convergent subsequence; a contradiction. The proof that $K$ is bounded below is similar.

($\Leftarrow$) Assume $K$ is closed and bounded. Let $(x_n)_{n=1}^{\infty}$ be a sequence of elements of $K$. The sequence is bounded because $K$ is bounded. By Bolzano-Weierstrass, it has a subsequence $(x_{n_k})_{k=1}^{\infty}$ that converges to some real $x$. For all $k$, $x_{n_k}\in K$, and $K$ is closed, therefore $x\in K$.
\color{black}

\textbf{T94:} Let $K\subseteq\mathbb{R}$ be compact and $f:K\to\mathbb{R}$ be a continuous function. Then $f$ is bounded.

\textbf{T95 (Extreme value theorem):} Let $K\subseteq\mathbb{R}$ be compact and $f:K\to\mathbb{R}$ be a continuous function. Then $f$ has a maximum and a minimum.

\textbf{D96:} An \textbf{open cover} of a set $S$ in a metric space is a family $(G_i)_{i\in I}$ of open sets such that $S\subseteq\bigcup_{i\in I}G_i$. A \textbf{subcover} of an open cover $(G_i)_{i\in I}$ is a sub-family $(G_i)_{i\in I'}$, where $I'\subseteq I$, such that $S\subseteq\bigcup_{i\in I'}G_i$.

\color{red}
\textbf{Example 1:} On the real line with the standard metric, let $S=\mathbb{Z}$. Define $G_x=(x-2,x+2)$, for $x\in\mathbb{R}$. Then $S\subseteq\bigcup_{x\in\mathbb{R}}G_x$ (open cover) and also $S\subseteq\bigcup_{\underset{n\text{ even }}{n\in\mathbb{Z}}}G_n$ (subcover).

\textbf{Example 2:} On the real line with the standard metric, let $S=[0,1]$. Define $G_n=(1/n,2)$, $n=1,2,...$, and $G_0=(-1,1/100)$. Then $S\subseteq\bigcup_{n=0}^{\infty}G_n$ (open cover) and also $S\subseteq\bigcup_{n=0}^{101}G_n$ (subcover).
\color{black}

\textbf{T[99]:} On the real line with the standard metric, a set $K$ is compact $\Leftrightarrow$ every open cover of $K$ has a finite subcover.

\textbf{L99:} Every open cover of the interval $[a,b]$, where $a,b\in\mathbb{R}$, $a\leq b$, has a finite subcover.

\textbf{T100:} On the real line with the standard metric, if $K$ is compact, then every open cover of $K$ has a finite subcover.

\textbf{T101:} Let $K\subseteq\mathbb{R}$ and assume that every open cover of $K$ has a finite subcover. Then $K$ is closed and bounded, hence compact.

\textbf{D102:} Let $(X,d)$ be a metric space and $K\subseteq X$.
\begin{enumerate}
    \item We say that $K$ is \textbf{sequentially compact} $\Leftrightarrow$ every sequence in $K$ has a subsequence that converges to an element of $K$. For $K=X$ this becomes: $X$ is compact $\Leftrightarrow$ every sequence in $X$ has a convergent subsequence.
    \item We say that $K$ is \textbf{compact} $\Leftrightarrow$ every open cover of $K$ has a finite subcover.
\end{enumerate}
\color{red}
\textbf{Example:} Let $d$ be the discrete metric on $X$ and $K\subseteq X$. Then, $K$ is sequentially compact $\Leftrightarrow$ $K$ is finite.

($\Rightarrow$) Assume $K$ is sequentially compact. If $K$ was infinite we would be able to find a sequence $(x_n)_{n=1}^{\infty}$ in $K$ with discrete terms. Then $d(x_n,x_m)=1$ for $n\neq m$. Therefore there is no convergent subsequence.

($\Leftarrow$) Assume $K$ is finite. Let $(x_n)_{n=1}^{\infty}$ be a sequence in $K$. Then $(x_n)_{n=1}^{\infty}$ has a constant subsequence, which trivially converges to an element of $K$.
\color{black}

\textbf{T105:} Let $(X,d)$ be a metric space and $K\subseteq X$ be sequentially compact. Then $K$ is closed and bounded.
\color{red}
\textbf{Proof:} Similar to the Heine-Borel theorem proof.
\color{black}

\textbf{T107:} Let $(X,d)$ be a sequentially compact metric space and $K\subseteq X$. The set $K$ is sequentially compact $\Leftrightarrow$ it is closed.

\color{red}
\textbf{Proof:} ($\Rightarrow$) Follows from theorem 105.

($\Leftarrow$) Assume $K$ is closed. Let $(x_n)_{n=1}^{\infty}$ be a sequence in $K$. Since $X$ is sequentially compact, there exists a subsequence $(x_{n_k})_{k=1}^{\infty}$ that converges to some limit $x\in X$. Since $x_{n_k}\in K$ for all $k$, and $K$ is closed, we have $x\in K$.
\color{black}

\textbf{T108:} If a metric space $(X,d)$ is sequentially compact, then it is complete.

\color{red}
\textbf{Proof:} Assume $(X,d)$ is sequentially compact and let $(x_n)_{n=1}^{\infty}$ be a Cauchy sequence in $X$. By sequential compactness, there is a convergent subsequence $(x_{n_k})_{k=1}^{\infty}$. A Cauchy sequence with a convergent subsequence is itself convergent.
\color{black}

\textbf{Remark:} The converse is not true. For example, $\mathbb{R}$ with the standard metric is complete but not compact.

\textbf{T110 (Extreme value theorem):} Let $(X,d)$ be a metric space, $K$ be a sequentially compact subset of $X$, and $f:K\to\mathbb{R}$ be a continuous function. Then $f$ has a maximum and a minimum. In particular, $f$ is bounded.

\textbf{D111:} Let $(X,d_X)$, $(Y,d_Y)$ be metric spaces. A function $f:X\to Y$ is said to be \textbf{uniformly continuous} $\Leftrightarrow$ $\forall\epsilon>0$, $\exists\delta>0$ such that $\forall x,x'\in X$ with $d_X(x,x')<\delta$ we have $d_Y(f(x),f(x'))<\epsilon$.

\color{red}
\textbf{Example:} Every Lipschitz function is uniformly continuous.
\color{black}

\textbf{T114:} Let $(X,d_X)$ be a sequentially compact metric space, $(Y,d_Y)$ be a metric space and $f:X\to Y$ be a continuous function. Then $f$ is uniformly continuous.

\textbf{T115:} If a metric space $(X,d)$ is compact, then it is sequentially compact.

\textbf{T116 (Lebesgue's lemma):} Let $(X,d)$ be a sequentially compact metric space and $X=\bigcup_{i\in I}G_i$ be an open cover of $X$. Then $\exists\delta>0$ such that $\forall x,y\in X$ with $d(x,y)<\delta$, there exists an $i$ such that $x,y\in G_i$. Any such $\delta$ is called a Lebesgue number of the open cover.

\textbf{D117:} A metric space $(X,d)$ is said to be \textbf{totally bounded} $\Leftrightarrow$ $\forall\delta>0$, $X$ can be covered by a finite number of open balls of radius $\delta$.

\textbf{Remark:} Bounded $\not\Rightarrow$ totally bounded. Take for example an infinite set $X$ with the discrete metric. Then $X$ is bounded. Take $\delta=1$. We can't cover $X$ with a finite number of open balls of radius $\delta$:
\begin{equation*}
    X \not\subseteq B(x_1,1) \cup \cdots \cup B(x_k,1) = \{x_1\} \cup \cdots \cup \{x_k\}.
\end{equation*}
Therefore the metric space is not totally bounded.

\textbf{T120:} If a metric space is sequentially compact, then it is totally bounded.

\textbf{T121:} Every sequentially compact metric space is compact.

\textbf{T122:} A metric space is compact $\Leftrightarrow$ it is complete and totally bounded.


\section{Separability}

\textbf{D123:} A set $S$ is said to be
\begin{itemize}
    \item \textbf{Infinitely countable} $\Leftrightarrow$ there is a bijection $f:\mathbb{N}\to S$;
    \item \textbf{Countable} $\Leftrightarrow$ it is finite or infinitely countable;
    \item \textbf{Uncountable} $\Leftrightarrow$ is is not countable.
\end{itemize}
\textbf{E45:} Let $(X,d)$ be a metric space and $D\subseteq X$. The following are equivalent:
\begin{enumerate}
    \item $D$ is dense.
    \item $\forall x\in X$ and $\forall\epsilon>0$, $\exists y\in D$ such that $d(x,y)<\epsilon$.
    \item $\forall x\in X$ there is a sequence $(y_n)_{n=1}^{\infty}$ of elements of $D$ such that $y_n\to x$.
    \item $\forall x\in X$ and every open nbhd $G$ of $x$, $G\cap D\neq\varnothing$.
    \item $D$ intersects every non-empty open set.
\end{enumerate}
\textbf{D125:} A metric space is said to be \textbf{separable} $\Leftrightarrow$ it has a countable dense subset.

\color{red}
\textbf{Examples:} $\mathbb{R}$ with the standard metric is a separable metric space because $\mathbb{Q}$ is dense and countable.

$\mathbb{R}^n$ with the Euclidean metric is a separable metric space because $\mathbb{Q}^n$ is dense and countable.

$\mathbb{C}$ with its standard metric is a separable metric space because $\{z\in\mathbb{C} : \text{Re}(z),\text{Im}(z)\in\mathbb{Q}\}$ is dense and countable.
\color{black}

\textbf{E128:} $\ell^2$ is separable.

\color{red}
\textbf{Proof:} Let $x=(x_1,...,x_n,...)\in\ell^2$ and $\epsilon>0$. Then $\sum_{n=1}^{\infty}|x_n|^2$ converges. There exists $N$ such that $\sum_{n=N+1}^{\infty}|x_n|^2<\epsilon^2/2$. Choose rationals $q_1,...,q_N$ such that $|x_i-q_i|<\epsilon/\sqrt{2N}$, $1\leq i\leq N$. Define $q=(q_1,...,q_N,0,0,...)$. Then
\begin{align*}
    d_2(x,q)^2 &= \sum_{n=1}^{\infty} |x_n-q_n|^2\\
    &= \sum_{n=1}^N |x_n-q_n|^2 + \sum_{n=N+1}^{\infty} |x_n-q_n|^2\\
    &< N\cdot\frac{\epsilon^2}{2N} + \frac{\epsilon^2}{2}.
\end{align*}
We have show that the set $D$ of all rational sequences whose terms are eventually zero is dense. It is countable because $D=D_1\cup \cdots \cup D_n \cup \cdots$, where
\begin{equation*}
    D_n = \{(q_1,...,q_n,0,0,...) : q_1,...,q_n\in\mathbb{Q}\},
\end{equation*}
and each $D_n$ is countable.
\color{black}

\textbf{Remark:} $\ell^2$ is separable for $1\leq p<\infty$ (but not $\infty$).

\color{red}
\textbf{Proof:} $\ell^{\infty}$ is not separable. Recall, for $x=(x_1,...,x_n,...)\in\ell^{\infty}$ and $y=(y_1,...,y_n,...)\in\ell^{\infty}$,
\begin{equation*}
    d_{\infty}(x,y) = \sup\{|x_n-y_n| : n\in\mathbb{N}\}.
\end{equation*}
Let $A=\{x\in\ell^{\infty} : \forall n \text{ } x_n=0 \text{ or } 1\}$. Then $A$ is uncountable and for all $a,a'\in A$ with $a\neq a'$, $d_{\infty}(a,a')=1$, therefore $B(a,1/2)\cap B(a',1/2)=\varnothing$. $(B(a,1/2))_{a\in A}$ is an uncountable family of disjoint open balls. If $D\subseteq\ell^{\infty}$ is dense, then, for each $a\in A$ there is an element $x_a\in D$ in $B(a,1/2)$. The set $\{x_a:a\in A\}$ is uncountable and is a subset of $D$, therefore $D$ is uncountable.
\color{black}

\textbf{T130 (Weierstrass approximation theorem):} Let $f:[a,b]\to\mathbb{R}$ be a continuous function and $\epsilon>0$. There exists a polynomial $p$ with real coefficients such that, $\forall x\in[a,b]$, $|f(x)-p(x)|<\epsilon$.

\textbf{T131:} Let $f:[a,b]\to\mathbb{R}$ be continuous and $\epsilon>0$. There exists a polynomial $p$ with rational coefficients such that, $\forall x\in[a,b]$, $|f(x)-p(x)|<\epsilon$.

\color{red}
\textbf{Proof:} We work with $[a,b]=[0,1]$. By theorem 130, there exists a polynomial with real coefficients $p(x)=c_nx^n+...+c_1x+c_0$, such that, $\forall x\in[0,1]$, $|f(x)-p(x)|<\epsilon$. Choose rationals $r_n,...,r_1,r_0$ such that
\begin{equation*}
    |r_i-c_i| < \frac{\epsilon}{2(n+1)}, \hspace{0.5cm} 0\leq i\leq n.
\end{equation*}
Define $q(x)=r_nx^n+...+r_1x+r_0$. Then, $\forall x\in[a,b]$,
\begin{align*}
    |p(x) - q(x)| &= |(c_n-r_n)x^n + ... + (c_1-r_1)x + (c_0-r_0)|\\
    &\leq |c_n-r_n|x^n + ... + |c_1-r_1|x + |c_0-r_0|\\
    &\leq \frac{\epsilon}{2(n+1)}(n+1) = \frac{\epsilon}{2},
\end{align*}
therefore; $|f(x)-q(x)|\leq|f(x)-p(x)|+|p(x)-q(x)|<\epsilon/2+\epsilon/2=\epsilon$.
\color{black}

\textbf{T132:} $C([a,b])$ is separable.

\color{red}
\textbf{Proof:} Let $D$ be the set of all polynomials with rational coefficients. Then $D$ is countable. By theorem 131, $\forall f\in C([a,b])$ and $\forall\epsilon>0$, $\exists p\in D$ such that, $\forall x\in[a,b]$, $|f(x)-q(x)|<\epsilon$, therefore $d_{\infty}(f,q)=\max\{|f(x)-q(x)| : x\in[a,b]\}<\epsilon$. This shows that $D$ is dense in $C([a,b])$.
\color{black}

\textbf{T133:} Let $(X,d)$ be a metric space, $A\subseteq X$, $A\neq\varnothing$ and $d_A$ be the induced metric on $A$. Then the metric space $(A,d_A)$ is separable.

\color{red}
\textbf{Proof:} Let $D\subseteq X$ be countable and dense. Then $\forall c\in D$ and $r\in\mathbb{Q}^+$, if $B(c,r)\cap A\neq\varnothing$, pick a point $a_{c,r}\in B(c,r)\cap A$. The set $\{a_{c,r}:c\in D, r\in\mathbb{Q}^+\}$ is a countable subset of $A$. We claim that it is dense in $A$. Let $a\in A$, $\epsilon>0$. Pick a positive rational $\epsilon'$ with $0<\epsilon'<\epsilon$. Since $D$ is dense in $X$, there exists a point $c\in D$ such that $d(c,a)<\epsilon'/2$. So $B(c,\epsilon'/2)\cap A\neq\varnothing$ (it contains $a$), therefore a point $a_{c,\epsilon'/2}$ was chosen in $B(c,\epsilon'/2)\cap A$. Therefore
\begin{equation*}
    d(a,a_{c,\epsilon'/2}) \leq d(a,c) + d(c,a_{c,\epsilon'/2}) < \frac{\epsilon'}{2} + \frac{\epsilon'}{2} = \epsilon' < \epsilon.
\end{equation*}
\color{black}

\textbf{T135:} Every compact metric space is separable.

\color{red}
\textbf{Consider adding proof}
\color{black}

\textbf{T136:} Let $(X,d)$ be a separable metric space and let $D$ be a countable dense subset of $X$. Let $\mathcal{B}=\{B(c,r):c\in D,r\in\mathbb{Q}^+\}$ be the set of all open balls with centers in $D$ and rational radii. Then, $\mathcal{B}$ is countable and every open set in $X$ can be written as a union of elements of $\mathcal{B}$.

\textbf{D137:} Let $(X,\mathcal{T})$ be a topological space. An \textbf{open base} (or simply base) for the topology $\mathcal{T}$ is a family $\mathcal{B}$ of open sets such that every open set in $\mathcal{T}$ can be written as a union of elements of $\mathcal{B}$.

\color{red}
\textbf{Example:} In a metric space, the family of all open balls is an open base.
\color{black}

\textbf{D139:} A topological space $(X,\mathcal{T})$ is said to satisfy the \textbf{second axiom of countability} (or simply to be \textbf{second countable}) $\Leftrightarrow$ it has a countable open base.

\color{red}
\textbf{E38:} If a metric space is separable then it is second countable.
\color{black}

\textbf{T140:} In a separable metric space, every family of pairwise disjoint non-empty open sets is countable.

\color{red}
\textbf{Proof:} Let $(X,d)$ be separable and $D$ be a countable dense subset. Let $(G_i)_{i\in I}$ be a family of pairwise disjoint non-empty open sets. For each $i\in I$, pick an element $x_i\in D\cap G_i$. The $x_i$'s are distinct because the $G_i$'s are disjoint. The function $I\to D$, $i\mapsto x_i$ is an injection. Let $D'$ be its range. Then $I\to D'$, $i\mapsto x_i$ is a bijection. $D'$ is countable, therefore $I$ is countable.
\color{black}

\textbf{T141:} On the real line with the standard metric, every open set can be written as a countable union of disjoint open intervals.

\textbf{T142:} Let $(X,d_X)$ and $(Y,d_Y)$ be metric spaces, $D\subseteq X$ be dense, $f,g:X\to Y$ continuous functions such that $f(x)=g(x)$ $\forall x\in D$. Then $f=g$.

\color{red}
\textbf{Proof:} Fix a point $x\in X$. Since $D$ is dense, there is a sequence $(x_n)_{n=1}^{\infty}$ of elements of $X$ such that $x_n\to x$. By continuity, $f(x_n)\to f(x)$ and $g(x_n)\to g(x)$. But $f(x_n)=g(x_n)$ for all $n$. Therefore $f(x)=g(x)$.
\color{black}

\textbf{T143:} Let $(X,d_X)$ and $(Y,d_Y)$ be metric spaces, $D\subseteq X$ be dense, $f:D\to Y$ be uniformly continuous, and assume that $(Y,d_Y)$ is complete. Then $f$ has a unique continuous extension $F:X\to Y$.

\section{Completeness (again)}

\textbf{T144:} Let $(X,d)$ be a metric space, $F$ be a non-empty subset of $X$ and $d_F$ be the induced metric on $F$. If the metric space $(F,d_F)$ is complete then $F$ is a closed subset of $X$.

\textbf{T145:} Let $(X,d)$ be a complete metric space, $F$ be a non-empty subset of $X$ and $d_F$ be the induced metric on $F$. If $F$ is a closed subset of $X$, then the metric space $(F,d_F)$ is complete.

\color{red}
\textbf{Proof:} Assume $F$ is closed. Let $(x_n)_{n=1}^{\infty}$ be Cauchy in $(F,d_F)$. Then $(x_n)_{n=1}^{\infty}$ is Cauchy in $(X,d)$. Therefore $d(x_n,x)\to0$ for some $x\in X$. Then $x\in F$ because $F$ is closed, and $d_F(x_n,x)\to0$, therefore $(x_n)_{n=1}^{\infty}$ converges to $x$ in $(F,d_F)$.
\color{black}

\textbf{T146:} Let $(X,d)$ be a complete metric space, $A\subseteq X$, $A\neq\varnothing$. Then
\begin{enumerate}
    \item The metric space $(\overline{A},d_{\overline{A}})$ is complete;
    \item If $A\subseteq B\subseteq X$ and $(B,d_B)$ is complete, then $\overline{A}\subseteq B$.
\end{enumerate}
\textbf{D147 (Isometries):} Let $(X,d_X)$ and $(Y,d_Y)$ be metric spaces. A function $f:X\to Y$ is called an \textbf{isometry} $\Leftrightarrow$ $\forall x_1,x_2\in X$, $d_Y(f(x_1),f(x_2))=d_X(x_1,x_2)$,

\textbf{T148:} Let $(X,d_X)$ and $(Y,d_Y)$ be metric spaces and $f:X\to Y$ be an isometry. Then $f$ is an injection. If, moreover, $f$ is a surjection (hence $f$ is a bijection), then $f^{-1}:Y\to X$ is an isometry as well.

\textbf{D149 (Isometric metric spaces):} The metric spaces $(X,d_X)$ and $(Y,d_Y)$ are said to be \textbf{isometric} $\Leftrightarrow$ there exists an isometry $f$ from $X$ onto $Y$. (By theorem 1483, $f$ is then a bijection and $f^{-1}:Y\to X$ is an isometry as well.

\textbf{E52:} If two metric spaces are isometric and one of them is complete, compact, connected,... then so is the other.

\textbf{T150:} Let $(X,d)$ be a bounded metric space and let $C(X,\mathbb{R})$ be the set of all bounded continuous functions $f:X\to\mathbb{R}$ equipped with the metric $d_{\infty}(f_1,f_2)=\sup\{|f_1(t)-f_2(t)|:t\in X\}$. For each $x\in X$ define $F_x:X\to\mathbb{R}$ by $F_x(x')=d(x,x')$. Then
\begin{enumerate}
    \item $F_x\in C(X,\mathbb{R})$;
    \item The map $X\to C(X,\mathbb{R})$, $x\mapsto F_x$ is an isometry;
    \item $X^*=\{F_x:x\in X\}$, equipped with the induced metric, is a subspace of $C(X,\mathbb{R})$ isometric to $X$;
    \item The closure $\overline{X^*}$ of $X^*$ in $C(X,\mathbb{R})$, equipped with the induced metric, is a complete metric space;
    \item $X^*$ is dense in $\overline{X^*}$.
\end{enumerate}
\textbf{D152:} Let $(X,d_X)$ be a metric space. A \textbf{completion} of $(X,d_X)$ is any metric space $(Y,d_Y)$ with the following properties:
\begin{enumerate}
    \item $(Y,d_Y)$ is complete;
    \item $(Y,d_Y)$ has a subspace $X^*$ isometric to $(X,d_X)$;
    \item $X^*$ is dense in $Y$.
\end{enumerate}
Theorem 150 shows that every metric space has a completion.


\section{Connectedness}

\textbf{D153:} A metric space $(X,d)$ is said to be \textbf{disconnected} $\Leftrightarrow$ there exist non-empty disjoint open sets $G_1$ and $G_2$ such that $X=G_1\cup G_2$. Otherwise the metric space is called \textbf{connected}. A non-empty subset $A$ of a metric space $(X,d)$ be said to be \textbf{disconnected} $\Leftrightarrow$ the metric space $(A,d_A)$, where $d_A$ is the induced metric, is disconnected.

\textbf{T154:} $\mathbb{R}$ with the standard metric is connected.

\color{red}
\textbf{Proof:} If not, then $\mathbb{R}=G_1\cup G_2$ where $G_1,G_2$ are open, non-empty, disjoint. Let $x_1\in G_1$, $x_2\in G_2$. We may assume $x_1<x_2$. Since $G_1$ is open, there exists an interval $(x_1-r,x_1+r)\subseteq G_1$. Let $s=\sup A$ where $A=\{x\in(x_1,x_2):(x_1,x)\subseteq G_1\}$.

\textbf{Case 1:} $s\in G_1$. There is an interval $(s-\delta,s+\delta)\subseteq G_1$. By the approximation property there is an element $a\in A$ in $(s-\delta,s+\delta)$. Then $(x_1,s+\delta)=(x_1,a)\cup(a,s+\delta)\subseteq G_1$, which shows that $A$ has elements $>s$, which is not the case.

\textbf{Case 2:} $s\in G_2$. There is an interval $(s-\delta,s+\delta)\subseteq G_2$. But then $A$ can't have any elements in $(s-\delta_1,s)$ which violates the approximation property.

We have shown that $s\not\in G_1,G_2$; a contradiction.
\color{black}

\textbf{E53:} On the real line with the standard metric all intervals are connected sets.

\textbf{T155:} A subset of the real line is connected $\Leftrightarrow$ it is an interval.

\color{red}
\textbf{Proof:} Let $A\subseteq\mathbb{R}$ be connected. Let $m=\inf A$ and $M=\sup A$. We'll show that $A$ is an interval with endpoints $m$ and $M$. We present a proof in the case $m,M\in\mathbb{R}$. It is enough to show that $(m,M)\subseteq A$. If not, then there exists an $x\in(m,M)$ such that $x\not\in A$. Then $A\subseteq\mathbb{R}\setminus\{x\} =G_1\cup G_2$ where $G_1=(-\infty,x)$ and $G_2=(x,\infty)$.

By the approximation property there is an element of $A$ in $(m,x)$ and an element of $A$ in $(x,M)$, therefore $A\cap G_1\neq\varnothing$ and $A\cap G_2\neq\varnothing$. Clearly $A\cap G_1\cap G_2=\varnothing$. It follows that $A$ is disconnected. The converse follows from exercise 53.
\color{black}

\textbf{T157:} A metric space $(X,d)$ is connected $\Leftrightarrow$ the only subsets of $X$ with empty boundary are $\varnothing$ and $X$.

\textbf{T158:} Let $(X,d_X)$ be a connected metric space, $(Y,d_Y)$ be a metric space and $f:X\to Y$ be a continuous surjection. Then $(Y,d_Y)$ is connected as well.

\color{red}
\textbf{Proof:} Assume for contradiction that $(Y,D_Y)$ is disconnected. Then $Y=G_1\cup G_2$, $G_1,G_2$ are open, non-empty, disjoint. Then $X=f^{-1}(Y)=f^{-1}(G_1)\cup f^{-1}(G_2)$. We have $f^{-1}(G_1)$, $f^{-1}(G_2)$ are open (because $f$ is continuous), non-empty (because $f$ is a surjection) and disjoint, contradicting the hypothesis that $X$ is connected.
\color{black}

\textbf{T159 (Intermediate value theorem):} Let $(X,d)$ be a connected metric space and $f:X\to\mathbb{R}$ be a continuous function. If $x_1,x_2\in X$ with $f(x_1)\neq f(x_2)$ and $y$ is a real number between $f(x_1)$ and $f(x_2)$, then there exists an $x\in X$ such that $f(x)=y$.

\textbf{T160:} A metric space $(X,d)$ is connected $\Leftrightarrow$ the only clopen subsets are $\varnothing$ and $X$.

\color{red}
\textbf{Proof:} If $(X,d)$ is connected and $A\subseteq X$ is clopen, then both $A$ and $A^c$ are open, and $X=A\cup A^c$. If both $A$ and $A^c$ were non-empty, then $X=A\cup A^c$ would be a disconnection of $X$. Therefore at least one of them is empty, therefore $A=\varnothing$ or $X$. Conversely, assume that the only clopen subsets of $X$ are $\varnothing$ and $X$. If $X=G_1\cup G_2$ where $G_1,G_2$ are open and disjoint, then $G_1$ is clopen, therefore $G_1=\varnothing$, or $G_1=X$ in which case $G_2=\varnothing$. Therefore no disconnection of $X$ exists, therefore $X$ is connected.
\color{black}

\textbf{T161:} The equivalence class of any point in $X$ is the largest connected subset of $X$ that contains that point.

\color{red}
\textbf{Consider adding stuff on connected components}
\color{black}

\textbf{D162:} Let $(X,d)$ be a metric space and $x_0,x_1\in X$. A \textbf{path} in $X$ from $x_0$ to $x_1$ is a continuous function $\gamma:[0,1]\to X$ such that $\gamma(0)=x_0$ and $\gamma(1)=x_1$. We say $(X,d)$ is \textbf{path-connected} $\Leftrightarrow$ for any two points $x_0,x_1\in X$ there is a path in $X$ from $x_0$ to $x_1$. A non-empty subset $A$ of $X$ is said to be \textbf{path-connected} $\Leftrightarrow$ the metric space $(A,d_A)$, where $d_A$ is the induced metric, is path-connected.

\textbf{T163:} Every path-connected metric space is connected.

\color{red}
\textbf{Proof:} Let $(X,d)$ be path-connected. Suppose, for contradiction, that $(X,d)$ is disconnected, and let $G_0,G_1$ be open subsets of $X$ such that $X=G_0\cup G_1$, $G_0\cap G_1=\varnothing$, $G_0\neq\varnothing$, $G_1\neq\varnothing$.

Pick $x_0\in G_0$, $x_1\in G_1$. If $\gamma:[0,1]\to X$ is any function such that $\gamma(0)=x_0$ and $\gamma(1)=x_1$, then the pair $G_0,G_1$ is a disconnection of the range of $\gamma$, therefore $\gamma$ can't be continuous. This show that there is no path in $X$ from $x_0$ to $x_1$, contradicting the hypothesis that $(X,d)$ is path-connected.
\color{black}

\textbf{Remark:} The converse of theorem 163 is not true.

\color{red}
\textbf{Proof:} In $\mathbb{R}^2$ with the Euclidean metric, let $A=\{(0,0)\}\cup\Gamma$, where $\Gamma$ is the graph of the function
\begin{equation*}
    f:(0,+\infty)\to\mathbb{R}, \hspace{0.5cm} f(x) = \sin(1/x).
\end{equation*}
There is no path from $(0,0)$ to any point on $\Gamma$, so $A$ is not path-connected.

We claim that $A$ is connected. Assume that $A$ is disconnected. Then there exists open subsets $G_1,G_2$ of $\mathbb{R}^2$ such that
\begin{equation*}
    A\subseteq G_1\cup G_2, \hspace{0.5cm} (A\cap G_1)\cap(A\cap G_2)=\varnothing, \hspace{0.5cm} A\cap G_1\neq\varnothing, \text{ } A\cap G_2\neq\varnothing.
\end{equation*}
$\Gamma$ is connected and $\Gamma\subseteq G_1\cup G_2$, therefore $\Gamma$ is entirely contained in one of $G_1,G_2$. We may assume that $\Gamma\subseteq G_1$. Then $(0,0)\in G_2$. Take a sequence $(x_n)_{n\in\mathbb{N}}$ of elements of $\Gamma$ that converges to $(0,0)$. Eventually are the $x_n$'s are in the open set $G_2$; a contradiction.
\color{black}

\end{document}
