\documentclass[12pt, a4paper]{article}
\usepackage{blindtext, titlesec, amsthm, thmtools, amsmath, amsfonts, scalerel, amssymb, graphicx, titlesec, xcolor, multicol, hyperref}
\usepackage[utf8]{inputenc}
\hypersetup{colorlinks,linkcolor={red!40!black},citecolor={blue!50!black},urlcolor={blue!80!black}}
\newtheorem{theorem}{Theorema}[subsection]
\newtheorem{lemma}[theorem]{Lemma}
\newtheorem{corollary}[theorem]{Corollarium}
\newtheorem{hypothesis}{Coniectura}
\theoremstyle{definition}
\newtheorem{definition}{Definitio}[section]
\theoremstyle{remark}
\newtheorem{remark}{Observatio}[section]
\newtheorem{example}{Exampli Gratia}[section]
\newcommand{\bb}[1]{\mathbb{#1}}
\renewcommand\qedsymbol{Q.E.D.}
\title{Handin 3}
\author{Harry Han}
\date{\today}
\begin{document}

\begin{center}
	METRIC SPACES 2023-24\\
	ASSIGNMENT 3\\
	Due date: Monday of week 7 by 4pm.
\end{center}

\bigskip

\section{Q1}

Let $X=\mathbb{R}^2$, $d$ be the Euclidean metric and define
\[
	G=\left\{(x,y)\in \mathbb{R}^2 : \frac{x^2}{9} + \frac{y^2}{4}<1\right\}.
\]
Prove that the set $G$ is open  in the metric space $(X,d)$.

\begin{proof}
	Let us define a homeomorphism $f: X \rightarrow X$ as $f(x,y) = (3x, 2y)$.

	It is clear the $f$ is bijective, and an inverse , $f^{-1}(x,y) = (\frac{x}{3}, \frac{y}{2})$, exists.
	To see $f$ is continuous, consider the map $g_1(x,y) = (3x, 0), g_2(x,y) = (0,2y)$. Both $g_1$ and $g_2$ are continuous, thus $f = g_1 + g_2 $ also is. Similar argument can be applied to $f^{-1}$.

	It is clear that $f(G) = \{(x,y) \in \bb{R} : x^2 + y^2 < 1 \} = B(0, 1)$, i.e., an open ball of radius one centered at (0,0). That is, $f^{-1}(B(0,1)) = G$. We have proved that all open balls are open sets, and apply open ball characterisation of continuous function, we conclude $G$ is open. Another way to think about it is that homeomorphism perserves open sets.
\end{proof}

\newpage

\section{Q2}

Let $(X,d)$ be a metric space and let $A$ be a non-empty subset of $X$.
Define $f:X\to\mathbb{R}$ by $f(x)=\inf\{d(x,a) : a \in A\}$. ($f(x)$ is the distance from the point $x$ to the set $A$.) Prove that $f$ is continuous.


\begin{proof}

Let us prove a lemma:

\begin{lemma}
For all $x, y \in X$
\begin{equation}\label{trianguleSetDistance}
	|f(x) - f(y)| \leq d(x,y)
\end{equation}
\end{lemma}

To prove this, consider the term $f(x)$ and $d(x,y) + f(y)$. By definition of infimum, for any $\epsilon > 0$, there exist $a \in X$ such that $d(a,y) < f(y) + \epsilon \implies d(a,y) + d(y, x) < d(x, y) + f(y) + \epsilon $. This is equivalent to $d(a,y) + d(y,x) \leq d(x,y) + f(y)$.
By definition of infimum again we have $f(x) \leq d(x ,a) \leq d(y,a ) + d(y,x)$. Combining these two equation we get $f(x) \leq f(y) + d(x,y)$ that is $f(x) - f(y) \leq d(x,y)$. 

If we interchange $x, y$, this equation becomes $f(y) - f(x) \leq d(x,y)$. As $|f(x) - f(y)|$ equals to $f(x) - f(y)$ or $f(y) - f(x)$, we conclude equation \eqref{trianguleSetDistance} is true.

To prove $f$ is continuous, we need to show that for all $\epsilon > 0$ there exist $\delta$ such that $d(x,y) < \delta \implies |f(x) - f(y)| < \epsilon$. Using this lemma it is trivial that $\delta = \epsilon$.
\end{proof}

\end{document}

