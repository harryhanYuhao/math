\documentclass[12pt, a4paper]{article}
\usepackage{blindtext, titlesec, amsthm, thmtools, amsmath,
amsfonts, scalerel, amssymb, graphicx, titlesec, xcolor, multicol, 
hyperref, tikz, bm}

\usepackage[utf8]{inputenc}
\hypersetup{colorlinks,linkcolor={red!40!black},citecolor={blue!50!black},urlcolor={blue!80!black}}
% \newtheorem{theorem}{Theorema}[subsection]
\newtheorem{lemma}[section]{Lemma}
\newtheorem{corollary}[section]{Corollarium}
\newtheorem{hypothesis}{Coniectura}
\newtheorem{note}{N.B.}[section]
% \theoremstyle{definition}
% \newtheorem{definition}{Definitio}[section]
\theoremstyle{remark}
\newtheorem{remark}{Observatio}[section]
\newtheorem{example}{Exampli Gratia}[section]
\newcommand{\bb}[1]{\mathbb{#1}}
\renewcommand\qedsymbol{Q.E.D.}

\usepackage{tcolorbox}
\tcbuselibrary{theorems}
\newtcbtheorem[number within=section]{definition}{Definitio}%
{fonttitle=\bfseries}{th}
\newtcbtheorem[number within=section]{theorem}{Theorema}%
{fonttitle=\bfseries}{th}

\newcommand{\mc}[1]{\mathcal{#1}}
\renewcommand{\emph}[1]{\textit{\textbf{#1}}}


\begin{document}
\textbf{NOTATIONS}
\begin{enumerate}
	\item $\mathcal{F}$ denotes $\bb{C}$ or $\bb{R}$
	\item $V \subseteq V_1, V_2, \cdots W \subseteq W_1, W_2, \cdots $ be vector spaces over $\mathcal{F}$
	\item $\mathcal{R}$ be a ring
\end{enumerate}

\section{Field, Rings, and Vector Space}
\begin{definition}{Ring}{}
	A \textbf{ring}, $\mathcal{R}$, is a set with two operations $+$ and $\cdot$ such that: 
	\begin{enumerate}
		\item $(\mathcal{R}, +)$ is an abelian group
		\item $(\mc{R},\cdot)$ is monoid. (i.e., $\cdot$ is associative and has an identity element) 
		\item $\cdot$ is distributive over $+$, i.e., $a(b+c) = ab + ac$.
	\end{enumerate}
\end{definition}
\begin{note}
	A ring does not needs to be commutative over $\cdot$. 
\end{note}

\begin{definition}{Field}{}
	A field is a non-zero commutative ring such that every non-zero element has a multiplicative inverse.
\end{definition}
\begin{note}
	For field $\mc{F}$, both $+$ and $\cdot$ commute. 
\end{note}

\begin{definition}{Integral Domain}{}
	A Integral Domain, $\mc{R}$, is a non-zero commutative ring such that for $a, b \in \mc{R}$ such that $ab = 0$, then $a = 0$ or $b = 0$.
\end{definition}
\begin{note}
	Every finite integral domain is a field.
\end{note}

\begin{definition}{Ideal}{}
	Let $\mc{R}$ be a ring. A subset $I \subseteq \mc{R}$ is an ideal if:
	\begin{enumerate}
		\item $I$ is a subgroup of $(\mc{R}, +)$
		\item For all $r \in \mc{R}$ and $i \in I$, $ri \in I$.
	\end{enumerate}

	$I$ is a \emph{principal ideal} if $I = \langle t \rangle$ for some $t \in \mc{R}$. That is, $I$ is generated by $t$.
\end{definition}
\begin{note}
	Subring does not need to be an ideal, nor does an ideal needs to be a subring.
\end{note}

\begin{definition}{Ring Homomorphism}{}
	Let $\mc{R}$ and $\mc{S}$ be rings. A function $\phi: \mc{R} \to \mc{S}$ is a ring homomorphism if:
	\begin{enumerate}
		\item $\phi(a + b) = \phi(a) + \phi(b)$
		\item $\phi(ab) = \phi(a)\phi(b)$
	\end{enumerate}
\end{definition}
\note{
	It is \emph{NOT} required that $\phi(1) = 1$. Yet $\phi(0) = 0$ . Ring homomorphism preserves subring and subgroup.
}

\begin{definition}{Left Modules}{}
	A left Module $\mc{M}$ over a ring $\mc{R}$ is an abelian group $(\mc{M}, +)$ with a scalar multiplication $\mc{R} \times \mc{M} \to \mc{M}$ such that for $a,b \in \mc{M}$ and $r, s \in \mc{R}$
	\begin{enumerate}
		\item $r(a+b) = ra + rb$
		\item $(s+r)a = sa + ra$
		\item $(rs)a = r(sa)$
		\item $1_Ra = a$
	\end{enumerate}
	\emph{Sub Module} is defined similarly as vector subspace: that is, a closed subgroup pf $\mc{M}$ under multiplication.
\end{definition}

\begin{definition}{R-homomorphism}{}
	Let $\mc{M}$ and $\mc{N}$ be left $\mc{R}$-modules. A function $\phi: \mc{M} \to \mc{N}$ is an $\mc{R}$-homomorphism if:
	\begin{enumerate}
		\item $\phi(a+b) = \phi(a) + \phi(b)$
		\item $\phi(ra) = r\phi(a)$
	\end{enumerate}
\end{definition}

\section{Inner Product}
\begin{definition}{Inner Product Space}{}
	An inner product space is a vector space $V$ over $\mc{F}$ with an inner product $\langle \cdot, \cdot \rangle: V \times V \to \mc{F}$ such that for $u, v, w \in V$ and $a \in \mc{F}$:
	\begin{enumerate}
		\item $\langle u, v \rangle = \overline{\langle v, u \rangle}$
		\item $\langle u, \lambda v + \mu w \rangle = \lambda \langle u, v \rangle + \mu \langle u, w \rangle$
		\item $\langle u, u \rangle \geq 0$ and $\langle u, u \rangle = 0$ if and only if $u = 0$.
	\end{enumerate}

	The standard complex inner product is: $\langle u, v \rangle = \sum_{i=1}^n u_i \overline{v_i}$.
\end{definition}

\begin{definition}{Adjoint}{}
	Operator $V, S$ are adjoint if $\langle u, Sv \rangle = \langle Su, v \rangle$. Denoted $V$ as $S^*$.
	In particular, if the vector space is $\bb{C}$, then $S^* = \overline{S^T}$.
\end{definition}

\begin{definition}{Hermitian}{}
	An complex operator $S$ is Hermitian if $S = S^*$.
\end{definition}

\begin{definition}{Unitary}{}
	An complex operator $S$ is unitary if $S^*S = SS^* = I$.
\end{definition}

\section{Jordan Normal Form}

\begin{definition}{Jordan Normal Form}{}
	Let $\mc{F}$ be an algebraically closed field. Let $V$ be a finite dimensional vector space over $\mc{F}$, and $\phi$ be endomorphism with characteristic polynomial $\chi = (x - \lambda_1)^{a_1}\cdot (x- \lambda_2)^{a_2} \cdots$. 
	Then there exist certain basis $\mc{B}$ such that $\mc{B}[\phi]\mc{B} = diag(J(r_{11}, \lambda_1), \cdots, J(r_{1m_1}, \lambda_1) \cdots)$
\end{definition}

\note{
	\emph{To Find Jordan Normal Form}
	For endomorphism $\phi$ of dimension $n$, with characteristic polynomial $\chi = (x - \lambda_1)^{a_1}\cdot (x- \lambda_2)^{a_2} \cdots$:
	\begin{enumerate}
		\item Find the generalized eigenspace for each eigenvalue $\lambda_i$. Denote it as $W$.

			If $(x - \lambda)^a$ is a term in $\chi$, then the generalized eigenspace is $\ker(\phi - \lambda I)^n$.
		\item For each $\lambda_i$, define morphism $\psi = (\phi- \lambda_i I)$. 

			Define $W_i = \ker{\psi ^i}$. We have $0 = W_0 \subseteq W_1 \subseteq \cdots W_n = W$. 
			Define homomorphism $\psi_i: \frac{W_i}{W_{i-1}} \rightarrow \frac{W_{i-1}}{W_{i-2}}$ by $\psi_i (w + W_{i-1}) = \psi(w) + W_{i-2}$. 

			This morphism is injective. 

			Let us apply this algorithm: 
			Start with $W_n$. Pick basis for $\frac{W_n}{W_{n-1}}$ $v_{n, 1}, v_{n,2} \cdots$. Define $\psi(v_{n,a}) = v_{n-1, a}$.
			$v_{n-1, 1} \cdots$ will be part (or possible whole) basis of $\frac{W_{n-1}}{W_{n-2}}$. Extend it to be the full basis $v_{n-1,1}, v_{n-1,2} \cdots$.

			Continue this process until you reach $W_1$. Then the obtained $v_{n,1},\cdots v_{n-1, 1}, \cdots v_{1,1}$ are basis for $W$.
			\textbf{Order} them by ascend by the last term. And if last term is same, order by ascend by the first term. So, $v_{n,1}, v_{n-1,2} \cdots v_{n,2}$. 

			Repeat this process for each $\lambda_i$.
	\end{enumerate}
}

\end{document}

