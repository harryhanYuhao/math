\documentclass[12pt, a4paper]{article}
\usepackage{blindtext, titlesec, amsthm, thmtools, amsmath, amsfonts, scalerel, amssymb, graphicx, titlesec, xcolor, multicol, hyperref}
\usepackage[utf8]{inputenc}
\hypersetup{colorlinks,linkcolor={red!40!black},citecolor={blue!50!black},urlcolor={blue!80!black}}
\newtheorem{theorem}{Theorema}[subsection]
\newtheorem{lemma}[theorem]{Lemma}
\newtheorem{corollary}[theorem]{Corollarium}
\newtheorem{hypothesis}{Coniectura}
\theoremstyle{definition}
\newtheorem{definition}{Definitio}[section]
\theoremstyle{remark}
\newtheorem{remark}{Observatio}[section]
\newtheorem{example}{Exampli Gratia}[section]
\newcommand{\bb}[1]{\mathbb{#1}}
\renewcommand\qedsymbol{Q.E.D.}
\title{}
\author{}
\date{}

\begin{document}
\maketitle
%\tableofcontents

Let us define the $n\times n$ matrix $A_n$ thus:

Fix arbitrary real numbers $x_1,\ldots,x_n$ which are pairwise distinct, i.e. so that $x_i\neq x_j$ for any pair $i\neq j$.  Let $A_n=(a_{ij})$ be the following $n \times n$ matrix:  its diagonal entries are given by the equation,

$$a_{ii} = \sum_{j\neq i}\frac{x_i}{x_i-x_j},$$

while its off-diagonal entries given by the equation,

$$a_{ij} = \frac{x_i}{x_i-x_j},$$

I claim that the matrix $A_n$ has \textbf{eigenvalues} $0,1,2, \cdots, n-1$.

Let us first define some notation.
For any terms $x_1,x_2,\cdots, x_n$ Define $\chi = \{x_1,x_2, \cdots, x_n\}$, $\Delta^p_{S} = \sum_{i_1, i_2, \cdots, i_p \in S}i_1\cdot i_2\cdot i_3\cdots i_p$.

For example:
$$\Delta^0_{\chi} = \sum_{i=1}^{n} 1 = |\chi|$$
$$\Delta^1_{\chi} = \sum_{i=1}^{n} x_i$$
$$\Delta^2_{\chi} = \sum_{1 \leq i<j \leq n} x_ix_j = x_1x_2 + x_1x_3 + \cdots x_1x_n + x_2x_3 \cdots$$

The left-eigenvector, $\vec{v}_0$ corresponding to the eigenvalue $0$ is
$$
	\vec{v}_0 =\left[\dfrac{x_1}{x_1}, \dfrac{x_1}{x_2}, \cdots ,\dfrac{x_1}{x_n}\right]
$$

The left-eigenvector, $\vec{v}_k$ corresponding to the eigenvalue $k$, with $1 \leq k \leq n-1$ is
$$
	\vec{v}_k =
	\left[
		\dfrac{\Delta^{n-1-k}_{\chi \smallsetminus x_1}}{\Delta^{n-1-k}_{\chi \smallsetminus x_1}},
		\dfrac{\Delta^{n-1-k}_{\chi \smallsetminus x_2}}{\Delta^{n-1-k}_{\chi \smallsetminus x_1}},
		\cdots ,
		\dfrac{\Delta^{n-1-k}_{\chi \smallsetminus x_n}}{\Delta^{n-1-k}_{\chi \smallsetminus x_1}},
		\right]
$$

\section{Problem 5}

\subsection{ Proof}

\subsubsection{v0}

To prove that
$$
	\vec{v}_0 =\left[\dfrac{x_1}{x_1}, \dfrac{x_1}{x_2}, \cdots ,\dfrac{x_1}{x_n}\right]
$$

is easy.

We simply compute with bruteforce:

$$
	\vec{v}_0 A_7 = [a_{11} + \sum_{1\neq j}{x_1}{x_j-x_1}, a_{22} + \sum_{2\neq j}{x_2}{x_j-x_2}, \cdot]
$$
I.e, the ith term is $a_{ii} +  \sum_{i\neq j}\frac{x_i}{x_j-x_i}$.
Notice $a_{ii} = \sum_{i\neq j} \frac{x_i}{-x_j+x_i}$, and we conclude $\vec{v}_0 A_7 = 0$.

\subsubsection{ v(n-1)}

$$
	\vec{v}_1 A_7 = [a_{ii} + \sum_{i\neq j}\frac{x_j}{x_j-x_i}]
$$

Notice
$$
	a_{ii} + \sum_{i\neq j}\frac{x_j}{x_j-x_i} = \sum_{i\neq j}\frac{x_j}{x_j-x_i} + \sum_{i\neq j}\frac{x_i}{-x_j+x_i} = \sum_{i\neq j}\frac{x_j-x_i}{x_j-x_i} = n-1
$$

$\vec{v}_1 A_7 = [n-1, n-1, \cdots] = (n-1) v_1$.

\subsubsection{ The rest}

Let us first notice 2 identity:

$$
	\frac{1}{x_i - x_j}(x_i \Delta^{n}_{\chi \smallsetminus x_i} - x_j \Delta^{n}_{\chi \smallsetminus x_j}) = \Delta^{n}_{\chi \smallsetminus \{x_i, x_j\}}  (1)
$$

$$
	\text{(Iterate through j)}\sum_{j \neq i} \Delta^{k}_{\chi \smallsetminus \{x_i, x_j\}} = (n-k-1) \Delta^{k}_{\chi \smallsetminus \{x_i\}} (2)
$$

They two can be proved by simply calculation.

Recall
$$
	\vec{v}_k =
	\left[
		\dfrac{\Delta^{n-1-k}_{\chi \smallsetminus x_1}}{\Delta^{n-1-k}_{\chi \smallsetminus x_1}},
		\dfrac{\Delta^{n-1-k}_{\chi \smallsetminus x_2}}{\Delta^{n-1-k}_{\chi \smallsetminus x_1}},
		\cdots ,
		\dfrac{\Delta^{n-1-k}_{\chi \smallsetminus x_n}}{\Delta^{n-1-k}_{\chi \smallsetminus x_1}},
		\right]
$$

Using these two identities, let us do some calculations for: $\vec{v}_k A_k = \vec{u}$

The first term of $u$ divided by $\frac{1}{\Delta^{n-1-k}_{\chi \smallsetminus x_1}}$is:

$$
	\frac{x_1}{x_1-x_2} \Delta^{n-1-k}_{\chi \smallsetminus x_1} -
	\frac{x_2}{x_1-x_2} \Delta^{n-1-k}_{\chi \smallsetminus x_2} +
	\frac{x_1}{x_1-x_3} \Delta^{n-1-k}_{\chi \smallsetminus x_1} -
	\frac{x_3}{x_1-x_3} \Delta^{n-1-k}_{\chi \smallsetminus x_3} + \cdots
$$

Which is equal to

$$
	\frac{1}{x_1-x_2} (x_1 \Delta^{n-1-k}_{\chi \smallsetminus x_1} - x_2 \Delta^{n-1-k}_{\chi \smallsetminus x_2}) +
	\frac{1}{x_1-x_3} (x_1 \Delta^{n-1-k}_{\chi \smallsetminus x_1} - x_3 \Delta^{n-1-k}_{\chi \smallsetminus x_3}) + \cdots
$$

Which is equal to

$$
	\Delta^{n-1-k}_{\chi \smallsetminus x_1, x_2} +
	\Delta^{n-1-k}_{\chi \smallsetminus x_1, x_3} + \cdots = k  \Delta^{n-1-k}_{\chi \smallsetminus x_1}
$$

Repeat this calculation we get

$$\vec{u} = k \vec{v}_k$$





\end{document}
