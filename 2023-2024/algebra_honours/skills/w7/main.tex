\documentclass[12pt, a4paper]{article}
\usepackage{blindtext, titlesec, amsthm, thmtools, amsmath, amsfonts, scalerel, amssymb, graphicx, titlesec, xcolor, multicol, hyperref}
\usepackage{bm}
\usepackage[utf8]{inputenc}
\hypersetup{colorlinks,linkcolor={red!40!black},citecolor={blue!50!black},urlcolor={blue!80!black}}
\newtheorem{theorem}{Theorema}[subsection]
\newtheorem{lemma}[theorem]{Lemma}
\newtheorem{corollary}[theorem]{Corollarium}
\newtheorem{hypothesis}{Coniectura}
\theoremstyle{definition}
\newtheorem{definition}{Definitio}[section]
\theoremstyle{remark}
\newtheorem{remark}{Observatio}[section]
\newtheorem{example}{Exampli Gratia}[section]
\newcommand{\bb}[1]{\mathbb{#1}}
\renewcommand\qedsymbol{Q.E.D.}
\begin{document}
%\tableofcontents


\section{Q2}

I claim that the dimension of the centrailizer of $J_n(\lambda)$ is $n$, more over, it has the form of

$$
	\left[
		\begin{array}{cccccc}
			a_1    & a_2    & a_3    & \cdots & a_n     \\
			0      & a_1    & a_3    & \cdots & a_{n-1} \\
			0      & 0      & a_1    & \cdots & a_{n-2} \\
			\vdots & \vdots & \vdots & \ddots & \vdots  \\
			0      & 0      & 0      & \cdots & a_1
		\end{array}
		\right]
$$

\begin{proof}
	We shall prove by induction.

	For the case $n=1$, $J_1(\lambda)$ is just real number, which commutes with any other real, i.e., its centrailizer has dimension 1.

	Assuming our conjecture is true for $J_{n-1}$, let us now consider the case for $J_n$.
	Notice that, for matrix $B \in \bb{R}^{1\times(n-1)}, D \in \bb{R}^{(n-1)\times(n-1)}$, we have

	$$
		J_n(\lambda)=
		\left[
			\begin{array}{c|c}
				\lambda & B \\
				\hline
				0       & D
			\end{array}
			\right]
	$$

	We can write our candidate for the centrailizer, $Z$, in this form, with $g \in \bb{R}$, $L \in \bb{R}^{1\times(n-1)}$, $M \in \bb{R}^{(n-1)\times1}$, $N \in \bb{R}^{(n-1)\times(n-1)}$.
	Notice $D = J_{n-1}(\lambda)$ and matrix $B = [1, 0, \cdots, 0]$.

	$$
		Z=
		\left[
			\begin{array}{c|c}
				g & L \\
				\hline
				M & N
			\end{array}
			\right]
	$$

	We have $ZJ_n(\lambda)=J_n(\lambda)Z$ if and only if

	\begin{equation}\label{Q2:1}
		\left[
			\begin{array}{c|c}
				g \lambda & gB+LD \\
				\hline
				M \lambda & MB+ND
			\end{array}
			\right]
		=
		\left[
			\begin{array}{c|c}
				\lambda g  + BM & \lambda L + BN \\
				\hline
				DM              & DN
			\end{array}
			\right]
	\end{equation}

	Let me show you that $M=0$.
	Since $M \lambda = DM$, write $M$ in the form of $[m_1, m_2, \cdots, m_{n-1}]^T$ this equation becomes
	(Recall the fact that $D = J_{n-1}(\lambda)$):
	$$[\lambda m_1, \lambda m_2, \cdots, \lambda m_{n-1}]^T = [\lambda m_1 + m_2, \lambda m_2 + m_3 \cdots \lambda m_{n-1}]^T$$
	which means $m_2, \cdots m_{n-1}$ are zero.
	Morever, notice $B = [1, 0, 0, \cdots ]$, we have $BM = m_1 \implies g\lambda  = \lambda g + m_1 $ (recall the first term of equation \ref{Q2:1}) .This means $ m_1 = 0$.

	Now equation 1 can be cleaned up to

	\begin{equation}
		\left[
			\begin{array}{c|c}
				g \lambda & gB+LD \\
				\hline
				0         & ND
			\end{array}
			\right]
		=
		\left[
			\begin{array}{c|c}
				\lambda g & \lambda L + BN \\
				\hline
				0         & DN
			\end{array}
			\right]
	\end{equation}
\end{proof}

This means that $N$ must be the centrailizer of $D$, which equals to $J_{n-1}(\lambda)$. Let us write $N$ in this form:

\begin{equation}
	N=
	\left[
		\begin{array}{ccccc}
			r_1    & r_2    & r_3    & \cdots & r_{n-1} \\
			0      & r_1    & r_3    & \cdots & r_{n-2} \\
			0      & 0      & r_1    & \cdots & r_{n-3} \\
			\vdots & \vdots & \vdots & \ddots & \vdots  \\
			0      & 0      & 0      & \cdots & r_1
		\end{array}
		\right]
\end{equation}

We further have the constrains that $gB + LD = L \lambda +BN$. Let us expand them, and write $L = [l_1, l_2, \cdots, l_{n-1}]^T$
We get:

\begin{equation}
	\left[
		\begin{array}{c}
			g + \lambda l_1   \\
			l_1 + \lambda l_2 \\
			\cdots            \\
			l_{n-2} + \lambda l_{n-1}
		\end{array}
		\right]
	=
	\begin{array} {c}
		\left[
			\begin{array}{c}
				l_1 \lambda + r_1 \\
				l_2 \lambda + r_2 \\
				\cdots            \\
				l_{n-1} \lambda + r_{n-1}
			\end{array}
			\right]
	\end{array}
\end{equation}

This means $g = r_1, l_1 = r_2, l_2 = r_3, \cdots l_{n-2} = r_{n-1}$, leaving no constrain for $l_{n-1}$, which we can set it to be $r_n$.


Thus, the centrailizer of $J_n(\lambda)$ has the form of

$$
	\left[
		\begin{array}{cccccc}
			r_1    & r_2    & r_3    & \cdots & r_{n}   \\
			0      & r_1    & r_2    & \cdots & r_{n-1} \\
			0      & 0      & r_1    & \cdots & r_{n-2} \\
			\vdots & \vdots & \vdots & \ddots & \vdots  \\
			0      & 0      & 0      & \cdots & r_1
		\end{array}
		\right]
$$

This is of dimension $n$ as desired.

Before moving into Q4, let us prove two lemma:

\begin{lemma}\label{lemma:1}
	For $J_n(\lambda_1) $ and $J_m(\lambda_2)$, there exist no non-zero matrix $Z$ such that $ZJ_n(\lambda_1) = J_m(\lambda_2)Z$.
\end{lemma}




\begin{proof}
	Assume such a matrix$Z$, exists. Let us define vector $v = [1, 0, \cdots, 0]^T$, then we have $ZJ_n(\lambda_1)v = J_m(\lambda_2)Zv$. Let the first column of $Z$ be denoted as $[z_1,z_2, z_3, \cdots, z_m ]^T$.
	By straight forward computation we get:
	\begin{equation}
		\left[
			\begin{array}{c}
				z_1 \lambda_1   \\
				z_2 \lambda_1   \\
				\vdots          \\
				z_{m} \lambda_1 \\
			\end{array}
			\right]
		=
		\left[
			\begin{array}{c}
				\lambda_2 z_1 + z_2 \\
				\lambda_2 z_2 + z_3 \\
				\vdots              \\
				\lambda_2 z_{m}
			\end{array}
			\right]
	\end{equation}

	This is clear that $z_1 = z_2 \cdots = 0$. The same argument can be applied to any other colomn of $Z$, which means $Z$ must be zero.
\end{proof}

\begin{lemma}\label{CentrailizerForNonSquare}
	For $J_n(\lambda), J_m(\lambda)$, there exist $Z_1 \in R^{n\times m}$ and $Z_2 \in R^{m\times n}$ such that $J_n(\lambda) Z_1 = Z_1 J_m(\lambda)$, and $Z_2 J_n(\lambda) = J_m(\lambda) Z_2$.

	Moreover, $Z_1$ and $Z_2$ are of the form of (assuming $n \leq m$):
	$$
		Z_1 = \left[
			\begin{array}{c}
				C \\
				0
			\end{array}
			\right], Z_2=
		\left[
			\begin{array}{c|c}
				C & 0
			\end{array}
			\right]
	$$
	Let us call $Z_1, Z_2$ \textbf{pseudo-centrailizer} of $J_n, J_m$.
	For the sake of simplicity, let pseudo-centrailizer of Jordan form of two different eigenvalue be $0$, and the pseudo-centrailizer of Jordan form of the same eigenvalue be the actual centrailizer. 
\end{lemma}

This can be proved similarly as Q3.


\section{Q4}

For matrix $$A = J_{n_1}(\lambda_1)\oplus \cdots \oplus J_{n_k}(\lambda_k),$$, with $\lambda_i$ be distinct.

I claim its centrailizer is of the form of

\begin{equation}
	\left[
		\begin{array}{cccc}
			Z_1    & 0      & \cdots & 0      \\
			0      & Z_2    & \cdots & 0      \\
			\vdots & \vdots & \ddots & \vdots \\
			0      & 0      & \cdots & Z_k
		\end{array}
		\right]
\end{equation}

Where $Z_i$ is the centrailizer of $J_{n_i}(\lambda_i)$. It has dimension of $n_1 + n_2 + \cdots + n_k$.

\begin{proof}
	This can be proved by striaght forward calculation.
	Let $Z$ be our candidate of the centrailizer of $A$.
	Let us write $Z$ in this form for submatrices of an apropriate size:
	$$
		Z =
		\left[
			\begin{array}{cccc}
				Z_{1,1} & Z_{1,2} & \cdots & Z_{1,k} \\
				Z_{2,1} & Z_{2,2} & \cdots & Z_{2,k} \\
				\vdots  & \vdots  & \ddots & \vdots  \\
				Z_{k,1} & Z_{k,2} & \cdots & Z_{k,k}
			\end{array}
			\right]
	$$

	Note $A$ can be written as:
	\[
		A =
		\left[
			\begin{array}{cccc}
				J_{n_1}(\lambda_1) & 0                  & \cdots & 0                  \\
				0                  & J_{n_2}(\lambda_2) & \cdots & 0                  \\
				\vdots             & \vdots             & \ddots & \vdots             \\
				0                  & 0                  & \cdots & J_{n_k}(\lambda_k)
			\end{array}
			\right]
	\]

	Carry out the computation $AZ = ZA$ we have:

	\begin{align}
		AZ   & =
		\left[
			\begin{array}{cccc}
				J_{n_1}(\lambda_1)Z_{1,1} & J_{n_1}(\lambda_1)Z_{1,2} & \cdots & J_{n_1}(\lambda_1)Z_{1,k} \\
				J_{n_2}(\lambda_2)Z_{2,1} & J_{n_2}(\lambda_2)Z_{2,2} & \cdots & J_{n_2}(\lambda_2)Z_{2,k} \\
				\vdots                    & \vdots                    & \ddots & \vdots                    \\
				J_{n_k}(\lambda_k)Z_{k,1} & J_{n_k}(\lambda_k)Z_{k,2} & \cdots & J_{n_k}(\lambda_k)Z_{k,k}
			\end{array}
		\right]  \\
		= ZA & =
		\left[
			\begin{array}{cccc}
				Z_{1,1}J_{n_1}(\lambda_1) & Z_{1,2}J_{n_2}(\lambda_2) & \cdots & Z_{1,k}J_{n_k}(\lambda_k) \\
				Z_{2,1}J_{n_1}(\lambda_1) & Z_{2,2}J_{n_2}(\lambda_2) & \cdots & Z_{2,k}J_{n_k}(\lambda_k) \\
				\vdots                    & \vdots                    & \ddots & \vdots                    \\
				Z_{k,1}J_{n_1}(\lambda_1) & Z_{k,2}J_{n_2}(\lambda_2) & \cdots & Z_{k,k}J_{n_k}(\lambda_k)
			\end{array}
			\right]
	\end{align}
	Applying \ref{lemma:1} to each block, we have $Z_{i,j} = 0$ for $i \neq j$. Moreover, $Z_{i,i}$ is the centrailizer of $J_{n_i}(\lambda_i)$.
\end{proof}

\section{Q5}

Now let us consider $$A = J_{n_1}(\lambda_1)\oplus \cdots \oplus J_{n_k}(\lambda_k),$$

Where $\lambda_i$ are not necessarily distinct.

Assuming there are $m$ distinct $\lambda_i$'s, we can write $A$, and each $\lambda_i$ corrosponds to $k_i$ Jordan forms of size $s_{i,1} \leq \cdots \leq s_{i,k_i}$,
I claim that the centrailizer has the dimension

$$ \sum_{a=1}^{m} \sum_{b=1}^{k_a} (1+2 \cdot (b-1)) s_{a,b} $$

Moreover, the centrailizer is of the form of
\[
	Z =
	\left[
		\begin{array}{cccc}
			Z_{1,1} & Z_{1,2} & \cdots & Z_{1,k} \\
			Z_{2,1} & Z_{2,2} & \cdots & Z_{2,k} \\
			\vdots  & \vdots  & \ddots & \vdots  \\
			Z_{k,1} & Z_{k,2} & \cdots & Z_{k,k}
		\end{array}
		\right]  \\
\]

Where $Z_{i,j}$ is the psucentrailizer (defined in \ref{CentrailizerForNonSquare} of $J_{n_i}(\lambda_i)$ and $J_{n_j}(\lambda_j)$.

The proof is exactly the same as problem 4. We are behooved to use bruteforc, with lemma \ref{CentrailizerForNonSquare}.

\end{document}
