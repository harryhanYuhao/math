\documentclass[12pt, a4paper]{article}
\usepackage{blindtext, titlesec, amsthm, thmtools, amsmath, amsfonts, scalerel, amssymb, graphicx, titlesec, xcolor, multicol, hyperref}
\usepackage[utf8]{inputenc}
\hypersetup{colorlinks,linkcolor={red!40!black},citecolor={blue!50!black},urlcolor={blue!80!black}}
\newtheorem{theorem}{Theorema}[subsection]
\newtheorem{lemma}[theorem]{Lemma}
\newtheorem{corollary}[theorem]{Corollarium}
\newtheorem{hypothesis}{Coniectura}
\theoremstyle{definition}
\newtheorem{definition}{Definitio}[section]
\theoremstyle{remark}
\newtheorem{remark}{Observatio}[section]
\newtheorem{example}{Exampli Gratia}[section]
\newcommand{\bb}[1]{\mathbb{#1}}
\renewcommand\qedsymbol{Q.E.D.}
\begin{document}
Let $R = \mathbb{F}_3[X]$.

\section{Q1}

I claim that there are 8 monic polynomial of degree 3 that has no root.
I can think of no better way then to list them all:
\begin{enumerate}
	\item $X^3+2X^2+1$
	\item $X^3 + 2X^2+ X+1$
	\item $X^3+2X+1$
	\item $X^3+2X^2+2X+1$
	\item $X^3+X^2+X+1$
	\item $X^3+X^2+2$
	\item $X^3+2X+2$
	\item $X^3+2X^2+2X+2$
\end{enumerate}

\section{Q2}

Let us pick $P = X^3+2X+1$. Let $I = \langle P \rangle$ be the ideal generated by it.

The quotient ring $R/I$ can be considered the set of all polynomials of degree less than 3. There are total of 27 such polynomials.

\section{Q3}

I claim that $R/I$ is a field.

\begin{proof}
	It is clear that $R/I$ is commutative, as all polynomials over a field is commutative.

	Let us prove it is a field by showing that there is no zero divisor in $R/I$.
	
	Assuming $f,g \neq 0$ are polynomials of degree less then 3, and $fg = X^3+2X+1 = 0$.  
	I claim this is impossible. As $X^3+2X+1$ is of degree 3, one of $f$ or $g$ must be degree 1. 
	Every degree 1 polynomial would have a root, this root would also be the root of $X^3+2X+1$, which has no root.

	Thus it is a contradiction and I claim that $R/I$ is an finite integral domain.
	Applying thoerem 3.2.17, I claim $R/I$ is a field.
\end{proof}

\section{Q4}

By Lagrange theorem the order of $X$ is a divisor of 26.

Let us try them by turn.

\begin{enumerate}
	\item $X^2 \neq 0$
	\item $X^3 = X^3 - X^3 -2X - 1 = X + 2 \neq 0$
	\item $X^6 = (X^3)^2 = X^2+4X+4 = X^2+X+1 $ (For computation's sake)
	\item $X^{12} = X^4 + X^2 + 1 + 2X^3+2X^2+2X = X^2+2$
	\item $X^{13} = X^3 + 2X = 2$
	\item $X^{26} = 1$
\end{enumerate}

Thus we conclude the order of $X$ is 26.

\end{document}

