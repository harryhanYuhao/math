\documentclass{beamer}
\usepackage[utf8]{inputenc}

\usetheme{Madrid}
\usecolortheme{default}

%------------------------------------------------------------
%This block of code defines the information to appear in the
%Title page
\title[Countable and Non-Countable Set] %optional
{Countable and Non-Countable Sets}

% \subtitle{A short story}

\author[H.~Han]
{H.~Han}

% \institute[VFU] % (optional)
% {
%   \inst{1}%
%   Faculty of Physics\\
%   Very Famous University
%   \and
%   \inst{2}%
%   Faculty of Chemistry\\
%   Very Famous University
% }

% \date[VLC 2021] % (optional)
% {Very Large Conference, April 2021}


%End of title page configuration block
%------------------------------------------------------------



%------------------------------------------------------------
%The next block of commands puts the table of contents at the 
%beginning of each section and highlights the current section:

% \AtBeginSection[]
% {
%   \begin{frame}
%     \frametitle{Table of Contents}
%     \tableofcontents[currentsection]
%   \end{frame}
% }
%------------------------------------------------------------


\begin{document}

%The next statement creates the title page.
\frame{\titlepage}


%---------------------------------------------------------
%This block of code is for the table of contents after
%the title page
% \begin{frame}
% \frametitle{Table of Contents}
% \tableofcontents
% \end{frame}
%---------------------------------------------------------


\section{First section}

%---------------------------------------------------------
%Changing visivility of the text
% \begin{frame}
% \frametitle{Sample frame title}
% This is a text in second frame. For the sake of showing an example.
%
% \begin{itemize}
%     \item<1-> Text visible on slide 1
%     \item<2-> Text visible on slide 2
%     \item<3> Text visible on slides 3
%     \item<4-> Text visible on slide 4
% \end{itemize}
% \end{frame}

%---------------------------------------------------------


%---------------------------------------------------------
%Example of the \pause command
% \begin{frame}
% In this slide \pause
%
% the text will be partially visible \pause
%
% And finally everything will be there
% \end{frame}
%---------------------------------------------------------

% \section{Second section}

%---------------------------------------------------------
\begin{frame}
\frametitle{Countability}

\begin{block}{Countability}
A set $S$ is countable if there exists a bijection $f: S \rightarrow \mathbb{N}$.
\end{block}

\begin{itemize}
    \item $\mathbb{N}$ is countable.
    \item $\mathbb{Z}$ is countable.
    \item $\mathbb{Q}$ is countable.
    \item $\mathbb{R}$ is uncountable.
\end{itemize}

In particular, a countable set must contain infinitely many elements.



% \begin{alertblock}{Important theorem}
% Sample text in red box
% \end{alertblock}
%
% \begin{examples}
% Sample text in green box. The title of the block is ``Examples".
% \end{examples}
\end{frame}

% \begin{frame}
% \frametitle{Countability}
%
% No finite set is countable.
%
% A countable set is infinite.
%
% Countable set is the ``Smallest'' infinite set. 
%
% \begin{block}{Infinte Set (My definition)}
%   A set $S$ is infinite, if there exists a surjection $f: S \rightarrow \mathbb{N}$.
% \end{block}
%
% It can be proved that for any infinite set, there exists a bijection $f: S \rightarrow S'$ where $S' \subset S$.
% \end{frame}

\begin{frame}
\frametitle{Countability}

Let $C_i$ be countable sets.

\begin{itemize}
  \item $C_0 \cup C_1$ is countable.
  \item By induction, $\bigcup_{i=0}^n C_i$ is countable, for finite $n$.
  \item $C_0 \times C_1$ is countable.
  \item By induction, $X_{i\in 1, \cdots , n} C_i = C_0\times C_1 \times \cdots \times C_n$ is countable, for finite $n$.
  \item $L=\bigcup_{i\in \mathbb{N}} C_i$ is countable: an surjection from $N\times N $ to $L$ 
  \item $G=X_{i\in \mathbb{N}} C_i$ is \alert{NOT} countable!
\end{itemize}
\end{frame}

\begin{frame}
\frametitle{Countability}
\begin{columns}
\column{0.5\textwidth}
Consider real number from 0 to 1, $[0,1]$.\\
$0.000000000000 \cdots$\\
$\cdots$\\
$0.100200000000 \cdots $\\ 
$\cdots$\\
$0.123456789101 \cdots$\\
$\cdots$\\

\column{0.5\textwidth}

Injection from $[0,1]$ to $G$: $G$ is not countable.

\end{columns}
\end{frame}

%---------------------------------------------------------
%Two columns
% \begin{frame}
% \frametitle{Two-column slide}
%
% \begin{columns}
%
% \column{0.5\textwidth}
% This is a text in first column.
% $$E=mc^2$$
% \begin{itemize}
% \item First item
% \item Second item
% \end{itemize}
%
% \column{0.5\textwidth}
% This text will be in the second column
% and on a second tought this is a nice looking
% layout in some cases.
% \end{columns}
% \end{frame}
%---------------------------------------------------------


\end{document}
