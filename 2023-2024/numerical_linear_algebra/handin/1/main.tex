\documentclass{article}
\usepackage{graphicx} % Required for inserting images
\usepackage{algpseudocode}
\usepackage{bm}

\title{NLA\_Assignment1}
\author{Harry Han}
\date{September 2023}

\begin{document}

\section{Question 1}
\subsection*{i}

With vectors $\bm{x}, \bm{y} \in R^n$, and matrix $A\in R^{n*n}$, such that $\bm{x}\bm{y}^T = A$ consider the following algorithm:

\begin{algorithmic}[1] 
\ForAll{$i \in \{1, 2, \cdots, n\}$} 
    \ForAll{$j \in \{1, 2, \cdots, n\}$} 
        \State $A_{i, j} = \bm{x}_i\bm{y}_j  $
    \EndFor
\EndFor
\end{algorithmic}

i.e, this algorithm loops over each entry of $A$ and compute its value by multiplying the corresponding values of $\bm{x}$ and $\bm{y}$. 

Each multiplication requires 1 FLOP, which is rpeated $n^2$ times, thus we conclude the complexity of this algorithme is $C(n) = n^2$.

\subsection*{ii}

We are to demonstrate that it is possible to compute 
\begin{equation}
\label{ii}
    (I-\bm{u}\bm{u}^T)\bm{x}
\end{equation}
with $O(n)$ complexity, given vectors $\bm{u}, \bm{x} \in R^n$.

Note operation \ref{ii} is equivalent to $\bm{x}-\bm{u}(\bm{u}^T\bm{x})$, which can be broken down into three steps:

\begin{enumerate}
    \item $\bm{u}^T\bm{x}=c$, for some constant c; the complexity is equivalent to that of dot product, thus $O(n)$
    \item $\bm{u}c = \bm{u}'$; this is a simple scaler multiplication, with complexity $O(n)$
    \item $\bm{x} + \bm{u}'$: vector addition: whose complexity is $O(n)$
\end{enumerate}

As the operation can be broken down to three steps each with complexity $O(n)$, the overall efficiency of the algorithm is $O(n)$.

\newpage 

\section{Question 2}

We are to prove that for all matrices $A \in R^{n*n}$, $||A||_\infty = ||A^T||_1 $

We know that $||A||_\infty$ is the maximum of the row sum, and $||A||_1$ is the maximum column sum. Since the maximum row sum of $A$ must be the same as the maximum column sum of $A^T$, we conclude they are the same.

\end{document}

