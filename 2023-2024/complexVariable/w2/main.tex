% available style: report
\documentclass[12pt]{article}

\usepackage[tmargin=2cm,rmargin=2.5cm,lmargin=2.5cm,bmargin=2cm,footskip=0.4cm]{geometry} 
% Top margin, right margin, left margin, bottom margin, footnote skip
\usepackage[utf8]{inputenc}
\usepackage{biblatex}
\addbibresource{./reference/reference.bib}
% linktocpage shall be added to snippets.
\usepackage{hyperref,theoremref}
\hypersetup{
	colorlinks, 
	linkcolor={red!40!black}, 
	citecolor={blue!50!black},
	urlcolor={blue!80!black},
	linktocpage % Link table of content to the page instead of the title
}

\usepackage{blindtext}
\usepackage{titlesec}
\usepackage{amsthm}
\usepackage{thmtools}
\usepackage{amsmath}
\usepackage{amssymb}
\usepackage{graphicx}
\usepackage{titlesec}
\usepackage{xcolor}
\usepackage{multicol}
\usepackage{hyperref}
\usepackage{import}
\usepackage{bm}


\newtheorem{theorem}{Theorema}[section]
\newtheorem{lemma}[theorem]{Lemma}
\newtheorem{corollary}{Corollarium}[section]
\newtheorem{proposition}{Propositio}[theorem]
\theoremstyle{definition}
\newtheorem{definition}{Definitio}[section]

\theoremstyle{definition}
\newtheorem{axiom}{Axioma}[section]

\theoremstyle{remark}
\newtheorem{remark}{Observatio}[section]
\newtheorem{hypothesis}{Coniectura}[section]
\newtheorem{example}{Exampli Gratia}[section]

% Proof Environments
\newcommand{\thm}[2]{\begin{theorem}[#1]{}#2\end{theorem}}

%TODO mayby proof environment shall have more margin
\renewenvironment{proof}{\vspace{0.4cm}\noindent\small{\emph{Demonstratio.}}}{\qed\vspace{0.4cm}}
% \renewenvironment{proof}{{\bfseries\emph{Demonstratio.}}}{\qed}
\renewcommand\qedsymbol{Q.E.D.}
\renewcommand{\chaptername}{Caput}
\renewcommand{\contentsname}{Index Capitum} % Index Capitum 
\renewcommand{\emph}[1]{\textbf{\textit{#1}}}
\renewcommand{\ker}[1]{\operatorname{Ker}{#1}}

%\DeclareMathOperator{\ker}{Ker}

% New Commands
\newcommand{\bb}[1]{\mathbb{#1}} %TODO add this line to nvim snippets

% ALGEBRA
\newcommand{\orb}[2]{\text{Orb}_{#1}({#2})}
\newcommand{\stab}[2]{\text{Stab}_{#1}({#2})}
\newcommand{\im}[1]{\text{im}{\ #1}}
\newcommand{\se}[2]{\text{send}_{#1}({#2})}

%STATISTICS
\newcommand{\var}[1]{\text{Var}(#1)}
\newcommand{\ud}[1]{\underline{#1}}
\newcommand{\cor}[1]{\text{Cor}(#1)}
\newcommand{\std}[1]{\text{Std}(#1)}
\newcommand{\ste}[1]{\text{S.E.}(#1)}


\title{Complex Variable}
\author{Harry}
\date{\today}
\begin{document}
\section{Q3}
We are to prove that there is no $z \in \mathbb{C}$ such that $|z| = |z-2(1-i)| = 1$.

Let us invoke triangular-inequality:

$$
|z-2(1-i)| \geq	||z| - |2(1-i)|| = 2\sqrt{2} - 1 > 1
$$

This is a contradiction and we conclude that there is no such $z$.

\section{Q4}
I claim that all $z$ such that $|Re(z)| + |Im(z)| = \sqrt{2} |z|$ are $r e^{i \pi/4}$ and $r e^{i 3\pi/4}$ for any $r \in \bb{R}$.

Backword diretion: easy to verify with computation.

Forward direction:

For any $z = re^{i \theta}$, our constrains gives us: $|\sin{\theta}| + |\cos{\theta}| = \sqrt{2}$.

Consider the following cases:

$\sin{\theta} + \cos{\theta} = \sqrt{2}$. 
Notice $\sin{\theta} + \cos{\theta} = \sqrt{2}\sin{(\frac{\pi}{4} + \theta)} = \sqrt{2}$. (Apply identity for $\sin(a+b)$)
It is clear that the equaility is only established with $\theta = \frac{\pi}{4}$ for $0\leq \theta \leq 2\pi$.

Similarly, for 
$-\sin{\theta}  -\cos{\theta} = \sqrt{2}$, we can assert the equality is only established with $\theta = \frac{5\pi}{4}$ for $0\leq \theta \leq 2\pi$.


For $-\sin{\theta} + \cos{\theta} = \sqrt{2}$. We have
$-\sin{\theta} + \cos{\theta} = \sqrt{2}\sin{(\frac{3\pi}{4} + \theta)} = \sqrt{2}$. 
It is clear that the equaility is only established with $\theta = \frac{7\pi}{4}$ for $0\leq \theta \leq 2\pi$.

And for $\sin{\theta} - \cos{\theta} = \sqrt{2}$, the equality is only established with $\theta = \frac{3\pi}{4}$.

Note for all four cases we have set no restrain for $r$. 

Thus we conclude that all $z$ such that $|Re(z)| + |Im(z)| = \sqrt{2} |z|$ are $r e^{i \pi/4}$ and $r e^{i 3\pi/4}$ for any $r \in \bb{R}$.

\section{Q5}

I claim that, for all $\alpha>0$, we have $\frac{|z-\alpha|}{|z+\alpha|} < 1$ iff $Re(z) > 0$.

Let us write $z = a+ bi$ and $|z|^2 = a^2 + b^2$.

Again, there are three cases:

1): $a=0$.
In such case we have $(\frac{|z-\alpha|^2}{|z+\alpha|^2}) = (\frac{\alpha^2 + b^2}{\alpha^2 + b^2}) = 1$. As modular is always positive, we conclude that $\frac{|z-\alpha|}{|z+\alpha|} = 1$

2): $a>0$. 
Since $a, \alpha>0$, it is clear that $|a-\alpha| < |a+\alpha| \implies (a-\alpha)^2 < (a+\alpha)^2 \implies (a-\alpha)^2 + b^2 < (a+\alpha)^2 + b^2$ thus
in such case we have $\frac{|z-\alpha|^2}{|z+\alpha|^2} = \frac{(a-\alpha)^2 + b^2}{(a+\alpha)^2 + b^2} < 1 \implies \frac{|z-\alpha|}{|z+\alpha|} < 1$ (as modular is positive). 

3): $a<0$.
Since $a<0, \alpha>0$, it is clear that $|a-\alpha| > |a+\alpha| \implies (a-\alpha)^2 > (a+\alpha)^2 \implies (a-\alpha)^2 + b^2 > (a+\alpha)^2 + b^2$ thus 
in such case we have $\frac{|z-\alpha|^2}{|z+\alpha|^2} = \frac{(a-\alpha)^2 + b^2}{(a+\alpha)^2 + b^2} > 1 \implies \frac{|z-\alpha|}{|z+\alpha|} > 1$.

Thus we conclude that, for all $\alpha>0$, we have $\frac{|z-\alpha|}{|z+\alpha|} < 1$ iff $Re(z) > 0$.

\section{Q6}
We have  $z = x+iy$ and $x>0$. I claim that for $\epsilon = \frac{x}{2}$, and $w $ such that $|w-z| < \epsilon$, we have $Re(w) > 0$.

Let us write $w = a+bi$. Notice $|w-z|<\epsilon \implies (a-x)^2 + (b-y)^2 < \epsilon^2 \implies (a-x)^2 < \epsilon^2 \implies |x-a| < \frac{x}{2} $. As $\frac{x}{2}>0$, apply definition of absolute value, we have $ \frac{x}{2}<a<\frac{3x}{2}$. As $x > 0$ we have $a>x>0$ as desired. 
\end{document} 
