\documentclass[12pt, a4paper]{article}
\usepackage{blindtext, titlesec, amsthm, thmtools, amsmath, amsfonts, scalerel, amssymb, graphicx, titlesec, xcolor, multicol, hyperref}
\usepackage[utf8]{inputenc}
\hypersetup{colorlinks,linkcolor={red!40!black},citecolor={blue!50!black},urlcolor={blue!80!black}}
\newtheorem{theorem}{Theorema}[subsection]
\newtheorem{lemma}[theorem]{Lemma}
\newtheorem{corollary}[theorem]{Corollarium}
\newtheorem{hypothesis}{Coniectura}
\theoremstyle{definition}
\newtheorem{definition}{Definitio}[section]
\theoremstyle{remark}
\newtheorem{remark}{Observatio}[section]
\newtheorem{example}{Exampli Gratia}[section]
\newcommand{\bb}[1]{\mathbb{#1}}
\renewcommand\qedsymbol{Q.E.D.}
\title{Complex Variable Handin 4}
\author{Harry Han}
\date{\today}
\begin{document}
\maketitle

\section{Q1} % (fold)
\label{sec:Q1}
\subsection{(a)}
I am to prove that 

\[
  \left|\int_{C_3(i)} \frac{1}{z} dz \right| \leq 3 \pi
\]

Let us invoke ML lemma. Apply the formula we learned from kindergarden, we find the arclength of $C_3(i)$ is $6 \pi$.

Notice that for certain $z$ on the curve such that $|\frac{1}{z}|$ attained its maximum, for the same $z$, $|z|$ attains its minimum.

For the circle centered at $i$ with radius 3, it is clear that the minimum of $|z|$ is attained at $z=-2i$, for which $|z| = 2$, which means the maximum of $|1/z|$ is $1/2$.

If you want furthur justification that the minimum is attained at $-2i$, notice that the modulus is the same as distance from the origin, and the point on the circle that is closest to the point $a$ that is not the center, lies on the line segment $ac$, where $c$ is the center of the circle. This is again the formula from primary school. 

Apply $ML$ lemma I claim 

\[
  \left|\int_{C_3(i)} \frac{1}{z} dz \right| \leq \max {\frac{1}{z}}\cdot 6 \pi = 3\pi 
\]
  
\subsection{(b)} 

I am to prove that, for $R > 2$, where $C_{R}^+$ is the upper semi circle centered at origin with radius $R$.

It is clear that the arc length of this curve is $R\pi$.

\[
  \left| \int_{C^+_{R}} \frac{1}{2+z^2} dz \right| \leq \frac{R\pi}{R^2-2}
\]

Again, the maximum of $|\frac{1}{2+z^2}|$ is attained at $z$ if and only if the minimum of $|2+z^2|$ is attained at the same $z$ along the curve.

Before finding the minimum of $|2+z^2|$, let us find the image of $z^2 + 2$ for $z \in C^+_{R}$. 
It is clear that the image is the full circle with the radius $R^2$ centered at the point 2.
Thus the minimum of the modulus on the image is the point $-R^2$, which is attained at $z = Ri \implies 2+z^2 = -R^2 +2 \implies |2+z^2| = R^2 - 2 $. 
(Recall we have the constrain that $R > 2$).

In this way we apply $ML$ lemma and find that the maximum of the integral is $\frac{R\pi}{R^2 - 2}$.

\section{Q2} % (fold)
\label{sec:Q2}

\subsection{Q2(a)}
\label{subsec:Q2(a)}

We are to show that there is no function $f: \bb{C} \symbol{92} \{0 \} \rightarrow \bb{C}$ such that $f' = 1/z$ and $f$ is holomorphic on all of its domain.

\begin{proof}
  Assuming such a $f$ exist. It is the anti derivative of $1/z$ in its domain. 

  Let us integrate $1/z$ along the unit circle. Since $f$ is its antiderivative, by Lemma 3.3.9 we find this integral is 0. However, by theorem 3.4.1, this integral equals to $2\pi i$. This is an contradiction and we conclude such an $f$ does not exist.
\end{proof}

\subsection{q2(b)}
\label{subsec:q2(b)}

Let us now demonstrate the there is no holomorphic function $l: \bb{C} \symbol{92} \{0\} \rightarrow \bb{C}$ such that $exp(l(z)) =z$ for all $z
$ in its domain.

Assuming such a function exist. Let us differentiate $exp(l) = z$ on both side. 

We find that $exp(l)' l' = 1 \implies l' = \frac{1}{exp(l)'} = \frac{1}{z}$. This $l$ is the antiderivative of $\frac{1}{z}$. In previous section we have shown that such a function does not exist.

%subsection fold end

\section{Q3} % (fold)
\label{sec:Q3}

I am to prove that the integral 

\[
  I = \displaystyle\int_{-\infty}^{\infty} \frac{1}{x^2+2} dx = \lim_{R \rightarrow \infty} \int_{-R}^{R} \frac{1}{x^2+2} dx= \frac{\pi}{\sqrt{2}} 
\]

Let us define the complex function $f(z) = \frac{1}{z^2 + 2}$. Let us define the curve $\Gamma_R$ be the upper circle centered at origin with radius $R$ and the curve $\Lambda_R$ be the line segment between $-R$ and $R$. 

Notice the contour integral $\int_{\Lambda_R} f(z) = \int_{-R}^{R} \frac{1}{x^2+2}dx$

Notice that for all $R > 0$, the curve $\Lambda_R + \Gamma_R$ becomes a loop that is the closed upper half semi circle centered at $0$ with radius $R$. 
Thus we can apply cauchy integral theorem to integrage $f(z)$ 

\[
  \int_{\Lambda_R +  \Gamma_R} f(z) dz = \int_{\Lambda_R + \Gamma_R}  \frac{1}{z+\sqrt{2} i} \frac{1}{z- \sqrt{2} i} dz= 2\pi i \cdot \frac{1}{2\sqrt{2}i} = \frac{\pi}{\sqrt{2}}
\]

Notice this expression is independent of $R$.

In the previous section we have shown that:

\[
  \left| \int_{\Gamma_{R}} \frac{1}{2+z^2} dz \right| \leq \frac{R\pi}{R^2-2} 
\]

This means $\lim_{R \rightarrow \infty} \int_{\Gamma_R} f(z) \leq \lim_{R \rightarrow \infty} | \int_{\Gamma_R} f(z)| dz = 0$ 

Thus:

\[
  I = \lim_{R \rightarrow \infty} \int_{\Lambda_R +  \Gamma_R} f(z)dz - \lim_{R \rightarrow \infty} \int_{\Gamma_R} f(z) dz= \frac{\pi}{\sqrt{2}} 
\]

% section Q3 (end)

\section{Q4} % (fold)
\label{sec:Q4}

We are to prove that if  $f$ is holomorphic on $D_1(0)$, and $\max_{z\in C_r(0)}|f(z)| = 0$ for $r \rightarrow 1$, then $f$ is identically 0. 

\begin{proof}
  Pick any $z_0 \in D_1(0)$. Pick a number $\rho$ such that $ \rho < 1 $ and $\rho >  \frac{1+|z|}{2}$. Apply cauchy integral formula we get: 

  \[
    f(z_0) = \frac{1}{2\pi i} \displaystyle\int_{C_{\rho}(0)} \frac{f(z)}{z-z_0} dz 
  \]
  
  Notice for each $z$, $|\frac{f(z)}{z-z_0}| = |f(z)| |\frac{1}{z-z_0}|$. 
  In particular, for $z \in C_\rho(0)$, $|\frac{1}{z-z_0}| \leq \frac{1}{1- |z_0|} $. 
  This means 
  $$\max_{z \in C_{\rho}(1)} \left|\frac{f(z)}{z-z_0}\right| \leq  \left|\frac{1}{1- z_0}\right| \max_{z \in C_{\rho}(1)} \left|f(z)\right|$$. 

  Moreover, it is clear that the arclength of $C_{\rho} (0) < 2\pi$.

  Let us apply ML lemma: 
  
  \[
    \lim_{\rho \rightarrow  1} \left| \frac{1}{2\pi i} \displaystyle\int_{C_{\rho}(0)} \frac{f(z)}{z-z_0} dz \right| 
    \leq  \lim_{\rho \rightarrow  1} \frac{1}{2 \pi}  \left|\frac{1}{1- z_0}\right| \max_{z \in C_{\rho}(1)} \left|f(z)\right| = 0
  \]
  This means that $|f(z_0)| = 0$. The only way it is possible is that $f(z_0) =0$.

  Notice we have set no other restriction on $z$ except that $z \in D_1(0)$. Thus I claim $z\in D_1(0) \implies f(z_0) = 0 $ as desired.
 
\end{proof}



% section Q4 (end)

\end{document}
