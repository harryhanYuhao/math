% available style: report
\documentclass[12pt]{article}

\usepackage[tmargin=2cm,rmargin=2.5cm,lmargin=2.5cm,bmargin=2cm,footskip=0.4cm]{geometry} 
% Top margin, right margin, left margin, bottom margin, footnote skip
\usepackage[utf8]{inputenc}
\usepackage{biblatex}
\addbibresource{./reference/reference.bib}
% linktocpage shall be added to snippets.
\usepackage{hyperref,theoremref}
\hypersetup{
	colorlinks, 
	linkcolor={red!40!black}, 
	citecolor={blue!50!black},
	urlcolor={blue!80!black},
	linktocpage % Link table of content to the page instead of the title
}

\usepackage{blindtext}
\usepackage{titlesec}
\usepackage{amsthm}
\usepackage{thmtools}
\usepackage{amsmath}
\usepackage{amssymb}
\usepackage{graphicx}
\usepackage{titlesec}
\usepackage{xcolor}
\usepackage{multicol}
\usepackage{hyperref}
\usepackage{import}
\usepackage{bm}


\newtheorem{theorem}{Theorema}[section]
\newtheorem{lemma}[theorem]{Lemma}
\newtheorem{corollary}{Corollarium}[section]
\newtheorem{proposition}{Propositio}[theorem]
\theoremstyle{definition}
\newtheorem{definition}{Definitio}[section]

\theoremstyle{definition}
\newtheorem{axiom}{Axioma}[section]

\theoremstyle{remark}
\newtheorem{remark}{Observatio}[section]
\newtheorem{hypothesis}{Coniectura}[section]
\newtheorem{example}{Exampli Gratia}[section]

% Proof Environments
\newcommand{\thm}[2]{\begin{theorem}[#1]{}#2\end{theorem}}

%TODO mayby proof environment shall have more margin
\renewenvironment{proof}{\vspace{0.4cm}\noindent\small{\emph{Demonstratio.}}}{\qed\vspace{0.4cm}}
% \renewenvironment{proof}{{\bfseries\emph{Demonstratio.}}}{\qed}
\renewcommand\qedsymbol{Q.E.D.}
\renewcommand{\chaptername}{Caput}
\renewcommand{\contentsname}{Index Capitum} % Index Capitum 
\renewcommand{\emph}[1]{\textbf{\textit{#1}}}
\renewcommand{\ker}[1]{\operatorname{Ker}{#1}}

%\DeclareMathOperator{\ker}{Ker}

% New Commands
\newcommand{\bb}[1]{\mathbb{#1}} %TODO add this line to nvim snippets

% ALGEBRA
\newcommand{\orb}[2]{\text{Orb}_{#1}({#2})}
\newcommand{\stab}[2]{\text{Stab}_{#1}({#2})}
\newcommand{\im}[1]{\text{im}{\ #1}}
\newcommand{\se}[2]{\text{send}_{#1}({#2})}

%STATISTICS
\newcommand{\var}[1]{\text{Var}(#1)}
\newcommand{\ud}[1]{\underline{#1}}
\newcommand{\cor}[1]{\text{Cor}(#1)}
\newcommand{\std}[1]{\text{Std}(#1)}
\newcommand{\ste}[1]{\text{S.E.}(#1)}


\title{Graph Theory}
\author{Harry Han}
\date{\today}
\begin{document}
\section{Basics}

\begin{definition}[Graph]
	Let $n, m, f, k$ denote number of vertices, edges, faces, components respectively. 
	\begin{enumerate}
		\item	\b{Simple Graph} No loops, no multiple edges.	
		\item	\b{Walk} Finite sequence of connected edges.
		\item	\b{Trail} walk with no repeated edges.
		\item	\b{Path} walk with no repeated vertices.
		\item 	\b{Cycle} Closed plath with no repeated edges. (any loop or multiple edges are cycles)
		\item	\b{Disconnecting Set} A set of edges whose removal disconnects the graph (increase the number of components).
		\item	\b{Cut Set} smallest disconnecting set.
		\item	\b{Edge Connectivity} The size of the smallest cut set, $\lambda (G)$. We say $G$ is $k$ connected if $\lambda (G) \geq k$.
		\item	\b{Bridge} An edge whose removal increases the number of components.
		\item	\b{Separating Set} A set of vertices whose removal disconnects the graph.
		\item	\b{Vertex Connectivity} The size of the smallest separating set, $\kappa (G)$. We say $G$ is $k$ connected if $\kappa (G) \geq k$. We have $\kappa (G) \leq \lambda (G)$
		\item	\b{Cut Vertex} A vertex whose removal increases the number of components.
		\item	\b{Eulerian} A graph with a closed trail containing all edges.
		\item	\b{Semi-Eulerian} A graph with a trail containing all edges.
		\item	\b{Hamiltonian} A graph with a closed trail passing through exactly once all vertices. This is Hamiltonian cycle. (No repeated edges or vertices, passing through all vertices)
		\item	\b{Forest} A graph with no cycles.
		\item	\b{Tree} A connected forest.
		\item	\b{Cutset Rank} Number of edges in a spanning tree. $n-k$.
		\item	\b{Cycle Rank} Number of edges removed to get a spanning forest. $m - (n-k)$.
		\item	\b{Connected Digraph} A digraph whose corresponding undirected graph is connected.
		\item	\b{Strongly Connected Digraph} A digraph such that for all $u,v \in V$, there is a $u-v$ path and a $v-u$ path.
		\item	\b{Orientable} A graph is orientable if it is possible to assign a direction to each edge such that the resulting digraph is strongly connected.
		\item	\b{Tournament} A digraph such that two vertices are joined by one directed edge.+
		\item	\b{Chromatic Number} Coloring index.
		\item	\b{Chromatic Polynomial} Coloring edge with minimum number of colors.
	\end{enumerate}
\end{definition}

\begin{theorem}[Adjacency matrix]
	Let $A$ be adjacency matrix.
	Trace of $A^2$ is twice the number of edges. Trace of $A^3$ is 6 times the number of triangles.
\end{theorem}

\begin{theorem}[Bound for edges]
	$G$ be a simple graph with $n$ vertices, $k$ components, $m$ edges. We have 
	\[ 
		n-k \leq m \leq \frac{(n-k)(n-k+1)}{2}
	\]
\end{theorem}

\begin{theorem}[Ore]
If $G$ is simple graph with $n \geq 3$ vertices such that $\deg (u) + \deg (v) \geq n$ for all non-adjacent vertices $u,v$, then $G$ is Hamiltonian.
\end{theorem}

\begin{definition}[Algorithms]
	\ 
\begin{enumerate}
	\item The shortest path problem: Dijkstra's algorithm.
	\item The minimum spanning tree problem: Greedy algorithm.
	\item Chinese postman problem: find the shortest closed walk containing all edges. (If the graph is Eulerian, then it is the Eulerian cycle. If Semi-Eulerian, this is the Semi-Eulerian path along the path of least length between the two odd vertices.
	\item Travelling salesman problem: find the shortest hamiltonian cycle. (Passing through all vertices exactly once and returning to the starting vertex) (Lower bound: find minimum spanning tree of $G - v$. Add the sum with two minimum edges of $v$).
\end{enumerate}
\end{definition}

\begin{theorem}[Counting Trees]
	There are $n^{n-2}$ labeled trees on $n$ vertices.
\end{theorem}

\begin{theorem}[Planarity]
	\ 
	\begin{enumerate}
		\item Every planar graph has a vertex of degree at most 5.
		\item For planar graphs, $ m \leq 3n - 6$. For plane graphs with no triangles, $m \leq 2n - 4$.
	\end{enumerate}
\end{theorem}

\begin{theorem}[Genus]
	For simple graph, $g(G) \geq \text{ceil} \{(m-3n)/6 +1 \}$. The equality holds if $G$ is complete graph.
\end{theorem}

\begin{theorem}[Euler]
	\begin{equation}
		n - m + f = 2 - 2g	
	\end{equation}
\end{theorem}

\begin{theorem}[Coloring]
	\begin{enumerate}
		\item Every planar graph is 4-colorable.
		\item Four color theorem for maps is equivalent to four color theorem for planar graphs.
	\end{enumerate}
\end{theorem}
\end{document}
