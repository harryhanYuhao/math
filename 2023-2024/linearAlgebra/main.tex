\documentclass[12pt]{report}

\input{macros}
\usepackage[utf8]{inputenc}
\usepackage{biblatex}
\addbibresource{./references.bib}
% linktocpage shall be added to snippets.
\usepackage{hyperref,theoremref}
\hypersetup{
	colorlinks, 
	linkcolor={red!40!black}, 
	citecolor={blue!50!black},
	urlcolor={blue!80!black},
	linktocpage % Link table of content to the page instead of the title
}

\usepackage{blindtext}
\usepackage{titlesec}
\usepackage{amsthm}
\usepackage{thmtools}
\usepackage{amsmath}
\usepackage{amssymb}
\usepackage{graphicx}
\usepackage{titlesec}
\usepackage{xcolor}
\usepackage{multicol}
\usepackage{hyperref}
\usepackage{import}
\usepackage{libertinus}           % Load the Libertinus font
\usepackage{mathbbol}
\usepackage{bm}
\usepackage{mathrsfs}

\usepackage{tikz}
\usepackage{tikz-cd}
\usetikzlibrary{arrows.meta, positioning, calc, arrows, decorations.pathreplacing}
% \usepackage{calrsfs}


\newtheorem{theorem}{Theorem}[section]
\newtheorem{lemma}[theorem]{Lemma}
\newtheorem{corollary}[theorem]{Corollarium}
\newtheorem{proposition}[theorem]{Proposition}
\theoremstyle{definition}
\newtheorem{definition}[theorem]{Definition}

\theoremstyle{definition}
\newtheorem{axiom}[theorem]{Axioma}

\theoremstyle{remark}
\newtheorem{remark}[theorem]{Remark}
\newtheorem{hypothesis}[theorem]{Coniectura}
\newtheorem{example}[theorem]{Example}
% Proof Environments
\newcommand{\thm}[2]{\begin{theorem}[#1]{}#2\end{theorem}}


\newcommand{\D}{\bm{\Delta}}
\renewcommand{\S}{\textbf{Set}}
\newcommand{\sS}{\textbf{sSet}}
\newcommand{\op}{^{\text{op}}}
\newcommand{\N}{\mathbb{N}}
\renewcommand{\L}{\Lambda}
\newcommand{\C}{\mathcal{C}}
\newcommand{\F}{\text{Fun}}
\newcommand{\HomU}{\underline{\text{Hom}}}

\let\H\relax
\DeclareMathOperator{\H}{\text{Hom}}

\renewcommand{\emph}{\textbf}


\title{Linear Algebra}
\author{Harry Han}
\date{\today}
\begin{document}
\maketitle
\tableofcontents
\chapter{Vector Spaces}

\section{Group, Field and, Vector Spaces}

\definition[group]{
A \textbf{group} is a set $G$ with an operation $\cdot$ such that
\begin{enumerate}
\item $\cdot$ is closed;
\item $\cdot$ is associative;
\item $\exists e\in G$ such that $\forall g\in G, eg=ge=g$;
\item $\forall g\in G, \exists g^{-1}\in G$ such that $gg^{-1}=g^{-1}g=e$;
\end{enumerate}
}

\definition[field]{
A \textbf{field} is a set $F$ with two operations, addition and multiplication, such that
\begin{enumerate}
\item $F$ is an abelian group under addition
\item $F\setminus\{0\}$ is an abelian group under multiplication
\item Multiplication distributes over addition
\end{enumerate}
}

\definition[vector space]{
A \textbf{vector space} over a field $F$ is a set $V$ with two operations, addition and scalar multiplication, such that
\begin{enumerate}
\item $V$ is an abelian group under addition
\item (Multiplicative identity) $\forall v\in V, 1 v=v$ (here 1 is the multiplicative identity of the field)
\item (Distributive Property) $\forall a,b\in F,$ and $v\in V, (a+b)v=av+bv$; and $\forall v, u \in V$ and $a\in F, a(v+u)=av+au$
\end{enumerate}
}

Note, the field $F$ has an additive identity, 0. 
The abelian group of vectors also has an additive identity.

\begin{theorem}
\begin{enumerate}
    \ 
    \item For $u\in V$, its additive inverse in $V$ is $-1u$, while $-1$ is the additive inverse of the multiplicative identity of the field F.
  \end{enumerate}
\end{theorem}
  
\end{document}
