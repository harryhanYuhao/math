% available style: report
\documentclass[12pt]{article}

\usepackage[utf8]{inputenc}
\usepackage{biblatex}
\addbibresource{./references.bib}
% linktocpage shall be added to snippets.
\usepackage{hyperref,theoremref}
\hypersetup{
	colorlinks, 
	linkcolor={red!40!black}, 
	citecolor={blue!50!black},
	urlcolor={blue!80!black},
	linktocpage % Link table of content to the page instead of the title
}

\usepackage{blindtext}
\usepackage{titlesec}
\usepackage{amsthm}
\usepackage{thmtools}
\usepackage{amsmath}
\usepackage{amssymb}
\usepackage{graphicx}
\usepackage{titlesec}
\usepackage{xcolor}
\usepackage{multicol}
\usepackage{hyperref}
\usepackage{import}
\usepackage{libertinus}           % Load the Libertinus font
\usepackage{mathbbol}
\usepackage{bm}
\usepackage{mathrsfs}

\usepackage{tikz}
\usepackage{tikz-cd}
\usetikzlibrary{arrows.meta, positioning, calc, arrows, decorations.pathreplacing}
% \usepackage{calrsfs}


\newtheorem{theorem}{Theorem}[section]
\newtheorem{lemma}[theorem]{Lemma}
\newtheorem{corollary}[theorem]{Corollarium}
\newtheorem{proposition}[theorem]{Proposition}
\theoremstyle{definition}
\newtheorem{definition}[theorem]{Definition}

\theoremstyle{definition}
\newtheorem{axiom}[theorem]{Axioma}

\theoremstyle{remark}
\newtheorem{remark}[theorem]{Remark}
\newtheorem{hypothesis}[theorem]{Coniectura}
\newtheorem{example}[theorem]{Example}
% Proof Environments
\newcommand{\thm}[2]{\begin{theorem}[#1]{}#2\end{theorem}}


\newcommand{\D}{\bm{\Delta}}
\renewcommand{\S}{\textbf{Set}}
\newcommand{\sS}{\textbf{sSet}}
\newcommand{\op}{^{\text{op}}}
\newcommand{\N}{\mathbb{N}}
\renewcommand{\L}{\Lambda}
\newcommand{\C}{\mathcal{C}}
\newcommand{\F}{\text{Fun}}
\newcommand{\HomU}{\underline{\text{Hom}}}

\let\H\relax
\DeclareMathOperator{\H}{\text{Hom}}

\renewcommand{\emph}{\textbf}


\title{Differential Equation Note}
\author{Harry Han}
\date{\today}

\begin{document}

\section{Theory Of Differential Equations}

\begin{definition}[Classification of Differential Equations] 
	Differential equations involves in relations of functions and their derivatives.
	We usually use $t$ as independent variable with physical meaning of time.

	If only ordinary derivatives (the function depends only in one variable) are involved, the equation is called an \textbf{ordinary differential equation} (ODE).
	If partial derivatives are involved, the equation is called a \textbf{partial differential equation} (PDE).

	\textbf{Order} of a differential equation is the order of the highest derivative.
	A $n$th order ODE can be written as $F(t, x, x'', \cdots, x^{(n)}) = 0$. If $F$ is a linear function, we call this a linear ODE.

	A linear ODE is \textbf{homogeneous} if it can be written as $a_{n}x^{(n)} + a_{n-1}x^{(n-1)} + \cdots + a_{1}x' + a_{0}x = 0$ for constants $a_i$.
\end{definition}

\begin{definition}[Initial Value Problem]
	%TODO:
\end{definition}

\begin{theorem}[Solutions To linear Homogeneous ODE]
	Solutions to $n$th degree Linear homogeneous ODE of one function forms a vector space of $n$ dimension.

	I.e., with ODE $a_{n}x^{(n)} + a_{n-1}x^{(n-1)} + \cdots + a_{1}x' + a_{0}x = 0$, the set of solutions forms a vector space of dimension $n$.
\end{theorem}

\begin{theorem}[Existance And Uniqueness Theore for Initial Value Problem]

\end{theorem}

\section{Easy Differential Equations}

\begin{theorem}
	The equation $\frac{dy}{dt} + p(t)y = g(t)$ is solved by $y(t) = \frac{1}{\mu(t)} (\int \mu(t)g(t)dt+C)$ where $\mu(t) = \exp{\int p(t)dt}$.

\end{theorem}


\section{Linear ODE}
\begin{definition}[ODE]
	todo
\end{definition}

\begin{enumerate}
	\item The the existance and uniqueness theorem for ODE with initial condition.
	\item Solve linear ODE system. Similar to a single ODE. Matrix exponent.
	\item Characterisation of non linear ODE: critical points, stability, phase diagram, limit cycle, Poincare-Bendixson theorem (only in 2 D).
	\item Solve non homogeneous ODE. Three methods: variation of parameters, undetermined coefficients, change of basis.
\end{enumerate}


\section{Laplace Transform}

\begin{definition}[Laplace Transform]
	\begin{equation}
		\L{f(t)} = \int_{0}^{\infty} e^{-st} f(t) dt
	\end{equation}
\end{definition}

\begin{theorem}[Tables of Laplace Transform]
	\ 
\begin{enumerate}
	\item $\L{f^{(n)}(t)} = s^n \L{f(t)} - s^{n-1}f(0) - s^{n-2}f'(0) - \cdots - f^{(n-1)}(0)$
	\item $\L{e^{at}f(t)} = \L{f}(s-a)$
	\item $\L{t^nf(t)} = (-1)^n \frac{d^n}{ds^n} \L{f}(s)$
	\item $\L{1} = \frac{1}{s}$
	\item $\L{e^{at}} = \frac{1}{s-a}$ for $s>a$
	\item $\L{\sin{at}} = \frac{a}{s^2+a^2}$ for $s>0$
	\item $\L{\cos{at}} = \frac{s}{s^2+a^2}$ for $s>0$
	\item $\L{e^{at}\sin{bt}} = \frac{b}{(s-a)^2+b^2}$ for $s>a$
	\item $\L{e^{at}\cos{bt}} = \frac{s-a}{(s-a)^2+b^2}$ for $s>a$
	\item $\L{t^n} = \frac{n!}{s^{n+1}}$ for $s>0$
	\item $\L{\sinh{at}} = \frac{a}{s^2-a^2}$ for $s>|a|$
	\item $\L{\cosh{at}} = \frac{s}{s^2-a^2}$ for $s>|a|$
	\item $\L{t^n e^{at}} = \frac{n!}{(s-a)^{n+1}}$ for $s>a$
\end{enumerate}	
\end{theorem}


\section{Exam}
\begin{enumerate}
	\item Three pages of double sided notes.
	\item No numerical method: no Euler, no Runge-Kutta, no error analyis.
\end{enumerate}
\end{document}
