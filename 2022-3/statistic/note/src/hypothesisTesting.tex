\section{Overview}
We never accept null hypothesis. We never reject alternative hypothesis. 
It is either we fail to reject null hypothesis or we reject null hypothesis in favor of some alternative.
\begin{definition}[Type I, II error]
Type I error is to incorrectly reject the null hypothesis when it is true. Type II error is to incorrectly fail to reject the null hypothesis when it is false.

The probability of incurring a Type 1 error is the significance level of the hypothesis and is denoted by $\alpha$. The probability of incurring a Type II error is denoted by $\beta$. $1-\beta$ is called the power of the test.
\end{definition}

\paragraph{Five Step of hypothesis testing}
\begin{enumerate}
	\item State the null and alternative hypothesis.
	\item Select appropriate test statistics ad determine the distribution under null hypothesis
	\item Compute the test statistics
	\item Quantifying the Evidence, compute the critical region. \small \textit{If the test statistics falls in the critical region, we reject the null hypothesis. If p value smaller than $\alpha$, reject null}
	\item Make a decision
\end{enumerate}

\subparagraph{Notation}
$\mu_0$ is given mean to be tested. $\sigma^2$ is given (known) variance. $\bar{X}$ is sample mean, (estimate by estimator). $n$ is sample size. $s^2$ is estimated sample variance. $\alpha$ is significance level. $C$ is critical region. 
\subparagraph{Z-test}
\emph{Use When Population Variance is Known}
$z_0$ is critical value. $z$ is test statistics. 
\begin{equation}H_0: \mu = \mu_0; \hspace{1cm} H_1: \mu\neq \mu_0\end{equation}	
\[
Z = \frac{\bar{X}-\mu_0}{\sqrt{\sigma^2/n}} \sim N(0,1),
C=\{z:|z|\geq z_0\},z_0 = \texttt{qnorm(1-\textalpha/2, mean=0, sd=1)}
\]

\subparagraph{One Sample T-test}
\emph{Use When Population Variance is Unknown}

Assumption: $X_1, \cdots, X_n$ are iid distributed $N(\mu, \sigma^2)$, where $\sigma^2$ is unknown.
\[
	H_0: \mu = \mu_0; \hspace{1cm} H_1: \mu\neq \mu_0
\]
$$T = \frac{\bar{X}-\mu_0}{\sqrt{S^2/n}} \sim t_{n-1},
C=\{t:|t|\geq t_0\},t_0 = \texttt{qt(1-\textalpha/2, df=n-1)}$$

\subparagraph{Paired T-Test}
Use only if the measurements of two samples are paired and related.

\subparagraph{Two Sample T-test}
\emph{Use When Population Variance is Unknown and Population size are different}

Assumption: $X_1, \cdots, X_n$ and $Y_1, \cdots, Y_m$ are iid distributed $N(\mu_1, \sigma^2)$ and $N(\mu_2, \sigma^2)$, where $\sigma^2$ is unknown. ($X_i, Y_i$ have the \emph{same variance})
\[
	H_0: \mu_1 = \mu_2; \hspace{1cm} H_1: \mu_1\neq \mu_2
\]
\[ 
	T = \frac{\bar{X}-\bar{Y}}{\sqrt{S^2_p(1/m+1/n)}} \sim t_{m+n-2}
\]
$S^2_p$ is the sample pooled variance estimator ($S^2_X$ and $S^2_Y$ are sample variance of $X$ and $Y$):
\[
	S^2_p = \frac{(n-1)S^2_X+(m-1)S^2_Y}{n+m-2}
\]

\subparagraph{F-test}
To test if two population have the same variance.
\[
	F = \frac{S^2_X}{S^2_Y} \sim F_{n_1-1, n_2-1}
\]
$S^2_X$ and $S^2_Y$ are sample variance of $X$ and $Y$.
At $\alpha=5\%$ confidence level, the critical region is $C=\{F: F\leq l \cup \geq u\}$, $l, u$ can be calculated by \texttt{qf(c(0.025, 0.975), df1=n1-1, df2=n2-1)}. 
