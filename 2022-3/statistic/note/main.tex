\documentclass[12pt, a4paper]{article}
\usepackage{blindtext, titlesec, amsthm, thmtools, amsmath, amsfonts, scalerel, amssymb, graphicx, titlesec, xcolor, multicol, hyperref}
\usepackage[utf8]{inputenc}
\hypersetup{colorlinks,linkcolor={red!40!black},citecolor={blue!50!black},urlcolor={blue!80!black}}
\newtheorem{theorem}{Theorema}[subsection]
\newtheorem{lemma}[theorem]{Lemma}
\newtheorem{corollary}[theorem]{Corollarium}
\newtheorem{hypothesis}{Coniectura}
\theoremstyle{definition}
\newtheorem{definition}{Definitio}[section]
\theoremstyle{remark}
\newtheorem{remark}{Observatio}[section]
\newtheorem{example}{Exampli Gratia}[section]
\renewcommand\qedsymbol{Q.E.D.}
\title{Statistics}
\author{Harry Han}
\date{\today}
\newcommand{\bb}[1]{\mathbb{#1}}
\newcommand{\var}[1]{\text{Var}(#1)}
\begin{document}
\maketitle
%\tableofcontents

\section{Linear Regression}

\subsection{Least Square Estimate}

\subsubsection{Single Variabel Linear Regression}
\begin{theorem}[Least Square Estimators]
	Assuming iid random variables $x_i$ with the regression model
	\begin{equation}
		Y_i=\alpha + \beta	x_i + \epsilon_i
	\end{equation}
	With the assumption that each $\epsilon_i$ is normally distributed, $\bb{E}[\epsilon_i]=0$ (Linearity of Expectation), and $\var{ \epsilon_i }=\sigma^2$ (homoscedasticity) 

	The least square estimator are 
	\begin{equation}
		\hat{\alpha} = \bar{Y}-\hat{\beta}\bar{x}; \hat{\beta} = \frac{\sum^{n}_{i=1}(x_i-\bar{x})(Y_i-\bar{Y}) }{\sum^{n}_{i=1}(x_i-\bar{x}^2) }
	\end{equation}
	Both estimator are consistent and unbiased. Moreover:
	\begin{equation}
		\var{\hat{\beta}} = \frac{\sigma^2}{(n-1)s^2_x}; \var{\hat{ \alpha }}=\sigma^2\left(\frac{1}{n}+\frac{\bar{x}^2}{(n-1)s^2_x}\right)
	\end{equation}
	Where $\sigma^2=\var{\epsilon_i}=\var{Y_i}$ and $s_x^2= \frac{\sum (x_i-x)^2 }{n-1}$ is the sample variance of the explanatory varaible.
\end{theorem}

\begin{remark}[Assumption of Linear Regression model]
	There are four assumption for the linear regression model:
	\begin{enumerate}
		\item Linearity of Expectation: $\bb{E}[Y_i]=\alpha + \beta x_i$
		\item Homoscedasticity (same variance): $\var{ \epsilon_i }=\sigma^2$
		\item Independence: $\epsilon_i$ and $x_i$ are independent
		\item Normality: $\epsilon_i$ is normally distributed
	\end{enumerate}
\end{remark}

\begin{definition}[Residue Sum of Square]
	The residue sun of square, SSE, is defined as:
	\begin{equation}
		SSE=\sum^{n}_{i=1}(y_i-\hat{y}_i)^2
	\end{equation}
	Its unbiased estimator is $S^2_E = \frac{1}{2}\sum^{n}_{i=1}(y_i-\hat{y}_i)^2$
\end{definition}

The regression equation for the response given the explanatory varaiable $x_0$ is
\begin{equation}
	\bb{E}[Y_0]=\alpha + \beta x_0
\end{equation}
\end{document}

