\documentclass[10pt]{article}

\input{macros}
\usepackage[utf8]{inputenc}
\usepackage{biblatex}
\addbibresource{./references.bib}
% linktocpage shall be added to snippets.
\usepackage{hyperref,theoremref}
\hypersetup{
	colorlinks, 
	linkcolor={red!40!black}, 
	citecolor={blue!50!black},
	urlcolor={blue!80!black},
	linktocpage % Link table of content to the page instead of the title
}

\usepackage{blindtext}
\usepackage{titlesec}
\usepackage{amsthm}
\usepackage{thmtools}
\usepackage{amsmath}
\usepackage{amssymb}
\usepackage{graphicx}
\usepackage{titlesec}
\usepackage{xcolor}
\usepackage{multicol}
\usepackage{hyperref}
\usepackage{import}
\usepackage{libertinus}           % Load the Libertinus font
\usepackage{mathbbol}
\usepackage{bm}
\usepackage{mathrsfs}

\usepackage{tikz}
\usepackage{tikz-cd}
\usetikzlibrary{arrows.meta, positioning, calc, arrows, decorations.pathreplacing}
% \usepackage{calrsfs}


\newtheorem{theorem}{Theorem}[section]
\newtheorem{lemma}[theorem]{Lemma}
\newtheorem{corollary}[theorem]{Corollarium}
\newtheorem{proposition}[theorem]{Proposition}
\theoremstyle{definition}
\newtheorem{definition}[theorem]{Definition}

\theoremstyle{definition}
\newtheorem{axiom}[theorem]{Axioma}

\theoremstyle{remark}
\newtheorem{remark}[theorem]{Remark}
\newtheorem{hypothesis}[theorem]{Coniectura}
\newtheorem{example}[theorem]{Example}
% Proof Environments
\newcommand{\thm}[2]{\begin{theorem}[#1]{}#2\end{theorem}}


\newcommand{\D}{\bm{\Delta}}
\renewcommand{\S}{\textbf{Set}}
\newcommand{\sS}{\textbf{sSet}}
\newcommand{\op}{^{\text{op}}}
\newcommand{\N}{\mathbb{N}}
\renewcommand{\L}{\Lambda}
\newcommand{\C}{\mathcal{C}}
\newcommand{\F}{\text{Fun}}
\newcommand{\HomU}{\underline{\text{Hom}}}

\let\H\relax
\DeclareMathOperator{\H}{\text{Hom}}

\renewcommand{\emph}{\textbf}


\title{CheatSheet}
\author{Harry Han}
\date{\today}
\begin{document}
\section{Preliminaries}
\theorem[Central Limit Theorem]{
	Let $X_1, \cdots X_n$ be iid random variable of any distribution with expectation $\mu$ and variance $\sigma^2$. Then the distribution of $\bar{X}_n$ approaches a normal distribution with mean $\mu$ and variance $\frac{\sigma^2}{n}$ as $n\to \infty$.	
	This is equivalent to 
	\begin{equation}
	\begin{split}
		P(\sum X_i \leq x) &\approx \Phi(\frac{x-n\mu}{\sigma\sqrt{n}}) \text{ as } n\to \infty \\
		\text{ or } \sum X_i &\sim N(n\mu, n\sigma^2) \text{ equivalently } \bar{X}_n = \frac{1}{n}\sum X_i \sim N(\mu, \frac{\sigma^2}{n})
	\end{split}
	\end{equation}

	$\Phi$ is standard normal cdf.
}

\begin{definition}[Normal Distribution]
	A random variable $X$ is said to be normally distributed with mean $\mu$ and variance $\sigma^2$ if its pdf is 
	\begin{equation}
		f_X(x) = \frac{1}{\sigma\sqrt{2\pi}}e^{-\frac{1}{2}(\frac{x-\mu}{\sigma})^2}
	\end{equation}
	We write $X\sim N(\mu, \sigma^2)$

	Its cdf is 
	\begin{equation}
		\Phi(x) = \frac{1}{\sigma\sqrt{2\pi}}\int^x_{-\infty}e^{-\frac{1}{2}(\frac{t-\mu}{\sigma})^2}dt 
	\end{equation}
\end{definition}
\section{Sampling}
\definition[Estimator and Estimate]{
	The Estimator is a random variable; The estimate is its realisation. 
	The estimator is denoted with $T$, and the estimate is denoted with $t$. 
	The intrinsic population parameter is denoted with $\theta$, the estimate of the parameter is denoted as $\hat{\theta}$
	\begin{enumerate}
		\item \textbf{Estimator} is some function of the sample random variable, $T=g(X_1, X_2, \cdots, X_n)$ used to estimate a population parameter $\theta$. The Estimator, being a transformation fo the sample random variable, is also a random variable.
		\item \textbf{Estimate} is the value of the estimator for a particular sample, $t=g(x_1, \cdots ,x_n)$. It is a realisation of the random variable $T$.
	\end{enumerate}
}

\definition[Unbiased Estimator]{
	An estimator $T$ is said to be unbiased if $\bb{E}[T]=\theta$. Otherwise it is biased. 

	An estimator is asymptotically unbiased if $\bb{E}[T]\rightarrow \theta$ as $n\rightarrow \infty$.
}

\definition[Consistent Estimator]{
	An estimator $T$ is said to be consistent if 
\begin{enumerate}
	\item $T\rightarrow \theta$ as $n\rightarrow \infty$. (It is asymptotically unbiased)
	\item $\var{T}\rightarrow 0$ as $n\rightarrow \infty$.
\end{enumerate}
}

\definition[Sample Mean and Variance Estimator]{
	The sample mean and variance estimators are functions of sample random variables used to estimate the mean of variance of the population.
 	\begin{enumerate}
		\item The sample mean estimator is $\bar{X}=\frac{1}{n}\sum_{i=1}^n X_i$.
		\item The sample variance estimator is $S^2=\frac{1}{n-1}\sum_{i=1}^n (X_i-\bar{X})^2$.
	\end{enumerate}
}

\theorem{
	The sample mean estimator is unbiased and consistent. $\var{\bar{E}}=\frac{\sigma^2}{n}$. The sample variance estimator is unbiased but not consistent.
}

\definition[Standard Error of The Estimator]{
	The standard error of an estimator is the square root of its variance.
$\ste{\bar{X}}=\frac{\sigma}{\sqrt{n}}$
}

\subsection{Method of Moments}
\definition{
Consider some (population) random variable $X$ which has the probability distribution function $f_X(x|\theta)$ depending on some population parameter $\theta$. For $k = 1,2,3, \cdots, n$ the $k^{th}$ \textit{population moment} is defined as the expectation:
\begin{equation}
 \bb{E}[X^k] = \int x^k f_X(x|\theta) dx
\end{equation}

Let random variables $X_1, X_2, \cdots, X_n$ be a random sample from the population. The $k^{th}$ \textit{sample moment} is defined as:
\begin{equation}
	\bar{X^k} = \frac{1}{n}\sum_{i=1}^n X_i^k 
\end{equation}

}

Let $n$ be population, $s^2$ be sample variance and $\bar{x}$ be sample mean, i.e., first sample moment. 
Then the observed second sample moment is $\frac{n-1}{n}s^2+\bar{x}^2$

The method of moment estimator may be biased or unbiased.

\paragraph{Distribution of Sample Mean and Variance}
For small population with unknown variance, we can use the chi-square distribution and t-distribution. 

The confidence interval for $\mu$ is given by

\begin{equation}
	(\bar{x} - t_{n-1, \alpha/2}\frac{s}{\sqrt{n}}, \bar{x} + t_{n-1, \alpha/2}\frac{s}{\sqrt{n}})
\end{equation}

where $t_{n-1, \alpha/2}$ is the $100(1-\alpha/2)$ quantile of the t-distribution with $n-1$ degrees of freedom, which can be found by the R command \texttt{qt(1-alpha/2, df = n-1)}.

Known $\frac{(n-1)S^2}{\sigma^2}\sim \chi^2_{n-1}$,
The confidence interval for $\sigma^2$ is given by

\begin{equation}
	(\frac{(n-1)s^2}{\chi^2_{n-1, \alpha/2}}, \frac{(n-1)s^2}{\chi^2_{n-1, 1-\alpha/2}})
\end{equation}

where $\chi^2_{n-1, \alpha/2}$ and $\chi^2_{n-1, 1-\alpha/2}$ are the $100\alpha/2$ and $100(1-\alpha/2)$ quantiles of the chi-square distribution with $n-1$ degrees of freedom, which can be found by the R command \texttt{qchisq(alpha/2, df = n-1)} and \texttt{qchisq(1-alpha/2, df = n-1)}.

Thus 
\begin{equation}
T=\frac{\bar{X}-\mu}{S/\sqrt{n}}\sim t_{n-1}
\end{equation}
$\bar{x}, s^2$ are the sample mean and sample variance estimate.

\section{Hypothesis Testing}
\paragraph{Five Step of hypothesis testing}
\begin{enumerate}
	\item State the null and alternative hypothesis.
	\item Select appropriate test statistics ad determine the distribution under null hypothesis
	\item Compute the test statistics
	\item Quantifying the Evidence, compute the critical region. \small \textit{If the test statistics falls in the critical region, we reject the null hypothesis. If p value smaller than $\alpha$, reject null}
	\item Make a decision
\end{enumerate}

\subparagraph{Notation}
$\mu_0$ is given mean to be tested. $\sigma^2$ is given (known) variance. $\bar{X}$ is sample mean, (estimate by estimator). $n$ is sample size. $s^2$ is estimated sample variance. $\alpha$ is significance level. $C$ is critical region. 
\subparagraph{Z-test}
\emph{Use When Population Variance is Known}
$z_0$ is critical value. $z$ is test statistics. 
\begin{equation}H_0: \mu = \mu_0; \hspace{1cm} H_1: \mu\neq \mu_0\end{equation}	
\[
Z = \frac{\bar{X}-\mu_0}{\sqrt{\sigma^2/n}} \sim N(0,1),
C=\{z:|z|\geq z_0\},z_0 = \texttt{qnorm(1-\textalpha/2, mean=0, sd=1)}
\]

\subparagraph{One Sample T-test}
\emph{Use When Population Variance is Unknown}

Assumption: $X_1, \cdots, X_n$ are iid distributed $N(\mu, \sigma^2)$, where $\sigma^2$ is unknown.
\[
	H_0: \mu = \mu_0; \hspace{1cm} H_1: \mu\neq \mu_0
\]
$$T = \frac{\bar{X}-\mu_0}{\sqrt{S^2/n}} \sim t_{n-1},
C=\{t:|t|\geq t_0\},t_0 = \texttt{qt(1-\textalpha/2, df=n-1)}$$

\subparagraph{Paired T-Test}
Use only if the measurements of two samples are paired and related.

\subparagraph{Two Sample T-test}
\emph{Use When Population Variance is Unknown and Population size are different}

Assumption: $X_1, \cdots, X_n$ and $Y_1, \cdots, Y_m$ are iid distributed $N(\mu_1, \sigma^2)$ and $N(\mu_2, \sigma^2)$, where $\sigma^2$ is unknown. ($X_i, Y_i$ have the \emph{same variance})
\[
	H_0: \mu_1 = \mu_2; \hspace{1cm} H_1: \mu_1\neq \mu_2
\]
\[ 
	T = \frac{\bar{X}-\bar{Y}}{\sqrt{S^2_p(1/m+1/n)}} \sim t_{m+n-2}
\]
$S^2_p$ is the sample pooled variance estimator ($S^2_X$ and $S^2_Y$ are sample variance of $X$ and $Y$):
\[
	S^2_p = \frac{(n-1)S^2_X+(m-1)S^2_Y}{n+m-2}
\]

\subparagraph{F-test}
To test if two population have the same variance.
\[
	F = \frac{S^2_X}{S^2_Y} \sim F_{n_1-1, n_2-1}
\]
$S^2_X$ and $S^2_Y$ are sample variance of $X$ and $Y$.
At $\alpha=5\%$ confidence level, the critical region is $C=\{F: F\leq l \cup \geq u\}$, $l, u$ can be calculated by \texttt{qf(c(0.025, 0.975), df1=n1-1, df2=n2-1)}. 

\section{Linear Regression}
	Assuming iid random variables $x_i$ with the regression model 
	, and assuming that each $\epsilon_i$ is normally distributed, $\bb{E}[\epsilon_i]=0$ (Linearity of Expectation), and $\var{ \epsilon_i }=\sigma^2$ (homoscedasticity) 
	\begin{equation}
		Y_i=\alpha + \beta	x_i + \epsilon_i
	\end{equation}

	The least square estimator are 
	\begin{equation}
		\hat{\alpha} = \bar{Y}-\hat{\beta}\bar{x}; \hat{\beta} = \frac{\sum^{n}_{i=1}(x_i-\bar{x})(Y_i-\bar{Y}) }{\sum^{n}_{i=1}(x_i-\bar{x})^2 }
	\end{equation}
	Both estimator are consistent and unbiased. Moreover:
	\begin{equation}
		\var{\hat{\beta}} = \frac{\sigma^2}{(n-1)s^2_x}; \var{\hat{ \alpha }}=\sigma^2\left(\frac{1}{n}+\frac{\bar{x}^2}{(n-1)s^2_x}\right)
	\end{equation}
	Where $\sigma^2=\var{\epsilon_i}=\var{Y_i}$ and $s_x^2= \frac{\sum (x_i-x)^2 }{n-1}$ is the sample variance of the explanatory varaible.

	\paragraph{Assumption of Linear Regression}
	\begin{enumerate}
		\item Linearity of Expectation: $\bb{E}[Y_i]=\alpha + \beta x_i$ (residue vs fitted)
		\item Homoscedasticity (same variance): $\var{ \epsilon_i }=\sigma^2$. (Scale location plot [Transformation of standardized residue vs fitted value] $\sqrt{e_i^*}$ vs $\hat{y}$)
		\item Independence: $\epsilon_i$ and $x_i$ are independent
		\item Normality: $\epsilon_i$ is normally distributed. 
	\end{enumerate}

	\begin{definition}[Residue Sum of Square]
	The residue sun of square, SSE, is defined as:
	\begin{equation}
		SSE=\sum^{n}_{i=1}(y_i-\hat{y}_i)^2
	\end{equation}
	Its unbiased estimator is $s^2_e = \frac{SSE}{n-2}	=\frac{1}{n-2}\sum^{n}_{i=1}(y_i-\hat{y}_i)^2$. (It is unbiased)
	The residue standard deviation is $s_e=\sqrt{s^2_e}$.
	The standard error of least-square estimate is 
	\begin{equation}
		\hat{S.E.}(\hat{\alpha})=s_e\sqrt{\frac{1}{n}+\frac{\bar{x}^2}{(n-1)s^2_x}}; \text{and } \hat{S.E.}(\hat{\beta})=\frac{s_e}{\sqrt{(n-1)s^2_x}}
	\end{equation}
\end{definition}

\paragraph{Distribution of Least Square Estimators}

Distribution of $\hat{\alpha}$ and $\hat{beta}$ given $\sigma$ is:
\begin{equation}
	\hat{\alpha} \sim N(\alpha, \sigma^2\left(\frac{1}{n}+\frac{\bar{x}^2}{(n-1)s^2_x}\right) \text{ and }
	\hat{\beta} \sim N(\alpha, \frac{\sigma^2}{(n-1)s^2_x})
\end{equation}

Distribution of residue variance estimator is:
\begin{equation}
	\frac{(n-2)S^2_E}{\sigma^2} \sim \chi^2_{n-2}
\end{equation}

Both $\hat{\alpha}$ and $\hat{\beta}$ are independent of $S^2_E$, thus their marginal distribution is:
\begin{equation}
	\frac{\hat{\alpha}-\alpha}{\hat{S.E.}(\hat{\alpha})} \sim t_{n-2} \text{ and }
	\frac{\hat{\beta}-\beta}{\hat{S.E.}(\hat{\beta})} \sim t_{n-2}
\end{equation}

\paragraph{Confidence and Prediction Interval}
The confidence interval is the estimate of the line of the best fit $\alpha + \beta x$ at a given $x_0$.
\begin{equation}
	\hat{\alpha} + \hat{\beta}x_0 \pm t_{n-2, \alpha/2} \cdot \sqrt{s^2_e\left(\frac{1}{n}+\frac{(x_0-\bar{x})^2}{(n-1)s^2_x}\right)}
\end{equation}
Prediction interval for $Y_0$ is a confidence interval for a future observation, and it is wider than the confidence interval.
\begin{equation}
	\hat{\alpha} + \hat{\beta}x_0 \pm t_{n-2, \alpha/2} \cdot \sqrt{s^2_e\left(1+\frac{1}{n}+\frac{(x_0-\bar{x})^2}{(n-1)s^2_x}\right)}
\end{equation}

\section{ANOVA}
\paragraph{Multiple Linear Regression}
Consider the multiple linear regression model $\ud{Y}=\bm{X}\ud{\beta}+\epsilon$, where $\ud{Y}$ is the response vector, $\bm{X}$ is the $n\times p$ design matrix ($p$=\#of explanatory variable), $\ud{\beta}$ is the parameter vector, and $\epsilon$ is the error vector.
The least square estimator is $\hat{\beta}=(\bm{X}^T\bm{X})^{-1}\bm{X}^T\ud{y}$.
The distribution of $\hat{\beta_j}$ is: $\frac{\hat{\beta_j}-\beta_j}{\ste{\hat{\beta}_j}} \sim t_{n-p}$. 

\subsection{ANOVA \textit{F}-Test}
\paragraph{Residue Sum of Square, RSS}
Consider the model $Y_i \sim N(\beta_1x_{i,1}+\cdots+\beta_{p_0}x_{i, p_0}+\cdots+ \beta_{p_1}x_{i, p_1})$
\[RSS_1=\sum (Y_i-\hat{y})^2=(n-p_1)S^2_1\], where $S^2_1$ is the variance residue estimator.
\begin{definition}[ANONA \textit{F}-Test]
	Consider the linear regression model $\ud{Y} = \bm{A} \ud{\beta}+\ud{\epsilon}$ where $\epsilon \sim N(0, \sigma) $. 
	We want to access whether some explanatory variable have negligible effect on the response variable.
	\[H_0: \beta_{p_0+1} + \cdots + \beta_{p_1}=0; \text{vs } \exists j \in \{p_{0}+1, \cdots p_1\}, \beta j \neq 0 \]
	The F-test statistic under $H_0$ is:
	\[F=\frac{(RSS_0-RSS_1)/(p_1-p_0)}{RSS_1/(n-p_1)} \sim F_{p_1-p_0, n-p_1}\]
	At $\alpha$ significance level, the critical region is $C=[F_{p_1-p_0, n-p_1; \alpha}, \infty]$
	The critical value can be computed by R as \texttt{qf(1-$\alpha$, p1-p0, n-p1)}.
\end{definition}

\begin{table}[ht]
	\centering
	\caption{ONE-WAY ANOVA Table (k treatments and total n samples)}
	\begin{tabular}{|c|c|c|c|c|}
			 & DF & RSS & Mean Sum Sq & F-Value \\
			\hline
			Categorical Variable & $k-1$ & $RSS_B=RSS_Y-RSS_E$ & $S^2_B=\frac{RSS_B}{k-1}$ & $F=S^2_B/S^2_E$\\
			Residue & $n-k$ & $RSS_E$ & $S_E^2=\frac{RSS_E}{n-k}$ & \\
			Total & $n-1$ & $RSS_Y$ & $S_Y^2=\frac{RSS_Y}{n-1}$& \\
	\end{tabular}
\end{table}

\begin{table}[ht]
	\centering
	\caption{TWO-WAY ANOVA Table (b blocks, k treatments and total n samples)}
	\begin{tabular}{|c|c|c|c|c|}
			 & DF & RSS & Mean Sum Sq & F-Value \\
			\hline
			Treatments & $k-1$ & $RSS_T$& $S^2_T=\frac{RSS_T}{k-1}$ & $F=S^2_T/S^2_E$\\
			Blocks & $b-1$ & $RSS_B$ & $S_B^2=\frac{RSS_B}{b-1}$ & $F=S^2_B/S^2_E$\\
			Residue & $n-k-b+1$ & $RSS_E$ & $S_E^2=\frac{RSS_E}{n-k-b+1}$ & \\
			Total & $n-1$ & $RSS_Y$ & $S_Y^2=\frac{RSS_Y}{n-1}$& \\
	\end{tabular}
\end{table}

\section{Common Distribution}
\begin{table}[h]
	\centering
	\caption{Common Distribution For Discrete Random Variable}
	\begin{tabular}{c|c|c|c|c}
		Distribution &  parameter & PMF & $\bb{E}[X]$ & $\var{X}$\\
		\hline
		Bernoulli & Probability $p$ & $p^x(1-p)^{1-x}$ & $p$ & $p(1-p)$ \\
		\hline
		Binomial & sample size, $n$, probability $p$ & $\binom{n}{x}p^x(1-p)^{n-x}$ & $np$ & $np(p-1)$ \\
		\hline
		Poisson & Rate, $\lambda>0$ & $\frac{\lambda^x}{x!}e^{-\lambda}$ & $\lambda$  & $\lambda$\\
		\hline
		Geometric & $p$ & $(1-p)^{x-1}p$ & $\frac{1}{p}$ & $\frac{1-p}{p}$ \\
	\end{tabular}
\end{table}

\end{document}
