\documentclass[12pt,a4paper]{article}
\usepackage[utf8]{inputenc}
\usepackage{amsthm, blindtext, titlesec, thmtools, amsmath, amsfonts, scalerel, amssymb, graphicx}

\newtheorem{theorem}{Theorem}[subsection]
\newtheorem{lemma}[theorem]{Lemma}
\newtheorem{corollary}[theorem]{Corollary}
\newtheorem{hypothesis}{Hypothesis}

\theoremstyle{definition}
\newtheorem{definition}{Definition}[section]

\title{Probability Note}
\author{Harry Han}
\date{Dec 2022}
\begin{document}

\maketitle
\tableofcontents
\newpage

\section{Basics Definitions}
\begin{definition}[Sample Spaces]
	A sample space is a set of all possible outcomes of an experiment.
\end{definition}

\begin{definition}[Events]
	An event is a subset of the sample space.
\end{definition}

\begin{definition}[Probability Mass Function]
	Let 
	$ \mathbb{S} = \{x_1, x_2, \dots \}$ be a sample space and domain of function $F.$
	$F$ is its probability mass function provided that: $0<F(x_i)<1$ and $\sum_{i=1} F(x_i) = 1$.
\end{definition}

\begin{definition}[Probability of an Event]
	Let $A$ be an event and $F$ be its probability mass function.
	Then $P(A) = \sum_{x_i \in A} F(x_i)$.
\end{definition}

\begin{definition}[Independence of an Event]
	Let $A$ and $B$ be two events.
	Then $A$ and $B$ are independent if $P(A \cap B) = P(A)P(B)$.
\end{definition}

\begin{definition}[Random Variable]
	A random variable is a function on sample space.
\end{definition}

\begin{definition}[Idenpendence of Random Variable]
	Two random variables $X$ and $Y$ are independent if and only if $P(X=x, Y=y) = P(X=x)P(Y=y)$ for all $x$ and $y$.
\end{definition}

\begin{definition}[Expected Value of a Random Variable]
	Let $X$ be a random variable.
	Its expected value is $E(X)= \sum_i P(x_i)x_i$	
\end{definition}

	
\end{document}
