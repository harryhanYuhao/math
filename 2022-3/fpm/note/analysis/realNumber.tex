\documentclass[../note.tex]{subfiles}
\begin{document}
\section{Real Number}
\subsection{Axioms}
\begin{axiom}[Archimedean Property]
	$\forall r \in \bb{R}, \exists n \in \bb{N}$ such that $n>a$.
\end{axiom}
\begin{axiom}[The Completeness of Real Number] 
	Let $\bb{D}\subseteq \bb{R}$. If $\bb{D}$ is bounded, there exists $s\&i, \in \bb{R}$ such that they are the supremum and infimum of $\bb{D}$.
\end{axiom}

\subsection{The Countable Sets}

\begin{axiom}[The "Smallest" Infinite Set]
	A set $\mathbb{S}$ is infinite iff $\mathbb{S} \succ \mathbb{N}$. 
\end{axiom}

\begin{remark}
	Although FPM is a pure mathematic class with emphasis on rigor, no rigorous definition for the infinite set has been proposed. This definition/axiom is of my own conception.
\end{remark}

\begin{definition}[Countable Set]
	A set $\mathbb{S}$ is countable iff $\mathbb{N} \asymp \mathbb{S}$ (there exists a bijection $f:\mathbb{N} \rightarrow \mathbb{S}$).
\end{definition}

\begin{theorem}[At Most Countable]
	Let $\mathbb{A}$ be an infinite set. \\($\mathbb{A} \prec \mathbb{N}$) iff ($\mathbb{A} \asymp \mathbb{N}$).
\end{theorem}

\begin{proof}
	We want to prove $\mathbb{A} \prec \mathbb{N}$ is equivalent to $\mathbb{A} \asymp \mathbb{N}$. $\mathbb{A} \asymp \mathbb{N} \rightarrow \mathbb{A} \prec \mathbb{N}$ is by definition. 
	We only need to prove the other direction; i.e., provided $\mathbb{A} \prec \mathbb{N}$, find a bijective function $h: \mathbb{A} \rightarrow \mathbb{N}$.

	Let $f: \mathbb{A} \rightarrow \mathbb{N}$ be an injective mapping. If $f$ is bijective, we are done. If $f$ is injective but not bijective, let $\mathbb{N}^-$ be the range of $f.$ As $\mathbb{A}$ is infinite, $\mathbb{N}^-$ is also infinite.  
	Let $f': \mathbb{A} \rightarrow \mathbb{N}^-$ such that $f(a) = f'(a).$ $f'$ is an bijective mapping. 
	
    Thus we only need to show there exists a mapping $g: \mathbb{N}^- \rightarrow \mathbb{N}$ that is bijective. 

	$g$ can be constructed by such: sort $\mathbb{N}^-$ and $\mathbb{N}$ in ascending order. Let the first element in the sorted $\mathbb{N}^-$ maps to the first in the sorted $\mathbb{N}$, the secound to secound, etc.
	As $\mathbb{N}^-$ is infinite, $g$ must be bijective.

	Indeed $h = g \circ f': \mathbb{A} \rightarrow \mathbb{N}$ is the bijective mapping we seek. 
\end{proof}

\begin{theorem}[List of Countable and Uncountable Sets]
	Any of the following sets are countable.
	\begin{enumerate}
		\item $\mathbb{Z}, \mathbb{Q}$ 
		\item Any infinite subset of countable sets.
		\item Any Unions of countable and finite sets.
		\item Any products of countable sets and finite sets. i.e., if $\mathbb{S}, \mathbb{T}$ are countable, $\{\mathbb{S}\times \mathbb{S}\}, \{\mathbb{S}\times \mathbb{T}\times \dots \times \mathbb{S} \}$ are also countable.
	\end{enumerate}
\end{theorem}

\begin{hypothesis}
Is the product of countable number of countable sets countable? (Proposed Feb 6)
\end{hypothesis}

\end{document}
