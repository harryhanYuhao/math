\documentclass[../note.tex]{subfiles}
\begin{document}
\section{Real Number}

\subsection{Algebraic Structures of Real Number}

\axiom{Real Number is a Field}{
	Real Number is a field denoted by $\bb{R}$ with operation addition and multiplication.
	It is a group under both addtion and multiplication. Moreover, distributive law holds. Namely,
	\begin{enumerate}
		\small
		\item F1: Closure. $\forall a,b \in \bb{R}, a+b \in \bb{R}, ab \in \bb{R}$
		\item F2: Associativity. $\forall a,b,c \in \bb{R}, (a+b)+c = a+(b+c), (ab)c = a(bc)$
		\item F3: Commutativity. $\forall a,b \in \bb{R}, a+b = b+a, ab = ba$
		\item F4: Existance pf Additive Identity. There exist $0\in \bb{R}$ such that $\forall a \in \bb{R}, a+0 = a$
		\item F5: Existance of Multiplicative Identity. There exist $1\in \bb{R}$ such that $\forall a \in \bb{R}, a \cdot 1 = a$
		\item F6: Existance of Additive Inverse. $\forall a \in \bb{R}, \exists b \in \bb{R}$ such that $a+b = 0$
		\item F7: Existance of Multiplicative Inverse. $\forall a \in \bb{R}, a \neq 0, \exists b \in \bb{R}$ such that $ab = 1$
		\item F8: Distributivity. $\forall a,b,c \in \bb{R}, a(b+c) = ab+ac, (a+b)c = ac+bc$
	\end{enumerate}
}

\axiom{Order Axiom}{
	There is a relation $<$ on $\bb{R}$ with the following properties:
	\begin{enumerate}
		\item O1: Trichotomy Property: one and only one of the following holds: \[a<b, a=b, a>b\]
		\item Transitive Property: If $a<b$ and $b<c$, then $a<c$.
		\item Additive Property: If $a<b$ and $c<d$, then $a+c < b+d$.
		\item Multiplicative Properties: $a<b, c>0 \implies ac<bc$ and $a<b, c<0 \implies ac>bc$.
	\end{enumerate}
}

% TODO the definition is quite confusing
\definition{Z, N}{
	Define function $s:\bb{R} \rightarrow \bb{R} $ as $s(n)=n+1$, and $p:\bb{R} \rightarrow \bb{R}$ as $p(n)=n-1$. The set $\bb{N}_0$ can be constructed as follows: 
	\begin{enumerate}
		\item $0\in \bb{N}_0$
		\item $\forall n \in \bb{N}_0, s(n) \in \bb{N}_0$
		\item If $s(n)=s(m)$ then $n=m$
		\item There exists not $n\in \bb{N}_0$ such that $s(n)=0$
		\item If $0\in A \subseteq \bb{N}$ and if $n\in A$ implies $s(n) = A$, then $A=\bb{N}_0$
	\end{enumerate}
	
	Define the set $\bb{Z}$ as a superset of $\bb{N}_0$ with the two extra properties:
	\begin{enumerate}
		\item Given $n\in \bb{Z}$ there exists $n \in \bb{Z}$ such that $p(n)\in \bb{Z}$.
		\item If $A \subseteq \bb{Z}$ and $s(n)\in A$ if and only if $n\in A$ then $A=\bb{Z}$.
	\end{enumerate}
}

\definition[Upper and Lower Bound]{
	Let $A \subseteq \bb{R}$ be a non-empty set. An upper bound of $A$ is a real number $b$ such that $a \leq b$ for all $a \in A$. A lower bound of $A$ is a real number $b$ such that $a \geq b$ for all $a \in A$.
	
	A set with an upper bound is called bounded above, and a set with a lower bound is called bounded below. A set with both upper and lower bounds is called bounded.
}

\definition[Supremum \& Infimum]{
	Let $A \subseteq \bb{R}$ be a non-empty set. The supremum of $A$ is the least upper bound of $A$, denoted by $\sup A$. 
	The infimum of $A$ is the greatest lower bound of $A$, denoted by $\inf A$.
}

\begin{axiom}[The Completeness of Real Number] 
	All subsets of real number that is bounded above contains a supremum in the real number.
	Similarly, all subsets of real number that is bounded below contains an infimum in the real number.
\end{axiom}

From these assumed axioms all properties of $\bb{R}, \bb{Z} \& \bb{Q}$ can be derived.

\subsection{Properties of Real Number}

\theorem[Approximation Property of Supremum]{
	If the set $\bb{E} \subseteq \bb{R}$ has a supremum $\sup{\bb{E}}$. Then for every $\epsilon > 0$ there exists $e \in \bb{E}$ such that $\sup{\bb{E}} - \epsilon < e$.
}

\definition{Absolute Value}{
	The absolute value of a real number $x$, denoted as $|x|$ is defined thus:
	\[
		|x|=
	\begin{cases}
		x \text{ if } x \geq 0 \\
		-x \text{ otherwise}
	\end{cases}
\]
}

\theorem{
\
\begin{enumerate}
	\item Absolute Value is multiplicative: $|ab|=|a||b|$
	\item $-|a|\leq a \leq |a|$
	\item $|a|=0 \iff a=0$
	\item $|a-b|=|b-a|$
	\item \emph{Triangular Inequality} $|a+b|\leq |a+b|$; $||a|-|b||\leq |a-b|$
\end{enumerate}
}

\theorem[Archimedean Property]{
	$\forall a,b \in \bb{R},$ and $a<b$, $ \exists n \in \bb{N}$ such that $an>b$.
	Archimedean Property is \emph{not} an Axiom.
}

\subsection{The Countable Sets}

\definition[Injective, Surjective, and Bijective]{
	Let $f:A \rightarrow B$ be a function. $f$ is injective if $\forall a_1, a_2 \in A, f(a_1)=f(a_2) \implies a_1=a_2$. $f$ is surjective if $\forall b \in B, \exists a \in A$ such that $f(a)=b$. $f$ is bijective if it is both injective and surjective.
	Injective is also called one-to-one, and surjective is also called onto.
}

\begin{definition}[Finite, Countable Set, and Uncountable]
	Let $E$ be a set.
	\begin{enumerate}
		\item $E$ is said to be finite if either $E=\emptyset$ or there is an injective function $f:E \rightarrow \{1,2,\dots,n\}$ for some $n \in \bb{N}$.
		\item $E$ is said to be countable if there is a bijective function $f:E \rightarrow \bb{N}$.
		\item $E$ is at most countable if it is finite or countable.
		\item $E$ is uncountable if it is not at most countable.
	\end{enumerate}
\end{definition}

\begin{axiom}[The "Smallest" Infinite Set]
	A set $\mathbb{S}$ is infinite iff $\mathbb{S} \succ \mathbb{N}$. 
\end{axiom}

\theorem[At Most Countable]{ 
	Let $\mathbb{A}$ be an infinite set. ($\mathbb{A} \prec \mathbb{N}$) iff ($\mathbb{A} \asymp \mathbb{N}$).
}

\proof{
	We want to prove $\mathbb{A} \prec \mathbb{N}$ is equivalent to $\mathbb{A} \asymp \mathbb{N}$. $\mathbb{A} \asymp \mathbb{N} \rightarrow \mathbb{A} \prec \mathbb{N}$ is by definition. 
	We only need to prove the other direction; i.e., provided $\mathbb{A} \prec \mathbb{N}$, find a bijective function $h: \mathbb{A} \rightarrow \mathbb{N}$.

	Let $f: \mathbb{A} \rightarrow \mathbb{N}$ be an injective mapping. If $f$ is bijective, we are done. If $f$ is injective but not bijective, let $\mathbb{N}^-$ be the range of $f.$ As $\mathbb{A}$ is infinite, $\mathbb{N}^-$ is also infinite.  
	Let $f': \mathbb{A} \rightarrow \mathbb{N}^-$ such that $f(a) = f'(a).$ $f'$ is an bijective mapping. 
	
    Thus we only need to show there exists a mapping $g: \mathbb{N}^- \rightarrow \mathbb{N}$ that is bijective. 

	$g$ can be constructed by such: sort $\mathbb{N}^-$ and $\mathbb{N}$ in ascending order. Let the first element in the sorted $\mathbb{N}^-$ maps to the first in the sorted $\mathbb{N}$, the secound to secound, etc.
	As $\mathbb{N}^-$ is infinite, $g$ must be bijective.

	Indeed $h = g \circ f': \mathbb{A} \rightarrow \mathbb{N}$ is the bijective mapping we seek. 
}

\theorem[List of Countable and Uncountable Sets]{
	Any of the following sets are countable.
	\begin{enumerate}
		\item $\mathbb{Z}, \mathbb{Q}$ 
		\item Any infinite subset of countable sets.
		\item Any Unions of countable and finite sets.
		\item Any products of countable sets and finite sets. i.e., if $\mathbb{S}, \mathbb{T}$ are countable, $\{\mathbb{S}\times \mathbb{S}\}, \{\mathbb{S}\times \mathbb{T}\times \dots \times \mathbb{S} \}$ are also countable.
	\end{enumerate}
}

\begin{hypothesis}
	Is the product of countable number of countable sets countable? (Proposed Feb 6). \emph{It is not Countable}. (Proved April 21)
\end{hypothesis}

\begin{proof}
	Let $L$ denote the product of countable number of countable sets. 
	The set of $D = \{x\in \bb{R}|x\in (0,1)\}$ , when in decimal expansion, is a subset of $L$. $D$ is uncountable; thus $L$.
\end{proof}

\end{document}
