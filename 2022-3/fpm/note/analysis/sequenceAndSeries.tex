\documentclass[../note.tex]{subfiles}
\begin{document}

\section{Sequence and Series}
\subsection{Sequence}
\begin{definition}[Sequence]
\end{definition}

\begin{definition}[Convergent and Divergent]
\end{definition}

\begin{definition}[Increasing and Decreasing Sequence]
\end{definition}

\begin{definition}[Limit of supremum \& infimum]\label{def:limiSupremum}
	For a sequence $(s_n)$, let $b_i$ denotes the supremum of $\{s_n|n>i\}$.
	If $(b_n)$ converges, the value it converges to is called the limit of supremum of $(s_n)$, and is denoted as $\mathcal{L}_s (s_n)$. 
	The limit of infimum is defined and denoted as $\mathcal{L}_i(s_n)$. (And the series be denoted as $(s_i)$.
	(b for big, s for small)
\end{definition}
\begin{remark}
	$(m_n)$, if exists, is a decreasing series; likewise $(i_n)$ is increasing.
\end{remark}
\begin{hypothesis} 
	A sequence $(s_n)$ converges if and only if $\mathcal{L}_s (s_n) = \mathcal{L}_i(s_n)$. 
	(Proposed Feb 8 2023)
\end{hypothesis}
\begin{definition}[Cauchy Sequence]\cite{Ross}
	A sequence $(s_n)$ is a Cauchy Sequence iff $(\forall \epsilon > 0)(\exists N)(\forall n,m>N)(|s_n-s_m|<\epsilon)$
\end{definition}
\begin{theorem}
A sequence converges if and only if it is a Cauchy Sequence.
\end{theorem}
\begin{proof}
Consider 
\end{proof}
\begin{remark}
	We can define a sudo Cauchy Sequence to be sequence $(s_n)$ such that $(\forall \epsilon > 0)(\exists N)(\forall n>N)(|s_n-s_{n+1}|<\epsilon)$ Indeed all convergent sequence are sudo Cauchy Sequence, but not all sudo Cauchy Sequence are convergent.
\end{remark}

\subsection{Series}
\begin{definition}[Series]
	A series can be expressed as $ \sum^{\infty}_{k=1} a_k$. 
\end{definition}
\begin{definition}[Convergent and Divergent]
	Consider the seires :\\
	$(s_n)=\sum^{n}_{k=1} a_k$. $(s_n)$ is called the partial sum of the series. The series $\sum^{\infty}_{k=1}a_k $ converges if and only if its partial sum converges; otherwise it diverges. 
\end{definition}
\begin{example}
List of Convergent and Divergent series:
\begin{enumerate}
	\item Harmonic Series.
\end{enumerate}
\end{example}

\begin{definition}[Cauchy Criterion]A series befits Cauchy Criterion if and only if its partial sum is a Cauchy Sequence. \end{definition}

\begin{definition}[Absolute Convergent] A series $\sum^{\infty}_{k=1}a_k$ converges absolutely if and only if $\sum^{\infty}_{k=1}|a_k|$ converges. Otherwise it converges non-absolutely
\end{definition}
\begin{theorem}[Convergence Reveries]\label{th:ConvergenceReveries}
\ 
\begin{enumerate}
	\item \label{th:ConvergenceReveries:en:absoluteconverge} Absolute Convergent
		If a series converge absolutely, it converges. The converse is not true. 
	\item Addition, Substraction, Multiplication, Division
		If $\sum^{\infty}_{k=1}a_k $ and $\sum^{\infty}_{k=1}b_k$ converges, the followings also converge:
		$\sum^{\infty}_{k=1}(a_k+b_k) $, $\sum^{\infty}_{k=1}(a_k-b_k) $ $\sum^{\infty}_{k=1}(a_k\cdot b_k)$
\end{enumerate}
\end{theorem}

% \begin{proof}
% 	To prove \ref{th:ConvergenceReveries:en:absoluteconverge} of theorem \ref{th:ConvergenceReveries}
% \end{proof}


\end{document}
