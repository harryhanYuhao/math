\documentclass[../note.tex]{subfiles}
\begin{document}

\section{Real Functions}

\subsection{Continuity}

\begin{definition}[Continuity of a Function]\label{def:Continuity}
	Function $f: \bb{D} \rightarrow \bb{R}$ (provided $\bb{D} \subseteq \bb{R}$) is continuous at $a$
	if and only if for all sequence $(x_i)$ (provided $x_i \in \bb{D}$) that converges to $a$ the sequence $(f(x_i))$ converges to $f(a)$.

	If the function $f$ is continuous for all $a \in \bb{A}$, we say $f$ is continous on the interval $\bb{A}$.
\end{definition}

\begin{theorem}[Equivalent Defintion]
	The definition \ref{def:Continuity} of function $f:\mathbb{D} \rightarrow \bb{R}$ is continuous at $a\in \bb{D}$ is equivalent to $\lim _{x\to a}f(x) = f(a)$, provided $\bb{D}$ is a non-degenerate interval in $\bb{R}$; 
	i.e. $\forall \epsilon \in \bb{D} \exists \delta $ such that $|x-a| < \delta \implies |f(x)-f(a)| < \epsilon$.
\end{theorem}

\begin{definition}[Extreme Value Theorem]
	$f:\bb{I} \rightarrow \bb{R}$ where $\bb{I}$ is a close interval of real number is bounded.
\end{definition}

\begin{proof}
	We shall prove Extreme Value Theorem (EVT) by showing that a continuous function on a finite domain must have finite number of local maxima. Then range of the function would be bounded, as guaranteed by EVT, by the supremum and infinum of the set containing all its local maximum and two values at the end points.

	A real number $f(a)$ is a local maximum of function $f$ if $f(a)>f(x)$ there exists $\delta > 0$ such that for all $x \in(a-\delta, a+\delta), f(a)>f(x)$. 

	Moreover, we claim that for all local maxima $f(a)$ attained at $a$, there exists two intervals $(s, a)$ and $(a, d)$ such that the function is strictly increasing in the interval $(s,a)$ while stricting decreasing in the interval $(a,d)$. 
	The interval $(s,d)$ is called the charateristic interval of local maxima attained at $a$, and $\lambda = (d-a)$ is its charateristic length. The characteristic intervals are disjoint.
	
	Assuming function $f$ is continuous in the domain $[l, h]$ and it achieves infinitely many distinct local maxima in the same domain. Such function would have infinite charateristic intervals, and, for all $\lambda$, the characteristic length of a characteristic intevarl, there exists $0<\lambda '<\lambda$ that is the characteristic length of another characteristic interval.

\end{proof}

\begin{remark}
	Here is the outline of the proof. 
	First difine local maximum as $f(x) > f(y)$ for all $y \in (x-\delta, x+\delta)$ for some $\delta > 0$. 
	
	Then prove the lemme: for all continuous function with finite domain, it will have finite local maximums. EVT follows immediately.

	Note we have not restrict the domain of function $f$ to be real number. EVT holds for rational number, irrational number, and many other.

	However, extreme value theorem does not hold when $\bb{I}$ is a open interval. For example, function $f(x)=\frac{1}{x}$ is continous but unbounded in the open interval $(0,1)$.
\end{remark}

\begin{hypothesis}[Valid Domain for EVT]
	Investigate the valid domain for EVT under the definition of continuity. 
\end{hypothesis}

\begin{theorem}[Intermediate Value Theorem]
	Let $f: \bb{D} \rightarrow \bb{R}$ be continuous on the interval $[a,b]\in \bb{D}$. $\forall c $ such that $f(a)<c<f(b)\ \exists d \in [a,b] $ such that $ f(d) = c$.  
\end{theorem}

\begin{proof}[Proof of Intermediate Value Theorem]
	Construct a set $\bb{E} = \{e\in [a,b]| f(e)<c\}$. Since $f(a)<c$, $\bb{E}$ is non empty.  
	As $\bb{E}$ is bounded, by the completeness of real number, there exists a supremum , which shall be denoted as $\sup{\bb{E}}=s$. 
	As $\bb{E}$ is bounded by a close inteval, $s$ is also bounded by the same intevral. Our aim is to show that $f(s)=c$.

	By our assumption $f$ is continous at $s$. Assuming $f(s) < c$. 
	Since $f$ is continous as $s$, there exists $\delta \in \bb{R}$ such that for all $s<x<\delta$, $|f(x)-f(s)|<c-f(s)$, i.e., $f(s)<f(x)<c$, contradicting our assumption that $s$ is the supremum of $\bb{E}$. 
	Thus we conlude $f(s)\geq c$. $f(s)\leq c$ can be proved similarly.
\end{proof}

\begin{remark}
	The sequence definition of continuity sets no restrains on the domains of the function. Indeed by this definition all discrete functions are continuous: contradicting our intuition about continuity.
	
	The more important consequence of continuity relies heavily on the properties of real number, i.e, the completeness of real number.
\end{remark}

\subsection{Bizare Functions}
\begin{hypothesis}
	\ 
\begin{enumerate}
	\item Is there a function $f: \bb{R} \rightarrow \bb{R}$ such that for all invertal $[a,b] \in \bb{R}$, $f$ is neither strictly increasing or decreasing? 

\end{enumerate}
\end{hypothesis}
\subsubsection{Unbounded Functions}
Commonly, unbounded functions are defined in a open inteval, e.g., $\frac{1}{x}$ in the interval $(0, 1]$. By extreme value theorem we know that there is no continuous function in close interval that is unbounded. However, there is function that is non-continuous that shows unbounded behaviour in close inteval.

Consider the function $f:[0,1] \rightarrow (\infty, 1]$ defined as thus:
		\[
    f(x)= 
\begin{cases}
	2^{n+1},& \text{if } x=2^{-n} \text{ for some natural number }n\\
    x,              & \text{otherwise}
\end{cases}
\]
\end{document}
