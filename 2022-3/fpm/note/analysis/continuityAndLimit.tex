\documentclass[../note.tex]{subfiles}
\begin{document}

\section{Real Functions}

\subsection{Continuity}

\begin{definition}[Continuity of a Function]\label{def:Continuity}
	Function $f: \bb{D} \rightarrow \bb{R}$ (provided $\bb{D} \subseteq \bb{R}$) is continuous at $a$
	if and only if for all sequence $(x_i)$ (provided $x_i \in \bb{D}$) that converges to $a$ the sequence $(f(x_i))$ converges to $f(a)$.

	If the function $f$ is continuous for all $a \in \bb{A}$, we say $f$ is continous on the interval $\bb{A}$.
\end{definition}

\begin{theorem}[Equivalent Defintion]
	The definition \ref{def:Continuity} of function $f:\mathbb{D} \rightarrow \bb{R}$ is continuous at $a\in \bb{D}$ is equivalent to $\lim _{x\to a}f(x) = f(a)$, provided $\bb{D}$ is a non-degenerate interval in $\bb{R}$; 
	i.e. $\forall \epsilon \in \bb{D} \exists \delta $ such that $|x-a| < \delta \implies |f(x)-f(a)| < \epsilon$.
\end{theorem}

\begin{definition}[Extreme Value Theorem]
	$f:\bb{I} \rightarrow \bb{R}$ where $\bb{I}$ is a close interval of real number is bounded and would attain supremum and infimum on $\bb{I}$.
\end{definition}

\begin{proof}
	We shall first show that a continuous function whose domain is a closed interval must have finite number of local maxima, and thus bounded above. Its maximum is the maximum of the set consisted of all local maxima and its value at the end points. 
	\footnote{
		Indeed a continuous function in a closed interval may have 0 local maximum or minimum: in such cases the statement still holds, though it requires more justifications.
	}

	In this proof let $f(x)$ be a continous function defined in a closed interval $[l, h]$.

	A real number $r = f(a)$ is a local maximum of function $f$ if there exists $\delta > 0$ such that for all $x \in(a-\delta, a+\delta), f(a)>f(x)$. 

	Moreover, we claim that for all local maxima $f(a)$ attained at $a$, there exists a interval $(s, a)$  such that the function is strictly increasing in the interval $(s,a)$, but not so for any interval $(s', a)$ with $s'<s$. 
	There also exists another interval $(a, d)$ such that the function is stricting decreasing in the interval but not so for any interval $(a. d')$ with $d<d'$.

	The interval $(s,d)$ is called the charateristic interval of local maxima attained at $a$, and $\lambda = (d-a)$ is its charateristic length. 
	We claim the characteristic intervals partitions $[l,h]$.
	Define function $\gamma: [l,h] \rightarrow \bb{R}$ as $\gamma(x) = \lambda$ if $x\in (a,d)$. Importantly, $\exists \beta > 0$ such that $\exists \zeta \in [x-\lambda,x+\lambda]$ and $|f(x)-f(\zeta)|>\beta$.
	
	If $f(x)$ attains infinitely many local maxima, we claim that $\exists \alpha \in [l, h]$ such that $\displaystyle \lim_{x\to \alpha} \gamma (x) = 0$.
	Thus there exists $\beta > 0$ such that for all $\delta>0$ there exists $0<\lambda<\delta$ such that there exists $\zeta \in [\alpha-\lambda, \alpha+\lambda]$ and $|f(\alpha)-f(\zeta)|>\beta$, which means $f$ is discontinuous at $\alpha$, contridicting our assumption, and we can conclude that $f$ is indeed bounded above.

	So far we have proved that a function that is continuous in a closed interval must be bounded above. It is similar to prove that it is bounded below.
\end{proof}
% TODO: The proof is a little sketchy, need to be improved.
\begin{remark}
	We have not restrict the domain of function $f$ to be real number to show it is bounded, although it may be required if we need to show it would attains sup and inf.

	Notice, extreme value theorem does not hold when $\bb{I}$ is a open interval. For example, function $f(x)=\frac{1}{x}$ is continous but unbounded in the open interval $(0,1)$.
\end{remark}

\begin{corollary}
	Let function $f: \bb{D} \rightarrow	\bb{R}$ be continuous on the inteval $I=[a,b]$. For all $\epsilon \in I$, there exists a subinterval $S\subseteq I$ and $\epsilon \in S$ such that in $S$ the function $f$ is either strictly increasing, decreasing, or remain constant.
\end{corollary}

\begin{hypothesis}[Valid Domain for EVT]
	Investigate the valid domain for EVT under the definition of continuity. 
\end{hypothesis}

\begin{theorem}[Intermediate Value Theorem]
	Let $f: \bb{D} \rightarrow \bb{R}$ be continuous on the interval $[a,b]\in \bb{D}$. $\forall c $ such that $f(a)<c<f(b)\ \exists d \in [a,b] $ such that $ f(d) = c$.  
\end{theorem}

\begin{proof}[Proof of Intermediate Value Theorem]
	Construct a set $\bb{E} = \{e\in [a,b]| f(e)<c\}$. Since $f(a)<c$, $\bb{E}$ is non empty.  
	As $\bb{E}$ is bounded, by the completeness of real number, there exists a supremum , which shall be denoted as $\sup{\bb{E}}=s$. 
	As $\bb{E}$ is bounded by a close inteval, $s$ is also bounded by the same intevral. Our aim is to show that $f(s)=c$.

	By our assumption $f$ is continous at $s$. Assuming $f(s) < c$. 
	Since $f$ is continous as $s$, there exists $\delta \in \bb{R}$ such that for all $s<x<\delta$, $|f(x)-f(s)|<c-f(s)$, i.e., $f(s)<f(x)<c$, contradicting our assumption that $s$ is the supremum of $\bb{E}$. 
	Thus we conlude $f(s)\geq c$. $f(s)\leq c$ can be proved similarly.
\end{proof}

\begin{remark}
	The sequence definition of continuity sets no restrains on the domains of the function. Indeed by this definition all discrete functions are continuous: contradicting our intuition about continuity.
	
	The more important consequence of continuity relies heavily on the properties of real number, i.e, the completeness of real number.
\end{remark}

\subsection{Bizare Functions}
\begin{hypothesis}
	\ 
\begin{enumerate}
	\item Is there a defined in close interval $[a,b]$ but is not strictly increasing nor decreasing for any interval in its domain? 
	\item Is there a function that is continuous everywhere but not differentiable at any point?
\end{enumerate}
\end{hypothesis}
\paragraph{Non Increasing Nor Decreasing}
Let function $f: \bb{R}\rightarrow \bb{R}$ be defined as such:
\[
	f(x)=
	\begin{cases}
		1 &\text{if } x \in \bb{Q}\\
		0 &\text{otherwise}
	\end{cases}
\]
As infinitely many number of rational numbers and irrational number are contained in any domain $[a,b]$, such funtion is not increasing nor decreasing for any intervals of real number.

\paragraph{Unbounded Functions}
Commonly, unbounded functions are defined in a open inteval, e.g., $\frac{1}{x}$ in the interval $(0, 1]$. By extreme value theorem we know that there is no continuous function in close interval that is unbounded. However, there is function that is non-continuous that shows unbounded behaviour in close inteval.

Consider the function $f:[0,1] \rightarrow (\infty, 1]$ defined as thus:
\[
    f(x)= 
\begin{cases}
	2^{n+1},& \text{if } x=2^{-n} \text{ for some natural number }n\\
    x,              & \text{otherwise}
\end{cases}
\]


\end{document}

