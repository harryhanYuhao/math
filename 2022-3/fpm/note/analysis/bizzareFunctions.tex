\documentclass[../note.tex]{subfiles}
\begin{document}
\section{Bizare Functions}
\begin{hypothesis}
	\ 
\begin{enumerate}
	\item Is there a defined in close interval $[a,b]$ but is not strictly increasing nor decreasing for any interval in its domain? 
	\item Is there a function that is continuous everywhere but not differentiable at any point?
	\item  
\end{enumerate}
\end{hypothesis}
\paragraph{A Continuous Function in Finite Domain with Infinite Local Maxima} \label{InfiniteLocalMaxima}
Define function $f:\bb{R}\rightarrow \bb{R}$ as such:
\[
	f(x)=
	\begin{cases}
		x\sin{\frac{1}{x}} & \text{if } x\neq 0\\
		0 & \text{if } x = 0
	\end{cases}
\]

It has infinitely many local maxima in the inteval $[-1,1]$.

Notice the function is continous at $0$. Morever, for $\forall \delta > 0$, $[-\delta, \delta]$ contains infinitly many local maxima. Denominate points with such property as transedental points. It is possible to have a function defined in domain $[a,b]$ while at same time have $\forall \epsilon \in [a,b], \epsilon$ is a transedental point? Can such function be continous? 
\paragraph{Non Increasing Nor Decreasing}
Let function $f: \bb{R}\rightarrow \bb{R}$ be defined as such:
\[
	f(x)=
	\begin{cases}
		1 &\text{if } x \in \bb{Q}\\
		0 &\text{otherwise}
	\end{cases}
\]
As infinitely many number of rational numbers and irrational number are contained in any domain $[a,b]$, such funtion is not increasing nor decreasing for any intervals of real number.

\paragraph{Unbounded Functions}
Commonly, unbounded functions are defined in a open inteval, e.g., $\frac{1}{x}$ in the interval $(0, 1]$. By extreme value theorem we know that there is no continuous function in close interval that is unbounded. However, there is function that is non-continuous that shows unbounded behaviour in close inteval.

Consider the function $f:[0,1] \rightarrow (\infty, 1]$ defined as thus:
\[
    f(x)= 
\begin{cases}
	2^{n+1},& \text{if } x=2^{-n} \text{ for some natural number }n\\
    x,              & \text{otherwise}
\end{cases}
\]


\paragraph{Riemann Function} \label{RiemannFunction} \hypertarget{RiemannFunction}{}
Consider function $f$ defined as such:
\[
	f(x)=
	\begin{cases}
		0 & \text{if } x \text{ is irrational or } x=0 \\
		\frac{1}{q} & \text{if } x=\frac{p}{q} \text{ for some } p \in \bb{Z}; q \in \bb{N} \text{ with } gcf(q, p)=1  \\ 
	\end{cases}
\]

Such function is continuous at every irrational point but not continuous at rational point.

\end{document}
