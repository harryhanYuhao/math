\documentclass[../note.tex]{subfiles}
\begin{document}
\section{Differentiability}
\begin{definition}[Differentiability]
	A real function $f$ is differntiable at point $a \in \bb{R}$ if $f$ is defined at some open interval containing $a$ and 
	\begin{equation}
		f'(a) = \lim_{x \to a} \frac{f(x)-f(a)}{x-a}
	\end{equation}
	exists. $f'(a)$ is called the derivative of $f$ at the point $a$.
\end{definition}

\begin{theorem}[Differentiability Implies Continuity]
	Let $f$ be a real function. If $f$ is differentiable at point $a$, then $f$ is continuous at $a$.
\end{theorem}

\begin{proof}
	Let $f$ be differentiable at $a$. 
	\[
		\lim_{x \to a}f(x)=
		\lim_{x \to a} \underbrace{\frac{f(x)-f(a)}{x-a}}_{Differentiability}
		\underbrace{(x-a)}_{0}
		+f(a) 
		= f(a)
	\]
\end{proof}

\begin{theorem}[Rolle's Theorem]
	Let $f$ be a real function. If it is continuous on the interval $[a,b]$, differentiable on $(a,b)$, and $f(a)=f(b)$, then for some $c \in (a,b)$ $f'(c)=0$.
\end{theorem}
\begin{proof}
	We claim the the derivative of the function is zero at its local extrema garunteed by the Extreme Value Theorem.
\end{proof}

\begin{theorem}[Mean Value Theorem]
	Let $f$ be a real function. If it is continuous on the interval $[a,b]$, differentiable on $(a,b)$, then for some $c \in (a,b)$ $f'(c)=\frac{f(b)-f(a)}{b-a}$.
\end{theorem}
\begin{proof}
	Let function $g$ be defined as $g(x)=f(x)-\frac{f(b)-f(a)}{b-a}(x-a)$. Then $g$ is continuous on $[a,b]$ and $g(a)=g(b)$. At $c$ garunteed by the Rolle's theorem where $g'(c)=0$, $f'(c) = \frac{f(b)-f(a)}{b-a}$.
\end{proof}
\end{document}
