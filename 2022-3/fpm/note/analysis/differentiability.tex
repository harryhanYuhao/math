\documentclass[../note.tex]{subfiles}
\begin{document}
\section{Differentiability}

\subsection{Defintions of Differentiability}

\begin{definition}[Differentiability]
	A real function $f$ is differntiable at point $a \in \bb{R}$ if $f$ is defined at some open interval containing $a$ and 
	\begin{equation}
		f'(a) = \lim_{x \to a} \frac{f(x)-f(a)}{x-a}
	\end{equation}
	exists. $f'(a)$ is called the derivative of $f$ at the point $a$.
\end{definition}

\begin{theorem}[Differentiability Implies Continuity]
	Let $f$ be a real function. If $f$ is differentiable at point $a$, then $f$ is continuous at $a$.
\end{theorem}

\begin{proof}
	Let $f$ be differentiable at $a$. 
	\[
		\lim_{x \to a}f(x)=
		\lim_{x \to a} \underbrace{\frac{f(x)-f(a)}{x-a}}_{Differentiability}
		\underbrace{(x-a)}_{0}
		+f(a) 
		= f(a)
	\]
\end{proof}

\definition[Differentiability]{
	A function is \emph{differentiable} on interval $I$ if its derivative exists and is finite for all $a\in I$.
	A function is \emph{continuously differentiable} on interval $I$ if its derivative exists and is continuous for all $a\in I$.
}

\subsection{Differentiability Theorems}

\theorem{
	Let $f,g$ be real fucntion that is differentiable on interval $I$. Then
	\begin{enumerate}
		\item $\alpha f'(a)=\alpha f(a)$
		\item $(f+g)'(a)= f'(a)+g'(a)$
		\item $(fg)'(a)= f(a)g'(a)+g(a)f'(a)$
		\item $(f/g)'(a)= \frac{f'(a)g(a)-f(a)g'(a)}{g(a)^2}$
	\end{enumerate}
}

\theorem[Chain Rule]{
	Let $f,g$ be real fucntion that is differentiable on interval $I$. Then
	\begin{equation}
		(f\circ g)'(a)=f'(g(a))g'(a)
	\end{equation}
}

\theorem[Power Rule]{
	\begin{equation}
		x^n = \frac{d}{dx}x^n = nx^{n-1}
	\end{equation}
}

\begin{theorem}[Rolle's Theorem]
	Let $f$ be a real function. If it is continuous on the interval $[a,b]$, differentiable on $(a,b)$, and $f(a)=f(b)$, then for some $c \in (a,b)$ $f'(c)=0$.
\end{theorem}
\begin{proof}
	We claim the the derivative of the function is zero at its local extrema garunteed by the Extreme Value Theorem.
\end{proof}

\begin{theorem}[Mean Value Theorem]
	Let $f$ be a real function. If it is continuous on the interval $[a,b]$, differentiable on $(a,b)$, then for some $c \in (a,b)$ $f'(c)=\frac{f(b)-f(a)}{b-a}$.

	\emph{Extended Version}: 
	If $f,g$ are continuous on $[a,b]$ and differentiable on $(a,b)$, then for some $c \in (a,b)$ $f'(c)(g(b)-g(a))=g'(c)(f(b)-f(a))$.
\end{theorem}

\begin{proof}
	Let function $g$ be defined as $g(x)=f(x)-\frac{f(b)-f(a)}{b-a}(x-a)$. Then $g$ is continuous on $[a,b]$ and $g(a)=g(b)$. At $c$ garunteed by the Rolle's theorem where $g'(c)=0$, $f'(c) = \frac{f(b)-f(a)}{b-a}$.
\end{proof}

\theorem[Bernoulli's Inequality]{
	Let $\alpha >0$ and $\delta l\geq -1$. Then 
	\begin{enumerate}
		\item $(1+\delta)^{\alpha} \leq 1+\delta\alpha$ if $\alpha \in (0,1]$,
		\item $(1+\delta)^{\alpha} \geq 1+\delta\alpha$ if $\alpha \in [1,\infty)$,
	\end{enumerate}
}

\end{document}
