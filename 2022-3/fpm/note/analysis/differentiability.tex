\documentclass[../note.tex]{subfiles}
\begin{document}
\section{Differentiability}

\subsection{Defintions of Differentiability}

\begin{definition}[Differentiability]
	A real function $f$ is differntiable at point $a \in \bb{R}$ if $f$ is defined at some open interval containing $a$ and 
	\begin{equation}
		f'(a) = \lim_{x \to a} \frac{f(x)-f(a)}{x-a}
	\end{equation}
	exists. $f'(a)$ is called the derivative of $f$ at the point $a$.
\end{definition}

\begin{theorem}[Differentiability Implies Continuity]
	Let $f$ be a real function. If $f$ is differentiable at point $a$, then $f$ is continuous at $a$.
\end{theorem}

\begin{proof}
	Let $f$ be differentiable at $a$. 
	\[
		\lim_{x \to a}f(x)=
		\lim_{x \to a} \underbrace{\frac{f(x)-f(a)}{x-a}}_{Differentiability}
		\underbrace{(x-a)}_{0}
		+f(a) 
		= f(a)
	\]
\end{proof}

\definition[Differentiability]{
	A function is \emph{differentiable} on interval $I$ if its derivative exists and is finite for all $a\in I$.
	A function is \emph{continuously differentiable} on interval $I$ if its derivative exists and is continuous for all $a\in I$.
}

\subsection{Differentiability Theorems}

\theorem{
	Let $f,g$ be real fucntion that is differentiable on interval $I$. Then
	\begin{enumerate}
		\item $\alpha f'(a)=\alpha f(a)$
		\item $(f+g)'(a)= f'(a)+g'(a)$
		\item $(fg)'(a)= f(a)g'(a)+g(a)f'(a)$
		\item $(f/g)'(a)= \frac{f'(a)g(a)-f(a)g'(a)}{g(a)^2}$
	\end{enumerate}
}

\theorem[Chain Rule]{
	Let $f,g$ be real fucntion that is differentiable on interval $I$. Then
	\begin{equation}
		(f\circ g)'(a)=f'(g(a))g'(a)
	\end{equation}
}

\theorem[Power Rule]{
	\begin{equation}
		x^n = \frac{d}{dx}x^n = nx^{n-1}
	\end{equation}
}

\begin{theorem}[Rolle's Theorem]
	Let $f$ be a real function. If it is continuous on the interval $[a,b]$, differentiable on $(a,b)$, and $f(a)=f(b)$, then for some $c \in (a,b)$ $f'(c)=0$.
\end{theorem}
\begin{proof}
	We claim the the derivative of the function is zero at its local extrema garunteed by the Extreme Value Theorem.
\end{proof}

\begin{theorem}[Mean Value Theorem]
	Let $f$ be a real function. If it is continuous on the interval $[a,b]$, differentiable on $(a,b)$, then for some $c \in (a,b)$ $f'(c)=\frac{f(b)-f(a)}{b-a}$.

	\emph{Extended Version}: 
	If $f,g$ are continuous on $[a,b]$ and differentiable on $(a,b)$, then for some $c \in (a,b)$ $f'(c)(g(b)-g(a))=g'(c)(f(b)-f(a))$.
\end{theorem}

\begin{proof}
	Let function $g$ be defined as $g(x)=f(x)-\frac{f(b)-f(a)}{b-a}(x-a)$. Then $g$ is continuous on $[a,b]$ and $g(a)=g(b)$. At $c$ garunteed by the Rolle's theorem where $g'(c)=0$, $f'(c) = \frac{f(b)-f(a)}{b-a}$.
\end{proof}

\theorem[Bernoulli's Inequality]{
	Let $\alpha >0$ and $\delta l\geq -1$. Then 
	\begin{enumerate}
		\item $(1+\delta)^{\alpha} \leq 1+\delta\alpha$ if $\alpha \in (0,1]$,
		\item $(1+\delta)^{\alpha} \geq 1+\delta\alpha$ if $\alpha \in [1,\infty)$,
	\end{enumerate}
}

\subsection{Monotone Functions}

\definition{
	\ 
\begin{enumerate}

	\item	A function $f$ is \emph{increasing} on interval $I$ if $x_1<x_2 \implies f(x_1) \leq f(x_2)$. It is \emph{strictly increasing} if $x_1<x_2 \implies f(x_1) < f(x_2)$.
	\item A function $f$ is \emph{decreasing} on interval $I$ if $x_1<x_2 \implies f(x_1) \geq f(x_2)$. It is \emph{strictly decreasing} if $x_1<x_2 \implies f(x_1) > f(x_2)$.
	\item A function $f$ is \emph{monotone} on interval $I$ if it is either increasing or decreasing.
	\item A function $f$ is \emph{strictly monotone} on interval $I$ if it is either strictly increasing or strictly decreasing.
\end{enumerate}
}

\remark{
	Note that for a function $f: \bb{R} \rightarrow \bb{R}$ $x_1<x_2 \implies f(x_1) < f(x_2) $
}

\theorem{
	Let $a<b$ be real and $f$ be continuous on $[a,b]$ and differentiable on $(a,b)$. 
	\begin{enumerate}
		\item $f'(x)>0$ for all $x \in (a,b)$ if and only if $f$ is strictly increasing on $[a,b]$.
		\item $f'(x)<0$ for all $x \in (a,b)$ if and only if $f$ is strictly decreasing on $[a,b]$.
		\item $f'(x)=0$ for all $x \in (a,b)$ if and only if $f$ is constant on $[a,b]$.
	\end{enumerate}
}

\theorem[Fermat's Theorem]{
	If a function attains a extrema at $a$, then either $f'(a)=0$ or $f'(a)$ does not exist.
}

\theorem[Darboux's Theorem]{
	If $f$ is differentiable on $(a,b)$, then $f'$ has the intermediate value property; i.e., if $f'(a)<c<f'(b)$, then there exists $x \in (a,b)$ such that $f'(x)=c$.
}

\remark{
	This theorem can not be mistaken as all derivative, if defined, are continuous. See bizzare function.
}

\corollary{
a
}

\theorem[L'Hopital's Rule]{
	Let $a \in \bb{R} \cup \{ \pm \infty \}$ and $f,g$ be real functions. If $A = \lim_{x \rightarrow a} f(x) = \lim_{x \rightarrow a} f(x)$ is either 0 or $\infty$ and $f,g$ are differentiable on $(a,b)\{a\}$, then 
	\begin{equation}
		\lim_{x \rightarrow a} \frac{f(x)}{g(x)} = \lim_{x \rightarrow a} \frac{f'(x)}{g'(x)}
	\end{equation}
}

\definition[Taylor Polynomial]{
	Taylor's polynomial $P_n^{f,x_0}(x)$ of degree $n$ of a function $f$ at $x_0$ is defined as 
	\begin{equation}
		P_n^{f,x_0}(x)=f(x_0)+\sum_{k=1}^{n}\frac{f^{(k)}(x_0)}{k!}(x-x_0)^k 
	\end{equation}

}

\theorem[Taylor's Formula]{
	Let $n\in \bb{N}$ and $a<b$ be real number. 
	If $f:(a,b) \rightarrow \bb{R}$ and $f^{(n+1)}$ exists on $(a,b)$, then for all each $x, x_0 \in (a,b)$ there exists $c \in (a,x)$ such that
	\begin{align}
	`	\begin{split}
			f(x)=f(x_0)+\sum_{k=1}^{n}\frac{f^{(k)}(x_0)}{k!}(x-x_0)^k + &\frac{f^{(n+1)}(c)}{(n+1)!}(x-x_0)^{n+1} = \\
			P_n^{f,x_0}(x)+ &\frac{f^{n+1}(c)}{(n+1)!}(x-x_0)^{n+1}
		\end{split}
	\end{align}
}
\end{document}
