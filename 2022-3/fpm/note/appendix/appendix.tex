\documentclass[../note.tex]{subfiles}
\begin{document}

\renewcommand{\thechapter}{\Roman{chapter}}
\chapter{On Polynomials}
\label{appendix:hypotheises}
\begin{hypothesis}
	Provided $n+1$ points in 2 dimensional spaces,there are exactly one $n$ degree polynomials that passes through all the points, provided that those $n+1$ points are not passed throught by another polynomial of lower degree.

	Extension: how about $n+1$ points in $d$ dimensional spaces? (Proposed 1 Mar 2023)
\end{hypothesis}

\begin{example}
	For points $(0,0), (1,1), (2,4),$ the only 2 degree polynomials that pass throught it is $y=x^2$.

	However, for points $(0,0), (1,1), (2,2)$, there are no degree 2 polynomials that pass through it. 

	This hypothesis is definitely related to linear independence.
\end{example}

\begin{proof}[A Quick proof using Fundamental Theorem of Algebra]

\end{proof}

\chapter{Latin and Abbreviations}
De Mathematica Pura \hfill On Pure Mathematics\\
Caput \hfill Chapter\\
Index Capitis \hfill Index of Chapters\\
Theorema, Theoremae \dotfill  Theorem\\
Definitio, Definitiones \hfill Definition\\
Propositio, Propositiones \hfill Proposition\\
Coniectura, Coniecturae \hfill Conjecture\\
Demonstratio, Demonstrationes \hfill Proof\\
Q.E.D. \hfill Quod Erat Demonstrandum \\
\rightline{Which was to be demonstrated, signify end of proof}\\
\\
\noindent Exampli gratia \hfill For (the sake of) example\\
SDU(sine detrimento universalitatis) \hfill without any loss of generosity\\


\chapter{Chronology of Proposed, Proved, and Disproved Hypotheses}
%TODO fix footnote note showing up in the table
\begin{table}[h!]
\centering
\begin{tabular}{|c|c|c|c|}
	Hypothesis/Theorem & Date of Proposition & Date of Resolvation & Outcome\\
	Theorem \ref{analysis:th:Convergence_and_limit_of_sup_in} & Feb 8, 2023 & Feb 9 & PROVED\\
	Theorem \ref{th:ConvergenceReveries}.\ref{IntegralTest} & Feb 14, 2023 & Feb 17 & PROVED\footnote{\label{note1}With Modification}\\
	Theorem \ref{th:ConvergenceReveries}.\ref{AlternatingSeriesTest} & Feb 14, 2023 &  & \\
	Hypothesis \ref{hypothesis:Finite Riamenn Sum} & Feb 16, 2023 & & \\
	Theorem \ref{th:ConvergenceReveries}.\ref{Raabe's Test} & Feb 17, 2023 & & \\
\end{tabular}
\end{table}
\end{document}
