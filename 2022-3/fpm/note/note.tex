\documentclass[12pt, a4paper]{report}
\usepackage{blindtext, titlesec, amsthm, thmtools, amsmath, amsfonts, scalerel, amssymb, graphicx, titlesec, xcolor}
\usepackage[utf8]{inputenc}
%\hypersetup{frenchlinks=true}
\newtheorem{theorem}{Theorem}[section]
\newtheorem{lemma}[theorem]{Lemma}
\newtheorem{corollary}{Corollary}[section]
\newtheorem{hypothesis}{Hypothesis}
\theoremstyle{definition}
\newtheorem{definition}{Definition}[section]

\theoremstyle{definition}
\newtheorem{axiom}{Axiom}[section]

\theoremstyle{remark}
\newtheorem{remark}{Remark}[section]
\title{Notes on Pure Mathematics}
\author{Harry Han}
\date{\today}
\begin{document}
\maketitle
\tableofcontents

\newpage

\begin{abstract}
	This is my notes when taking the class \emph{Fundamentals of Pure Mathematics} at the University of Edinburgh. These notes do not follow the course lecture notes; instead, they are my thoughts and works inspired by the class.
\end{abstract}
\chapter{Notation}

\begin{itemize}
	\item The \verb|\mathbb{}| fonts are used to denote sets. ($\mathbb{S}, \mathbb{Y},$ etc.)
	\item $\mathbb{A} \succ \mathbb{B}$ denotes there exits a surjective function $f:A\rightarrow B$. $\prec$, $\asymp$ denotes injective, bijective, respectively.
\end{itemize}

\chapter{Analysis}
\begin{axiom}[The "Smallest" Infinite Set]
	A set $\mathbb{S}$ is infinite iff $\mathbb{S} \succ \mathbb{N}$. 
\end{axiom}

\begin{remark}
	Although FPM is a pure mathematic class with emphasis on rigor, no rigorous definition for the infinite set has been proposed. This definition/axiom is of my own conception.
\end{remark}

\begin{definition}[Countable Set]
	A set $\mathbb{S}$ is countable iff $\mathbb{N} \asymp \mathbb{S}$ (there exists a bijection $f:\mathbb{N} \rightarrow \mathbb{S}$).
\end{definition}

\begin{corollary}[At Most Countable]
	Let $\mathbb{A}$ be an infinite set. \\($\mathbb{A} \prec \mathbb{N}$) iff ($\mathbb{A} \asymp \mathbb{N}$).
\end{corollary}

\begin{proof}
	We want to prove $\mathbb{A} \prec \mathbb{N}$ is equivalent to $\mathbb{A} \asymp \mathbb{N}$. $\mathbb{A} \asymp \mathbb{N} \rightarrow \mathbb{A} \prec \mathbb{N}$ is by definition. 
	We only need to prove the other direction; i.e., provided $\mathbb{A} \prec \mathbb{N}$, find a bijective function $h: \mathbb{A} \rightarrow \mathbb{N}$.

	Let $f: \mathbb{A} \rightarrow \mathbb{N}$ be an injective mapping. If $f$ is bijective, we are done. If $f$ is injective but not bijective, let $\mathbb{N}^-$ be the range of $f.$ As $\mathbb{A}$ is infinite, $\mathbb{N}^-$ is also infinite.  
	%TODO: f' is not bijective, construct the correct bijective f'.
	Let $f': \mathbb{A} \rightarrow \mathbb{N}^-$ such that $f(a) = f'(a).$ $f'$ is an bijective mapping. 
	
    Thus we only need to show there exists a mapping $g: \mathbb{N}^- \rightarrow \mathbb{N}$ that is bijective. 

	$g$ can be constructed by such: sort $\mathbb{N}^-$ and $\mathbb{N}$ in ascending order. Let the first element in the sorted $\mathbb{N}^-$ maps to the first in the sorted $\mathbb{N}$, the secound to secound, etc.
	As $\mathbb{N}^-$ is infinite, $g$ must be bijective.

	Indeed $h = f' \circ g: \mathbb{A} \rightarrow \mathbb{N}$ is the bijective mapping we seek. QED. 
\end{proof}

\begin{theorem}[List of Countable and Uncountable Sets]
	Any of the following sets are countable.
	\begin{enumerate}
		\item $\mathbb{Z}$ and any of its infinite subsets.
		\item $\mathbb{R}$ and any of its infinite subsets.
		\item If $\mathbb{S}$ is countable, $\{\mathbb{S}\times \mathbb{S}\}, \{\mathbb{S}\times \mathbb{S}\times \dots \times \mathbb{S}) \}$ are also countable.
	\end{enumerate}
\end{theorem}

\chapter{Algebra}

\begin{definition}[Group]
Group is a set $\mathbb{S}$ with an operation $\odot$ that fulfills the following four properties:
\begin{enumerate}
	\item Closure
	\item Associtivity: $(a \odot b) \odot c = a \odot (b \odot c)$;
	\item Identity
	\item Inverse
\end{enumerate}
\end{definition}
\begin{theorem}[Consequence of the Definition]
There are many unobvious properties that directly follows the definition.
\begin{enumerate}
	\item General Associtivity: Parenthesis does not matter, as long as the order is the same: 
		$a \odot b \odot c \odot d \odot e \odot f \odot g \cdots  = (a \odot ((b \odot c) \odot e (\odot f \odot g)\cdots) = \cdots$
\end{enumerate}
\end{theorem}
\end{document}
