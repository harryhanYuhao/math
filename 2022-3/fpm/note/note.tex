\documentclass[12pt, a4paper]{report}
\usepackage{blindtext, titlesec, amsthm, thmtools, amsmath, amsfonts, scalerel, amssymb, graphicx, titlesec, xcolor, multicol, hyperref}
\usepackage[utf8]{inputenc}
\usepackage{biblatex}
\addbibresource{./reference/reference.bib}
\hypersetup{colorlinks, linkcolor={red!40!black}, citecolor={blue!50!black}, urlcolor={blue!80!black}}
\newtheorem{theorem}{Theorema}[section]
\newtheorem{lemma}[theorem]{Lemma}
\newtheorem{corollary}{Corollarium}[section]
\newtheorem{proposition}{Propositio}[theorem]
\theoremstyle{definition}
\newtheorem{definition}{Definitio}[section]

\theoremstyle{definition}
\newtheorem{axiom}{Axioma}[section]

\theoremstyle{remark}
\newtheorem{remark}{Observatio}[section]
\newtheorem{hypothesis}{Coniectura}[section]
\newtheorem{example}{Exampli Gratia}[section]

\renewenvironment{proof}{{\bfseries\emph{Demonstratio.}}}{\qed}
\renewcommand\qedsymbol{\bfseries\emph{Q.E.D.}}
\renewcommand{\chaptername}{Caput}
\renewcommand{\contentsname}{Index Capitum} % Index Capitum 
\usepackage{subfiles}
\title{
	DE MATHEMATICA PURA \\
	\large On Pure Mathematics
}
\author{Harry Han}
\date{\today}
\begin{document}
\maketitle
\tableofcontents

\newpage

\begin{abstract}
	These are my notes when taking the class \emph{Fundamentals of Pure Mathematics} at the University of Edinburgh. They are not a replicate of the lecture notes: they are my thoughts and explorations. 
	Most importantly, all proofs presented in this document are of my own conception. 

	Terms like ``Theorem, Proposition'' are coined in Latin. As the English terms descended from Latin, most of them are self-explanatory. 
\end{abstract}
\chapter{Notation}
\begin{itemize}
	\item The \verb|\mathbb{}| fonts are used to denote sets. ($\mathbb{S}, \mathbb{Y},$ etc.)
	\item $\mathbb{A} \succ \mathbb{B}$ denotes there exits a surjective function $f:\mathbb{A}\rightarrow \mathbb{B}$. $\prec$, $\asymp$ denotes injective, bijective, respectively.
	\item $e$ is used to denote the identity of a group.
	\item When there is no ambiguity, the notation for the operation of group is ommited. (i.e., $a \odot b = ab$).
		$a^{-1}$ is used to denote the inverse of $a$.
	\item $\mathbb{H}\leq \mathbb{S}$ donotes that $\mathbb{H}$ is a subgroup of $\mathbb{S}$. If $\mathbb{H}\neq \mathbb{S}$, it is a proper subgroup and is denoted as $\mathbb{H}<\mathbb{S}$. See \hyperlink{def:subgroup}{definition}.
	\item Sequence and series are denoted as $(s_n)$ and $\sum^{\infty}_{k=1}s_k$ respectively.
	\item $\mathcal{L}_s(s_n), \mathcal{L}_s(s_n)$ is the limit of supremum \& infimum. See definition \ref{def:limiSupremum}.

\end{itemize}

\chapter{Analysis}
\subfile{./analysis/realNumber.tex}

\subfile{./analysis/sequenceAndSeries.tex}

\chapter{Algebra}

\subfile{./algebra/group.tex}

\appendix
\subfile{appendix/appendix.tex}

\printbibliography
\end{document}
