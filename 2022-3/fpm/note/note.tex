\documentclass[12pt, a4paper]{report}
\usepackage{blindtext, titlesec, amsthm, thmtools, amsmath, amsfonts, scalerel, amssymb, graphicx, titlesec, xcolor, multicol, hyperref}
\usepackage[utf8]{inputenc}
\usepackage{biblatex}
\addbibresource{./reference/reference.bib}
% linktocpage shall be added to snippets.
\hypersetup{colorlinks, linkcolor={red!40!black}, citecolor={blue!50!black}, urlcolor={blue!80!black}, linktocpage}
\newtheorem{theorem}{Theorema}[section]
\newtheorem{lemma}[theorem]{Lemma}
\newtheorem{corollary}{Corollarium}[section]
\newtheorem{proposition}{Propositio}[theorem]
\theoremstyle{definition}
\newtheorem{definition}{Definitio}[section]

\theoremstyle{definition}
\newtheorem{axiom}{Axioma}[section]

\theoremstyle{remark}
\newtheorem{remark}{Observatio}[section]
\newtheorem{hypothesis}{Coniectura}[section]
\newtheorem{example}{Exampli Gratia}[section]

%TODO mayby proof environment shall have more margin
\renewenvironment{proof}{\small{\emph{Demonstratio.}}}{\qed}
% \renewenvironment{proof}{{\bfseries\emph{Demonstratio.}}}{\qed}
\renewcommand\qedsymbol{Q.E.D.}
\renewcommand{\chaptername}{Caput}
\renewcommand{\contentsname}{Index Capitum} % Index Capitum 
\renewcommand{\emph}[1]{\textbf{\textit{#1}}}
\renewcommand{\ker}[1]{\text{ker}{\ #1}}

% New Commands
\newcommand{\bb}[1]{\mathbb{#1}} %TODO add this line to nvim snippets
\newcommand{\orb}[2]{\text{Orb}_{#1}({#2})}
\newcommand{\stab}[2]{\text{Stab}_{#1}({#2})}
\newcommand{\im}[1]{\text{im}{\ #1}}

\usepackage{subfiles}
\title{
	DE MATHEMATICA PURA \\
	\large On Pure Mathematics
}
\author{Harry Han}
\date{\today}
\begin{document}
\maketitle
\tableofcontents

\newpage

\begin{abstract}
	These are my notes when taking the class \textit{Fundamentals of Pure Mathematics} at the University of Edinburgh. They are not a replicate of the lecture notes: they are my thoughts and explorations. 
	Most importantly, all proofs presented in this document are of my own conception. 

	Terms like ``Theorem, Proposition'' are coined in Latin. As the English terms descended from Latin, most of them are self-explanatory. 
\end{abstract}

\chapter{Analysis}
\subfile{./analysis/realNumber.tex}

\subfile{./analysis/sequenceAndSeries.tex}

\subfile{./analysis/continuityAndLimit.tex}

\subfile{./analysis/differentiability.tex}

\chapter{Algebra}

\subfile{./algebra/groupDefinition.tex}

\subfile{./algebra/betweenGroup.tex}

\appendix
\subfile{appendix/appendix.tex}

\printbibliography
\end{document}
