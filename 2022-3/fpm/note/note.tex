\documentclass[12pt, a4paper]{report}
\usepackage{blindtext, titlesec, amsthm, thmtools, amsmath, amsfonts, scalerel, amssymb, graphicx, titlesec, xcolor}
\usepackage[utf8]{inputenc}
%\hypersetup{frenchlinks=true}
\newtheorem{theorem}{Theorema}[section]
\newtheorem{lemma}[theorem]{Lemma}
\newtheorem{corollary}{Corollarium}[section]
\newtheorem{propositio}{Propositio}[section]
\theoremstyle{definition}
\newtheorem{definition}{Definitio}[section]

\theoremstyle{definition}
\newtheorem{axiom}{Axioma}[section]

\theoremstyle{remark}
\newtheorem{remark}{Observatio}[section]
\newtheorem{hypothesis}{Coniectura}[section]

\renewcommand\qedsymbol{Q.E.D.}
\title{Notes on Pure Mathematics}
\author{Harry Han}
\date{\today}
\begin{document}
\maketitle
\tableofcontents

\newpage

\begin{abstract}
	This is my notes when taking the class \emph{Fundamentals of Pure Mathematics} at the University of Edinburgh. These notes do not follow the course lecture notes; instead, they are my thoughts and works inspired by the class.
	Terms like ``Theorem, Proposition'' are coined in Latin. Here is the list: theorema for theorem, Corollarium for corollary, Propositio for proposition, Definitio for definition, Conietura for hypothesis (conjecture).	
\end{abstract}
\chapter{Notation}

\begin{itemize}
	\item The \verb|\mathbb{}| fonts are used to denote sets. ($\mathbb{S}, \mathbb{Y},$ etc.)
	\item $\mathbb{A} \succ \mathbb{B}$ denotes there exits a surjective function $f:A\rightarrow B$. $\prec$, $\asymp$ denotes injective, bijective, respectively.
	\item $e$ is used to denote the identity of a group.
	\item When there is no ambiguity, the notation for the operation of group is ommited. (i.e., $a \odot b = ab$).
		$a^{-1}$ is used to denote the inverse of $a$.

\end{itemize}

\chapter{Analysis}
\begin{axiom}[The "Smallest" Infinite Set]
	A set $\mathbb{S}$ is infinite iff $\mathbb{S} \succ \mathbb{N}$. 
\end{axiom}

\begin{remark}
	Although FPM is a pure mathematic class with emphasis on rigor, no rigorous definition for the infinite set has been proposed. This definition/axiom is of my own conception.
\end{remark}

\begin{definition}[Countable Set]
	A set $\mathbb{S}$ is countable iff $\mathbb{N} \asymp \mathbb{S}$ (there exists a bijection $f:\mathbb{N} \rightarrow \mathbb{S}$).
\end{definition}

\begin{corollary}[At Most Countable]
	Let $\mathbb{A}$ be an infinite set. \\($\mathbb{A} \prec \mathbb{N}$) iff ($\mathbb{A} \asymp \mathbb{N}$).
\end{corollary}

\begin{proof}
	We want to prove $\mathbb{A} \prec \mathbb{N}$ is equivalent to $\mathbb{A} \asymp \mathbb{N}$. $\mathbb{A} \asymp \mathbb{N} \rightarrow \mathbb{A} \prec \mathbb{N}$ is by definition. 
	We only need to prove the other direction; i.e., provided $\mathbb{A} \prec \mathbb{N}$, find a bijective function $h: \mathbb{A} \rightarrow \mathbb{N}$.

	Let $f: \mathbb{A} \rightarrow \mathbb{N}$ be an injective mapping. If $f$ is bijective, we are done. If $f$ is injective but not bijective, let $\mathbb{N}^-$ be the range of $f.$ As $\mathbb{A}$ is infinite, $\mathbb{N}^-$ is also infinite.  
	%TODO: f' is not bijective, construct the correct bijective f'.
	Let $f': \mathbb{A} \rightarrow \mathbb{N}^-$ such that $f(a) = f'(a).$ $f'$ is an bijective mapping. 
	
    Thus we only need to show there exists a mapping $g: \mathbb{N}^- \rightarrow \mathbb{N}$ that is bijective. 

	$g$ can be constructed by such: sort $\mathbb{N}^-$ and $\mathbb{N}$ in ascending order. Let the first element in the sorted $\mathbb{N}^-$ maps to the first in the sorted $\mathbb{N}$, the secound to secound, etc.
	As $\mathbb{N}^-$ is infinite, $g$ must be bijective.

	Indeed $h = f' \circ g: \mathbb{A} \rightarrow \mathbb{N}$ is the bijective mapping we seek. 
\end{proof}

\begin{theorem}[List of Countable and Uncountable Sets]
	Any of the following sets are countable.
	\begin{enumerate}
		\item $\mathbb{Z}$ and any of its infinite subsets.
		\item $\mathbb{R}$ and any of its infinite subsets.
		\item If $\mathbb{S}$ is countable, $\{\mathbb{S}\times \mathbb{S}\}, \{\mathbb{S}\times \mathbb{S}\times \dots \times \mathbb{S}) \}$ are also countable.
	\end{enumerate}
\end{theorem}

\chapter{Algebra}

\begin{definition}[Group]
Group is a set $\mathbb{S}$ with an operation $\odot$ that fulfills the following four properties:
\begin{enumerate}
	\item Closure
	\item Associtivity: $(a \odot b) \odot c = a \odot (b \odot c)$;
	\item Identity
	\item Inverse
\end{enumerate}
\end{definition}
\begin{theorem}[Consequence of the Definition]
There are many non-obvious properties that directly follows the definition.
\begin{enumerate}
	\item General Associtivity: Parenthesis does not matter, as long as the order is the same: 
		$a \odot b \odot c \odot d \odot e \odot f \odot g \cdots  = (a \odot ((b \odot c) \odot e (\odot f \odot g)\cdots) = \cdots$
	\item Order of Inverse: $(a \odot b)^{-1}=b^{-1}\odot a^{-1}$.  
\end{enumerate}
\end{theorem}

Here are some examples of groups. 
\begin{enumerate}
	\item $\mathbb{S}=\{e\}$
	\item $\mathbb{S} = \{e, a, b, c\}.$ With the following operation:
		1. All elements are their own inverse; 
		2. The group is abelian.
		2. $a \odot b = c, a \odot c = b, b \odot c = a$.
\end{enumerate}
\begin{hypothesis}
	These are some of my hypothesis and thoughts.
	\begin{enumerate}
		\item different properties of odd finite groups and even finite groups
		\item If defining the reverto of the operation $\odot $ to be $\oslash $ as such: $a \odot b = a \oslash b^{-1}$. What are the sets such that it would be a group under both $\odot \& \oslash$?
		\item Can we have a set $\mathbb{S}$, such that under the operation $\odot $ we have $\forall a, b \in \mathbb{S}, a \odot b = b \odot  a$ but without associtivity? (Community without associtivity?)
	\end{enumerate}
\end{hypothesis}
\begin{definition}[Order of Group and element]
The order of the group $\mathbb{S}$ is $|\mathbb{S}|$ (How many elements it has). \\
The order of an element $s \in \mathbb{S}$ is the smallest integer $i$ such that $s^i = e.$ (If such $i$ exists)
\end{definition}

\begin{definition}[Cyclic Group]
Let $\mathbb{G}$ be a group and $g$ one of its element. Considering the set:
\[
	\mathbb{S} = \{\cdots g^{-2}, g^{-1}, e, g, g^1, g^2 \cdots\}  
\]
If $\mathbb{S}$ is finite, it is called a cyclic group. (It can be shown that it must be a subgroup of $\mathbb{G}$.
\end{definition}
\end{document}
