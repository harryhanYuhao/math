\documentclass[12pt, a4paper]{report}
\usepackage{blindtext, titlesec, amsthm, thmtools, amsmath, amsfonts, scalerel, amssymb, graphicx, titlesec, xcolor, multicol}
\usepackage[utf8]{inputenc}
\usepackage{biblatex}
\addbibresource{./reference/reference.bib}
%\hypersetup{frenchlinks=true}
\newtheorem{theorem}{Theorema}[section]
\newtheorem{lemma}[theorem]{Lemma}
\newtheorem{corollary}{Corollarium}[section]
\newtheorem{propositio}{Propositio}[section]
\theoremstyle{definition}
\newtheorem{definition}{Definitio}[section]

\theoremstyle{definition}
\newtheorem{axiom}{Axioma}[section]

\theoremstyle{remark}
\newtheorem{remark}{Observatio}[section]
\newtheorem{hypothesis}{Coniectura}[section]
\newtheorem{example}{Exampli Gratia}[section]

\renewcommand\qedsymbol{Q.E.D.}
\renewcommand{\chaptername}{Caput}
\renewcommand{\contentsname}{Index Capitum} % Index Capitum 
\usepackage{subfiles}
\title{
	DE MATHEMATICA PURA \\
	\large On Pure Mathematics
}
\author{Harry Han}
\date{\today}
\begin{document}
\maketitle
\tableofcontents

\newpage

\begin{abstract}
	These are my notes when taking the class \emph{Fundamentals of Pure Mathematics} at the University of Edinburgh. They are not a replicate of the lecture notes: they are my thoughts and explorations. 
	Most importantly, all proofs presented in this document are of my own conception. 

	Terms like ``Theorem, Proposition'' are coined in Latin. As the English terms descended from Latin, most of them are self-explanatory. 
\end{abstract}
\chapter{Notation}

\begin{itemize}
	\item The \verb|\mathbb{}| fonts are used to denote sets. ($\mathbb{S}, \mathbb{Y},$ etc.)
	\item $\mathbb{A} \succ \mathbb{B}$ denotes there exits a surjective function $f:A\rightarrow B$. $\prec$, $\asymp$ denotes injective, bijective, respectively.
	\item $e$ is used to denote the identity of a group.
	\item When there is no ambiguity, the notation for the operation of group is ommited. (i.e., $a \odot b = ab$).
		$a^{-1}$ is used to denote the inverse of $a$.

\end{itemize}

\chapter{Analysis}
\section{The Countable Sets}

\begin{axiom}[The "Smallest" Infinite Set]
	A set $\mathbb{S}$ is infinite iff $\mathbb{S} \succ \mathbb{N}$. 
\end{axiom}

\begin{remark}
	Although FPM is a pure mathematic class with emphasis on rigor, no rigorous definition for the infinite set has been proposed. This definition/axiom is of my own conception.
\end{remark}

\begin{definition}[Countable Set]
	A set $\mathbb{S}$ is countable iff $\mathbb{N} \asymp \mathbb{S}$ (there exists a bijection $f:\mathbb{N} \rightarrow \mathbb{S}$).
\end{definition}

\begin{corollary}[At Most Countable]
	Let $\mathbb{A}$ be an infinite set. \\($\mathbb{A} \prec \mathbb{N}$) iff ($\mathbb{A} \asymp \mathbb{N}$).
\end{corollary}

\begin{proof}
	We want to prove $\mathbb{A} \prec \mathbb{N}$ is equivalent to $\mathbb{A} \asymp \mathbb{N}$. $\mathbb{A} \asymp \mathbb{N} \rightarrow \mathbb{A} \prec \mathbb{N}$ is by definition. 
	We only need to prove the other direction; i.e., provided $\mathbb{A} \prec \mathbb{N}$, find a bijective function $h: \mathbb{A} \rightarrow \mathbb{N}$.

	Let $f: \mathbb{A} \rightarrow \mathbb{N}$ be an injective mapping. If $f$ is bijective, we are done. If $f$ is injective but not bijective, let $\mathbb{N}^-$ be the range of $f.$ As $\mathbb{A}$ is infinite, $\mathbb{N}^-$ is also infinite.  
	%TODO: f' is not bijective, construct the correct bijective f'.
	Let $f': \mathbb{A} \rightarrow \mathbb{N}^-$ such that $f(a) = f'(a).$ $f'$ is an bijective mapping. 
	
    Thus we only need to show there exists a mapping $g: \mathbb{N}^- \rightarrow \mathbb{N}$ that is bijective. 

	$g$ can be constructed by such: sort $\mathbb{N}^-$ and $\mathbb{N}$ in ascending order. Let the first element in the sorted $\mathbb{N}^-$ maps to the first in the sorted $\mathbb{N}$, the secound to secound, etc.
	As $\mathbb{N}^-$ is infinite, $g$ must be bijective.

	Indeed $h = g \circ f': \mathbb{A} \rightarrow \mathbb{N}$ is the bijective mapping we seek. 
\end{proof}

\begin{theorem}[List of Countable and Uncountable Sets]
	Any of the following sets are countable.
	\begin{enumerate}
		\item $\mathbb{Z}, \mathbb{Q}$ 
		\item Any infinite subset of countable sets.
		\item Any Unions of countable and finite sets.
		\item Any products of countable sets and finite sets. i.e., if $\mathbb{S}, \mathbb{T}$ are countable, $\{\mathbb{S}\times \mathbb{S}\}, \{\mathbb{S}\times \mathbb{T}\times \dots \times \mathbb{S} \}$ are also countable.
	\end{enumerate}
\end{theorem}

\begin{hypothesis}
Is the product of countable number of countable sets countable? (Proposed Feb 6)
\end{hypothesis}

\section{Sequence and Series}
\subsection{Sequence}
\begin{definition}[Sequence]
\end{definition}

\begin{definition}[Convergent and Divergent]
\end{definition}

\begin{definition}[Increasing and Decreasing Sequence]
\end{definition}

\begin{definition}[Cauchy Sequence]\cite{Ross}
	A sequence $(s_n)$ is a Cauchy Sequence iff $(\forall \epsilon > 0)(\exists N)(\forall n,m>N)(|s_n-s_m|<\epsilon)$
\end{definition}
\begin{theorem}[Properties of Cauchy Sequence]
Here are two important propeties:
\begin{enumerate}
	\item All convergent sequences are Cauchy Sequences;
	\item Cauchy Sequences are bounded;
\end{enumerate}
\end{theorem}

\subsection{Series}
\begin{definition}[Series]
\end{definition}
\begin{definition}[Convergent and Divergent]
\end{definition}
\begin{example}
List of Convergent and Divergent series:
\begin{enumerate}
	\item Harmonic Series.
\end{enumerate}
\end{example}
\begin{definition}[Cauchy Criterion]
\end{definition}
\begin{theorem}[Convergent Tests]
Here are the most common convergent test:
\begin{enumerate}
	\item Comparison test
	\item Ratio test
	\item Root test
\end{enumerate}
\end{theorem}



\chapter{Algebra}
\section{Group}

\begin{definition}[Group]
Group is a set $\mathbb{S}$ with an operation $\odot$ that fulfills the following four properties:
\begin{enumerate}
	\item Closure
	\item Associtivity: $(a \odot b) \odot c = a \odot (b \odot c)$;
	\item Identity
	\item Inverse
\end{enumerate}
\end{definition}
\begin{theorem}[Consequence of the Definition]
There are many non-obvious properties that directly follows the definition.
\begin{enumerate}
	\item General Associtivity: Parenthesis does not matter, as long as the order is the same: 
		$a \odot b \odot c \odot d \odot e \odot f \odot g \cdots  = (a \odot ((b \odot c) \odot e (\odot f \odot g)\cdots) = \cdots$
	\item Order of Inverse: $(a \odot b)^{-1}=b^{-1}\odot a^{-1}$.  
\end{enumerate}
\end{theorem}

Here are some examples of groups. 
\begin{enumerate}
	\item $\mathbb{S}=\{e\}$
	\item $\mathbb{S} = \{e, a, b, c\}.$ With the following operation:
		1. All elements are their own inverse; 
		2. The group is abelian.
		2. $a \odot b = c, a \odot c = b, b \odot c = a$.
\end{enumerate}
\begin{hypothesis}
	These are some of my hypothesis and thoughts.
	\begin{enumerate}
		\item different properties of odd finite groups and even finite groups
		\item If defining the reverto of the operation $\odot $ to be $\oslash $ as such: $a \odot b = a \oslash b^{-1}$. What are the sets such that it would be a group under both $\odot \& \oslash$?
		\item Can we have a set $\mathbb{S}$, such that under the operation $\odot $ we have $\forall a, b \in \mathbb{S}, a \odot b = b \odot  a$ but without associtivity? (Community without associtivity?)
	\end{enumerate}
\end{hypothesis}
\begin{definition}[Order of Group and element]
The order of the group $\mathbb{S}$ is $|\mathbb{S}|$ (How many elements it has). \\
The order of an element $s \in \mathbb{S}$ is the smallest integer $i$ such that $s^i = e.$ (If such $i$ exists)
\end{definition}

\begin{definition}[Cyclic Group]
Let $\mathbb{G}$ be a group and $g$ one of its element. Considering the set:
\[
	\mathbb{S} = \{\cdots g^{-2}, g^{-1}, e, g, g^1, g^2 \cdots\}  
\]
If $\mathbb{S}$ is finite, it is called a cyclic group. (It can be shown that it must be a subgroup of $\mathbb{G}$.
\end{definition}

\begin{theorem}[Properties of Cyclic Group]
Here are some properties immediately follows the definition.
\begin{enumerate}
	\item Any subgroup of a cyclic group is also cyclic.
\end{enumerate}	
\end{theorem}

\begin{theorem}[Lagrange Theorem]
Consider finite group $\mathbb{G}$ and its subgroup $\mathbb{S}$. $|\mathbb{S}|$ divides $|\mathbb{G}|.$	
\end{theorem}
\begin{example}
	The followings demonstrate Lagrange Theorem.
	\begin{enumerate}
		\item $\mathbb{Z}_{10}$ under addition modula 10 and its subgroup $\mathbb{S} = \{0,2,4,6,10\}$. 
		$|\mathbb{Z}_{10}|=10, |\mathbb{S}|=5$.  
	\end{enumerate}
\end{example}
\begin{proof}[Proof of Lagrange Theorem]
	Let $\mathbb{G}=\{g_1, g_2, g_3, \cdots \}$ be a group and $\mathbb{S}= \{s_0, s_1, s_2, \cdots\}$(let $s_0=e$) be its subgroup. 
	If $\mathbb{S}=\mathbb{G},$ we are done. If not, sine detrimento universalitatis(without loss of generality), let  $g_i \notin \mathbb{S}$. 
	Consider the set: $\mathbb{D}_1=\{g_1s|s \in \mathbb{S}\}$. The set $\mathbb{D}_1$ has the following properties:
	\begin{enumerate}
		\item $g_1s \in \mathbb{D}_1 \rightarrow g_1s \in \mathbb{G}$ 
		\item $|\mathbb{D}_1| = |\mathbb{S}|$. 
		\item $(\forall d \in \mathbb{D}_1)$ the set $\mathbb{D}_1'=\{ds|s\in \mathbb{S}\}=\mathbb{D}_1$
		\item $g_1s \in \mathbb{D}_1 \rightarrow g_1s \notin \mathbb{S}.$ 
	\end{enumerate}
	Property I is true because $\mathbb{G}$ is a group with the property closure.\\
	By claiming that $g_1s_i \neq g_1s_j$ for $i \neq j$ it is sufficently to show property II is true.
	
	To prove property III, we shall prove statement 1) $\mathbb{D}_1\subseteq \mathbb{D}_1'$ and 2) $\mathbb{D}_1'\subseteq \mathbb{D}_1$.
	To prove statement 1), consider $a\in \mathbb{D}_1$, $\exists s_1 \in \mathbb{S}$ such that $g_1s_1$. 
	Let $\mathbb{D}_1'$ be defined as $\mathbb{D}_1'=\{bs|s\in \mathbb{S}\}$. 
	and $b$ can be written in the form of $g_1s_2$. 
	Indeed $bs_2^{-1}s_1=a \rightarrow a\in \mathbb{D}_1' \rightarrow \mathbb{D}_1 \subseteq \mathbb{D}_1'.$ Statement 2) can be proved similarly.

	Property IV can be proved by contradiction. Assuming $\exists g_1s \in \mathbb{D}_1$ and $ g_1s \in \mathbb{S}.$ 
	We have $g_1ss^{-1} \in \mathbb{S}$ (by Inverse and Closure property of group) $\rightarrow g_1 \in \mathbb{S},$(by associtivity property of group) contradicting our assumption that $g \notin \mathbb{S}$.

	If $\mathbb{G}=\mathbb{S}\cup \mathbb{D}_1$, we are done, as $|\mathbb{G}| = 2|\mathbb{S}|$. 

	If $\exists g_2 \in \mathbb{G} \lor g_2 \notin \mathbb{S}, \mathbb{D}_1$. 
	Construct the set $\mathbb{D}_2=\{g_ws|s \in \mathbb{S}\}$. All elements in $\mathbb{D}_2$ have properties I, II of $\mathbb{D}_1$, and a stronger IV property: 
	$g_1s \in \mathbb{D}_2 \rightarrow g_1s \notin \mathbb{S}, \mathbb{D}_1.$.

	Thus by same reasoning, if $\mathbb{G}=\mathbb{S}\cup \mathbb{D}_1 \cup \mathbb{D}_2$, $|\mathbb{G}| = 3|\mathbb{S}|$.
	If not, we can constuct more disjoined sets $\mathbb{D}_3, \mathbb{D}_4, \cdots \mathbb{D}_n$ until the union of them and $\mathbb{S}$ forms $\mathbb{G}$. This can always be done as $\mathbb{G}$ is finite, and will have an order of $(n+1)\cdot |\mathbb{S}|. $
\end{proof}



\appendix
\subfile{appendix/appendix.tex}

\printbibliography
\end{document}
