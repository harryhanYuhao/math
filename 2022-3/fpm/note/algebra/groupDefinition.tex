\documentclass[../note.tex]{subfiles}
\begin{document}

\section{Group: Definitions}

\begin{definition}[Group]
Group is a set $\mathbb{S}$ with an operation $\odot$ that fulfills the following four properties:
\begin{enumerate}
	\item Closure: $\forall a,b\in \mathbb{S}, a\odot b\in \mathbb{S}$.
	\item Associtivity: $\forall a, b, c \in \mathbb{S}, (a \odot b) \odot c = a \odot (b \odot c)$;
	\item Identity: $\exists e\in \mathbb{S}$ such that $a\odot e = e\odot a = a$ for all $a\in \mathbb{S}$;
	\item Inverse: $\forall a\in \mathbb{S}, \exists a^{-1}\in \mathbb{S}$ such that $a\odot a^{-1} = e$.
\end{enumerate}

\begin{remark}
	Operation, $\odot $, within the set $\bb{S}$ can be defined as a function: $\odot: \bb{S}\times \bb{S} \rightarrow \bb{S}$ with the notation $\odot (a,b)=a \odot b$.

	The definition of identity can not be simplified to $\exists e\in \mathbb{S}$ such that $a\odot e = a$ for all $a\in \mathbb{S}$. 
	Nothing has prevented us from arbitrating that for a certain set $\bb{S}$ and $a, b \in \bb{S}$, $a \odot b = a$ while $b \odot a\neq a$.

	In contrast, the definition of inverse needs not such emphasis. If a set $\bb{S}$ has the property of closure, associtivity, and identity, $\forall a\in \mathbb{S} \exists a^{-1}\in \mathbb{S}$ such that $a\odot a^{-1} = e$ would implies that $a^{-1}a = e$. Here is a quick proof.

	Let $b = a^{-1}$.
	$a b=e \implies a b b = b a b = b$ (multiply both sides of $e$ with $b$). We know $b$ itself has an inverse, denoted as $c$; thus $a b b c = b a b c = b c$, and by associtivity, $a b (b c) = a b b c = b a b c = ba(bc) \implies a b = b a$ = $e$.  
\end{remark}

\emph{Note On Notation}: When there is no confusion, the symbol $\odot$ is omitted. \label{notation:Omission_Of_Group_Operator}

\end{definition}
\begin{theorem}[Consequence of the Definition]
There are many non-obvious properties that directly follow the definition.
\begin{enumerate}
	\item $\forall a, p \in \bb{G},$ there exists a unique $b \in \bb{G}$ such that $a b = p$.
	\item $\forall a, p \in \bb{G},$ there exists a unique $b \in \bb{G}$ such that $b a = p$.
	\item The Equation Rule: $\forall a, b, c \in \bb{G},$ $a b = a c \implies b = c$.
	\item The uniqueness of Inverse: All element of group has one and only one inverse.
	\item General Associtivity: Parenthesis does not matter, as long as the order is the same: 
		$a b c d e f g \cdots  = (a ((b c) e (f g)\cdots) = \cdots$
	\item Order of Inverse: $(a b)^{-1}=b^{-1}a^{-1}$. Indeed $(a_1a_2a_3\dots a_n)^{-1} = a_n^{-1}a_{n-1}^{-1}\dots a_1^{-1}$.
\end{enumerate}
\end{theorem}

\begin{hypothesis}
	These are some of my hypothesis and thoughts.
	\begin{enumerate}
		\item different properties of odd finite groups and even finite groups
		\item If defining the reverto of the operation $\odot $ to be $\oslash $ as such: $a \odot b = a \oslash b^{-1}$. What are the sets such that it would be a group under both $\odot \& \oslash$?
		\item Can we have a set $\mathbb{S}$, such that under the operation $\odot $ we have $\forall a, b \in \mathbb{S}, a \odot b = b \odot  a$ but without associtivity? (Community without associtivity?)
	\end{enumerate}
\end{hypothesis}
\begin{definition}[Order of Group and element]
The order of the group $\mathbb{S}$ is $|\mathbb{S}|$ (How many elements it has). \\
The order of an element $s \in \mathbb{S}$ is the smallest integer $i$ such that $s^i = e.$ (If such $i$ exists)
\end{definition}

\begin{definition}[Abelian Group]
	An abelian group is a group $\mathbb{S}$ such that $\forall a, b \in \mathbb{S}, a  b = b  a$.
\end{definition}

Here are some examples of groups. 
\begin{enumerate}
	\item The trivial group: $\mathbb{S}=\{e\}$
	\item The Group $\bb{Z}_n$. (Which is the only cyclic group after isomorphism)
	\item Dihedral Group 
	\item Permutation Group
\end{enumerate}

\begin{definition}[Subgroup]
	\hypertarget{def:subgroup}{A} subgroup of a group $\mathbb{S}$ is a subset $\mathbb{H}$ of $\mathbb{S}$ that is also a group under the same operation $\odot$. It is denoted as $ \mathbb{H}\leq \mathbb{S}$. If $\mathbb{H} \neq \mathbb{G}$, we call it a proper subgroup with notation $\mathbb{H}< \mathbb{S}$.
\end{definition}

\begin{theorem}[Test For Subgroup] 
	For group $\mathbb{S}$ and it subset$\mathbb{H} \subseteq \mathbb{S}$, $\mathbb{H}$ is a subgroup of $\mathbb{S}$ if and only if 
	\begin{enumerate}
		\item $\mathbb{H}$ is not empty 
		\item $\forall h,k \in \mathbb{H}, h\odot k^{-1} \in \mathbb{H}$.
	\end{enumerate}
\end{theorem}

\begin{theorem} [Immediate Consequence of Subgroup definition]
	Let $\bb{G}$ be a group and $\bb{S}\leq \bb{G}$. Let $g_i\in \bb{G}$ and $s_i \in \bb{S}$:
	\begin{enumerate}
		\item $g\notin \bb{S} \implies gs, sg \notin \bb{S}$; however, $g_1, g_2 \notin \bb{S}$ may not imply $g_1g_2 \notin \bb{S}$
	\end{enumerate}
\end{theorem}

Graphs can help us to construct/discover more examples of groups.

\begin{definition}[Graph]
	A graph is a finite set of vertices and edges connecting the vertices;
	or, with abstraction of set theory, a graph consists of two sets $\mathbb{V}$ and $\mathbb{E}$, where each element of $\mathbb{E}$ is an element of $\mathbb{V}\times \mathbb{V}.$
\end{definition}

\begin{definition}[Dihedral Group]
	The dihedral group of order $2n$ is the group of symmetries of a regular $n$-gon. It is the direct product of two copies of the cyclic group of order $n$.
\end{definition}

\begin{definition}[Cyclic Group]
Let $\mathbb{G}$ be a group and $g$ one of its element. Considering the set:
\[
	\mathbb{S} = \{\cdots g^{-2}, g^{-1}, e, g, g^1, g^2 \cdots\}  
\]
If $\mathbb{S}$ is finite, it is called a cyclic group. Most importantly, all sets of such form with inheritted operation must be a group.
\end{definition}

\begin{theorem}[Properties of Cyclic Group]
Here are some properties immediately follows the definition.
\begin{enumerate}
	\item All set in the form of the defintion of the cyclic group is a subgroup.
	\item \label{Cyclic_Inheritance}Any subgroup of a cyclic group is also cyclic.
\end{enumerate}	
\end{theorem}

\begin{proof}
	The first property would be obvious after the introduction of homomorphism \& isomophisim, as all cyclic group with $n$ elements are isomorphic to $\mathbb{Z}_n$.
	
	To prove the second property. Consider the cyclic group in the form of $e, g, g^2, \dots, g^n$. Let $\bb{S}$ be its subgroup, let $m$ be the smallest integer such that $g^m \in \bb{S}$ (such $m$ can always be found as $\bb{S}$ must not be empty and all cyclic groups are finite). We claim $m|n$, thus $\bb{S}$ is cyclic.

	If $m$ does not divides $n$, by eucilidean algorithm we can find $m'<m$ such that $am-bn=m'$, where $a,b$ are integers, which means $g^{m'}\in \bb{S}$, which contradicts our assumption.
\end{proof}

\begin{example}
An example of cyclic group is $\mathbb{Z}_n$. (Integer under modular $n$ addition);
\end{example}

\begin{proposition}[Number of Subgroup for $\mathbb{Z}_n$]
	The number of the subgroup of cyclic group $\mathbb{Z}_n$ is equal to the number of the divisor of $n$. (Proposed 27 Feb 2023)
\end{proposition}


\subsection{Equivalence Relation And Lagrange Theorem}

\begin{definition}[Relation]\label{def:relation}
	A relation on a set $X$ is a subset $R \subseteq X\times X$. $R$ contains the ordered pairs $(x,y)$ and we call $x$ and $y$ related with notation $x \sim y$.
\end{definition}

\begin{definition}[Equivalence Relation] 
An equivalence relation on a set $X$ is a relation $R$ on $X$ such that:
\begin{enumerate}
\item $R$ is reflexive: $x \sim x$ for all $x \in X$.
\item $R$ is symmetric: $x \sim y$ implies $y \sim x$ for all $x,y \in X$.
\item $R$ is transitive: $x \sim y$ and $y \sim z$ implies $x \sim z$ for all $x,y,z \in X$.
\end{enumerate}
\end{definition}


\begin{theorem}
	Equivalence relation partitions the set.
\end{theorem}

\emph{Note On Notation}
Let $A, B$ be subset of group $\bb{G}$ and let $g \in \bb{G}$. Then:
$AB = \{ab|a\in A, b\in B\}$ and $gA = \{ga|a\in A\}$. 

\begin{definition}[Coset]
	The left coset of a group $G$ is a set of the form $gG$ for some $g\in G$, and right coset is defined as $Gg$.
\end{definition}

\begin{lemma}
	Let $\bb{G}$ be a group and $S$ its subgroup. Define relation $a \sim b$ if and only if there exists $s \in S$ such that $as = b$. (i.e., $a,b$ belongs to the same cosets). This is an equivlence relation.  
\end{lemma}

\begin{theorem}
Cosets partitions the group. 
\end{theorem}

\begin{theorem}[Lagrange Theorem]
Consider finite group $\mathbb{G}$ and its subgroup $\mathbb{S}$. $|\mathbb{S}|$ divides $|\mathbb{G}|.$	
\end{theorem}


\begin{proof}[Proof of Lagrange Theorem]
	As cosets partitions the group, and the number of the element in each coset equals to the order of the subgroup, Lagrange Theorem follows.
\end{proof}

The following statement somewhat similar in form of Lagrange Theorem is \emph{FALSE}:
Provided  $n$ divides $  |\bb{G}|$, there exists $S \leq \bb{G}$ and $|S|=n$.

\begin{proposition}[Some Application of Lagrange Theorem]\label{prop:Applic_lagrange}
	Lagrange theorem is very powerful, here are some of its application.
\begin{enumerate}
	\item \label{orderOfElement}For a group $\mathbb{G}$ with order $p$ and $k \in \mathbb{G}$ with order $q$; then $q$ divides $p$.
	\item \label{prime_cyclic} For a group $\mathbb{G}$ with prime order (i.e., $|\mathbb{G}|$ is prime), it must be a cyclic group.
\end{enumerate}
\end{proposition}

\begin{proof}
	\ref{prop:Applic_lagrange}.\ref{orderOfElement} implies \ref{prop:Applic_lagrange}.\ref{prime_cyclic}.
	To prove proposition \ref{prop:Applic_lagrange}.\ref{orderOfElement}, let $\mathbb{G}$ be a group and $\langle g\rangle$ be a cyclic group containing $g \in \mathbb{G}$, by langrange theorem $|\langle g\rangle|$ must divides $|\mathbb{G}|$.
\end{proof}

\end{document}
