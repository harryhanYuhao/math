\documentclass[../note.tex]{subfiles}
\begin{document}

\section{Between Groups}

\subsection{Homomorphism and Isomorphism}

\begin{definition}[Homomorphism]
	Group $\bb{G}\ \& \ \bb{P}$ are isomorphic to each other if there exists a function (shall we add the following[proposed 14 mar]  defined for all $a \in \bb{G}$)
	$\phi: \bb{G}\rightarrow \bb{P}$ such that	$ a, b \in \bb{G} \implies \phi(ab) = \phi(a)\phi(b)$.

	The function $\phi$ is denominated as a homomorphism of the group $\bb{G}\&\bb{P}$.
\end{definition}

\begin{definition}[Isomorphism]
	A homomorphism $\phi \bb{G} \rightarrow \bb{P}$ that is also a bijection is a isomorphism of group $\bb{B}\&\bb{P}$, and we denote two such isomorphic group with $\bb{G} \cong \bb{P}$.
\end{definition}

\begin{theorem}[Consequence of homomorphism]
	Let $\phi: \bb{G} \rightarrow \bb{P}$ denote a homomorphism. 
\begin{enumerate}
	\item $\phi(e_{\bb{G}}) = e_{\bb{P}}$. 
	\item $\phi(a^{-1}) = \phi(a)^{-1}$ for $n\in \bb{N}$ (Including negative numbers).
	\item If $\phi$ is injection, the order of $\phi(a)$ is the same as the order of $a$.
\end{enumerate}
\end{theorem}

\subsection{Kernel and Image}
\begin{definition}[Image]
	The image of a homomorphism $\phi: \bb{G} \rightarrow \bb{P}$ is the set $\{\phi(c) |c \in \bb{G}\}$.
	Image is denoted as $\im{\phi}$.
	\emph{Image is a subset of $\bb{P}$.}
\end{definition}
\begin{definition}[Kernel] 
	The kernel of a homomorphism $\phi: \bb{G} \rightarrow \bb{P}$ is the set $\{c \in \bb{G} | \phi(c) = e_{\bb{P}}\}$.
	Kernel is denoted as $\ker{\phi}$.
	\emph{Kernel is a subset of $\bb{G}$.}
\end{definition}

\begin{theorem}[Properties of Kernel and Image]
	Let $\phi: \bb{G} \rightarrow \bb{P}$ be a homomorphism.
	\begin{enumerate}
		\item Both Kernel and Image are groups.
		\item Kernel is a \hyperlink{def:normal_group}{normal subgroup} of $\bb{G}$.
		\item For all $a\in \im{\phi}$, there exists exactly $n$ number of $b\in \bb{G}$ such that $\phi(b) = a$, where $n=|\ker{\phi}|$.
	\end{enumerate}
\end{theorem}

\begin{proof}
	The first two properties are obvious.

	To prove the third property: consider $\phi(g)=h$, then for all $e' \in \ker{\phi}$, $\phi( ge' )=h$. Thus there are at least $|\ker{\phi}|$ number of element in $\bb{G}$ such that $\phi(g)=h$. 
	If there exists another $\epsilon \notin \ker{\phi}$ such that $\phi(g \epsilon) = h$, we have $\phi(g \epsilon) = \phi(g)\phi(\epsilon) = h \implies \phi{\epsilon}=e_h$, i.e, $\epsilon \in \ker{\phi}$, a contradiction.
\end{proof}

\begin{hypothesis}[Proposed 21 Mar 2023]
	Let $\phi: G \rightarrow H$ be a homomorphism. Then there exists a subgroup $S < G$ such that $\phi$ is a isomorphism between $S$ and $im \phi$.
\end{hypothesis}

\subsection{Product and Polymorphism}

\begin{theorem}
	Let $\bb{H}, \bb{K} \leq \bb{G}$ be subgroups of $\bb{G}$ with the union of $\bb{H}, \bb{K} = \{e\}$
	\begin{enumerate}
		\item The map $\phi: \bb{H}\times \bb{K} \rightarrow \bb{H}\bb{K}$ given by $\phi(h,k) = hk$ is a bijection.
		\item $\bb{H}\bb{K}$ is a subgroup of $\bb{G}$ if and only if for all $h, k: $ $\exists h' k'$ such that $hk = k'h'$.
		\item $\bb{H}\bb{K}$ is isomorphic to $\bb{H} \times \bb{K}$ if and only if every element of $\bb{H}$ commutes with every element of $\bb{K}$, i.e. $\forall h, k: hk=kh$.
	\end{enumerate}
\end{theorem}

\begin{hypothesis}[Abelian Group]
	What is the relationship between abelian group and cyclic group? 

	I propose many abelian groups are isomorphic to cyclic group or product of cyclic group.
\end{hypothesis}

\section{Group Actions}
\begin{definition}[Group Action]\label{def:group_action} \hypertarget{def:group_action}{}
	A group action between a group $\bb{G}$ and a set $S$ is a function $f: \bb{G}\times S \rightarrow S$. 
	Let $\gamma \in \bb{G} \ \&\ s \in S$. Let $f(g,s)$ be denoted as $\gamma\cdot s$. Group action must fulfill the two following property:
	\begin{enumerate}
		\item $\epsilon\cdot s=s$ for all $s\in S$, with $\epsilon$ being the identity of the group $\bb{G}$
		\item $\gamma_1\cdot(\gamma_2 \cdot s) = (\gamma_1 \gamma_2) \cdot s$ for all $\gamma_1, \gamma_2 \in \bb{G}\ \&\ s \in S$
	\end{enumerate}
\end{definition}

\emph{Note On Notation:}
In this session, Let $G$ be a group actting on the set $S$. Let Greek letters $\alpha, \beta \dots \in G$, while Latin letter $a,b,c, \dots \in S$. 

Group actions are denoted by $\cdot$, while the sign for operation within the group are ommited when clear.

\begin{example}[Examples of Group Action]
	\ 
\begin{enumerate}
	\item A group $\bb{G}$ acts on itself in multiple ways, such as right multiplication, $g\cdot h=\gamma h$, left multiplication, $\gamma\cdot h = h\gamma^{-1}$, and conjugation, $\gamma\cdot h = \gamma h\gamma^{-1}$.
\end{enumerate}
\end{example}
\begin{theorem}[Immediate Consequence of the Definition]
	\ 
	\begin{enumerate}
		\item $\gamma \cdot a = b \implies \gamma^{-1} \cdot b = a$ 
	\end{enumerate}
\end{theorem}
\begin{definition}[Orbit] \label{def:orbit} \hypertarget{def:orbit_stabilizer}{}
	Let $G$ act on $S$ and $s \in S$. The orbit of $s$ is the set $\{\gamma\cdot s | \gamma \in G\}$. Orbit is denoted as $\orb{G}{s}$.
	\emph{Orbit is a subset of $S$}.
\end{definition}

\begin{definition}[Stabilizer]\label{def:stabilizer} 
	Let $G$ act on $S$ and $s \in S$. The stabilizer of $s$ is the set $\{\gamma \in G | \gamma\cdot s = s\}$. Stabilizer is denoted as $\stab{G}{s}$.
	\emph{Stabilizer is a subset of $G$}. 
\end{definition}

\begin{theorem}[Properties of Stabilizer and Orbit]
	\ 
	\begin{enumerate}
		\item $\stab{G}{s} \leq G$ (Stabilizer is a subgroup of $G$; N.B. Orbit is \emph{not} a group).
	\end{enumerate}
\end{theorem}

\begin{theorem}[Orbit Equivalence Relation]
	Let $G$ be a group acting on the set $S$.
	Define the relationship $a\sim b$ if $b\in \orb{G}{a}$.
	This is a equivalent relation; thus the orbits of the element of $S$ partitions $S$.
\end{theorem}

% TODO find a proper name for send x
\begin{definition}[Send x]
	If $a,b$ are in the same orbit. Define the set $\se{a}{b}=\{\gamma\in G| \gamma \cdot a=b \}$.
\end{definition}

\begin{lemma}
$\se{a}{b}$ is left coset of $\stab{\bb{G}}{a}$
\end{lemma}

\begin{proof}
	Assuming $a,b$ are in the same orbit, then $\exists \gamma \in \bb{G}$ such that $\gamma \cdot a = b$. 
	For all $\epsilon' \in \stab{\bb{G}}{a}$, we have $\gamma \epsilon' \cdot a = g $. Thus $\gamma\stab{\bb{G}}{a} \subseteq \se{a}{b}$.
	For all $\lambda \in \se{\bb{G}}{a}$, we can have  $\lambda = \gamma\lambda'$ and $\gamma\lambda' \cdot a = b \implies \lambda' \cdot a = \gamma^{-1} \cdot b=a$, which means $\lambda' \in \stab{\bb{G}}{a}$ and $\se{a}{b} \subseteq \gamma\stab{\bb{G}}{a}$.
\end{proof}

\begin{theorem}[Orbit Stabilizer Theorem]
	Let $G$ act on $X$ and $x \in X$. The stabilizer of $x$ is the set $\{g \in G | g\cdot x = x\}$.
	$|\orb{G}{x}||\stab{G}{x}|=|G|$
\end{theorem}

To prove Orbit Stabilizer Theorem, we need lemma \ref{lemmaOrbitStabilizerTheorem}.

\begin{lemma}\label{lemmaOrbitStabilizerTheorem}
	Let $G$ be a group acting on the set $S$.
\begin{enumerate}
	\item For $a$ and $b\in \orb{G}{a}$, $|\stab{G}{a}|=|\stab{G}{b}|$.
	\item Let set $S_b=\{g\in G| ga=b \}$. For all $b_1, b_2 \in \orb{G}{a}, |S_{b_1}|=|S_{b_2}|$.
\end{enumerate}
\end{lemma}

To prove the first part of the lemma, consider $\{a,b \cdots\}\in \orb{G}{a}$. By definition of orbit, we know $\exists \beta \in G $ such that $\beta a = b$.
Let $\stab{G}{a} = \{e_g, e_1, e_2 \cdots\}$. Notice that $\forall \epsilon \in \stab{G}{a}$, $\beta \epsilon \beta^{-1} \beta a = \beta \epsilon \beta^{-1} b = b$; i.e, $\beta \epsilon \beta^{-1} \in \stab{G}{b}$. 
Since $\epsilon_1 \neq \epsilon_2 \implies \beta \epsilon_1 \beta^{-1} \neq \beta \epsilon_2 \beta ^{-1}$, we conclide that $|\stab{G}{a}| \leq |\stab{G}{b}| $.

Such argument can be repeated for all $a',b' \in \orb{G}{a}$ such that $|\stab{G}{a'}| \leq |\stab{G}{b'}|$; the only possibility that is it true is that indeed they are all equal to each other.

\begin{proof}[Orbit Stabilizer Theorem]
	Let $G$ be a group acting on the set $S$. 
	Consider the pseudo set (that allows the repetition of the elements) $L=\{ag| g\in G\}$. 
	By lemma \ref{lemmaOrbitStabilizerTheorem} we claim $L=\orb{G}{a} \times |\stab{G}{a}|$, i.e, $L$ is made up of all element of orbit of $a$ repeated $|\stab{G}{a}|$ times. Thus $|G|=|L|=|\stab{G}{a}||\orb{G}{a}|$.
\end{proof}

\begin{theorem}[Cauchy's Theorem]
	Let $G$ be a group. Let $p$ be prime. If $p$ divides $|G|$, then $G$ contains a element of order $p$.
\end{theorem}

\end{document}
