\documentclass[../note.tex]{subfiles}
\begin{document}

\section{Between Groups}

\begin{definition}[Homomorphism]
	Group $\bb{G}\ \& \ \bb{P}$ are isomorphic to each other if there exists a function (shall we add the following[proposed 14 mar]  defined for all $a \in \bb{G}$)
	$\phi: \bb{G}\rightarrow \bb{P}$ such that	$ a, b \in \bb{G} \implies \phi(ab) = \phi(a)\phi(b)$.

	The function $\phi$ is denominated as a homomorphism of the group $\bb{G}\&\bb{P}$.
\end{definition}

\begin{definition}[Isomorphism]
	A homomorphism $\phi \bb{G} \rightarrow \bb{P}$ that is also a bijection is a isomorphism of group $\bb{B}\&\bb{P}$, and we denote two such isomorphic group with $\bb{G} \cong \bb{P}$.
\end{definition}

\begin{proposition}[Consequence of homomorphism]
	Let $\phi: \bb{G} \rightarrow \bb{P}$ denote a homomorphism. 
\begin{enumerate}
	\item $\phi(e_{\bb{G}}) = e_{\bb{P}}$. 
	\item $\phi(a^{-1}) = \phi(a)^{-1}$.
	\item If $\phi$ is injection, the order of $\phi(a)$ is the same as the order of $a$.
	\item $\{\phi(c) |c \in \bb{G}\}\leq \bb{P}$.
\end{enumerate}
\end{proposition}

\begin{definition}[Image]
	The image of a homomorphism $\phi: \bb{G} \rightarrow \bb{P}$ is the set $\{\phi(c) |c \in \bb{G}\}$.
\end{definition}
\begin{definition}[Kernel] 
	The kernel of a homomorphism $\phi: \bb{G} \rightarrow \bb{P}$ is the set $\{c \in \bb{G} | \phi(c) = e_{\bb{P}}\}$.
\end{definition}

\begin{definition}[Normal Group]
	A subgroup $\bb{H}$ of a group $\bb{G}$ is called normal if $\forall a \in \bb{G}$, the left and right coset of $\bb{H}$ are equal, i.e. $aG=Ga$.
\end{definition}

\end{document}
