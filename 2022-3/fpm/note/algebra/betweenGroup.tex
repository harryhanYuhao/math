\documentclass[../note.tex]{subfiles}
\begin{document}

\section{Between Groups}

\begin{definition}[Homomorphism]
	Group $\bb{G}\ \& \ \bb{P}$ are isomorphic to each other if there exists a function (shall we add the following[proposed 14 mar]  defined for all $a \in \bb{G}$)
	$\phi: \bb{G}\rightarrow \bb{P}$ such that	$ a, b \in \bb{G} \implies \phi(ab) = \phi(a)\phi(b)$.

	The function $\phi$ is denominated as a homomorphism of the group $\bb{G}\&\bb{P}$.
\end{definition}

\begin{definition}[Isomorphism]
	A homomorphism $\phi \bb{G} \rightarrow \bb{P}$ that is also a bijection is a isomorphism of group $\bb{B}\&\bb{P}$, and we denote two such isomorphic group with $\bb{G} \cong \bb{P}$.
\end{definition}

\begin{proposition}[Consequence of homomorphism]
	Let $\phi: \bb{G} \rightarrow \bb{P}$ denote a homomorphism. 
\begin{enumerate}
	\item $\phi(e_{\bb{G}}) = e_{\bb{P}}$. 
	\item $\phi(a^{-1}) = \phi(a)^{-1}$.
	\item If $\phi$ is injection, the order of $\phi(a)$ is the same as the order of $a$.
	\item $\{\phi(c) |c \in \bb{G}\}\leq \bb{P}$.
\end{enumerate}
\end{proposition}

\begin{definition}[Image]
	The image of a homomorphism $\phi: \bb{G} \rightarrow \bb{P}$ is the set $\{\phi(c) |c \in \bb{G}\}$.
\end{definition}
\begin{definition}[Kernel] 
	The kernel of a homomorphism $\phi: \bb{G} \rightarrow \bb{P}$ is the set $\{c \in \bb{G} | \phi(c) = e_{\bb{P}}\}$.
\end{definition}

\begin{definition}[Normal Group]
	A subgroup $\bb{H}$ of a group $\bb{G}$ is called normal if $\forall a \in \bb{G}$, the left and right coset of $\bb{H}$ are equal, i.e. $aG=Ga$.
\end{definition}

\begin{theorem}
	Let $\phi : G \rightarrow H$ be a group homomorphism.
\begin{enumerate}
	\item $\ker{\phi}$ is a normal subgroup of $G$. 
	\item $\phi: G \rightarrow H$ is injective if and only if $\ker{\phi} = \{e_{G}\}$.
	\item If $\phi$ is injective, it gives an isomorphism between $G$ and $im{\phi}$.
\end{enumerate}
\end{theorem}

\begin{hypothesis}[Proposed 20 Mar 2023]
	Let $\phi: G \rightarrow H$ be a homomorphism.
	Define kernel of $\phi$ respect to $a \in H$ as $\ker_a{\phi}=\{g\in G| \phi(g)=a\}$.
	For all $a, b$ such that $\ker_a{\phi}, \ker_b{\phi}$ is not an empty set, $\ker_a{\phi}=\ker_a{\phi}$ 
\end{hypothesis}

\begin{hypothesis}[Proposed 21 Mar 2023]
	Let $\phi: G \rightarrow H$ be a homomorphism. Then there exists a subgroup $S < G$ such that $\phi$ is a isomorphism between $S$ and $im \phi$.
\end{hypothesis}

\section{Group Actions}
\begin{definition}[Group Action]
	A group action between a group $\bb{G}$ and a set $S$ is a function $f: \bb{G}\times S \rightarrow S$. 
	Let $g\in \bb{G} \& s \in s$. Let $f(g,s)$ be denoted as $g\cdot s$. Group action must fulfill the two following property:
	\begin{enumerate}
		\item $e\cdot s=s$ for all $s\in S$, with $e$ being the identity of the group $\bb{G}$
		\item $g_1\cdot(g_2 \cdot s) = (g_1 g_2) \cdot s$ for all $g_1, g_2 \in \bb{G} \& s \in S$
	\end{enumerate}
\end{definition}
\begin{definition}[Orbit]
	Let $G$ act on $X$ and $x \in X$. The orbit of $x$ is the set $\{g\cdot x | g \in G\}$.	
\end{definition}

\begin{definition}[Stabilizer]
	Let $G$ act on $X$ and $x \in X$. The stabilizer of $x$ is the set $\{g \in G | g\cdot x = x\}$.
\end{definition}

\begin{theorem}[Orbit Stabilizer Theorem]
	Let $G$ act on $X$ and $x \in X$. The stabilizer of $x$ is the set $\{g \in G | g\cdot x = x\}$.
	$|\orb{G}{x}||\stab{G}{x}|=|G|$
\end{theorem}

\begin{theorem}[Cauchy's Theorem]
	Let $G$ be a group. Let $p$ be prime. If $p$ divides $|G|$, then $G$ contains a element of order $p$.
\end{theorem}

\end{document}
