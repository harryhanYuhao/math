\documentclass[12pt, a4paper]{article}
\usepackage{blindtext, titlesec, amsthm, thmtools, amsmath, amsfonts, scalerel, amssymb, graphicx, titlesec, xcolor}
\usepackage[utf8]{inputenc}
\newtheorem{theorem}{Theorem}[subsection]
\newtheorem{lemma}[theorem]{Lemma}
\newtheorem{corollary}[theorem]{Corollary}
\newtheorem{hypothesis}{Hypothesis}
\theoremstyle{definition}
\newtheorem{definition}{Definition}[section]

\theoremstyle{remark}
\newtheorem{remark}{Remark}[section]
\renewcommand\qedsymbol{Q.E.D.}

\title{Title}
\author{Harry Han; S2162783}
\date{\today}

\begin{document}
%\tableofcontents
\subparagraph{Q1}
We are to prove the sequence $\left(\displaystyle \frac{5n^7}{7^n}\right)_{n \in \mathbb{N}}$ converges. 
\begin{proof}
	We shall first prove the statement (1): $(n>11)\rightarrow (5n^7<5^n)$
	Statement 1 can be proved by induction. \\
	Base case: for $n=12$, $5n^7 = 179159040<5^n=244140625.$ \\
	Induction: Assuming $5n^7<5^n$.
	$(\forall n>11)$ $(\frac{n+1}{n}<\frac{12}{11}\approx 1.0909<1.2585\approx 5^{\frac{1}{7}})$
	$\rightarrow (\frac{n+1}{n})^7<5 \rightarrow 5(n+1)^7 < 5\cdot 5n^7<5\cdot 5^n=5^{n+1}.$
	And statement 1 is proved. 

	We also need statement 2, which is obvious: $(0<a<1, 0<n<m) \rightarrow (a^m<a^n$)

	$\forall \epsilon > 0$, let $l_1 = \displaystyle \frac{\ln{\epsilon}}{\ln{\frac{5}{7}}}$ and $l_2 = 12$. 
	Let $N$ be the greater of $l_1$ and $l_2$. 
	$$
	(n>N) \rightarrow 
	\underbrace{\frac{5n^7}{7^n}}_{\alpha}
	<\underbrace{\frac{5^n}{7^n}}_{\beta}
	<\underbrace{\frac{5^N}{7^N}}_{\gamma}
	\leq \underbrace{
	\left(\frac{5}{7}\right)^{\frac{\ln{\epsilon}}{\ln{\frac{5}{7}}}}
	}_{\delta}
	=\epsilon
	$$
	$\alpha<\beta$ is of statement 1.
	$\beta<\gamma\leq \delta$ is of statement 2.


	Thus we have proved that $(\forall \epsilon > 0) (\exists N \in \mathbb{N}) (\forall n>N) (\displaystyle \frac{5n^7}{7^n}<\epsilon)$.
\end{proof}

\newpage

\subparagraph{Q2}

Let $\mathbb{S}$ be a countable set, and let $\mathbb{E}_s$ be a countable sets for all $s \in \mathbb{S}$.
Define the set 
\[\mathbb{E} =\displaystyle \bigcup_{s\in \mathbb{S}}\mathbb{E}_s \]
We are to prove that $\mathbb{E}$ is countable by constructing a bijective mapping $\mathfrak{C}: \mathbb{N}\rightarrow \mathbb{E}$.

\begin{proof}
	By definition there exists a bijective function $f:\mathbb{N} \rightarrow \mathbb{S}$, and bijective function $g_s :\mathbb{N}\rightarrow \mathbb{E}_s$. Moreover, let us denote each element of the set $\mathbb{E}_s$ to be $E_s^{1}, E_s^2,\cdots.$ As there exists a bijective function $g$ between $\mathbb{N}$ and $\mathbb{E}_s$, $\{E_s^{i}|i\in \mathbb{N}\}=\mathbb{E}_s$

	We also know that there exists a bijective mapping $h:\mathbb{N}\rightarrow \mathbb{N}\times \mathbb{N}$. 
	
	Now consider the function defined thus: $m: \mathbb{N}\times \mathbb{N}\rightarrow E$ such that $m(a,b) = E^a_{f(b)}$. $m$ is bijective.

	The function $\mathfrak{C}=m\circ h: \mathbb{N} \rightarrow \mathbb{E}$ is bijective, as all of the function used to construct it are bijective.

	% Injectivity: Different $a\in \mathbb{N}$ maps to a different $(a', a'')\in \mathbb{N}\times \mathbb{N}$ which in in terms maps to a different $E_{f(a'')}^{a'}$.
	%
	% Surjectivity: for all $E_{f(a'')}^{a'}$, there exists $(a'', a') \in \mathbb{N}\times \mathbb{N}$ that maps to it, which is maps to by 


\end{proof}


\end{document}

