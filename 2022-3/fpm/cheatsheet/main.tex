\documentclass[12pt]{article}
\usepackage[tmargin=2cm,rmargin=2.5cm,lmargin=2.5cm,bmargin=2cm,footskip=0.4cm]{geometry} 
% Top margin, right margin, left margin, bottom margin, footnote skip
\usepackage[utf8]{inputenc}
\usepackage{biblatex}
\addbibresource{./reference/reference.bib}
% linktocpage shall be added to snippets.
\usepackage{hyperref,theoremref}
\hypersetup{
	colorlinks, 
	linkcolor={red!40!black}, 
	citecolor={blue!50!black},
	urlcolor={blue!80!black},
	linktocpage % Link table of content to the page instead of the title
}

\usepackage{blindtext}
\usepackage{titlesec}
\usepackage{amsthm}
\usepackage{thmtools}
\usepackage{amsmath}
\usepackage{amssymb}
\usepackage{graphicx}
\usepackage{titlesec}
\usepackage{xcolor}
\usepackage{multicol}
\usepackage{hyperref}
\usepackage{import}
\usepackage{bm}


\newtheorem{theorem}{Theorema}[section]
\newtheorem{lemma}[theorem]{Lemma}
\newtheorem{corollary}{Corollarium}[section]
\newtheorem{proposition}{Propositio}[theorem]
\theoremstyle{definition}
\newtheorem{definition}{Definitio}[section]

\theoremstyle{definition}
\newtheorem{axiom}{Axioma}[section]

\theoremstyle{remark}
\newtheorem{remark}{Observatio}[section]
\newtheorem{hypothesis}{Coniectura}[section]
\newtheorem{example}{Exampli Gratia}[section]

% Proof Environments
\newcommand{\thm}[2]{\begin{theorem}[#1]{}#2\end{theorem}}

%TODO mayby proof environment shall have more margin
\renewenvironment{proof}{\vspace{0.4cm}\noindent\small{\emph{Demonstratio.}}}{\qed\vspace{0.4cm}}
% \renewenvironment{proof}{{\bfseries\emph{Demonstratio.}}}{\qed}
\renewcommand\qedsymbol{Q.E.D.}
\renewcommand{\chaptername}{Caput}
\renewcommand{\contentsname}{Index Capitum} % Index Capitum 
\renewcommand{\emph}[1]{\textbf{\textit{#1}}}
\renewcommand{\ker}[1]{\operatorname{Ker}{#1}}

%\DeclareMathOperator{\ker}{Ker}

% New Commands
\newcommand{\bb}[1]{\mathbb{#1}} %TODO add this line to nvim snippets

% ALGEBRA
\newcommand{\orb}[2]{\text{Orb}_{#1}({#2})}
\newcommand{\stab}[2]{\text{Stab}_{#1}({#2})}
\newcommand{\im}[1]{\text{im}{\ #1}}
\newcommand{\se}[2]{\text{send}_{#1}({#2})}

%STATISTICS
\newcommand{\var}[1]{\text{Var}(#1)}
\newcommand{\ud}[1]{\underline{#1}}
\newcommand{\cor}[1]{\text{Cor}(#1)}
\newcommand{\std}[1]{\text{Std}(#1)}
\newcommand{\ste}[1]{\text{S.E.}(#1)}


\begin{document}
\section{Algebra}
\subsection{Graph}
\definition{
	Isomorphism between two graphs are bijection btween them that preserve all edges. 
	A symmytry is a isomorphism to itself.
}

\definition[Group]{
Group is a set $\mathbb{S}$ with an operation $\odot$ that fulfills the following four properties:
\begin{enumerate}
	\item Closure: $\forall a,b\in \mathbb{S}, a\odot b\in \mathbb{S}$.
	\item Associtivity: $\forall a, b, c \in \mathbb{S}, (a \odot b) \odot c = a \odot (b \odot c)$;
	\item Identity: $\exists e\in \mathbb{S}$ such that $a\odot e = e\odot a = a$ for all $a\in \mathbb{S}$;
	\item Inverse: $\forall a\in \mathbb{S}, \exists a^{-1}\in \mathbb{S}$ such that $a\odot a^{-1} = e$.
\end{enumerate}
}

\definition[Order of Group and element]{
The order of the group $\mathbb{S}$ is $|\mathbb{S}|$ (How many elements it has). \\
The order of an element $s \in \mathbb{S}$ is the smallest integer $i$ such that $s^i = e.$ (If such $i$ exists)
}

\definition[Abelian Group]{
	An abelian group is a group $\mathbb{S}$ such that $\forall a, b \in \mathbb{S}, a  b = b  a$.
}

\definition[Normal Group]{\label{def:normal_group}\hypertarget{def:normal_group}{}
	A subgroup $\bb{H}$ of a group $\bb{G}$ is called normal if $\forall a \in \bb{G}$, the left and right coset of $\bb{H}$ are equal, i.e. $a \bb{H}=\bb{H}a$.
}

\subsection{Group Property and Lagrange Theorem}

\begin{theorem}[Lagrange Theorem]
Consider finite group $\mathbb{G}$ and its subgroup $\mathbb{S}$. $|\mathbb{S}|$ divides $|\mathbb{G}|.$	
\end{theorem}

\theorem[Useful Facts]{
\begin{enumerate}
	\item \label{orderOfElement}For a group $\mathbb{G}$ with order $p$ and $k \in \mathbb{G}$ with order $q$; then $q$ divides $p$.
	\item \label{prime_cyclic} For a group $\mathbb{G}$ with prime order (i.e., $|\mathbb{G}|$ is prime), it must be a cyclic group.
	\item All groups with order smaller than 6 are abelian.
\end{enumerate}
}

\subsection{Homomorphism and Isomorphism}

\definition[Homomorphism]{
	Group $\bb{G}\ \& \ \bb{P}$ are isomorphic to each other if there exists a function 
	$\phi: \bb{G}\rightarrow \bb{P}$ such that	$ a, b \in \bb{G} \implies \phi(ab) = \phi(a)\phi(b)$.

	The function $\phi$ is denominated as a homomorphism of the group $\bb{G}\&\bb{P}$.
}

\definition[Isomorphism]{
	A homomorphism $\phi \bb{G} \rightarrow \bb{P}$ that is also a bijection is a isomorphism of group $\bb{B}\&\bb{P}$, and we denote two such isomorphic group with $\bb{G} \cong \bb{P}$.
}

\theorem[Consequence of homomorphism]{
	Let $\phi: \bb{G} \rightarrow \bb{P}$ denote a homomorphism. 
\begin{enumerate}
	\item $\phi(e_{\bb{G}}) = e_{\bb{P}}$. 
	\item $\phi(a^{-1}) = \phi(a)^{-1}$ for $n\in \bb{N}$ (Including negative numbers).
	\item If $\phi$ is injection, the order of $\phi(a)$ is the same as the order of $a$.
\end{enumerate}
}

\definition[Image]{
	The image of a homomorphism $\phi: \bb{G} \rightarrow \bb{P}$ is the set $\{\phi(c) |c \in \bb{G}\}$.
	Image is denoted as $\im{\phi}$.
	Image is a \emph{subgroup} of $\bb{P}$.
}

\definition[Kernel]{ 
	The kernel of a homomorphism $\phi: \bb{G} \rightarrow \bb{P}$ is the set $\{c \in \bb{G} | \phi(c) = e_{\bb{P}}\}$.
	Kernel is denoted as $\ker{\phi}$.
	Kernel is a \emph{normal subgroup} of $\bb{G}$.
}

\theorem[Properties of Kernel and Image]{
	Let $\phi: \bb{G} \rightarrow \bb{P}$ be a homomorphism.
	\begin{enumerate}
		\item For all $a\in \im{\phi}$, there exists exactly $n$ number of $b\in \bb{G}$ such that $\phi(b) = a$, where $n=|\ker{\phi}|$.
	\end{enumerate}
}

\theorem{
	Let $\bb{H}, \bb{K} \leq \bb{G}$ be subgroups of $\bb{G}$ with the union of $\bb{H}, \bb{K} = \{e\}$
	\begin{enumerate}
		\item The map $\phi: \bb{H}\times \bb{K} \rightarrow \bb{H}\bb{K}$ given by $\phi(h,k) = hk$ is a bijection.
		\item $\bb{H}\bb{K}$ is a subgroup of $\bb{G}$ if and only if for all $h, k: $ $\exists h' k'$ such that $hk = k'h'$.
		\item $\bb{H}\bb{K}$ is isomorphic to $\bb{H} \times \bb{K}$ if and only if every element of $\bb{H}$ commutes with every element of $\bb{K}$, i.e. $\forall h, k: hk=kh$.
	\end{enumerate}
}

\subsection{Group Action}
\definition[Group Action]{\label{def:group_action} \hypertarget{def:group_action}{}
	A group action between a group $\bb{G}$ and a set $S$ is a function $f: \bb{G}\times S \rightarrow S$. 
	Let $\gamma \in \bb{G} \ \&\ s \in S$. Let $f(g,s)$ be denoted as $\gamma\cdot s$. Group action must fulfill the two following property:
	\begin{enumerate}
		\item $\epsilon\cdot s=s$ for all $s\in S$, with $\epsilon$ being the identity of the group $\bb{G}$
		\item $\gamma_1\cdot(\gamma_2 \cdot s) = (\gamma_1 \gamma_2) \cdot s$ for all $\gamma_1, \gamma_2 \in \bb{G}\ \&\ s \in S$
	\end{enumerate}
}

\definition[Kernal Of Action]{
	Let $G$ be a group acting on set $X$.
	The set $\bb{N}:=\{g\in G | g\cdot x=x \text{ for all } x\in X\}$
	
	If the Kernel $\bb{N}=\{e\}$, then the action is \emph{faithful}.
}

\definition[Transitive Group Action]{
	An group action of $G$ on $X$ is transitive if for all $x,y\in X$, there exists $g\in G$ such that $g\cdot x=y$.
}

\begin{definition}[Orbit] \label{def:orbit} \hypertarget{def:orbit_stabilizer}{}
	Let $G$ act on $S$ and $s \in S$. The orbit of $s$ is the set $\{\gamma\cdot s | \gamma \in G\}$. Orbit is denoted as $\orb{G}{s}$.
	Orbit is a \emph{subset of $S$} and is \emph{NOT} a group.\\
\end{definition}

\begin{definition}[Stabilizer]\label{def:stabilizer} 
	Let $G$ act on $S$ and $s \in S$. The stabilizer of $s$ is the set $\{\gamma \in G | \gamma\cdot s = s\}$. Stabilizer is denoted as $\stab{G}{s}$.
	Stabilizer is a \emph{subgroup} of $G$. 
\end{definition}

\theorem[Orbit Counting Theorem]{
	Let $G$ be a group acting on set $S$. 
	Orbits \emph{partition} $S$ into disjoint sets.

	Number of Orbits = $\frac{1}{|G|}\sum_{s\in X} \stab{G}{s} $.
}

\begin{theorem}[Orbit Stabilizer Theorem]
	Let $G$ act on $X$ and $x \in X$. The stabilizer of $x$ is the set $\{g \in G | g\cdot x = x\}$.
	$|\orb{G}{x}||\stab{G}{x}|=|G|$
\end{theorem}

\begin{theorem}[Cauchy's Theorem]
	Let $G$ be a group. Let $p$ be prime. If $p$ divides $|G|$, then $G$ contains a element of order $p$.
\end{theorem}

\begin{theorem}[Sylow' First Theorem]
	Let $G$ be a group. Let $p$ be prime. If $p^k$ divides $|G|$, then $G$ contains a subgroup of order $p^k$.
\end{theorem}

\section{Analysis}

\subsection{Real Number}
\emph{Triangular Inequality} $|a+b|\leq |a+b|$; $||a|-|b||\leq |a-b|$

\begin{axiom}[The Completeness of Real Number] 
	All subsets of real number that is bounded above contains a supremum in the real number.
	Similarly, all subsets of real number that is bounded below contains an infimum in the real number.
\end{axiom}

\theorem[Archimedean Property]{
	$\forall a,b \in \bb{R},$ and $a<b$, $ \exists n \in \bb{N}$ such that $an>b$.
	Archimedean Property is \emph{not} an Axiom.
}

\definition[Injective, Surjective, and Bijective]{
	Let $f:A \rightarrow B$ be a function. $f$ is injective if $\forall a_1, a_2 \in A, f(a_1)=f(a_2) \implies a_1=a_2$. $f$ is surjective if $\forall b \in B, \exists a \in A$ such that $f(a)=b$. $f$ is bijective if it is both injective and surjective.
	Injective is also called one-to-one, and surjective is also called onto.
}

\begin{definition}[Finite, Countable Set, and Uncountable]
	Let $E$ be a set.
	\begin{enumerate}
		\item $E$ is said to be finite if either $E=\emptyset$ or there is an injective function $f:E \rightarrow \{1,2,\dots,n\}$ for some $n \in \bb{N}$.
		\item $E$ is said to be countable if there is a bijective function $f:E \rightarrow \bb{N}$.
		\item $E$ is at most countable if it is finite or countable.
		\item $E$ is uncountable if it is not at most countable.
	\end{enumerate}
\end{definition}

\theorem[At Most Countable]{ 
	If $f: N \rightarrow E$ is surjective, then $E$ is at most countable.
}

\subsection{Sequences and Series}

\theorem{
	A sequence converges if and only if all of its subsequence converges.
}
\theorem[Bolzano-Weierstrass Theorem]{
	Every bounded sequence has a convergent subsequence. 
}

\begin{theorem}[Convergence Reveries]\label{th:ConvergenceReveries}
\ 
\begin{enumerate}
	\item For convergent series $\sum^{\infty}_{k=1} s_k$, $\sum^{\infty}_{k=1}s'_k$, and constant $c$, all of the following sequence converges:
		$\sum^{\infty}_{k=1}-s_k$ $\sum^{\infty}_{k=1}c\cdot s_k$, $\sum^{\infty}_{k=1}s_k+s'_k$, $\sum^{\infty}_{k+1}s_k\cdot s'_k$.\\
		In particular, $\sum^{\infty}_{k=1}\frac{1}{s_k}$ diverges.\\
		$\sum^{\infty}_{k=1}\frac{s_k}{s'_k}$ may diverge or converge.
	\item \textbf{Convergence Test}:\\
		If $\lim_{n \rightarrow \infty} s_n \neq 0$, $\sum s_n$ diverges.
	\item \label{th:ConvergenceReveries:en:absoluteconverge} \textbf{Absolute Convergent}:\\
		If a series converges absolutely, it converges. The converse is not true. 
	\item \textbf{Comparison Test}:\\
		Suppose $0<a_k$ and $0<b_k$ for large $k$ and $L = \lim_{n \rightarrow \infty} \frac{a_n}{b_n}$ exists as an extended real number. Then:
		\begin{itemize}
			\item If $L=0$ and $\sum b_k$ converges, $\sum a_k$ converges.
			\item If $L=\infty$ and $\sum b_k$ diverges, $\sum a_k$ diverges.
			\item If $L \in (0, \infty)$, then $\sum a_k$ converges absolutely if and only if $\sum b_k$ converges absolutely.
		\end{itemize}
	\item \textbf{Ratio Test}:\\
		For series $\sum^{\infty}_{k=1}s_k$, let $d=\lim_{k \to \infty} |\frac{s_k}{s_{k-1}}|$.\\
		If $d<1$, the series converges absolutely.\\
		If $d>1$, the series diverges.\\
		If $d=1$, the series may converge or diverge.
	\item \textbf{Root Test}:\\
		For series $\sum^{\infty}_{k=1}s_k$, let $d=\lim_{k \to \infty} |s_k|^{1/k}$.\\
		If $d<1$, the series converges absolutely.\\
		If $d>1$, the series diverges.\\
		If $d=1$, the series may converge or diverge.
	\item \textbf{Alternating Series Test}: \label{AlternatingSeriesTest}\\
		For series in the form $\sum^{\infty}_{k=1}(-1)^{k}s_k$. If $s_k$ is decreasing and $\lim_{k\to \infty} s_k = 0$, the sereis converge. 
	\item \label{Cauchy_Condensation_Test} \textbf{Cauchy's Condenstation Test}:\\
		Consider series $\sum^{\infty}_{k=1}s_k$. If ${s_k}$ is decreasing and greater than zero, the seires converge if and only if $\sum^{\infty}_{k=1}s_{2^k}2^k $ converges. 
	\item \textbf{Integral Test}\label{IntegralTest}
		For nonnegative and decreasing $s_k$, the series $\sum^{\infty}_{k=1}s_k$ converge if and only if $\int_{a}^{\infty}S(k)dk$ converge for some constant $a$, provided $\forall k \in \mathbb{N}, S(k)=s_k$.
		(Proposed Feb 14 2023, modified and proved 16 Feb)
	\item \textbf{Raabe's Test}\label{Raabe's Test}
		For series $\sum^{\infty}_{k=1}s_k$, let $l= \displaystyle n\left(1-\frac{s_{n+1}}{s_n}\right)$. The series converge if $l>1$, diverges if $l<1$, and is inconclusive if $l=1$.

\end{enumerate}
\end{theorem}
\subsection{Continuity And Differentiability}

\begin{theorem}[Mean Value Theorem]
	If $f,g$ are continuous on $[a,b]$ and differentiable on $(a,b)$, then for some $c \in (a,b)$ we have
	\[
		f'(c)(g(b)-g(a))=g'(c)(f(b)-f(a))
	\]
	.
\end{theorem}

\theorem[Fermat's Theorem]{
	If a function attains a extrema at $a$, then either $f'(a)=0$ or $f'(a)$ does not exist.
}

\theorem[Darboux's Theorem]{
	If $f$ is differentiable on $(a,b)$, then $f'$ has the intermediate value property; i.e., if $f'(a)<c<f'(b)$, then there exists $x \in (a,b)$ such that $f'(x)=c$.
}

\theorem[L'Hopital's Rule]{
	Let $a \in \bb{R} \cup \{ \pm \infty \}$ and $f,g$ be real functions. If $A = \lim_{x \rightarrow a} f(x) = \lim_{x \rightarrow a} f(x)$ is either 0 or $\infty$ and $f,g$ are differentiable on $(a,b)\{a\}$, then 
	\begin{equation}
		\lim_{x \rightarrow a} \frac{f(x)}{g(x)} = \lim_{x \rightarrow a} \frac{f'(x)}{g'(x)}
	\end{equation}
}

\definition[Taylor Polynomial]{
	Taylor's polynomial $P_n^{f,x_0}(x)$ of degree $n$ of a function $f$ at $x_0$ is defined as 
	\begin{equation}
		P_n^{f,x_0}(x)=f(x_0)+\sum_{k=1}^{n}\frac{f^{(k)}(x_0)}{k!}(x-x_0)^k 
	\end{equation}

}

\theorem[Taylor's Formula]{
	Let $n\in \bb{N}$ and $a<b$ be real number. 
	If $f:(a,b) \rightarrow \bb{R}$ and $f^{(n+1)}$ exists on $(a,b)$, then for all each $x, x_0 \in (a,b)$ there exists $c \in (a,x)$ such that
	\begin{align}
	`	\begin{split}
			f(x)=f(x_0)+\sum_{k=1}^{n}\frac{f^{(k)}(x_0)}{k!}(x-x_0)^k + &\frac{f^{(n+1)}(c)}{(n+1)!}(x-x_0)^{n+1} = \\
			P_n^{f,x_0}(x)+ &\frac{f^{n+1}(c)}{(n+1)!}(x-x_0)^{n+1}
		\end{split}
	\end{align}
}

\end{document}
