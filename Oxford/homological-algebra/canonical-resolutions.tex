\section{Q6. Canonical Resolutions}

This section aims to solve the following problem.

\begin{problem}
	Let $R$ be a ring and $(F, U)$ the free-forgetful adjunction between the category between $\RM$ and $\S$.
	$\perp := FU$, therefore, is a comonad on $\RM$. 
	Pick an $M\in \RM$, we can construct a simplicial set $\perp^\bullet M$.
	Let $N$ be a right $R$-module, and define $E$ as $-\otimes_{R}$. Therefore $E(\perp^\bullet M)$ is a also an simplicial object.

	Recall Dold-Kan correspondence that an simplicial object induces a normalised chain complex via the functor $N$.
	Define $H_n(M, E)$ as the $n$-th homology of the normalised chain complex $N(E(\perp^\bullet M))$.
	Prove that 
	\begin{equation}\label{p6}
		H_n(M, E) \cong \T^R_n(M, N)
	\end{equation}
\end{problem}

The key to solve this problem is the fact that the chain complex $N(\perp^\bullet M)$ is a free resolution of $M$. 
Therefore \eqref{p6} follows from the definition of Tor-functor.

Recall each term in $N(\perp^\bullet M)$ is given by
\[
	N(\perp^\bullet M)_n = \bigcap_{i=1}^n \ker(d_i) \subset \perp^{n+1} M = (FU)^{n+1} M
\]
Since $F$ is free functor, each term in the chain complex is a free $R$-module.
So the only obstacle is to show the chain complex is exact.
\begin{theorem}\label{canonical-resolution-exact}
	For any $M \in \RM$, the normalised chain complex $N(\perp^\bullet M)$ is exact. 
\end{theorem}

This theorem is can be proved in several steps via the method in contractible simplicial objects.

\begin{definition}[Contractible]
	A simplicial object $X$ is \emph{contractible} if there exists maps $s_{f}: X_n \to X_{n+1}$ for each $n\geq 0$ such that
	\begin{align}
		\d_{n+1}f_{n} &= \id \label{contractible1} \\
		\d_{i}f_{n} &= f_{n-1}\d_{i} \quad \text{for } 0 \leq i \leq n \label{contractible2} 
	\end{align}
\end{definition}

The following lemma is evident.

\begin{lemma}\label{contractible-exact}
	If $X$ is a contractible simplicial object, then the associated normalised chain complex $N(X)$ is exact.
\end{lemma}

\begin{proof}
	Let $X$ be a contractible simplicial object with contracting homotopies $f_n: X_n \to X_{n+1}$.
	The associated normalised chain complex $N(X)$ has differential $d_n = \d_n|_{N(X)_n}$.
	
	Combining these we have the following diagram.
	\begin{equation*}
	\begin{tikzcd}
		\cdots \arrow{r} & 
		N_{n+1}(X) \arrow{r}{\d_{n+1}}&
		N_n(X) \arrow{r}{\d_n} \arrow[shift left]{l}{f_n}&
		N_{n-1}(X) \arrow{r}       \arrow[shift left]{l}{f_{n-1}}&
		\cdots 
	\end{tikzcd}
	\end{equation*}
	Here, we have use some abuse of notation as $\im{f_n}$ may not lie in $N_{n+1}(X)$. 

	Assuming $x \in N_n(X)$ with $\d_n x = 0$. 
	Since $N_n(X) = \bigcap_{i=0}^{n-1}\ker \d_i$, we conclude $\d_i x = 0$ for all $0 \leq i \leq n$. 
	By \eqref{contractible2}, we have $\d_i f_n(x) = f_{n-1} \d_i x = 0$ for all $0 \leq i \leq n$, that is $f_n(x) \in N_{n+1}(X)$.
	Again, applying \eqref{contractible1}, we have $\d_{n+1} f_n(x) = x$, so $x \in \im{\d_{n+1}}$.

	This shows that $\ker{\d_n} \subset \im{\d_{n+1}}$.
	Therefore the chain complex $N(X)$ is exact.
\end{proof}

The following lemma is also crucial.
\begin{lemma}\label{adjoint-contractible}
	Suppose that $U: \A \rightarrow \B$ has a left adjoint $F: \B \rightarrow \A$.
	Then for any $A \in \A$, the simplicial object $U(\perp_\bullet A)$ is contractible.
\end{lemma}

\begin{proof}
	Define $f_n = \eta U\perp^n$, we can see all the required conditions hold.
\end{proof}

From now the argument is easy.
Using the notation of lemma \ref{adjoint-contractible}, let $\A = \RM$, $\B = \S$ and $A = M$. 
Therefore $U(\perp_\bullet M)$ is a contractible simplicial set.
Since forgetful functor does not change the mapping or the underlying set structure, but only forgets the module structure, therefore $\perp_\bullet M$ is also contractible as a simplicial set.
Therefore, applying lemma \ref{contractible-exact}, the normalised chain complex $N(\perp^\bullet M)$ is exact, which completes the proof for theorem \ref{canonical-resolution-exact}.

\section{Q7. Relative Homology and Hochschild Homology}

Given a ring homomorphism $f: R \to k$, there is a forgetful functor $U: \MR \to \Mk$.
It has a left adjoint $F: \Mk \to \MR$ defined by $F(M) = M \otimes_k R$.
Therefore $(F, U)$ constittutes a comonad $\perp = FU$ on $\MR$, and thus we can construct a simplicial object $\perp_\bullet M$ for any $M \in \MR$.

By definition of $(F, U)$, it is evident that $\perp M = M \otimes_k R$, and thus $\perp_n = \perp^{n+1} M = M \otimes_k R^{\otimes n}$.
Note that Lemma \ref{adjoint-contractible} still holds in this case, so the normalised chain complex $N(\perp_\bullet M)$ is exact considered as complex of $k$-modules.

Based on this observation, we define Bar resolution as follows.

The relative $\T$ group $\T^{R/k}(M, N)$ is defined as the homology of the normalised chain complex $N(\perp_\bullet M \otimes_R N)$.
Similarly, the relative $\E$ group $\E_{R/k}^n(M, N)$ is defined as the cohomology of the normalised cochain complex $N(\H_R(\perp_\bullet M, N))$.

\begin{definition}[Bar Resolution] 
	Given a ring homomorphism $k \to R$ and an $R$-module $M$, and its associated comonad $\perp = FU$ on $\MR$ as defined above,
	the \emph{Bar resolution} of $M$ over $k$ is the unnormalised chain complex a $\perp_\bullet M$ considered as a chain complex of $k$-modules.

	Precisely, each term in the Bar resolution is given by 
	\[
		\beta_n(R, M) = \perp_n M = M \otimes_k R^{\otimes (n+1)}
	\]
\end{definition}

We are now ready to define Hochschild homology and cohomology.
\begin{definition}[Hochschild Homology and Cohomology]
	Let $k$, $R$ be rings with a ring homomorphism $k \to R$. 
	Then we have the free-forgetful adjunction $(F, U)$ between $\MR$ and $\Mk$ as above, and a comonad $\perp = FU$ on $\MR$.

	Define $R^e$ as the enveloping algebra $R \otimes_k R^{op}$.
	The \emph{Hochschild homology} $HH_n(R, M)$ is defined as the relative $\T$ group $\T^{R/k}_n(R^e, N)$.
	The \emph{Hochschild cohomology} $HH^n(R, M)$ is defined as the relative $\E$ group $\E_{R/k}^n(R^e, N)$.
\end{definition}

We want to solve the following problem. 
\begin{problem}
	If $R$ is flat as a $k$-module prove that 
	$$
	HH_n(R, N) \cong \T_n^{R^e}(R, N)
	$$
	where the right hand side is the usual Tor-functor over the ring $R^e$.
	If $R$ is projective as a $k$-module prove that 
	$$
	HH^n(R, N) \cong \E^n_{R^e}(R, N)
	$$
	where the right hand side is the usual Ext-functor over the ring $R^e$.
\end{problem}

\begin{proof}
	If each $R$ is flat as a $k$-module, then every $R^{\otimes n}$ is also a flat $k$ module, as tensor product preserves flatness.
	Therefore each term in the Bar resolution $\beta_n(R, R) = R^e \otimes_k R^{\otimes n}$ is also flat as an $R^e$-module by the change of ring flatness property.
	Therefore the Bar resolution is a flat resolution of $R^e$ as an $R^e$-module, and $HH^n(R, N)$ computes the $n$-th homology group of the complex $\beta_\bullet(R, R^e) \otimes_{R} N$, which is the same as $\T_n^{R^e}(R, N)$.

	One caveat is that $\T$ is usually defined via projective resolutions, but here we have flat resolution. 
	This creates no problem since derived functors forms a universal $\delta$-functor, and computation via flat resolutions will gives the same result.

	The proof for cohomology case is exactly the same by replacing flat modules with projective modules, and tensor product with Hom-functor.
\end{proof}

\section{Further Discussion}

This section provides some further discussion on the relation between Hochschild homology and Kahler differentials, and introduces Andre-Quillen cohomology.
Most of the content here is based on \cite{weibel}.

Let us first define Kahler differentials. 

\begin{definition}[Kahler Differentials]
	Let $k \to R$ be a ring homomorphism.
	The module of \emph{Kahler differentials} $\Omega_{R/k}$ is defined as the $R$-module generated by symbols $d r$ for each $r \in R$, subject to the following relations:
	\begin{align*}
		d(r_1 + r_2) &= dr_1 + dr_2 \\
		d(r_1 r_2) &= r_1 dr_2 + r_2 dr_1 \\
		d(k) &= 0 \quad \text{for all } k \in k
	\end{align*}
\end{definition}

If $M$ is a $k$-derivation is a $k-$module homomorphism $D: R \to M$ such that $D(r_1 r_2) = r_1 D(r_2) + r_2 D(r_1)$ for all $r_1, r_2 \in R$.
The map $d: R \to \Omega_{R/k}$ defined by $r \mapsto dr$ is a $k$-derivation.
Denote the set of all $k$-derivations from $R$ to $M$ as $\text{Der}_k(R, M)$. 
It is an $R$ module via $(rD)(r') = r D(r')$ for all $r, r' \in R$ and $D \in \text{Der}_k(R, M)$.


We have the following universal property. 
\begin{theorem}[Universal Property of Kahler Differentials]
	For any $R$-module $M$, there is a natural isomorphism
	\[
		\H_R(\Omega_{R/k}, M) \cong \text{Der}_k(R, M)
	\]
\end{theorem}

\subsection{HKR Theorem} 

We have the following statement at degree one. 

\begin{theorem}
For a $k$-algebra $R$, its module of Kahler differentials $\Omega_{R/k}$ is isomorphic to the first Hochschild homology $HH_1(R, R)$.
\end{theorem}

For degree greater than one, define degree-$n$ Kahler forms as follows.
\begin{definition}[Kahler n-forms]
	Let $k \to R$ be a ring homomorphism.
	The module of \emph{Kahler n-forms} $\Omega^n_{R/k}$ is defined as the $n$-th exterior power of $\Omega_{R/k}$:
	\[
		\Omega^n_{R/k} := \bigwedge^n_R \Omega_{R/k}
	\]
\end{definition}

By restricting to a well behaved class of rings, we have the following beautiful result connecting Hochschild homology and Kahler differentials.

\begin{theorem}[Hochschild-Kostant-Rosenberg theorem]
	If $k$ is a field and $A$ is a commutative $k$-algebra satisfying:
	\begin{enumerate}
		\item $A$ is finitely regenerated 
		\item The $A$ module of Kahler differentials $\Omega_{A/k}$ is projective
	\end{enumerate}
	Then there is an isomorphism of $A$-algebras 
	\[
		HH_n(A, A) \cong \Omega^n_{A/k}
	\]

	Dually, there is a isomorphism of Hochschild cohomology and wedge product of Kahler differentials:
	\[
		HH^n(A, A) \cong \bigwedge^n_A Der_k(A, A)
	\]
\end{theorem}

This theorem is proved by Weibel \cite{weibel} in Theorem 9.4.7.

\subsection{Andre-Quillen Cohomology}

We are now ready to state Andre-Quillen cohomology. 

\begin{definition}[Andre-Quillen Cohomology]
	Let $k \to R$ be a ring homomorphism, and $M$ an $R$-module.
	The \emph{Andre-Quillen cohomology} $D^n(R/k, M)$ with values in an $R-$module $M$ is a cotriple cohomology of $R$ wtih coefficients in $Der_k(-, M)$:
	\[
		D^n(R/k, M) := H^n(R; Der_k(-, M))
	\]
\end{definition}

Its relation to Hochschild theory is via the following theorm by Barr. 

\begin{theorem}[Barr]
	Suppose $C_\bullet(R)$ is an $R-$module chain complex, natural in $R$ for each commutative $k-$algebra $R$, such that 
	\begin{enumerate}
		\item $H_0(C_\bullet(R)) \cong \Omega(R/k)$ for all $R$
		\item If $R$ is a polynomial algebra over $k$, $C_\bullet(R) \rightarrow \Omega(R/k)$ is a split exact resolution 
		\item For each $p$ there is a functor $F_p: k-\textbf{mod} \rightarrow k-\textbf{mod}$ such that $C_p(R) \cong R\otimes_k F_p(UR)$, where $UR$ is the underlying $k-$module of $R$.
	\end{enumerate}
	Then there is a natural isomorphism 
	\[
		D^n(R/k, M) \cong H^n(\H_R(C_\bullet(R), M))
	\]
\end{theorem}
