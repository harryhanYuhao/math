\section{Q1. Existence of Hilbert Polynomial}

The aim of this section is to prove Hilbert polynomial of an variety exists.

We shall first define Hilbert's function. 
\begin{definition}[Hilbert's function]
	Let $R = k[x_0, \cdots, x_n]$ be a polynomial ring graded by degree over a field $k$.
	Let $N$ be a finitely generated graded $R$-module with graded decomposition 
	$$
		N = \bigoplus_{d \in \mathbb{Z}} N_d
	$$
	The Hilbert's function of $N$ is defined as
	$$
		H_N(k) = \dim_k(N_k)
	$$
	for all integers $d \in \mathbb{Z}$, where dimension of $N_k$ is calculated as a vector space over the field $k$.
\end{definition}

The surprising theorem is that there exists a polynomial $P_N(t) \in \mathbb{Q}[t]$ such that $P_N(k) = H_N(k)$ for all large enough integers $k$.

\begin{theorem}\label{thm:hilbert-polynomial}
Let $N$ be a finitely generated graded module over the polynomial ring $R_n := k[x_0, \dots, x_n]$. Then there exists a polynomial $P_N(t) \in \mathbb{Q}[t]$ of degree at most $n$ such that for all sufficiently large integers $k$:
\[ H_N(k) = P_N(k) \]
where $H_N(k)$ is the Hilbert function of $N$.
\end{theorem}

Our proof was inspired by \cite{blog}.

\begin{proof}
We proceed by induction on the number of variables $n$.

\textbf{Base Case:} Since the ring $R$ has $n+1$ indeterminants, we shall start with $n = -1$, where the ring is $R_{-1} = K$. 
Since $N$ is a finitely generated module over a field, it is a finite-dimensional vector space. 
By a standard result in linear algebra, 
$$
\dim(N \oplus M) = \dim(N) + \dim(M),
$$
the graded components $N_k$ must be zero for all sufficiently large $k$. 
Thus, the dimension function becomes eventually zero, which corresponds to the zero polynomial $P_N(t) \equiv 0$. 
This polynomial has degree $-\infty$ (or $0$), which satisfies the condition $\deg(P_N) \le -1$.

\textbf{Inductive Step:}
Assume the theorem holds for $n-1$. Let $N$ be a finitely generated graded module over $R_n$. Consider the module homomorphism defined by multiplication by $x_n$:
\[ \phi: N \to N, \quad m \mapsto x_n m \]
Since $\phi$ maps the graded component $N_k$ to $N_{k+1}$, we can examine the kernel and cokernel of this map. Let:
\begin{align*}
N' &= \ker(\phi) = \{m \in N \mid x_n m = 0\} \\
N'' &= \operatorname{coker}(\phi) = N / x_n N
\end{align*}
Both $N'$ and $N''$ are finitely generated graded $R_n$-modules. Crucially, the element $x_n$ annihilates both modules. Therefore, they can be viewed as finitely generated graded modules over the quotient ring $R_n / \langle x_n \rangle \cong k[x_0, \dots, x_{n-1}] = R_{n-1}$.

By the induction hypothesis, there exist polynomials $P_{N'}$ and $P_{N''}$ of degree at most $n-1$ such that for $k \gg 0$:
\[ \dim(N'_k) = P_{N'}(k) \quad \text{and} \quad \dim(N''_k) = P_{N''}(k) \]

Now, consider the restriction of multiplication by $x_n$ to the $k$-th graded component. We have an exact sequence of $\mathbb{K}$-vector spaces:
\[ 0 \longrightarrow N'_k \longrightarrow N_k \xrightarrow{\cdot x_n} N_{k+1} \longrightarrow N''_{k+1} \longrightarrow 0 \]

Using the additivity of dimension in exact sequences, we obtain:
\[ \dim(N_{k+1}) - \dim(N_k) = \dim(N''_{k+1}) - \dim(N'_k) \]

For sufficiently large $k$, we can substitute the Hilbert polynomials for the dimensions on the right-hand side:
\[ \dim(N_{k+1}) - \dim(N_k) = P_{N''}(k+1) - P_{N'}(k) \]

Let $f(k) = P_{N''}(k+1) - P_{N'}(k)$. Since $P_{N'}$ and $P_{N''}$ are polynomials of degree at most $n-1$, their difference $f(t)$ is a polynomial in $\mathbb{Q}[t]$ of degree at most $n-1$.

Lemma \ref{ap:diff-int-val} states that if a function $H(k)$ satisfies $H(k+1) - H(k) = f(k)$ for $k \gg 0$, where $f(t)$ is a polynomial of degree $d$, then $H(k)$ agrees with a polynomial of degree $d+1$ for large $k$.
This is our desired Hilbert polynomial $P_N(t)$ of degree at most $n$.
\end{proof}

For a projective variety $X \subset \P^n$, we can consider its homogeneous coordinate ring $S(X) = k[x_0, \cdots, x_n]/I(X)$, which is a finitely generated graded module over the polynomial ring $R_n = k[x_0, \cdots, x_n]$.
So our theorem above applies.
Let the Hilbert polynomial of this module be the Hilbert polynomial of the variety $X$.
In addition, we denote the Hilbert polynomial of $X$ as $P_X(t)$.

When regarding the Hilbert polynomial of a projective variety $X \subset \P^n$, if we apply an automorphism of $\P^n$, the Hilbert polynomial remains unchanged, as automorphisms of $\P^n$ induce isomorphisms of the corresponding homogeneous coordinate rings, which preserve the graded structure and dimensions of the graded components.
This is an important \emph{invariance} property of Hilbert polynomials.

However, as will be shown in the example \ref{iso-diff-hil}, two isomorphic varieties may have different Hilbert polynomials.

This is summarized in the following theorem. 

\begin{theorem}
	Hilbert polynomial of a projective variety $X \subset \P^n$ is unchanged under automorphisms of $\P^n$.
	However, two isomorphic varieties may have different Hilbert polynomials.
\end{theorem}

Since dimension is always a positive integer, the Hilbert polyomial must take positive values for large enough integers.
In the appendix we have studied in details the behaviours of such polynomials. In particular, theorem \ref{ap:th:int-val-decomp} and \ref{ap:le:int-val-poly} applies, and we have the following theorem.

\begin{theorem}
	Hilbert polynomial must take integer values for all integers. 
	A degree $d$ Hilbert polynomial can be expressed as 
	$$
		P(k) = c_d \binom{k}{d} + c_{d-1} \binom{k}{d-1} + \cdots + c_1 \binom{k}{1} + c_0 \binom{k}{0}
	$$
	where $c_0, c_1, \cdots, c_d$ are integers.
\end{theorem}

Note that $\binom{k}{d}$ can be regarded as a polynomial with indeterminant $k$ of degree $d$.
So the constant term of $P(k)$ is $c_0$.

\begin{corollary}[Integral Constant Term]\label{cor:int-const-term}
	Then $P(0) = c_0$ is an integer.
\end{corollary}

\subsection{Examples of Hilbert Polynomials}

\begin{example}[Hilbert Polynomial for a Point]\label{ex:hilbert-p-point}
	Let $X$ be a point in $\P^n$.
	The homogeneous ideal of $X$ is generated by $n$ linear polynomials.
	Therefore, the homogeneous coordinate ring of $X$ is 
	$$
		M = k[x_0, \cdots, x_n]/(f_1, \cdots, f_n) \cong k[y]
	$$
	where $y$ is an indeterminant.
	The grading of $M$ is inherited from the grading of $k[y]$, i.e., the $i$-th graded component $M_i$ is spanned by $y^i$.

	Therefore, for all non-negative integer $d$, 
	$$
		\dim{M_d} = 1
	$$
	and the Hilbert polynomial is
	$$
		P_X(k) = 1
	$$
	The polynomial has degree $0$.
\end{example}

\begin{example}[Hilbert Polynomial for $\P^n$]\label{example:hilbert-polynomial-pn}
	The Hilbert polynomial of the projective space $\P^n$ is the Hilbert polynomial associated with the module $M = R_n = k[x_0, \cdots, x_n]$ on by $n+1$ inderminate coordinates $[x_0: x_1: \ldots : x_n]$.
	The $i$-th graded component $M_i$ is the vector space spanned by all monomials of degree $i$ in the $n+1$ inderminants $x_0, \cdots, x_n$.

	Therefore 
	\begin{align*}
		M_0 &= \text{span} \{1 \} & \implies \dim{M_0} = 1 \\ 
		M_1 &= \text{span} \{x_0, x_1, \ldots, x_n \} & \implies \dim{M_1} = n+1 \\
		M_2 &= \text{span} \{x_0^2, x_0x_1, \ldots, x_n^2 \} & \implies \dim{M_2} = \frac{(n+1)(n+2)}{2}
	\end{align*}
	In general, the dimension of $M_d$ is the number of ways to distribute $d$ items into $n+1$ bins while allowing empty bins. 
	Standard combinatorial arguments therefore shows, as a function of the grading $k$,
	\begin{equation}\label{HP-Pn}
		P_{\P^n}(k) = \dim{M_k} = \binom{n+k}{n} = \frac{(n+k)!}{n!k!} = \frac{(n+k)(n+k-1)\cdots(k + 1)}{n!}
	\end{equation}
	And this is our desired Hilbert polynomial.
	In fact, the evaluation of this polynomial at any non-negative integer $d$ matches the dimension of $M_d$.

	In particular, $P_{\P^n}(k)$ is a polynomial of degree $n$, anmd it has leading term
	$$ \frac{k^n}{n!} $$

	It is clear that the above arguments also works if we consider the trivil inclusion $\P^n \subset \P^m$ for any $m \geq n$, where $\P^n$ is identified with the variety $V(x_{n+1}, \cdots, x_m) \subset \P^m$.
	So the Hilbert polynomial in this case is also given by equation \eqref{HP-Pn}.
\end{example}

\begin{example}\label{ex:hilbert-poly-hypersurface}
	Let $X \in \P^n$ be the variety $V(f)$, where $f$ is a homogeneous polynomial of degree $d$.
	The Hilbert polynomial of $X$ is associated to the module $M = k[x_0, \cdots, x_n]/(f)$.

	Let $R$ denote the ring $k[x_0, \cdots, x_n]$, and $R_k$ denote the $k$-th graded component of $R$, i.e., the vector space spanned by all monomials of degree $k$.
	We have the following short exact sequence of graded $R$-modules
	\[
		0 \rightarrow R_{k-d} \xrightarrow{\cdot f} R_k \rightarrow M_k \rightarrow 0
	\]
	where the middle map is the multiplication by $f$.
	Therefore,
	$$
		P_M(k) = \dim{M_k} = \dim{R_k} - \dim{R_{k-d}} = \binom{n+k}{n} - \binom{n+k-d}{n}
	$$
	Since $n, d$ are constants, this is a polynomial of $k$, and therefore the desired Hilbert polynomial.

	Let us compute its leading term.
	Write $\binom{n+k}{n} = \sum_{i = 0}^{n} a_i k^i$. 
	By previous example, $a_n = \frac{1}{n!}$.
	Therefore 
	$$
		P_M(k) = \sum_{i = 0}^{n} a_i k^i - \sum_{i = 0}^{n} a_i (k - d)^i = \sum_{i = 0}^{n} a_i \left( k^i - (k - d)^i \right)
	$$
	
	When $i = n$, the term in the summand is 
	$$
	\frac{1}{n!} \left( k^n - (k - d)^n \right) = \frac{1}{n!} \left( k^n - k^n + n d k^{n-1} - \binom{n}{2} d^2 k^{n-2} + \ldots \right)
	$$
	Which gives the term of highest degree
	$$
	\frac{dk^{n-1}}{(n-1)!}
	$$
	Since all other terms in the summand will not give a term with higer degree, this is the leading term.
\end{example}

\begin{example}\label{iso-diff-hil}
	The image of the degree-2 Veronese map of $\P$ to $\P^2$ is a conic defined by a single quadric equation.
	\[
		\nu_2: \P^1 \to \P^2, \quad [s:t] \mapsto [s^2 : s t : t^2]
	\]
	Its homogeneous ideal is generated by three quadrics: 
	\[
		I = (xy - z^2)
	\]

	The $k$-th grading is generated by all monimials of degree $k$ in $x, y, z$, modulo the relation $xy = z^2$, which is precisely the monomials involving $x$, $y$, and at most degree one of $z$.
	There are $k+1$ such monomials without $z$, and $k$ such monomials with $z$.
	So the hilbert polynomial is
	\[
		P(k) = 2k + 1
	\]

	It is clear the $k[x, y, z]/ (xy -z^2) \cong k[x, y]$, so this variety is isomorphic to $\P^1$. 
	Take note, however, $P_{\P_1}(k) = \binom{k+1}{1} = k + 1$, which is different from the hilbert polynomial computed above.
\end{example}

\begin{example}[Twisted Cubic]\label{ex:hilbert-poly-twisted-cubic}
	The twisted cubic is defined as the image of the degree-3 Veronese map of $\P^1$ into $\P^3$
	\[
		\nu_3: \P^1 \to \P^3, \quad [s:t] \mapsto [s^3 : s^2 t : s t^2 : t^3]
	\]
	Its homogeneous ideal is generated by three quadrics: 
	\[
		I = (x_1 x_3 - x_2^2, x_0 x_2 - x_1^2, x_0 x_3 - x_1 x_2 )
	\]

	Similar to the arguments in \ref{iso-diff-hil}, the coordinate ring $k[x_0, x_1, x_2, x_3]/I$ is spanned by $x_0, x_3$, and at most degree 1 of $x_1$ or $x_2$.
	
	This gives it hilbert polynomial to be $3k+1$.
\end{example}
