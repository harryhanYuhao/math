\section{Q1. Existence of Hilbert Polynomial}

The aim of this section is to prove Hilbert polynomial of an variety exists and has degree equal to the dimension of the variety.

We shall first define Hilbert's function. 
\begin{definition}[Hilbert's function]
	Let $R = k[x_0, \cdots, x_n]$ be a polynomial ring over a field $k$ graded by degree.
	Let $M$ be a finitely generated graded $R$-module with decomposition 
	$$
		M = \bigoplus_{d \in \mathbb{Z}} M_d
	$$
	The Hilbert's function of $M$ is defined as
	$$
		H_M(k) = \dim_k(M_k)
	$$
	for all integers $d \in \mathbb{Z}$.
\end{definition}

The surprising theorem is that there exists a polynomial $P_M(t) \in \mathbb{Q}[t]$ such that $P_M(k) = H_M(k)$ for all large enough integers $k$.

\begin{theorem}
Let $M$ be a finitely generated graded module over the polynomial ring $R_n := \mathbb{K}[x_0, \dots, x_n]$. Then there exists a polynomial $P_M(t) \in \mathbb{Q}[t]$ of degree at most $n$ such that for all sufficiently large integers $k$:
\[ H_M(k) = P_M(k) \]
where $H_M(k)$ is the Hilbert function of $M$.
\end{theorem}

\begin{proof}
We proceed by induction on the number of variables $n$.

\textbf{Base Case:} Since the ring $R$ has $n+1$ indeterminants, we shall start with $n = -1$, where the ring is $R_{-1} = \mathbb{K}$. 
Since $M$ is a finitely generated module over a field, it is a finite-dimensional vector space. 
By a standard result in linear algebra, 
$$
\dim(N \oplus M) = \dim(N) + \dim(M),
$$
the graded components $M_k$ must be zero for all sufficiently large $k$. 
Thus, the dimension function becomes eventually zero, which corresponds to the zero polynomial $P_M(t) \equiv 0$. 
This polynomial has degree $-\infty$ (or $0$), which satisfies the condition $\deg(P_M) \le -1$.

\textbf{Inductive Step:}
Assume the theorem holds for $n-1$. Let $M$ be a finitely generated graded module over $R_n$. Consider the module homomorphism defined by multiplication by $x_n$:
\[ \phi: M \to M, \quad m \mapsto x_n m \]
Since $\phi$ maps the graded component $M_k$ to $M_{k+1}$, we can examine the kernel and cokernel of this map. Let:
\begin{align*}
M' &= \ker(\phi) = \{m \in M \mid x_n m = 0\} \\
M'' &= \operatorname{coker}(\phi) = M / x_n M
\end{align*}
Both $M'$ and $M''$ are finitely generated graded $R_n$-modules. Crucially, the element $x_n$ annihilates both modules. Therefore, they can be viewed as finitely generated graded modules over the quotient ring $R_n / \langle x_n \rangle \cong \mathbb{K}[x_0, \dots, x_{n-1}] = R_{n-1}$.

By the induction hypothesis, there exist polynomials $P_{M'}$ and $P_{M''}$ of degree at most $n-1$ such that for $k \gg 0$:
\[ \dim(M'_k) = P_{M'}(k) \quad \text{and} \quad \dim(M''_k) = P_{M''}(k) \]

Now, consider the restriction of multiplication by $x_n$ to the $k$-th graded component. We have an exact sequence of $\mathbb{K}$-vector spaces:
\[ 0 \longrightarrow M'_k \longrightarrow M_k \xrightarrow{\cdot x_n} M_{k+1} \longrightarrow M''_{k+1} \longrightarrow 0 \]

Using the additivity of dimension in exact sequences, we obtain:
\[ \dim(M_{k+1}) - \dim(M_k) = \dim(M''_{k+1}) - \dim(M'_k) \]

For sufficiently large $k$, we can substitute the Hilbert polynomials for the dimensions on the right-hand side:
\[ \dim(M_{k+1}) - \dim(M_k) = P_{M''}(k+1) - P_{M'}(k) \]

Let $f(k) = P_{M''}(k+1) - P_{M'}(k)$. Since $P_{M'}$ and $P_{M''}$ are polynomials of degree at most $n-1$, their difference $f(t)$ is a polynomial in $\mathbb{Q}[t]$ of degree at most $n-1$.

Lemma \ref{ap:diff-int-val} states that if a function $H(k)$ satisfies $H(k+1) - H(k) = f(k)$ for $k \gg 0$, where $f(t)$ is a polynomial of degree $d$, then $H(k)$ agrees with a polynomial of degree $d+1$ for large $k$.
This is our desired Hilbert polynomial $P_M(t)$ of degree at most $n$.
\end{proof}

\begin{lemma}[Graded prime filtration]
Let $R = \bigoplus_{d \ge 0} R_d$ be a Noetherian $\mathbb{Z}$-graded ring and let
$N$ be a finitely generated graded $R$-module. Then there exists a finite
ascending chain of graded submodules
\[
0 = N_0 \subset N_1 \subset \cdots \subset N_t = N
\]
such that for each $j = 1,\dots,t$ there exist a homogeneous prime ideal
$\mathfrak{q}_j \subset R$ and an integer $a_j \in \mathbb{Z}$ with
\[
	N_j / N_{j-1} \cong (R/\mathfrak{q}_j)[a_j]
\]
as graded $R$-modules, where $M[a_j]$ denotee the shifting of grading by $a_j$, i.e., $M[c]_l = M_{c+j}$.
\end{lemma}

\begin{proof}
	Let $\Sigma$ denote the set of all submodules of $N$ which admits such a grading. 
	$\Sigma$ is non-empty since it contains the zero submodule. 
	Apply Zorn's lemma to get a maximal element $M$ in $\Sigma$.

	Let $N/M$ be non-zero, otherwise there are nothing to prove.
	Consider the set of annihilators of non-zero homogeneous elements in $N/M$:
	\[	\mathcal{A} = \{ \operatorname{Ann}_R(x) \mid x \in N/M \text{ is homogeneous and } x \neq 0 \}. \]
	Since $R$ is Noetherian, $\mathcal{A}$ has a maximal element, say $\mathfrak{q} = \operatorname{Ann}_R(x)$ for some homogeneous $x \in N/M$.
	We claim that $\q$ is a prime ideal. 
	
	First note that $\q$ must be a homogeneous idea. 
	Therefore we only need to consider homogeneous elements in $R$.
	Let $a,b \in R$ be homogeneous elements such that $ab \in \q$. 
	If $b \notin \q$, then by definition $a\in \Ann_R(bx) \supset \q$. 
	By maximality, therefore, $\Ann(bx) = \Ann_(x) = \q$ thus $a \in \q$.

	This prove the $\q$ is a homogeneous prime ideal. 
	Suppose $m \in N_l$. 
	Define the module $A$ as the maximal module such that $Am \subset N/M$.
	Therefore $A\cong (R/\q)[-l]$. Let $M' \in N$ be the preimage of $A$ in $N$, 
	we have 
	$$
	M'/M \cong (R /\q)[-l]
	$$
	Therefore there is a filtration $0 \subset M \subset M'$. This contradicts the maximality of $M$. Therefore $N/M = 0$ and the proof is complete.
\end{proof}

We will prove the following generalised theorem, which includes the existence of Hilbert polynomial for projective varieties as a special case.
This theorem is the \emph{answer} to both Q1 and Q2.

\begin{theorem}
	Let $N$ be a finitely generated graded module over the Neotherian graded ring $R = k[x_0, \cdots, x_n]$. 
	For large enough integers $k$, there exists a polynomial $P_N(t) \in \mathbb{Q}[t]$ such that $P_N(k) = H_N(k)$.
	Moreover, the degree of the Hilbert polynomial $P_N(t)$ equals $\dim V(\Ann(N))$.
\end{theorem}

\begin{proof}
	Let $\q$ be a homogeneous prime ideal of $R$. 
	We shall first prove the theorem for $N = R/\q$.
	There are two cases, $\q = (x_0, \cdots, x_n)$ or otherwise. 

	If $\q = (x_0, \cdots, x_n)$, then $P_M \equiv 0$. The convention is that the zero polynomial and the empty set both have dimension $-1$, so the theorem holds in this case.

	Assume $\q \neq (x_0, \cdots, x_n)$. Pick $x_i \notin \q$.
	Consider the exact sequence 
	$$
		0 \rightarrow R/\q[-1] \xrightarrow{\cdot x_i} R/\q \rightarrow R/(\q, x_i) \rightarrow 0.
	$$

	All the above homomorphisms are of degree 0, therefore we have 
	$$
		H_{R/\q}(t) - H_{R/\q}(t-1) = H_{R/(\q, x_i)}(t).
	$$
	and 
	$$
		V(\Ann(R/(\q, x_i))) = V(\q, x_i) = V(\q) \cap V(x_i).
	$$
	Since $x_i \notin \q$, we have $\dim V(\q, x_i) = \dim V(\q) - 1$. 
	By induction on the dimension, we have
	$$
		\deg(P_{R/(\q, x_i)}) = \dim V(\q, x_i) = \dim V(\q) - 1.
	$$
	Therefore 
	$$
		\deg(P_{R/\q}) = \deg(P_{R/(\q, x_i)}) + 1 = \dim V(\q, x_i) + 1 = \dim V(\q).
	$$
	Given an exact sequence  $0 \rightarrow N' \rightarrow N \rightarrow N'' \rightarrow 0$, and assume the theorem holds for $N'$ and $N''$. 
	Then we have $P_{N'} + P_{N''} = P_{N}$.
	Therefore the theorem holds for all $N = R/\q$.

	Furthermore
	$$
		V(\Ann(N)) =  V(\Ann(N')) \cup V(\Ann(N'')).
	$$
	Therefore 
	$
	\dim V(\Ann(N)) = \max\{\dim V(\Ann(N')), \dim V(\Ann(N''))\} = \deg(P_{N'} + P_{N''}) = \deg(P_N).
	$.


\end{proof}

