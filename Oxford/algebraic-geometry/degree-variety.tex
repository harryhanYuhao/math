\section{Degree of a Variety}

The degree if a variety is an important invariant in algebraic geometry. 
Its definition is
\begin{definition}[Degree of a Variety]\label{def:degree-variety}
	Let $V$ be a $k$-dimensional projective variety in $\mathbb{P}^n$.
	The degree of $V$, denoted $\deg(V)$, is defined as the largest number of points in the intersection of $V$ with a subspace $L$ of codimension $k$ in $\mathbb{P}^n$.
\end{definition}

Before dive deep into the theoretical discussion of the degree, let us first compute some examples.

\subsection{Examples of Degree}

\begin{example}[Degree of Hypersurface]
	Let $X \in P^n$ be a hypersurface corresponding to a degree $d$ polynomial $f$.
	The dimension of $X$ is $n-1$, so we investigate its intersection with a $1$-dimensional linear subspace, denoted as $L$, which is spaned by $n-1$ different linear polynomials $l_1, \cdots, l_{n-1}$.
	
	It is clear that to find the intersection of $L \cap X$ is to find the solution to the following system
	\[l_1 = \cdots = l_{n-1} = f = 0\]
	We can simply substite $l_i$ into $f$, which gives us a polynomial of two variables of degree $d$.

	Say the reduced polynomial is $g(x, y)$. The problem is equivalent to find number of its solution in $[x, y] \in \P^1$. 
	There are two cases to consider.

	If $[0, 1]$ is a solution, then $g$ can be written as $y^sf(x)$, where $f(x)$ has degree at most $d - 1$. By fundamental theorem of algebra, $f(x)$ has at most $d - 1$ solutions in $\P^1$, so in total there are at most $d$ solutions.

	If $[0, 1]$ is not a solution, then we can set $y = 1$ and consider $g(x, 1)$ as a polynomial of one variable $x$ with degree $d$.
	By fundamental theorem of algebra, it has at most $d$ solutions in $\P^1$.

	In both cases, we have at most $d$ solutions in $L \cap X$. 
	The maximum number of solutions is attained when we choose $L$ general linear position, so we have $\deg(X) = d$.
\end{example}

\subsection{Properties of Degree}

We have the following important theorem relating degree of varieties to their Hilbert polynomials.

\begin{theorem}\label{thm:degree-hilbert-polynomial}
	Let $V$ be a $k$-dimensional projective variety in $\mathbb{P}^n$ with Hilbert polynomial $P_V(t)$.
	If the leading coefficient of $P_V(t)$ is $c_k$, then
	\[\deg(V) = k! \cdot c_k\]
	where $k!$ denotes the factorial of $k$.
\end{theorem}

Lemma \ref{ap:int-n-fac} in the appendix proves the for any degree $k$ Hilbert polynpmial, the leading coefficient multiplied by $k!$ is always an integer. 
So this theorem makes sense from the first glance.

The proof of this theorem requires some sophisticated techniques in birational geometry, which was not covered in this course. 
So we only provide a sketch of the proof here.

\begin{proof}[Sketch of proof]
The idea of the proof is to reduce the general case to the hypersurface case via a birational map.

In example
\[
\dim_k (R)_t=\binom{t+n}{n}-\binom{t+n-d}{n}=\frac{d\,t^{\,n-1}}{(n-1)!}+\text{(lower)}.
\]
Thus a hypersurface of degree \(d\) has Hilbert polynomial whose leading term
is \(d\,t^{n-1}/(n-1)!\); i.e. \(\deg(X)=d\) appears as the coefficient times \((n-1)!\).\medskip

2. \emph{Reduction to the hypersurface case via generic linear projection.}
Choose a generic linear projection
\(\pi:\P^n\setminus H' \to H\cong\P^{m+1}\) with center a linear subspace \(H'\)
of codimension \(m+2\), so that \(\pi\) restricts to a birational map
\(\pi|_X: X \dashrightarrow X'=\pi(X)\subset H\) and \(X'\) is a hypersurface
in \(\P^{m+1}\). (Such a projection exists for generic choice of \(H'\) and \(H\).)
By birationality and standard results on linear projections, the degree of \(X\)
equals the degree of the hypersurface \(X'\), i.e. \(\deg(X)=\deg(X')\).\medskip

3. \emph{Comparison of Hilbert polynomials under finite integral extensions.}
Let \(R'=k[y_0,\dots,y_{m+1}]/I(X')\) be the homogeneous coordinate ring of
\(X'\). The inclusion $k[y_0,\dots,y_{m+1}]\hookrightarrow k[x_0,\dots,x_n]$
induces an inclusion $R'\hookrightarrow R_X$ which makes $R_X$ a finitely generated
$R'$-module (integral extension). From this one shows that
\[
\dim_k (R_X)_t = \dim_k (R')_t + O(t^{m-1})\qquad(t\to\infty),
\]
so the leading coefficient (the coefficient of $t^m$) of $P_X(t)$ equals that of
$P_{X'}(t)$. Combining this with step (1) applied to the hypersurface \(X'\)
gives
\[
P_X(t)=\deg(X')\,\frac{t^m}{m!}+\text{(lower degree terms)}=\deg(X)\,\frac{t^m}{m!}+\text{(lower)}.
\]
\end{proof}
