\section{Degree of a Variety}

The degree if a variety is an important invariant in algebraic geometry. 
Its definition is the following.
\begin{definition}[Degree of a Variety]\label{def:degree-variety}\cite{Invitation-AG}
	Let $V$ be a $k$-dimensional projective variety in $\mathbb{P}^n$.
	The degree of $V$, denoted $\deg(V)$, is defined as the largest number of points in the intersection of $V$ with a subspace $L$ of codimension $k$ in $\mathbb{P}^n$.
\end{definition}

Before dive deep into the theoretical discussion of the degree, let us first compute some examples.

\subsection{Examples of Degree}

\begin{example}[Degree of Point]\label{ex:degree-point}
	Let $P$ be a point in $\P^n$. It is of dimension $0$, its hypersurface of codimension $0$ is $\P^n$ itself. 
	Clearly there is always one intersection point, and $\deg(P) = 1$.

	Note that the Hilbert polynomial of a point is $P_P(t) = 1$, whose leading coefficient is $1 = c$.
	Thus $\deg(P) = 0! \cdot c$.
\end{example}

\begin{example}[Degree of Projective Space]
	Consider the projective space $\P^n$ as a subvariety of $\P^m$ via the trivial embedding $\P^n \cong Z(x_{n+1}, \cdots, x_m) \subset \P^m$ for any $m \geq n$.
	Let $L$ be the linear subspace of dimension $m - n$ defined by $L = Z(x_1, \cdots, x_n)$.
	
	There intersectin is precisely one point $[1, 0, \cdots, 0]$. So we have $\deg(\P^n) = 1$.
	
	Take note that the leading coefficient of the Hilbert polynomial of $\P^n$ is $\frac{1}{n!} = c$, and $\deg(\P^n) = n! \cdot c$.

	Note that the leading coefficient of the Hilbert polynomial of $\P^n$ is $\frac{1}{n!} = c$, and $\deg(\P^n) = n! \cdot c$.
\end{example}

\begin{example}[Degree of Hypersurfaces]\label{ex:degree-hypersurface}
	Let $X \in P^n$ be a hypersurface corresponding to a degree $d$ polynomial $f$.
	The dimension of $X$ is $n-1$, so we investigate its intersection with a $1$-dimensional linear subspace, denoted as $L$, which is spaned by $n-1$ different linear polynomials $l_1, \cdots, l_{n-1}$.
	
	It is clear that to find the intersection of $L \cap X$ is to find the solution to the following system
	\[l_1 = \cdots = l_{n-1} = f = 0\]
	We can simply substite $l_i$ into $f$, which gives us a polynomial of two variables of degree $d$.

	Say the reduced polynomial is $g(x, y)$. The problem is equivalent to find number of its solution in $[x, y] \in \P^1$. 
	There are two cases to consider.

	If $[1, 0]$ is a solution, then $g$ can be written as $y^sf(x)$, where $f(x)$ has degree at most $d - 1$. By fundamental theorem of algebra, $f(x)$ has at most $d - 1$ solutions in $\P^1$, so in total there are at most $d$ solutions.

	If $[1, 0]$ is not a solution, then we can set $y = 1$ and consider $g(x, 1)$ as a polynomial of one variable $x$ with degree $d$.
	By fundamental theorem of algebra, it has at most $d$ solutions in $\P^1$.

	In both cases, we have at most $d$ solutions in $L \cap X$. 
	The maximum number of solutions is attained when we choose $L$ general linear position, so we have $\deg(X) = d$.

	Take note that the leading coefficient of the Hilbert polynomial of $X$ is $\frac{d}{(n-1)!} = c$, and $\deg(X) = (n-1)! \cdot c$.

	In example \ref{ex:hilbert-poly-hypersurface}, we have shown that the leading term of Hilbert polynomial of $X$ is 
	$\frac{d}{(n-1)!}$.
	So we have $\deg(X) = (n-1)! \cdot c$.
\end{example}

\begin{example}[Degree of the Twisted Cubic]
	As shown in the previous example, the twisted cubic is the 1-dimensional projective variety defined as the image of the degree-3 Veronese map of $\P^1$ into $\P^3$
	\[
		\nu_3: \P^1 \to \P^3, \quad [s:t] \mapsto [s^3 : s^2 t : s t^2 : t^3]
	\]

	On the first coordinatec chart, the map is 
	\[
		\nu_3: \A^1 \to \A^3, \quad [s:1] \mapsto [s^3, s^2, s, 1]
	\]

	In fact, $V = \{(s^3, s^2, s^1)\} \in \A^3$ is the affine variety corresponding to the idea $(z^2-y, z^3 - x)$.
	Take a affine plane, say $z = x$. 
	There are three intersections, namely $(0, 0, 0), (1, 1, 1), (-1, 1, -1)$.

	So we conclude that the degree of the twisted cubic is $3$.

	Take note that the leading coefficient of the Hilbert polynomial of the twisted cubic is $\frac{3}{1!} = c$, and $\deg(X) = 1! \cdot c$.
\end{example}


\subsection{Properties of Degree}

In all the above examples, we found that the degree of the variety equals the leading coefficient of its Hilbert polynomial multiplied by the factorial of its dimension.
We give a sketch of the proof of this important theorem below.

\begin{theorem}\label{thm:degree-hilbert-polynomial}
	Let $V$ be a $k$-dimensional projective variety in $\mathbb{P}^n$ with Hilbert polynomial $P_V(t)$.
	If the leading coefficient of $P_V(t)$ is $c_k$, then
	\[\deg(V) =  c_k \cdot k!\]
	where $k!$ denotes the factorial of $k$.
\end{theorem}

Lemma \ref{ap:int-n-fac} in the appendix proves the for any degree $k$ Hilbert polynpmial, the leading coefficient multiplied by $k!$ is always an integer. 
So this theorem makes sense from the first glance.

The proof of this theorem requires some sophisticated techniques in birational geometry, which was not covered in this course. 
So we only provide a sketch of the proof here.

\begin{proof}[Sketch of proof]
The idea of the proof is to reduce the general case to the hypersurface case via a birational map.

In example \ref{ex:degree-hypersurface}, we have shown that the degree of a hypersurface corresponding to degree $d$ polynomial in \(\P^n\) is \(d\). 
In example \ref{ex:hilbert-poly-hypersurface}, we showed that its Hilbert polynomial is
\[
\binom{t+n}{n}-\binom{t+n-d}{n}=\frac{d\,t^{\,n-1}}{(n-1)!}+\text{(lower terms)}.
\]
In $\P^n$, the dimension of a hypersurface is $n-1$, so the leading coefficient multiplied by $(n-1)!$ equals $d$, which is the degree of the hypersurface.

We reduce the general case to hypersurfaces via general linear projections. 
For every pair of linear subspace $(H, H')$ in $\P^n$ such that $\dim(H) + \dim(H') = n - 1$ and $H' \cap H = \emptyset$. 
Can define a projection map \(\pi:\P^n\setminus H' \to H\cong\P^{m+1}\) which maps $x\in \P^n / H'$ to the unique point in $H$ that intersects with the subspace spanned by $x$ and $H'$.

Let $X$ be an irreducible subvariety of $\P^n$ with dimension $\leq n - 2$, let $H$ be an generic $m + 1$-dimensional linear subspace of $\P^n$, and let $H '$ be $ n - m -2$ dimensional linear subspace disjoint from $H$ and $X$, then the image of $X$ under the projection $\pi$ is birational to $X$. 
Also note that the image $X' = \pi(X)$ is a hypersurface in $\P^{m+1}$. 
So what it remains to show is that the degree and leading coefficient of Hilbert polynomial are preserved under such birational map.

Let $R$ and $R'$ be the homogeneous coordinate rings of $X \subset \mathbb{P}^n$
and $X' \subset \mathbb{P}^{m+1}$, respectively.  
The linear projection $\pi : X \dashrightarrow X'$ induces an inclusion of graded
rings
\[
R' \hookrightarrow R.
\]
Since $\pi$ is birational, $R$ is integral over $R'$, hence a finite graded
$R'$-module.

Therefore there exist homogeneous elements $f_1,\dots,f_r \in R$ such that
\[
R = \sum_{i=1}^r R' f_i.
\]
Consequently, there exists an integer $s>0$ such that for all $k \gg 0$,
\[
\dim_k R_{k-s} \;\le\; \dim_k R'_k \;\le\; \dim_k R_k.
\]
Equivalently, for the Hilbert polynomials,
\[
P_X(k-s) \;\le\; P_{X'}(k) \;\le\; P_X(k)
\quad \text{for all } k \gg 0.
\]

These inequalities imply that $P_X$ and $P_{X'}$ have the same degree and the
same leading coefficient.  Since $X'$ is a hypersurface of degree $d$ in
$\mathbb{P}^{m+1}$, we conclude
\[
P_X(t) = \frac{d}{m!} t^m + \text{(lower terms)}.
\]
\end{proof}


