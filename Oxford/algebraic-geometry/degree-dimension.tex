\section{Q2. Degree of Hilbert Polynomial and Dimension of Variety}

Q2 asks to prove that degree of Hilbert polynomial associated with a variety equalsto its dimension.

We first need the following lemma about graded prime filtration.

\begin{lemma}[Graded prime filtration]\label{graded-prime-filtration}
Let $R = \bigoplus_{d \ge 0} R_d$ be a Noetherian $\mathbb{Z}$-graded ring and let
$N$ be a finitely generated graded $R$-module. Then there exists a finite
ascending chain of graded submodules
\[
0 = N_0 \subset N_1 \subset \cdots \subset N_t = N
\]
such that for each $j = 1,\dots,t$ there exist a homogeneous prime ideal
$\mathfrak{q}_j \subset R$ and an integer $a_j \in \mathbb{Z}$ with
\[
	N_j / N_{j-1} \cong (R/\mathfrak{q}_j)[a_j]
\]
as graded $R$-modules, where $M[a_j]$ denotee the shifting of grading by $a_j$, i.e., $M[c]_l = M_{c+j}$.
\end{lemma}

\begin{proof}
	Let $\Sigma$ denote the set of all submodules of $N$ which admits such a grading. 
	$\Sigma$ is non-empty since it contains the zero submodule. 
	Apply Zorn's lemma to get a maximal element $M$ in $\Sigma$.

	Let $N/M$ be non-zero, otherwise there are nothing to prove.
	Consider the set of annihilators of non-zero homogeneous elements in $N/M$:
	\[	\mathcal{A} = \{ \operatorname{Ann}_R(x) \mid x \in N/M \text{ is homogeneous and } x \neq 0 \}. \]
	Since $R$ is Noetherian, $\mathcal{A}$ has a maximal element, say $\mathfrak{q} = \operatorname{Ann}_R(x)$ for some homogeneous $x \in N/M$.
	We claim that $\q$ is a prime ideal. 
	
	First note that $\q$ must be a homogeneous idea. 
	Therefore we only need to consider homogeneous elements in $R$.
	Let $a,b \in R$ be homogeneous elements such that $ab \in \q$. 
	If $b \notin \q$, then by definition $a\in \Ann_R(bx) \supset \q$. 
	By maximality, therefore, $\Ann(bx) = \Ann_(x) = \q$ thus $a \in \q$.

	This prove the $\q$ is a homogeneous prime ideal. 
	Suppose $m \in N_l$. 
	Define the module $A$ as the maximal module such that $Am \subset N/M$.
	Therefore $A\cong (R/\q)[-l]$. Let $M' \in N$ be the preimage of $A$ in $N$, 
	we have 
	$$
	M'/M \cong (R /\q)[-l]
	$$
	Therefore there is a filtration $0 \subset M \subset M'$. This contradicts the maximality of $M$. Therefore $N/M = 0$ and the proof is complete.
\end{proof}

We will prove the following generalised theorem, which includes the existence of Hilbert polynomial for projective varieties as a special case.
This theorem is the \emph{answer} to Q2.

\begin{theorem}
	Let $N$ be a finitely generated graded module over the Neotherian graded ring $R = k[x_0, \cdots, x_n]$. 
	The degree of the Hilbert polynomial $P_N(t)$ equals dimension of the variety defined by the annihilator ideal of $N$, i.e.,
	$$		\deg(P_N) = \dim V(\Ann(N))$$.
\end{theorem}

\begin{proof}
	Let $\q$ be a homogeneous prime ideal of $R$. 
	We shall first prove the theorem for $N = R/\q[l]$.

	Let us consider $N = R/\q$ first.
	There are two cases, $\q = (x_0, \cdots, x_n)$ or otherwise. 

	If $\q = (x_0, \cdots, x_n)$, then $P_M \equiv 0$. The convention is that the zero polynomial and the empty set both have dimension $-1$, so the theorem holds in this case.

	Assume $\q \neq (x_0, \cdots, x_n)$. Pick $x_i \notin \q$.
	Consider the exact sequence 
	$$
		0 \rightarrow R/\q[-1] \xrightarrow{\cdot x_i} R/\q \rightarrow R/(\q, x_i) \rightarrow 0.
	$$

	All the above homomorphisms are of degree 0, therefore we have 
	$$
		P_{R/\q}(t) - P_{R/\q}(t-1) = P_{R/(\q, x_i)}(t).
	$$

	It is evident that
	$$
		V(\Ann(R/(\q, x_i))) = V(\q, x_i) = V(\q) \cap V(x_i).
	$$
	Since $x_i \notin \q$, we have $\dim V(\q, x_i) = \dim V(\q) - 1$. 
	By an induction argument on the dimension, we may assume
	$$
		\deg(P_{R/(\q, x_i)}) = \dim V(\q, x_i) = \dim V(\q) - 1.
	$$
	Therefore, the polynomial $P_{R/\q}(t) - P_{R/\q}(t-1)$ has degree $\dim V(\q) - 1$, and so $P_{R/\q}(t)$ has degree $\dim V(\q)$ by lemma \ref{ap:diff-int-val}.

	It is evident that $P_{N}(t+l) = P_{N[l]}(t)$, so we conclude that the theorem holds for $N = R/\q[l]$.

	Given an exact sequence with $0$-degree maps $0 \rightarrow N' \rightarrow N \rightarrow N'' \rightarrow 0$. 
	Assume the theorem holds for $N'$ and $N''$,
	By simple arguments of dimension we must have $P_{N'} + P_{N''} = P_{N}$.
	Therefore the theorem holds for all $N = R/\q$.

	Furthermore
	$$
		V(\Ann(N)) =  V(\Ann(N')) \cup V(\Ann(N'')).
	$$
	Therefore 
	$
	\dim V(\Ann(N)) = \max\{\dim V(\Ann(N')), \dim V(\Ann(N''))\} = \deg(P_{N'} + P_{N''}) = \deg(P_N).
	$

	Let us now induct on the length of the graded prime filtration of $N$ as in 
	When the length is $1$, lemma \ref{graded-prime-filtration} states $N \cong R/\q[l]$, which is already proved. 

	If $N$ has filtration of length greater than $1$, let its filtration be 
	$$
	0 = N_0 \subset N_1 \subset \cdots \subset N_{t-1} \subset N_t = N.
	$$

	Consider the exact sequence
	$$
	0 \rightarrow N_{t-1} \rightarrow N \rightarrow N/N_{t-1} \rightarrow 0.
	$$

	$N_{t-1}$ has a graded prime filtration of length $t-1$, so induction hypothesis applies. 
	$N/N_{t-1} \cong R/\q[l]$, which is also proved. So by the previous argument the theorem holds for $N$.
\end{proof}
