\section{Bezout's Theorem}

In this section we will prove Bezout's theorem in $\P^2$

\begin{theorem}[Bezout's Theorem]
	Let $X, Y \subseteq \P^2$ be projective varieties over an algebraically closed field $k$ having degrees $\deg(X)$ and $\deg(Y)$ respectively.
	If $X$ and $Y$ intersect transversely, then
	Then $X\cap Y$ contains exactly $\deg(X) \cdot \deg(Y)$ points counting properly with multiplicities.
\end{theorem}

We will make the definition of intersection multiplicity precise using only algebraic tools. 
Althought this may lose some geometric intuition, it has the advantage of being precise and easy to define.

We first need an extension of lemma \ref{graded-prime-filtration}.

\begin{lemma} \label{graded-prime-filtration-2}
	Let $M$ be a finitely generated graded module over the Neotherian graded ring $R = k[x_0, \cdots, x_n]$.
	$M$ admits a filtration  
	$$
		0 = M^{(0)} \subset M^{(1)} \subset \cdots \subset M^{(r)} = M
	$$
	such that for each $i$, $M^{(i)}/M^{(i-1)} \cong (R/\q_i)[l_i]$ for some homogeneous prime ideal $\q_i$ and integer $l_i$.

	The filtration is not unique. However we have the following criterion 
	\begin{enumerate}
		\item If $\p$ is a homogeneous prime ideal, then $\p \subset \Ann(M) \iff \p \subset \q_i$ for some $i$. 
			In particular, the minimal element of set $\{\q_i\}$ is unique and equals the minimal prime ideal containing $\Ann(M)$.
		\item For each minimal prime ideal $\p$ containing $\Ann(M)$, the length of filtration $M_{\p}$ over $R_{\p}$ is the number of indices $i$ such that $\q_i = \p$.
	\end{enumerate}
\end{lemma}

\begin{proof}
	For the first part, it is clear that $\p \subset \Ann M \iff \p \subset \Ann (M_i/M_{i-1})$ for some $i$, which is equivalent to $\p \subset \q_i$.
 
	To prove the second part, note that $\p$ is minimal in the set $\{\q_i\}$, so after localisation, we have $M_{i,\p} = M_{i-1, \p}$ except when $\q_i = \p$, and in this case $M_{i,\p}/M_{i-1, \p} \cong (R/\p)_{\p} \cong k(\p)$, which is the quotient field of the local ring $R_{\p}$.
\end{proof}

\begin{definition}
	The \emph{multiplicity} of the module $M$ at the homogeneous prime ideal $\p$ is defined as the length of the localisation $M_{\p}$ over the local ring $R_{\p}$. It is denoted as $\mu_{\p}(M)$.
\end{definition}

Let $Y \subset \P^n$ be a projective variety of dimension $r$ defined by homogeneous ideal $I$.
Let $H$ be a hypersurface in not containing $Y$ defined by homogeneous ideal $J = (f)$.
We have $Y \cap H = Z_1 \cup Z_2 \cup \cdots Z_s$ of dimension $r-1$. 
Let $\p_i$ be the homogeneous prime ideal defining $Z_i$.
We define the \emph{intersection multiplicity} of $Y$ and $H$ at $Z_i$ as 
$$
	I(Y, H; Z_i) := \mu_{\p_i}(R/(I + J)).
$$

We have the following theorem. 

\begin{theorem}
	Let $Y$ be projective variety of dimension greater than 1 in $\P^n$, let $H$ be a hypersurface not containing $Y$.
	Let $Y \cap H = Z_1 \cup Z_2 \cup \cdots \cup Z_s$ be the irreducible components of dimension $\dim(Y) -1$
	$$
	\sum_{j=i}^s I(Y, H; Z_j) \cdot \deg(Z_j) = \deg(Y) \cdot \deg(H).
	$$
\end{theorem}
\begin{proof}
	Let $H = Z(f)$, where $f$ is a homogeneous polynomial of degree $\deg(H)$.
	Let $J$ be the homogeneous ideal generated by $f$.
	Let $Y = Z(I)$, where $I$ is a homogeneous ideal.
	We have the following exact sequence 
	$$
	0 \rightarrow R/I [-\deg(H)] \xrightarrow{\cdot f} R/I \rightarrow R/(I + J) \rightarrow 0.
	$$

	Since this sequence is exact, we have the following relation of their Hilbert polynomials 
	$$
		P_{R/I}(t) - P_{R/I}(t-\deg(H)) = P_{R/(I + J)}(t).
	$$

	$P_{R/I}(t)$ has the leading term
	$$
	\frac{\deg{Y}}{\dim(Y)!} t^{\dim(Y)}.	
	$$

	By simple computation we get the leading term of $P_{R/I}(t) - P_{R/I}(t-d)$ is 
	\begin{equation}\label{hilbert-polynomial-sum1}
	\frac{\deg{H} \cdot \deg{Y}}{(\dim(Y) - 1)!} t^{\dim(Y) - 1}.
	\end{equation}

	Consider the module $R/(I + J)$.
	By lemma \ref{graded-prime-filtration-2}, it admits a filtration
	$$
	0 = M^{(0)} \subset M^{(1)} \subset \cdots \subset M^{(r)} = R/(I + J)
	$$
	whose quotients are of the form $(R/\q_i)[l_i]$ for some homogeneous prime ideal $\q_i$ and integer $l_i$.
	In our proof of theorem \ref{graded-prime-filtration}, we showed that 
	\begin{equation}\label{hilbert-polynomial-sum2}
		P_{R/(I + J)}(t) = \sum_{i=1}^r P_{(R/\q_i)[l_i]}(t).
	\end{equation}

	If $Z(\q_i)$ is a projective variety of dimension $r_i$ and degree $d_i$, we have 
	$$
		P_{(R/\q_i)[l_i]}(t) = \frac{d_i}{r_i!} t^{r_i} + \text{lower degree terms}.
	$$
	As observed earlier, the shift in grading does not affect the leading term of the Hilbert polynomial.
	So we are left with those $P_{R/\q_i}$ such that $\q_i$ are minimal primes of $M$, i.e., one of $\p_i$ corresponding to filtration of $Z_i$. 
	Each of them takes place exaclty $\mu_{\p_i}(M) = I(Y, H; Z_i)$ times in the filtration by lemma \ref{graded-prime-filtration-2}.

	So we have the leading term of right hand side of equation \eqref{hilbert-polynomial-sum2} is
	\begin{equation}
		\sum_{j=1}^s I(Y, H; Z_j) \cdot \deg(Z_j) \frac{1}{(\dim(Y) - 1)!} t^{\dim(Y) - 1}
	\end{equation}
	Comparing this with equation \eqref{hilbert-polynomial-sum1}, we get the desired result.
\end{proof}

\begin{theorem}[Bezout's Theorem: Restatement]
	Let $X = Z(f)$, $Y = Z(g)$ be curves in $\P^2$ defined by homogeneous polynomials $f, g$ of degrees $\deg(f)$ and $\deg(g)$ respectively.
	Assume that $X \cap Y = \{P_1, \cdots, P_s\}$ consists of $s$ points.
	We have 
	$$
		\sum_{i=1}^s I(X, Y; P_i) = \deg(f) \cdot \deg(g).	
	$$
\end{theorem}

\begin{proof}[Proof of Bezout's Theorem]
	In example \ref{ex:degree-hypersurface} we have shown that the degree of the hypersurface coincides with the degree of their corresponding polynomials.
	Example \ref{ex:degree-point} shows the degree of a point is 1. 
	Plug in to the previous theorem gives us the result.
\end{proof}
