\section{Examples of Computing Hilbert Polynomials}

\begin{example}[Hilbert Polynomial for $\P^n$]

	The Hilbert polynomial of the projective space $\P^n$ is the Hilbert polynomial associated with the module $M = R_n = k[x_0, \cdots, x_n]$ on by $n+1$ inderminate coordinates $[x_0: x_1: \ldots : x_n]$.
	The $i$-th graded component $M_i$ is the vector space spanned by all monomials of degree $i$ in the $n+1$ inderminants $x_0, \cdots, x_n$.

	Therefore 
	\begin{align*}
		M_0 &= \text{span} \{1 \} & \implies \dim{M_0} = 1 \\ 
		M_1 &= \text{span} \{x_0, x_1, \ldots, x_n \} & \implies \dim{M_1} = n+1 \\
		M_2 &= \text{span} \{x_0^2, x_0x_1, \ldots, x_n^2 \} & \implies \dim{M_2} = \frac{(n+1)(n+2)}{2}
	\end{align*}
	In general, the dimension of $M_d$ is the number of ways to distribute $d$ items into $n+1$ bins while allowing empty bins. 
	Standard combinatorial arguments therefore shows, as a function of the grading $k$,
	\begin{equation}\label{HP-Pn}
		P_{\P^n}(k) = \dim{M_k} = \binom{n+k}{n} = \frac{(n+k)!}{n!k!} = \frac{(n+k)(n+k-1)\cdots(k + 1)}{n!}
	\end{equation}
	And this is our desired Hilbert polynomial.
	In fact, the evaluation of this polynomial at any non-negative integer $d$ matches the dimension of $M_d$.

	In particular, $P_{\P^n}(k)$ is a polynomial of degree $n$ with leading term
	$ \frac{k^n}{n!} $.
	The degree of the Hilbert polynomial matches the dimension of $\P^n$. 
	As we will define later, the degree of the projective variety is defined as $n!$ times the leading coefficient of the Hilbert polynomial, which is $1$ for $\P^n$.
	The arithmetic genus of a variety $X$ is defined as $(-1)^{\dim(X)}(P_X(0) - 1)$. 
	For $\P^n$, the evaluation of $P_{\P^n}$ at $0$ is $\binom{n}{n} = 1$, and the arithmetic genus is $0$.
\end{example}

\begin{example}
	Let $X \in \P^n$ be the variety $V(f)$, where $f$ is a homogeneous polynomial of degree $d$.
	The Hilbert polynomial of $X$ is associated to the module $M = k[x_0, \cdots, x_n]/f$.

	Let $R$ denote the ring $k[x_0, \cdots, x_n]$, and $R_k$ denote the $k$-th graded component of $R$, i.e., the vector space spanned by all monomials of degree $k$.
	We have the following short exact sequence of graded $R$-modules
	\[
		0 \rightarrow R_{k-d} \xrightarrow{\cdot f} R_k \rightarrow M_k \rightarrow 0
	\]
	where the middle map is the multiplication by $f$.
	Therefore,
	$$
		P_M(k) = \dim{M_k} = \dim{R_k} - \dim{R_{k-d}} = \binom{n+k}{n} - \binom{n+k-d}{n}
	$$
	Since $n, d$ are constants, this is a polynomial of $k$, and therefore the desired Hilbert polynomial.
	As in the previous example, the degree of this polynomial is $n$.

	Let us compute the its leading term.
	Write $\binom{n+k}{n} = \sum_{i = 0}^{n} a_i k^i$. 
	By previous example, $a_n = \frac{1}{n!}$.
	Therefore 
	$$
		P_M(k) = \sum_{i = 0}^{n} a_i k^i - \sum_{i = 0}^{n} a_i (k - d)^i = \sum_{i = 0}^{n} a_i \left( k^i - (k - d)^i \right)
	$$
	
	When $i = n$, the term in the summand is 
	$$
	\frac{1}{n!} \left( k^n - (k - d)^n \right) = \frac{1}{n!} \left( k^n - k^n + n d k^{n-1} - \binom{n}{2} d^2 k^{n-2} + \ldots \right)
	$$
	Which gives the term of highest degree $\frac{dk^{n-1}}{(n-1)!}$.
	Since all other terms in the summand will not give a term with higer degree, this is the leading term, and the degree of the variety $X$ is $dn$.

	Finally, the arithmetic genus of $X$ is 
	$$
		(-1)^{n-1} (P_M(0) - 1) = (-1)^{n-1} \left( \binom{n}{n} - \binom{n - d}{n} - 1 \right) = (-1)^{n} \binom{n - d}{n}
	$$

	By lemma \ref{ap:binom-neg}, if $ 0 \leq d \leq n$,  $\binom{n - d}{n} = 0$, and the arithmetic genus is $0$.
	If $d > n$, by the same lemma $\binom{n - d}{n} = (-1)^n \binom{d - 1}{n}$, and the arithmetic genus is $\binom{d - 1}{n}$.
\end{example}

